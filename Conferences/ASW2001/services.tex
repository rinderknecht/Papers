\section{Services to be studied}
 
As we have mentioned in the previous sections, one of the objectives of the 
\textsc{Platonis} project is to define a methodology and an
architecture for the validation and experimentation of services
related to users mobility. Once this methodology and architecture
will be defined, we plan to study new services, and particularly those
based on WAP and IP. We plan to use terminals (cellular phones and
PDAs) or terminal emulators allowing a direct access to terminal
functions. 

Two services will be studied, one based on the subscriber location,
and another based on a distant network management. This latter has
been proposed by one of the industrial partners of the project. From
these services, a set of scenarios will be generated automatically or
manually. These scenarios will allow to test the interoperability and
also to detect errors related to non expected or erroneous
messages. For instance, in the case of the service of a distant
network management, the visualization of the equipment's deployment
can be troubled by a connection cut. If the user continues to move, he
will need to have a good synchronization between the visualization of
the equipment's deployment and his new geographical location.

The proposed methodology will allow through these two examples to extend the 
possibilities of testing other types of services. We envisage to develop a 
demonstrator to show how to validate the proposed services. These services 
will be in the demonstrator and will be experimented by the partners. A 
pedagogical evaluation is also planned. Students and researchers will 
receive a training of the use of the platform. Practical works and projects 
in the framework of this training will be contributed to evaluate the use of 
the platform. 

It is also expected that service providers will be able to test their
services and configurations (for instance, services described using
WML) through the use of the \textsc{Platonis} platform.
