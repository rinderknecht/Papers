\section{Test methodology}
\subsection{Interoperability test architectures}

We have mentioned in section~\ref{validation} that the \textsc{Platonis}
project aims at two kinds of tests: interoperability
testing and layer conformance testing. Figure~\ref{fig:fig1} shows test
architectures for interoperability testing in the
framework of the WAP. It is designed to validate and experiment new
services related to mobility.

This study is focused on the WAP architecture because this technology
is nowadays available but it is possible to extend our methodology to
other ones, like GPRS and UMTS, when they are widely available.

The proposed test architectures use several distributed access points
with a local WAP gateway. The behavior of each entity is
observed through a Point of Observation (PO) and controlled through a
Point of Control and Observation (PCO). Three levels of
interoperability tests will be performed: the first one uses a PCO to
control and observe the terminal exchanges, and two POs in the heart of
the network, between the components (figure~\ref{fig:fig1}~(a)), to
detect transmission errors and to perform some traffic analysis.
%\footnote{In ideal case we should analyze all data
%frames. Unfortunately, in practice, this procedure is complicated and
%expensive, so we will just check the header of each frame.}. 
This architecture can also be used for the network performance
evaluation. In the second one we will observe and analyse the log
files. A PCO is (still) on the terminal side, and a PO is located in
the server side (figure~\ref{fig:fig1}~(b)) to check the behavior of
the server. The last one considers the network as a black box that is
observed and controlled through the PCO of the terminal
(figure~\ref{fig:fig1}~(c)). This test architecture will be adapted to
test the application use-cases and QoS properties.


\subsection{A layer conformance test architecture}

In the \textsc{Platonis} project another type of test will be studied:
the conformance testing of a layer of the WAP stack. For this, a
test architecture is proposed in figure~\ref{fig:fig2}.

According to the WAP specification it is possible to
access each WAP layer directly through service access points (SAPs),
which facilitates the observation of the provided services of each
layer~\cite{wapforum}. However, it depends on the
availability of application programming interfaces (APIs) of each
layer. If such APIs are not available, we will consider embedded
system testing where target protocol is accessed through
context~\cite{hit}. Concerning the specifications, France T�l�com R\&D
will modelise the WSP and WTP layers in SDL.


