\section{Migration to GPRS}
 
For the first step of the project we will use the GSM network because
this technology is nowadays used but when the GPRS and UMTS 
are available we will migrate to those technologies. 

The following step will be the incorporation of the GPRS packet-based
interface on the existing circuit-switched GSM network. This
incorporation will keep the use of the existing services. It will
also facilitate several new applications that have not previously been
available due to the limitation in speed of the circuit-switched data
(9.6~kbps)\footnote{A theoretical maximum speed up to~171.2~kbps is
achievable with GPRS.}. In case of WAP application, the difference
between GPRS and pure GSM network is the transport and the physical
layers, as shown in figure~\ref{fig:fig3}~(a). In general, the WAP
services will not change.

The GPRS also will allow Internet applications to be executed on
mobile terminals \emph{without the WAP stack}, as shown in
figure~\ref{fig:fig3}~(b). It means that many services which are used
over the wired Internet today will be available over the mobile 
network~\cite{GPRS}.

%The GPRS also enables mobile Internet functionality by allowing
%interworking between the existing Internet and the new GPRS
%network. 

