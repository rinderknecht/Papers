\section{Introduction}

\subsection{Objectives}

The new trends in network technology (Internet, ATM) lead to the
design of new protocols and services. In most of cases the latter
shall interconnect heterogeneous elements which must be tested in
order to certify their interoperability. Both interoperability testing
and experimentation with these new products are hence becoming
strategic activities in the telecom software industry, not only for
the operators but also for equipment vendors and tool providers.

Companies have to develop an important activity in order to warranty
the correct behaviours of their software implementing the protocols
and services. Due to this validation effort and experimentation on
real platforms, trustable services will be produced, and the
time-to-market reduced.

The development and implementation of a validation and experimentation
platform (multi-protocols and multi-services), in close relationship
between academic world and industry, will address the companies
concern of assessing applications in a systematic way, and will allow
the academic members to experiment their innovations on a real-world
test bed.

Mobility is the main focus on the platform. In particular, we plan to
study protocols and services including mobility and the WAP, and also
the technologies that allow the deployment of these services (GSM, GPRS,
UMTS). Note though that the platform is generic enough to be used with
other protocols and services, such as those of wired
network.

A platform is a set of software tools and equipments offering
complementary functions and allowing real experimentations. In
particular, the \textsc{Platonis} platform is oriented to the cover
all the aspects of conformance and interoperability testing (from
specification, test selection, automated test generation to test
execution).

Moreover, when designing, implementing and deploying a system, it must
be checked that it not only supplies the expected functionalities, but
that it also provides acceptable performances, in terms of loading
rates, processing capacity, response time, etc. Such a study allows
also to preview the collapse of the system by identifying the
potential bottle-necks, and to determine an optimal configuration for
the ressources (buffer sizes, number of servers, distribution of
processes on the different machines, network topologies, etc.)
according to criteria like ``quality over cost''.

The \textsc{Platonis} platform will be opened to other users: 

\begin{itemize}
  \item companies wishing to experiment the new functionalities
        proposed by a given service;

  \item universities for research and teaching purposes (students'
        training to these new technologies).
\end{itemize}

In this context, the project has the following main objectives:

\begin{enumerate}

  \item the implementation of a platform for validating and
        experimentating new protocols and services;
  
  \item the validation and experimentation of the platform related to
        the terminals mobility. We plan to check whether some protocol
        exchanges in WAP over GSM are correct and if different
        entities may interwork properly;

  \item a method supported by tools, allowing to analyse some quality
        of service (QoS) properties, in consistency with the
        functional validation;

  \item the openness: the platform will be used at the beginning for
        teaching GET\footnote{Groupe des �coles des
        T�l�communications} and university students of
        the academic partners, and also the engineers of the member
        companies. It is planned to widen the use of the platform to
        other companies and academic institutions to a legal and
        commercial framework defined by the \textsc{Platonis}
        consortium.

\end{enumerate}


\subsection{Partnership and project management}

The \textsc{Platonis} project started in March 2001 and the duration
of the project is two years.  The project is managed by
INT\footnote{Institut National des T\'el\'ecommunications} which is in
charge of the coordination between the sub-projects and is the
representative to the ministry. The project is divided into four
sub-projects:

  \begin{enumerate}

  \item \emph{Platform and network implementation.}  Partner in
    charge: INT. The INT owns a test laboratory~\cite{besse,ASE,hit}.

  \item \emph{Protocols design and coding.}  Partner in charge: France
    T\'el\'ecom R\&D. France T\'el\'ecom R\&D brings its experience on
    the modeling, the QoS, the protocol validation and is a
    contributor to standardization institutions (ITU, ETSI
    etc.)~\cite{Alabau,Monin}.

  \item \emph{Experimentation and validation of proposed
      applications.}  The partner in charge is LaBRI-Universit\'e de
    Bordeaux. The LaBRI\footnote{Laboratoire Bordelais de Recherche en
      Informatique} has a significant experience in the area of
    interoperability testing~\cite{kone,castanet,rafiq}.

  \item \emph{Implementation of a demonstrator.}  Partner in charge:
    Kaptech. Kaptech is a new telecom operator for companies, and will
    bring its experience in industrial applications.

  \item Another partner, LIMOS\footnote{Laboratoire d'Informatique de
      Mod\'elisation et d'Optimisation des Syst\`emes.} - Universit\'e
    Blaise Pascal, is also associated to this project. The LIMOS has
    already an experience in protocol testing, and will cooperate
    mainly in the first and third sub-projects~\cite{laurencot}.

  \end{enumerate}



\subsection{Formal models}
   
One main activity associated closely with the \textsc{Platonis}
platform is the formal modelisation of part of the system architecture
and its behaviour: some protocol layers (the highest), some APIs and
the service factory infrastructure. This should allow to integrate in
several steps concepts like OSA (\emph{Open Service Access}) and VHE
(\emph{Virtual Home Environment}). We chose the \emph{Specification
  and Description Language} (SDL) for the behaviour formalisation and
the \emph{Unified Modeling Language} (UML) for the architecture.

     The rationale for formalisation is the following:

     \begin{itemize}

        \item automatic production of test suites for different
              interfaces;

        \item QoS analysis starting from the service design and
              specification, taking into account the concepts of
              negociation (essential in the UMTS) between different
              peers;

        \item functional validation (simulation and animation) of
              protocols, of APIs and services in the context of the
              given architecture, but independently of a protocol or
              of a peculiar platform;

        \item it is important to make sure that the model, which
              includes the concepts of OSA (and APIs) and VHE, is
              really based on the independence of a layer respectively
              to different possible implementations. We believe
              that formalisation is the best way.

    \end{itemize}
  
  \subsection{Conformance and interoperability\label{validation}}

     The validation platform provides two kinds of test:
     \emph{interoperability testing} (in a first step
     between the clients and the server, and in a second step
     between two terminals), and \emph{conformance} of a protocol
     layer (if there is no direct access to the layer under test, then
     \emph{embedded testing} will be envisaged, i.e. testing the layer
     through the upper layers). Moreover loading and robustness tests
     are taken into account.

     These tests are supposed to validate that different
     implementations interwork correctly, i.e. they provide the
     expected global service, while complying with the standards. Two
     kinds of standards are used: one is based upon the use of mobile
     phones using the WAP, and the second is based on the use of
     \emph{Personal Digital Assistants} (PDA) either on the WAP stack
     or a WML/UDP/IP stack over GPRS or UMTS. The security layer is not
     considered in a first step.

     The tests should be generated automatically, if possible, which
     implies a preliminary formal specification of the protocols and
     services. As mentionned previously we chose in that aim to
     modelise in SDL (standard of the ITU-TS). Once the protocol and
     the services are formally described, it will allow the automatic
     generation of the test sequences guaranting a coverage of the
     given specification, and allowing to detect different kind of
     faults, like output faults. The last step is the implementation
     of these test cases in a test architecture by a demonstrator.

     There are three kind of services to be tested:

     \begin{itemize}

       \item \emph{Terminal services.} The mobile terminal are
             characterized by data like the size of the screen, the
             character set, the available colours, telephonic abilities
             etc. These features are likely to be tested with methods
             from the Java Mexe (\emph{Mobile station application
             EXecution Environment}). The WAP Forum yet proposes some
             tests validating the mobile phones functionalities. It is
             indeed possible to run on-line these tests on the
             terminals, thus they will be skipped in a first step.

       \item \emph{Protocol-layer services.} These services are
             supplied by some protocol layers and are specified by
             event-driven diagrams.

       \item \emph{Application services.} Thery are of two kind. The
             first kind is \emph{general services}. These correspond
             to the execution of programmes interacting either
             directly or not with the end-users' mobile terminal. A
             mobile location service will be defined. Note that this
             kind of service requires the agreement of the operators,
             since access to this information is restricted. The
             second kind of application services is \emph{specific
             services}. We plan to define and test a service of
             network management and equipment maintenance provided by
             Kaptech. The application services are harder to formalise
             than the previous ones.

     \end{itemize}

     The tests and procedures likely to be implemented under the
     proposed test architectures are the following:

     \begin{itemize}

       \item \emph{Protocol-layer conformance testing.} These tests address
             the conformance of a given WAP layer. This assumes that
             the consortium owns its own WAP gateway.

       \item \emph{Interoperability testing.} These
             tests allows to check, for instance, the interoperability
             between an application running on a terminal and an
             application on a server. They will be carried out with a
             terminal simulator, using log files. In a second step,
             and only if the emulation software can run on the
             wireless PDAs, then the tests will be executed on these
             latter.

%             This validation implies the observation of the HTTP
%             responses to the terminal requests. The formal modeling
%             of request schemes is possible, allowing thus automatic
%             test generation, but the complete modeling of an
%             application is far beyond the scope of the
%             \textsc{Platonis} project objectives. These tests have to
%             be implemented in one hand with an open-source gateway,
%             and, in another hand, with an operator gateway (that can
%             be considered as a reference). The demonstrator has to
%             allow the test execution.

     \end{itemize}

%  \subsection{Performance evaluation}
%
%  The simulation techniques and real tests are complementary when
%  assessing the performance of a complex system. Simulation consists
%  in modeling the system behaviour with a system of queues, and to
%  exercise its transitions. It is both easier and more realistic than
%  the analytic method which looks for mathematical formul{\ae} with
%  strong hypothesis (like independence of events). It is also much
%  less expensive in hardware and software than the ``real test''
%  approach, because it exploits a model instead of a real system. The
%  execution time of a simulation does not depend on the real
%  processing-time of the application, hence it allows to explore a
%  great number of scenarios just varying the parameters corresponding
%  to the different configurations of the system. The major difficulty
%  of this approach is to have the precisest knowledge as possible on
%  the bottle-necks. in order to get a performance model we ought own a
%  precise system specification (either formal or not) and some
%  quantitative informations on the duration of each elementary
%  task. In number of cases, the former is owned; on the contrary, the
%  latter is remains always a problem. We can envisage, when a formal
%  specification is available --- in SDL for instance ---, to
%  automatically generate the required quantitative informations by
%  previewing points of measurement when producting the implementation
%  code. This would make compulsory to complete the specifications with
%  temporal annotations.
