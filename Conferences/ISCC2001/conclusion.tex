%%-*-latex-*-

\section{Conclusion\label{conclusion}}

In this paper we have presented a global test architecture for  
distributed services including the generation of test sequences for
service components. Our approach was validated by a case study: a
France Telecom \audio service.

The full service was described using the SDL language, and the running
of the Hit-or-Jump algorithm showed no deadlocks and produced the test
sequences for all the components of the studied service.

For sake of simplicity, we have selected a component of the \audio service
that is at the very heart of the service, the conference bridge, and
which coordinates the other components and illustrates clearly what
one imagines a \audio service is. The other components were generic
components that can be present in other kind of telecommunication
services, and for which we also generated the corresponding tests. We
have produced the tests for the bridge component in its context, and
we translated them in the TTCN and MSC formats.

We have also defined an architecture for the tester, which combines an
active part (based on a stimulation of the implementation) and a
passive one (based on the observation of the exchanges between the
CORBA objects). 

The results we got show that the use of formal methods considerably
eases the task of the service designers and developers, and that they
are usable for real services. Since the design phase to the
implementation and test phases, we used formal description techniques
(SDL, TTCN, MSC) and a formal test methodology. Moreover, we showed it
is possible to test the service components in the context of the
others (and not artificially in isolation). We think this is a notable
step towards the validation and the design of reusable
service-components.

