%%-*-latex-*-

\section{Introduction}

The telecommunication operators are facing the dramatic growth of new
services that utilize the telephonic network but also mix voice
transmission with image and sound transmission. The inherent
difficulty of this kind of services lies on their fast and reliable
integration, in order to satisfy the users' requirements.

The Intelligent Network (IN) has been the first attempt to solve this
sort of problems and, in the meanwhile, some others architectures have
been proposed, like TINA, in order to cope efficiently with the design
and the implementation of new services. These latter must meet the
users' expectations and be definitely reliable. With this aim, after
the design phase, the services should be validated and tested in order
to guarantee that the proposed implementations comply with the
requirements. Moreover, because of modularity and reusability
constraints, these services are to be defined from base components
which may further be combined and reused.
%\begin{figure}[!ht]
%\begin{center}
%\begin{bigbox}
%\epsfxsize=8cm
%\epsfbox{test_process.eps}
%\end{bigbox}
%\caption{Process of test production}
%\label{test_process}
%\end{center}
%\end{figure}

In this paper we present a method for service-component testing and a
Corba test architecture. To illustrate the application of the method we
present a case study: a \audio service. 
%The general architecture of this test environment is shown at
%figure~\ref{test_process}. 
Our input is an SDL
model of the service we want to test. Starting from this model, we
generate a test sequence by using the component test algorithm,
Hit-or-Jump. This sequence is built from the SDL model and is
thus expressed in terms of SDL entities. The values associated to
the exchanged signals have types represented within SDL. In order to
step forward the test execution on a CORBA environment, this sequence
must be transformed to be defined in terms of CORBA objects, and hence
mapped to the IDL language. Last, an execution environment on
a CORBA architecture is proposed.

The contribution of this work lies on the methods definition for the
service-component testing and the test execution architecture on a
CORBA environment. At the CFIP99~\cite{cfip99}, we have presented yet
a method for the embedded testing, but this method was based upon the
generation of the whole accessibility graph of the system under test.
The method presented here is new, it is based on the production of
partial accessibility graphs and in this way avoids the combinatorial
explosion of the state number of the global system.

We also present in this work the test execution architecture on a
distributed CORBA environment. We give the principles allowing the
test of a CORBA implementation by combining interactive and purely
observational techniques (also called \emph{passive testing}). This
approach is innovative since, to our knowledge, there exists very few
works on validation and test execution for CORBA environments.

This work was achieved in the CASTOR RNRT French
project~\cite{castor}. The specification of the \audio service was
provided by France Telecom R\&D.

This article is organized as follows. In
section~\ref{generation_de_tests} we give a short overview of the
algorithm used for the derivation of the service-components
tests. Section~\ref{descriptif_specification} presents the case study,
the \audio and its components. Moreover, in this section we 
present the results of the experiments we have made. Then
section~\ref{CORBA} presents a description of the CORBA environment
devoted to test execution, and explains the principles which guided
its design. Last, section~\ref{conclusion} gives the conclusions of
this work.




