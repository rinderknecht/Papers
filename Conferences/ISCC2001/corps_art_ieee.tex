%%-*-latex-*-

\section{A service-component testing method\label{generation_de_tests}} 

\subsection{Preliminaries}

In this section we present briefly the method used for the generation
of component tests. A more complete presentation can be found
in~\cite{chine}. A comparison with the existing
methods~\cite{bourhfir,IEEE} is presented in~\cite{cfip00}. Some tools
devoted to the embedded testing (or \emph{testing in context}) are
presented in~\cite{testcomposer,groz}.

The aim of the method is to test components in their context, i.e. the
complement of the component in the system, because usually there is
no direct access to the component. The system is defined in terms of
\emph{Communicating Extended Finite-State Machines} (CEFSMs), which
are based upon \emph{Extended Finite-State Machines} (EFSM) defined as
follows. \\

\noindent \textbf{Definition.} An EFSM is a 5-tuple
$M=(I,O,S,\vec{x},T)$ where $I$, $O$, $S$, 
$\vec{x}$, and $T$ are respectively finite sets of input symbols,
output symbols, transitions. Each transition $t$ in the set $T$ is a
6-tuple $t=(s_t,q_t,a_t,o_t,P_t,A_t)$ where $s_t$, $q_t$, $a_t$, and
$o_t$ are respectively the start (current) state, the next state, one
input and one output. $P_t(\vec{x})$ is a predicate on the current
values of the variables and $A_t(\vec{x})$ defines an action on the
values of the variables. Initially the machine is in a state
$s^{(0)}\in S$ with the values of variables $\vec{x}^{(0)}$. Let the
machine be in state $s_t$ with the current values of the variables
$\vec{x}$. When the input $a_t$ occurs, if $\vec{x}$ is valid for
$P_t$, i.e. $P_t(\vec{x})={\textsf{true}}$, then the machine achieves
the transition $t$, outputs $o_t$, changes the current values of the
variables by means of action $\vec{x}:=A_t(\vec{x})$, and finally
reaches the state~$q_t$.

The case study we present in this paper (\audio service) is specified
with the SDL language~\cite{itu}. The specification of behaviours in
SDL is based upon CEFSMs. The actions associated to a transition 
may include the running of tasks, procedure calls, dynamic creation of
processes (SDL has the concepts of \emph{types} and \emph{type
instance}) and  also the enabling and disabling of timers.

Until now the research on automatic test generation for components was
based on the \emph{exhaustive simulation} of these SDL specifications,
i.e. the exploration of the whole state-space (or \emph{accessibility
graph}). The first drawback is that, when testing real-world service
components, the number of states is huge; the second one is that
redundant parts may be explored. To avoid these problems, we have
designed and implemented a new algorithm, called \emph{Hit-or-jump},
based on the generation of partial accessibility graphs.

\subsection{Outline of the Hit-or-Jump algorithm\label{hit}}

The algorithm presented here allows to cover all the interactions of
the component in its context. The essence of our approach is as
follows. At any moment we conduct a local search from the current
state in a neighborhood of the accessibility graph. If an untested
part of the component is found (a Hit), we keep it for the final test
sequence, and then continue the search process from there. Otherwise,
we move randomly to the frontier of the neighborhood searched (Jump),
and resume the process from there. This procedure avoids the
building of a whole system accessibility graph. Accordingly, the space
required is determined by the user, e.g. a depth limit or a maximum
number of states, and it is independent of the system under
consideration. On the other hand, a random walk may get ``trapped''
at certain part of the component under test~\cite{IEEE}.
Our algorithm is designed to ``jump'' out of the ``trap'' and
pursue the exploration further. to build at each step a
partial accessibility graph to avoid the state-number explosion
problem mentioned before. The algorithm finally produces a test
sequence as a transition tour of the component in its
context. \\

\noindent
{\bf Initial condition}. The environment machine $C$ is in an initial state
$s_C^{(0)}$, the component machine under test $A$ is in an initial
state $s_A^{(0)}$, and the system variables have initial values
$\vec{x}^{(0)}$.

%\vspace{1mm}
\noindent
{\bf Termination}.
The algorithm terminates when all the transitions of $A$ have been
marked off.

%\vspace{1mm}
\noindent
{\bf Execution}.
\begin{enumerate}
\item {\bf HIT}\label{hit1}
From the current node $(s_C^{(k)},s_A^{(k)},\vec{x}^{(k)})$
conduct a search in $C\times A$ until (a) or (b) occurs:
\begin{enumerate}
\item \label{a} Reach an edge which is associated with unmarked transitions
of the component machine $A$: a Hit.\label{search}
Then: 
\begin{enumerate}
	\item Include the path from the current node to the edge (inclusive) in the test sequence under construction;
	\item Mark off the newly exercised transitions of~$A$;
	\item Arrive at a node $(s_C^{(k+1)},s_A^{(k+1)},\vec{x}^{(k+1)})$;
	\item Erase the searched graph;
	\item Repeat from~\ref{hit1}.
\end{enumerate}

\item \label{b} Reach a search depth or space limit without hitting any unmarked
transition of $A$. Then move to~\ref{jump}. 
\end{enumerate}
\item {\bf JUMP} \label{jump}
\begin{enumerate}
\item We have constructed a search tree, rooted at 
$(s_C^{(k)},s_A^{(k)},\vec{x}^{(k)})$.
\item Examine all the leaf nodes of the tree, 
and select one uniformly at random.
\item Include the path from the root to the selected leaf node in the test
sequence.
\item We arrive at the selected leaf node
$(s_C^{(k+1)},s_A^{(k+1)},\vec{x}^{(k+1)})$: a Jump.
\item Repeat from \ref{hit1}.
\end{enumerate}
\end{enumerate}

This algorithm avoids looping on the same state, as far as we choose
uniformly and randomly in the spanning tree. Notice that Hit-or-Jump
assumes that the specification is correct in the sense that it does
not imply run-time dead-locks, non accessible states, etc.

\section{A case study: A Conference Call Service\label{descriptif_specification}}

The \audio service is specified by means of the SDL language, as we
mentioned in the previous section. This service allows a given set of
users to establish a mutual communication. A subscriber, initiator of
the communication, plays a special role: the moderator. He has in
charge the conference-bridge reservation for a given hour and a given
day. The conference cannot start before the arriving of the moderator.

This specification lies upon an Intelligent Network (IN) architecture
and on the concept of reusable components, the \emph{Service
Independent Building Blocks} (SIBs). The \mbox{Q.1201} recommendation
defines the conceptual model of the IN. This model is divided in four
plans, each plan being an abstract view of the system. The
specification we considered stresses more the global functional plan,
which is responsible for the service creation, and put the accent also
on the service plan itself. We are here particularly interested in the
user's view of the network and the services. This plan models the
network as a unique entity, which executes the service.

In the system architecture described in figure~\ref{archi} we find
three main blocks: (1)~the \emph{Users} block, which consists of
three processes: \emph{users}, the \emph{conference} and
\emph{term\_manager} (2)~the SSP block (3)~the SCF block. The
\emph{users} process (the sole process that interacts with the
environment) models the user's behaviour (to off-hook, to on-hook, to
dial, to hear the busy-tone and the ring-back, to speak) which is
finely defined. The \emph{term\_manager} process manages the SDL data
of the block, and copes with other process creations. The SSP block
is responsible for the switching function; it is divided into four
processes: the \emph{obcsm} process for handling the originating half
calls, the \emph{tbcsm} process for the terminating half calls, the
CCF process and lastly the SSF process. The CCF process establishes,
handles and releases the calls. The SSF makes the connection, by
association with CCF, between a user and the Service Control Function
(SCF).

\begin{figure}[htbp]
\begin{center}
\epsfxsize=8cm
\epsfbox{architecture.eps}
\caption{Architecture of the specification}
\label{archi}
\end{center}
\end{figure}

The network communications, linked to the INAP protocol, are shortly
considered here (only a few primitives are specified). In this
specification the accent is put on the modular aspect of the service
design, and on its SIB decomposition. A SIB is a piece of logic (or
building block) which comprehends instantiation parameters as input,
and returns a result. 

%The figure~\ref{brique} describes a service-independent building block.

%\begin{figure}[htbp]
%\begin{center}
%\epsfxsize=8cm
%\epsfbox{brique_eng.eps}
%\caption{Service-independent building block}
%\label{brique}
%\end{center}
%\end{figure}

\subsection{Test generation for the service components}

A SIB is a monolithic, reusable and standard component,
used for building services. The SIBs take as input some static
parameters called SSD (\emph{Service Support Data}) and CID
(\emph{Call Instance Data}); it can be activated at an activation
point POI (\emph{Point of Initiation}) and may have one or more
terminations POR (\emph{Point of Return}).

The SIBs of the intelligent network are designed in an interprocedural
way. Each component is mapped to a SDL procedure. Hence, in order to
supply the service, a process is used, which has in charge the
chaining of the building blocks (this chaining is sequential). This
process plays the role of glue between the components.\\


\textbf{The results obtained}

The \audio service is supplied by a specific chaining of building
blocks. The different components are generic, each of them models a
function. With the aim of building up an executable service logic,
they have to be instantiated by some parameters. These latter allow
the definition of different behaviour of the same component, which are
the following:

\begin{itemize}
  \item DCL supplies the call triggering;\vspace*{-1mm}
  \item EDA allows the call establishing;\vspace*{-1mm}
  \item MVl\_ann issues the play announcements to the users;\vspace*{-1mm}
  \item MVL\_exp has in charge the network play announcements;\vspace*{-1mm}
  \item RSC\_incr, RSC\_decr and RSC\_obs manage respectively the
        \emph{conference bridge} resource incrementation, its
        decrementation and the monitoring.\vspace*{-1mm}
\end{itemize}

Figure~\ref{mesures} presents some metrics on the service
specification under study: number of lines of code, of blocks, of
processes, etc. Figure~\ref{resultats} presents our results: the first
column points the coverage rate of each component and the
second one gives the length of the derived sequence. Note that the
coverages are not always 100\% because the components are generic and
have been instanciated partially (for this application). The maximum
execution time on a Sun Ultra~5 running SunOS~5.5.1 is 120~seconds.

\begin{figure}[htbp]
 \begin{minipage}[b]{.38\linewidth}
 \centering
  \begin{tabular}{|l|c|}
  \hline
  Lines of SDL code & 9.873\\
  \hline
  Blocks & 3\\
  \hline
  Processes  & 13\\
  \hline
  Procedures & 49\\
  \hline
  States & 307\\
  \hline
  Signals & 67\\
  \hline
  Macro definitions & 29\\
  \hline
  Timers & 1\\
  \hline
  \end{tabular}
  \caption{The specification metrics}
  \label{mesures}
  \end{minipage}\hfill
 \begin{minipage}[b]{.46\linewidth} 
 \begin{tabular}{|l|c|c|}
  \hline
  DCL & 100\% & 32\\
  \hline
  EDA & 65\% & 119\\
  \hline
  MVL\_ann & 50\% & 55\\
  \hline
  MVL\_exp & 80\% & 197\\
  \hline
  RSC\_incr & 100\% & 36\\
  \hline
  RSC\_decr & 100\% & 131\\
  \hline
  RSC\_obs & 80\% & 140\\
  \hline
  \end{tabular}
  \caption{Test components results}
  \label{resultats}
 \end{minipage}
\end{figure}

\vspace*{-4mm} \noindent \textbf{The \emph{conference bridge} module}

We have also applied the algorithm to the \emph{conference bridge} module.
It seemed to us that it would be wiser to show the sequence computed
for this module, as far as it allows a better understanding of a
conference bridge --- the sequence for the components being less
clear. We got a sequence of length~247, expressed in the
MSC~\cite{mscitu} format and in the TTCN~\cite{ISO1991} notation. 

Figure~\ref{msc} is the simplified MSC for the test sequence obtained
for the module representing the bridge-conference terminal. This
sequence brings to light all the interactions of this module (it
indeed covers all the transitions). We only show here the signal
exchanges between the two peers, i.e. the switch (SSF) and the user,
with which the bridge interacts. The environment here is schematized
by the frame of the figure. The scenarios that can be found in the
signal exchanges of the figure~\ref{msc} are the following:

The user 2 asks for a connection to the \audio by dialing
        number \textsf{08.36.00.00.01}. At the conference level, this is
        translated into the signal input \textsf{setupreq}. The
        conference module sends an \textsf{offhook} signal to the
        switch. This latter, after a chaining of exchanges with the PCS
        and a series of component activations, sends a
        \textsf{setupresp} signal which means that the communication
        can be established. 
	First, the user receives a waiting
        message and, as soon as the communication is established, a
        welcome message. Once this latter is received, the user
        decides to on-hook: this sends a \textsf{releaseind} signal to
        the switch; this latter relays it to the conference module by
        means of the \textsf{releasereq}. This latter module
        acknowledges the disconnection request by the
        \textsf{releaseind} signal; then it receives a
        \textsf{line\_free} message, which indicates that the line is
        free again.

\begin{figure}[h]
\begin{center}
\begin{bigbox}
\epsfig{file=msc_final.eps, width=8cm}
\end{bigbox}
\caption{Simplified test sequence of the conference-bridge module}
\label{msc}
\end{center}
\end{figure}

        \vspace*{-10mm} The second part of the figure brings to the fore the following
        scenario: the user 1 calls the bridge, he receives the
        waiting signal and then the conference is opened; the waiting
        subscribers receive the welcome message; consequently the
        communication is established, they can talk to each other:
        this exchange is revealed by the reception of a
        \textsf{msg\_term} signal.


%We also supply the reader with a part of the TTCN  we got (see
%figure~\ref{ttcnseq}). The computed sequence corresponds to a
%black-box testing configuration, where the points of control and
%observation are located at the system level. Thus the exchanges
%between the environment and the user's block are visible.
%
%\begin{figure*}[!htbp]
%\begin{center}
%\begin{tabular}{l l}
%{\sf Test Case Name} &{\sf: network\_RI\_0}\\
%{\sf Group} &{\sf: network\_RI/}\\
%{\sf Purpose} &{\sf:}\\
%{\sf Default} &{\sf: DEF\_0}\\
%{\sf Comments} &{\sf: Generated by test oriented simulation of the test} \\
%&{\sf \- purpose network\_RI.msc}\\
%\end{tabular}
%\newline
%\scriptsize{
%\begin{tabular}{|p{5mm}|p{75mm}|p{15mm}|p{7mm}|}
%\hline
%{\sf Nr} &{\sf Behaviour Description} & {\sf Constraints Ref}&Verdict\\
%\hline
%{\sf 1} &{\sf env2users !offhook output} &&\\
%{\sf 2} &{\sf \hspace{0.1in}env2users !dial START TAC} & {\sf dial}&\\
%{\sf 3} &{\sf \hspace{0.2in}users ?msg CANCEL TAC} & {\sf msg\_3}&\\ 
%{\sf 4} &{\sf \hspace{0.3in}env2users !dial} & {\sf dial\_4}&\\
%{\sf 5} &{\sf \hspace{0.4in}env2users !releaseind} &&\\
%{\sf 6} &{\sf \hspace{0.5in}env2users !offhook} &&\\
%{\sf 7} &{\sf \hspace{0.6in}env2users !dial START TAC} & {\sf dial\_2}&\\
%{\sf 8} &{\sf \hspace{0.7in}env2users ?msg CANCEL TAC} & {\sf msg\_3}&\\
%{\sf 9} &{\sf \hspace{0.8in}env2users !dial} & {\sf dial\_4}&\\
%{\sf 10} &{\sf \hspace{0.9in}env2users !talk} & {\sf talk\_6}&\\
%{\sf 11} &{\sf \hspace{1.0in}env2users !offhook} &&\\
%{\sf 12} &{\sf \hspace{1.1in}env2users !dial START TAC} & {\sf dial\_7}&\\
%{\sf 13} &{\sf \hspace{1.2in}env2users ?msg CANCEL TAC} & {\sf msg\_3}&\\
%{\sf 14} &{\sf \hspace{1.3in}env2users !dial START} & {\sf dial\_7}&\\
%{\sf 15} &{\sf \hspace{1.4in}env2users ?msg CANCEL} & {\sf msg\_3}&(P)\\
%{\sf 16} &{\sf \hspace{1.4in} ? TIMEOUT TAC} &&F\\
%{\sf 17} &{\sf \hspace{1.2in} ? TIMEOUT TAC} &&F\\
%{\sf 18} &{\sf \hspace{0.6in} ? TIMEOUT TAC} &&F\\
%{\sf 19} &{\sf \hspace{0.1in} ? TIMEOUT TAC} &&F\\
%\hline
%\end{tabular}
%}
%\caption{TTCN sequence}
%\label{ttcnseq}
%\end{center}
%\end{figure*}


