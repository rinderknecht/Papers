%%-*-latex-*-

\begin{theorem}[Soundness]\hfill
\label{f0:soundness}
\begin{center}
If \(\smat{t}{p} \xrightarrow{+} \sigma\) 
then \(\subst{p}{\sigma} \sqsubseteq t\).
\end{center}
\end{theorem}

\paragraph{Proof.}

Let \(\ind{P}(\match{f}{p}{\sigma'}, n, \sigma)\) be the proposition
\begin{center}
\emph{If} \(\match{f}{p}{\sigma'} \xrightarrow{n} \sigma\)
\emph{then} \(\sigma' \subseteq \sigma\)
\emph{and} \(\subst{p}{\sigma} \sqsubseteq f\).
\end{center}
Then \(\forall p, f, \sigma', n, \sigma.\ind{P}(\match{f}{p}{\sigma'},
n, \sigma)\) implies the soundness property by choosing
\(\dom{\sigma'} = \varnothing\) and \(f = [t]\). Therefore, let us
assume
\begin{equation}
\match{f}{p}{\sigma'} \xrightarrow{n} \sigma \label{f0:sound:n}
\end{equation}
that is to say, there exists a successful series \(\Delta\) of \(n\)
rewrite steps (called a \emph{rewrite}) from \(\match{f}{p}{\sigma'}\)
to \(\sigma\). (If not, i.e., if \(\Delta\) is empty, then the theorem
is trivially true, since it has the form \(A \Rightarrow B\), which is
equivalent to \(\neg{A} \vee B\).) We hence can reason by induction on
length of \(\Delta\), that is to say, we assume that \(\ind{P}(\kappa,
n, \sigma)\) holds for any \(n\) (that is the \emph{induction
  hypothesis}) and then prove that all the ways to extend backwards
\(\Delta\) with one more step result in a configuration \(\kappa'\)
such that \(\ind{P}(\kappa', n+1, \sigma)\) holds too. Indeed, by
definition of \((\rightarrow)\), we have
\[
\kappa' \rightarrow \underbrace{\kappa \xrightarrow{n} \sigma}_{\Delta}
\quad \Longrightarrow \quad
\kappa' \xrightarrow{n+1} \sigma
\]
The complete\hyp{}induction principle implies then that \(\forall p,
f, \sigma', n, \sigma.\ind{P}(\match{f}{p}{\sigma'}, n, \sigma)\),
thus the soundness. We discriminate first on the length of \(\Delta\),
then case by case on the shape of the configuration
\(\match{f}{p}{\sigma'}\).
\begin{enumerate}

  \item \(\Delta\) is made of one rewrite step.\\ In this case,
    \(\Delta\) is made of \textsf{END}, since \textsf{END} is the
    unique terminating rule. Its syntax implies that \(f = \el\), \(p
    = \el\) and \(\sigma' = \sigma\). These definitions lead to
    \(\sigma' \subseteq \sigma\) and \(\subst{p}{\sigma} =
    \subst{\el}{\sigma} \eqn{1} \el \sqsubseteq \el = f\). Therefore
    \(\ind{P}(\match{f}{p}{\sigma'}, 1, \sigma)\) holds.

  \item \(\Delta\) is made of at least two rewrite steps.\\ A quick
    examination of the rules reveals that all the ways to extend
    backwards \(\Delta\) can be characterised by discriminating only
    upon the element (lexeme, parenthese or meta\hyp{}variable) added
    to the pattern \(p\). Assuming the induction hypothesis
    \(\ind{P}(\match{f}{p}{\sigma'}, n, \sigma)\) and
    \eqref{f0:sound:n}, we get
    \begin{align}
      \sigma' &\subseteq \sigma \label{f0:sound:6}\\
      \subst{p}{\sigma} &\sqsubseteq f \label{f0:sound:7}
    \end{align}
    we have the following cases to consider.
    \begin{enumerate}

      \item Case where \textsf{ELIM} extends \(\Delta\).\\ This means
        that we have \(l \in {\cal L} \cup \{\rpar\}\) in
        \begin{align}
            \match{\cons{l}{f}}{\cons{l\!}{\!p}}{\sigma'}
          &\rightarrow
            \match{f}{p}{\sigma'}
            &\textsf{ELIM} \notag\\
          &\xrightarrow{n} \sigma 
            & (\Delta) \notag\\
            \match{\cons{l}{f}}{\cons{l\!}{\!p}}{\sigma'}
          &\xrightarrow{n+1} \sigma \label{f0:sound:ELIM}
            \intertext{Furthermore, we have the following equalities:}
           \subst{\cons{l}{p}}{\sigma}
          &\eqn{2} \cons{l}{\subst{p}{\sigma}}\notag\\
          &\sqsubseteq \cons{l}{f}
          &\text{by} \,\; \eqref{f0:sound:7} \,\; \text{and} \,\;
           \textsf{EQ}\notag\\
             \subst{\cons{l}{p}}{\sigma}
          &\sqsubseteq \cons{l}{f} \label{f0:sound:ELIM:conc}
        \end{align}
        As a conclusion, the induction hypothesis and
        \eqref{f0:sound:n} imply \eqref{f0:sound:ELIM},
        \eqref{f0:sound:6} and \eqref{f0:sound:ELIM:conc}, which,
        together, imply
        \(\ind{P}(\match{\cons{l}{f}}{\cons{l\!}{\!p}}{\sigma'}, n+1,
        \sigma)\).

      \item Case where \textsf{UNPAR} extends \(\Delta\).\\ The syntax
        of the rule implies that there exists a sub\hyp{}forest \(f_2\)
        of \(f\) and a tree \(c(f_1)\) such that
        \begin{enumerate}

          \item \label{f0:sound:1} \(f \triangleq f_1 \cdot
            \cons{\rpar}{f_2}\)

        \end{enumerate}
        and
        \begin{align}
          \match{\cons{c(f_1)\!}{\!f_2}}{\cons{\lpar\!}{\!p}}{\sigma'}
          &\rightarrow
          \match{f_1 \cdot \cons{\rpar\!}{\!f_2}}{p}{\sigma'}
          \notag\\
          & \qquad\qquad\quad\ \textsf{UNPAR} \notag\\
          &\xrightarrow{n} \sigma
          \qquad\qquad\quad(\Delta) \notag\\
          \match{\cons{c(f_1)\!}{\!f_2}}{\cons{\lpar\!}{\!p}}{\sigma'}
          &\xrightarrow{n+1} \sigma \label{f0:sound:UNPAR}
        \end{align}
        Furthermore, we have the following equalities:
        \begin{align}
          \subst{\cons{\lpar}{p}}{\sigma}
          &\eqn{3} \cons{\lpar}{\subst{p}{\sigma}}\notag\\
          &\sqsubseteq \cons{\lpar}{f}
          &\text{by} \,\; \eqref{f0:sound:7}\notag\\
          &= \cons{\lpar}{f_1 \cdot \cons{\rpar}{f_2}}
          &\text{by} \,\; \text{\ref{f0:sound:1}}\notag\\
          &= \cons{\lpar}{f_1} \cdot \cons{\rpar}{f_2}
          &\text{by} \,\; \eqref{model:conc:2}\notag\\
          &= \cons{\lpar\!}{\!\el\!\cdot\!f_1}
             \!\cdot\!\cons{\rpar\!}{\!\el\!\cdot\!f_2}
          &\text{by} \,\; \eqref{model:conc:1}\notag\\
          &= ([\lpar] \!\cdot\! f_1) \!\cdot\! ([\rpar] \!\cdot\! f_2)
          &\text{by} \,\; \eqref{model:conc:2}\notag\\
          \subst{\cons{\lpar}{p}}{\sigma}
          &\sqsubseteq \cons{c(f_1)}{f_2}
          &\text{by} \,\; \textsf{SUB} \label{f0:sound:8}
        \end{align}
        As a conclusion, the induction hypothesis and
        \eqref{f0:sound:n} imply \eqref{f0:sound:UNPAR},
        \eqref{f0:sound:6} and \eqref{f0:sound:8}, which, together,
        imply \(\ind{P}(\match{\cons{c(f_1)}{f_2}}{\cons{\lpar}{p}}%
        {\sigma'}, n+1, \sigma)\).

      \item Case where \textsf{BIND} extends \(\Delta\).\\ The syntax
        of the rule implies that there exists a substitution
        \(\sigma_1\) such that
        \begin{enumerate}

          \item \(\sigma_1 \subseteq \sigma_1 \oplus x \mapsto t \triangleq
          \sigma'\) \label{f0:sound:incl1}

        \end{enumerate}
        and
        \begin{align}
           \match{\cons{t}{f}}{\cons{\meta{x}}{p}}{\sigma_1}
          &\rightarrow
           \match{f}{p}{\sigma'}
          &\textsf{BIND} \notag\\
          &\xrightarrow{n} \sigma
          & (\Delta) \notag\\
           \match{\cons{t}{f}}{\cons{\meta{x}}{p}}{\sigma_1}
           &\xrightarrow{n+1} \sigma \label{f0:sound:BIND}
        \end{align}
        Inclusions \ref{f0:sound:incl1} and \eqref{f0:sound:6} imply, by
        \eqref{model:incl:trans}:
        \begin{align}
          \sigma_1 \subseteq \sigma \label{f0:sound:incl2}\\
          \intertext{Moreover \eqref{f0:sound:6} and \ref{f0:sound:incl1}
            imply}
          \sigma_1 \oplus x \mapsto t \subseteq \sigma\notag
        \end{align}
        which is equivalent, by definition \eqref{model:incl}, to
        \[
        \forall y \in \dom{\sigma_1 \oplus x \mapsto t}.((\sigma_1
        \oplus x \mapsto t)(y) \!=\! \sigma(y))
        \] 
        By \eqref{model:dom}, \(x \in \dom{\sigma_1 \oplus x \mapsto
          t}\), so the previous equation can be instantiated by taking
        \(y = x\) and we draw \((\sigma_1 \oplus x \mapsto t)(x) =
        \sigma(x)\), which, by means of \eqref{model:oplus}, leads to
        \begin{equation}
          t = \sigma(x) \label{f0:sound:t}
        \end{equation}
        We can now derive the following:
        \begin{align}
            \subst{\cons{\meta{x}}{p}}{\sigma}
          &\eqn{2} \cons{\sigma(x)}{\subst{p}{\sigma}}\notag\\
          &= \cons{t}{\subst{p}{\sigma}}
          &\text{by} \,\; \eqref{f0:sound:t}\notag\\
          &\sqsubseteq \cons{t}{f}
          &\text{by} \,\; \eqref{f0:sound:7}\notag\\
            \subst{\cons{\meta{x}}{p}}{\sigma}
          &\sqsubseteq \cons{t}{f} \label{f0:sound:9}
        \end{align}
        As a conclusion, the induction hypothesis and
        \eqref{f0:sound:n} imply \eqref{f0:sound:BIND},
        \eqref{f0:sound:incl2} and \eqref{f0:sound:9}, which,
        together, imply 
        \(\ind{P}(\match{\cons{t}{f}}%
                        {\cons{\meta{x}}{p}}%
                        {\sigma_1},%
                  n+1, \sigma)\). \hfill \(\Box\)

    \end{enumerate}

\end{enumerate}
