%%-*-latex-*-

\section{Algorithm \textit{F}(0)}
\label{f0}

\subsection{Definitions}

\paragraph{Abstract Syntax Trees and Unparsed Patterns.}

Trees have been formally defined in paragraph \emph{Abstract Syntax
  Trees} at section~\ref{model}. Here, forests are inductively defined
here as the smallest set \(\cal F\) such that
\begin{itemize}

  \item \(\el \in {\cal F}\);

  \item if \(t \in {\cal H}\) and \(f \in {\cal F}\) then
    \(\cons{t}{f} \in {\cal F}\);

  \item if \(f \in {\cal F}\) then \(\cons{\lpar}{f} \in {\cal F}\) and
    \(\cons{\rpar}{f} \in {\cal F}\).

\end{itemize}
The terms \lpar and \lpar correspond respectively to the opening and
closing parentheses. They can be present in the syntax of the
programming language, i.e., \(\lpar, \rpar \in {\cal L}\), or not, the
point here being that the user is also allowed to insert them in the
unparsed patterns in order to help matching against the programme tree
by specifying that the enclosed unparsed pattern must match the same
subtree. They can be added anywhere in the unparsed pattern as long as
they are properly paired (the native parentheses follow the rules of
the source language). Unparsed patterns are formally defined as the
smallest set \(\cal P\) such that
\begin{itemize}

  \item \(\el \in {\cal P}\);

  \item if \(l \in {\cal L}\) and \(p \in {\cal P}\) then
    \(\cons{l}{p} \in {\cal P}\);

  \item if \(x \in {\cal V}\) and \(p \in {\cal P}\) then
    \(\cons{\meta{x}}{p} \in {\cal P}\);

  \item if \(p_1, p_2 \in {\cal P}\) then \([\lpar] \cdot p_1 \cdot
    [\rpar] \cdot p_2 \in {\cal P}\).

\end{itemize}
The reason why parenthesis are explicitly present in the definition of
\(\cal F\) is technical: during the pattern matching, the
extra\hyp{}parenthesis found in the unparsed pattern are injected in
the forest.


\paragraph{Substitutions and Ground Patterns.} 

Let us extend substitutions to cope with patterns, not just
meta\hyp{}variables. The effect of a substitution on a pattern is to
replace \emph{all} the meta\hyp{}variables it contains by abstract
syntax trees. In order to distinguish a substitution which applies to
a meta\hyp{}variable \(x\) and a substitution which applies to an
unparsed pattern \(p\), we shall note \(\sigma(x)\) the
former and \(\subst{p}{\sigma}\) the latter. The result of
a substitution on a pattern is called a \emph{ground pattern}, whose
set is defined inductively as the smallest set \(\hat{\cal P}\) such
that
\begin{itemize}

  \item \(\el \in \hat{\cal P}\);

  \item if \(t \in {\cal H}\) and \(\hat{p} \in \hat{\cal P}\) then
    \(\cons{t}{p} \in \hat{\cal P}\);

  \item if \(\hat{p}_1, \hat{p}_2 \in \hat{\cal P}\) then
    \([\lpar] \cdot \hat{p}_1 \cdot [\rpar] \cdot \hat{p}_2 \in
    \hat{\cal P}\).

\end{itemize}
The formal definition of substitutions on patterns is given in
figure~\ref{f0_subst_def}.
\begin{figure}
\framebox[\columnwidth]{%
\(
\begin{aligned}
\subst{\el}{\sigma} 
&\eqn{1} \el\\
\subst{\cons{\meta{x}}{p}}{\sigma}
&\eqn{2} \cons{\sigma(x)}{\subst{p}{\sigma}}\\
\subst{\cons{l}{p}}{\sigma}
&\eqn{3} \cons{l}{\subst{p}{\sigma}} 
\qquad \text{if} \; l \in {\cal L} \cup \{\lpar, \rpar\}
\end{aligned}
\)
}
\caption{Substitution \textit{F}(0)\label{f0_subst_def}}
\end{figure}
The first equation means that the substitution on the empty pattern is
always the empty forest. The second equation defines the substitution
of a meta\hyp{}variable by its associated tree: the tree is added to
the resulting forest and the substitution proceeds recursively over
the remaining pattern. The third and last equation specifies that the
substitutions always let the lexemes in the pattern invariant.

\paragraph{Tree Matching.}

We need a relation between a ground pattern and an abstract syntax
tree which captures the concept of \emph{tree matching}. Intuitively,
a ground pattern matches a tree if and only if the elements of the
pattern are embedded in the tree, in the same order. In case the
element is a pattern tree, its subtrees must match the same subtree in
the abstract syntax tree. The elements of a ground pattern contain
either abstract syntax trees or pattern trees whose leaves are leaves
of the abstract syntax tree. Let us note \(\hat{p} \sqsubseteq t\) the
relation `The ground pattern \(\hat{p}\) matches the abstract syntax
tree \(t\).' The formal definition we propose is based on ordered
inference rules (see formal model), given in
figure~\ref{f0_tree_matching_def}. Rule \textsf{ONE} states that a
pattern matches a tree if the same pattern matches the corresponding
singleton forest. This allows to relate a pattern and a tree. Rule
\textsf{EMP} says that the empty pattern always matches the empty
forest. Rule \textsf{EQ} specifies that if the head of the pattern and
the forest is the same abstract syntax tree, then the pattern matches
the forest if the pattern and the tree without their heads match. Rule
\textsf{SUB}, to be always considered after rule \textsf{EQ} when
selecting a rule based on the conclusion (the system is ordered),
states that if the two heads are different trees, there is a match if
the pattern matches the catenation of the subtrees \(f_1\) of the tree
in the forest, embedded in a pair of parenthesis, and the remaining
forest \(f_2\).
\begin{figure}
\framebox[\columnwidth]{
\begin{mathpar}
\inferrule*{}{\el \sqsubseteq \el}
\;\TirName{\text{\sf EMP}}

\inferrule
  {t \in {\cal H} \cup \{\rpar\}\and \hat{p} \sqsubseteq f}
  {\cons{t}{\hat{p}} \sqsubseteq \cons{t}{f}}
\,\TirName{\text{\sf EQ}}

\inferrule*[right=\text{\sf SUB}]
  {\hat{p} \sqsubseteq f_1 \cdot [\rpar] \cdot f_2}
  {\cons{\lpar}{\hat{p}} \sqsubseteq \cons{c(f_1)}{f_2}}
\quad
\inferrule*[right=\text{\sf ONE}]
  {\hat{p} \sqsubseteq [t]}
  {\hat{p} \sqsubseteq t}
\end{mathpar}
}
\caption{Tree matching \textit{F}(0)\label{f0_tree_matching_def}}
\end{figure}

\paragraph{Pattern Matching.} 

Pattern matching is defined by the rewrite system given in
figure~\ref{f0_match_def}, where \(l \in {\cal L} \cup \{\rpar\}\),
\(t \in {\cal T}\) and \(\sigma \subseteq \sigma \oplus x \mapsto t\).
\begin{figure}
\framebox[\columnwidth]{
\(
\begin{aligned}
\match{\el}{\el}{\sigma} 
& \rightarrow \sigma
& \!\!\textsf{END}\\
\match{\cons{l}{f}}{\cons{l}{p}}{\sigma} 
& \rightarrow \match{f}{p}{\sigma}
& \!\!\textsf{ELIM}\\
\match{\cons{c(f_1)\!}{\!f_2}}{\cons{\lpar\!}{\!p}}{\sigma}
& \rightarrow
\match{f_1 \cdot \cons{\rpar\!}{\!f_2}}{p}{\sigma}
& \!\!\textsf{UNPAR}\\
\match{\cons{t}{f}}{\cons{\meta{x}}{p}}{\sigma}
& \rightarrow
\match{f}{p}{\sigma \oplus x \mapsto t}
& \!\!\textsf{BIND}
\end{aligned}
\)
}
\caption{Pattern matching \textit{F}(0)\label{f0_match_def}}
\end{figure}
A term \(\match{f}{p}{\sigma}\) is called a \emph{configuration}. The
rewrite system above is syntax\hyp{}directed, i.e., the shape alone of
each configuration selects either no rewrite rule or only one. Let us
define \(\smat{t}{p} \triangleq \match{[t]}{p}{\sigma}\) whenever
\(\dom{\sigma} = \varnothing\). The pattern matching relation
\((\rightarrow)\) is defined by means of a rewrite system which is
syntax\hyp{}directed, therefore the relation is actually a
function. Rule \textsf{END} is the rule that must terminate any valid
series of rewrite steps. Rule \textsf{ELIM} eliminates the same lexeme
from the heads of the pattern and the forest. Rule \textsf{UNPAR}
unparses the abstract syntax tree in the head of the forest when an
opening parenthesis \lpar is matched against it. This means that the
tree is replaced by its subtrees between a pair of parentheses. Note
that it would have been more natural to define \textsf{UNPAR} as
\begin{align*}
\match{\cons{c(f_1)\!}{\!f_2}}{\cons{\lpar\!}{\!p}}{\sigma}
&\rightarrow
\match{[\lpar] \! \cdot f_1 \! \cdot \! [\rpar] \! \cdot \!
  f_2}{\cons{\lpar\!}{\!p}}{\sigma}
\intertext{but since the next step would necessarily be
  \textsf{ELIM}:}
&\rightarrow \match{f_1 \! \cdot \! [\rpar] \! \cdot \!
  f_2}{p}{\sigma}
\end{align*}
 we chose to optimise a bit by defining \textsf{UNPAR} as the
 composition of both rules. Finally, rule \textsf{BIND} handles the
 case of a meta\hyp{}variable: it must be bound to a tree (not a
 lexeme), by means of a substitution update.


\subsection{Soundness}
%%-*-latex-*-

\begin{theorem}[Soundness]\hfill
\label{f0:soundness}
\begin{center}
If \(\smat{t}{p} \xrightarrow{+} \sigma\) 
then \(\subst{p}{\sigma} \sqsubseteq t\).
\end{center}
\end{theorem}

\paragraph{Proof.}

Let \(\ind{P}(\match{f}{p}{\sigma'}, n, \sigma)\) be the proposition
\begin{center}
\emph{If} \(\match{f}{p}{\sigma'} \xrightarrow{n} \sigma\)
\emph{then} \(\sigma' \subseteq \sigma\)
\emph{and} \(\subst{p}{\sigma} \sqsubseteq f\).
\end{center}
Then \(\forall p, f, \sigma', n, \sigma.\ind{P}(\match{f}{p}{\sigma'},
n, \sigma)\) implies the soundness property by choosing
\(\dom{\sigma'} = \varnothing\) and \(f = [t]\). Therefore, let us
assume
\begin{equation}
\match{f}{p}{\sigma'} \xrightarrow{n} \sigma \label{f0:sound:n}
\end{equation}
that is to say, there exists a successful series \(\Delta\) of \(n\)
rewrite steps (called a \emph{rewrite}) from \(\match{f}{p}{\sigma'}\)
to \(\sigma\). (If not, i.e., if \(\Delta\) is empty, then the theorem
is trivially true, since it has the form \(A \Rightarrow B\), which is
equivalent to \(\neg{A} \vee B\).) We hence can reason by induction on
length of \(\Delta\), that is to say, we assume that \(\ind{P}(\kappa,
n, \sigma)\) holds for any \(n\) (that is the \emph{induction
  hypothesis}) and then prove that all the ways to extend backwards
\(\Delta\) with one more step result in a configuration \(\kappa'\)
such that \(\ind{P}(\kappa', n+1, \sigma)\) holds too. Indeed, by
definition of \((\rightarrow)\), we have
\[
\kappa' \rightarrow \underbrace{\kappa \xrightarrow{n} \sigma}_{\Delta}
\quad \Longrightarrow \quad
\kappa' \xrightarrow{n+1} \sigma
\]
The complete\hyp{}induction principle implies then that \(\forall p,
f, \sigma', n, \sigma.\ind{P}(\match{f}{p}{\sigma'}, n, \sigma)\),
thus the soundness. We discriminate first on the length of \(\Delta\),
then case by case on the shape of the configuration
\(\match{f}{p}{\sigma'}\).
\begin{enumerate}

  \item \(\Delta\) is made of one rewrite step.\\ In this case,
    \(\Delta\) is made of \textsf{END}, since \textsf{END} is the
    unique terminating rule. Its syntax implies that \(f = \el\), \(p
    = \el\) and \(\sigma' = \sigma\). These definitions lead to
    \(\sigma' \subseteq \sigma\) and \(\subst{p}{\sigma} =
    \subst{\el}{\sigma} \eqn{1} \el \sqsubseteq \el = f\). Therefore
    \(\ind{P}(\match{f}{p}{\sigma'}, 1, \sigma)\) holds.

  \item \(\Delta\) is made of at least two rewrite steps.\\ A quick
    examination of the rules reveals that all the ways to extend
    backwards \(\Delta\) can be characterised by discriminating only
    upon the element (lexeme, parenthese or meta\hyp{}variable) added
    to the pattern \(p\). Assuming the induction hypothesis
    \(\ind{P}(\match{f}{p}{\sigma'}, n, \sigma)\) and
    \eqref{f0:sound:n}, we get
    \begin{align}
      \sigma' &\subseteq \sigma \label{f0:sound:6}\\
      \subst{p}{\sigma} &\sqsubseteq f \label{f0:sound:7}
    \end{align}
    we have the following cases to consider.
    \begin{enumerate}

      \item Case where \textsf{ELIM} extends \(\Delta\).\\ This means
        that we have \(l \in {\cal L} \cup \{\rpar\}\) in
        \begin{align}
            \match{\cons{l}{f}}{\cons{l\!}{\!p}}{\sigma'}
          &\rightarrow
            \match{f}{p}{\sigma'}
            &\textsf{ELIM} \notag\\
          &\xrightarrow{n} \sigma 
            & (\Delta) \notag\\
            \match{\cons{l}{f}}{\cons{l\!}{\!p}}{\sigma'}
          &\xrightarrow{n+1} \sigma \label{f0:sound:ELIM}
            \intertext{Furthermore, we have the following equalities:}
           \subst{\cons{l}{p}}{\sigma}
          &\eqn{2} \cons{l}{\subst{p}{\sigma}}\notag\\
          &\sqsubseteq \cons{l}{f}
          &\text{by} \,\; \eqref{f0:sound:7} \,\; \text{and} \,\;
           \textsf{EQ}\notag\\
             \subst{\cons{l}{p}}{\sigma}
          &\sqsubseteq \cons{l}{f} \label{f0:sound:ELIM:conc}
        \end{align}
        As a conclusion, the induction hypothesis and
        \eqref{f0:sound:n} imply \eqref{f0:sound:ELIM},
        \eqref{f0:sound:6} and \eqref{f0:sound:ELIM:conc}, which,
        together, imply
        \(\ind{P}(\match{\cons{l}{f}}{\cons{l\!}{\!p}}{\sigma'}, n+1,
        \sigma)\).

      \item Case where \textsf{UNPAR} extends \(\Delta\).\\ The syntax
        of the rule implies that there exists a sub\hyp{}forest \(f_2\)
        of \(f\) and a tree \(c(f_1)\) such that
        \begin{enumerate}

          \item \label{f0:sound:1} \(f \triangleq f_1 \cdot
            \cons{\rpar}{f_2}\)

        \end{enumerate}
        and
        \begin{align}
          \match{\cons{c(f_1)\!}{\!f_2}}{\cons{\lpar\!}{\!p}}{\sigma'}
          &\rightarrow
          \match{f_1 \cdot \cons{\rpar\!}{\!f_2}}{p}{\sigma'}
          \notag\\
          & \qquad\qquad\quad\ \textsf{UNPAR} \notag\\
          &\xrightarrow{n} \sigma
          \qquad\qquad\quad(\Delta) \notag\\
          \match{\cons{c(f_1)\!}{\!f_2}}{\cons{\lpar\!}{\!p}}{\sigma'}
          &\xrightarrow{n+1} \sigma \label{f0:sound:UNPAR}
        \end{align}
        Furthermore, we have the following equalities:
        \begin{align}
          \subst{\cons{\lpar}{p}}{\sigma}
          &\eqn{3} \cons{\lpar}{\subst{p}{\sigma}}\notag\\
          &\sqsubseteq \cons{\lpar}{f}
          &\text{by} \,\; \eqref{f0:sound:7}\notag\\
          &= \cons{\lpar}{f_1 \cdot \cons{\rpar}{f_2}}
          &\text{by} \,\; \text{\ref{f0:sound:1}}\notag\\
          &= \cons{\lpar}{f_1} \cdot \cons{\rpar}{f_2}
          &\text{by} \,\; \eqref{model:conc:2}\notag\\
          &= \cons{\lpar\!}{\!\el\!\cdot\!f_1}
             \!\cdot\!\cons{\rpar\!}{\!\el\!\cdot\!f_2}
          &\text{by} \,\; \eqref{model:conc:1}\notag\\
          &= ([\lpar] \!\cdot\! f_1) \!\cdot\! ([\rpar] \!\cdot\! f_2)
          &\text{by} \,\; \eqref{model:conc:2}\notag\\
          \subst{\cons{\lpar}{p}}{\sigma}
          &\sqsubseteq \cons{c(f_1)}{f_2}
          &\text{by} \,\; \textsf{SUB} \label{f0:sound:8}
        \end{align}
        As a conclusion, the induction hypothesis and
        \eqref{f0:sound:n} imply \eqref{f0:sound:UNPAR},
        \eqref{f0:sound:6} and \eqref{f0:sound:8}, which, together,
        imply \(\ind{P}(\match{\cons{c(f_1)}{f_2}}{\cons{\lpar}{p}}%
        {\sigma'}, n+1, \sigma)\).

      \item Case where \textsf{BIND} extends \(\Delta\).\\ The syntax
        of the rule implies that there exists a substitution
        \(\sigma_1\) such that
        \begin{enumerate}

          \item \(\sigma_1 \subseteq \sigma_1 \oplus x \mapsto t \triangleq
          \sigma'\) \label{f0:sound:incl1}

        \end{enumerate}
        and
        \begin{align}
           \match{\cons{t}{f}}{\cons{\meta{x}}{p}}{\sigma_1}
          &\rightarrow
           \match{f}{p}{\sigma'}
          &\textsf{BIND} \notag\\
          &\xrightarrow{n} \sigma
          & (\Delta) \notag\\
           \match{\cons{t}{f}}{\cons{\meta{x}}{p}}{\sigma_1}
           &\xrightarrow{n+1} \sigma \label{f0:sound:BIND}
        \end{align}
        Inclusions \ref{f0:sound:incl1} and \eqref{f0:sound:6} imply, by
        \eqref{model:incl:trans}:
        \begin{align}
          \sigma_1 \subseteq \sigma \label{f0:sound:incl2}\\
          \intertext{Moreover \eqref{f0:sound:6} and \ref{f0:sound:incl1}
            imply}
          \sigma_1 \oplus x \mapsto t \subseteq \sigma\notag
        \end{align}
        which is equivalent, by definition \eqref{model:incl}, to
        \[
        \forall y \in \dom{\sigma_1 \oplus x \mapsto t}.((\sigma_1
        \oplus x \mapsto t)(y) \!=\! \sigma(y))
        \] 
        By \eqref{model:dom}, \(x \in \dom{\sigma_1 \oplus x \mapsto
          t}\), so the previous equation can be instantiated by taking
        \(y = x\) and we draw \((\sigma_1 \oplus x \mapsto t)(x) =
        \sigma(x)\), which, by means of \eqref{model:oplus}, leads to
        \begin{equation}
          t = \sigma(x) \label{f0:sound:t}
        \end{equation}
        We can now derive the following:
        \begin{align}
            \subst{\cons{\meta{x}}{p}}{\sigma}
          &\eqn{2} \cons{\sigma(x)}{\subst{p}{\sigma}}\notag\\
          &= \cons{t}{\subst{p}{\sigma}}
          &\text{by} \,\; \eqref{f0:sound:t}\notag\\
          &\sqsubseteq \cons{t}{f}
          &\text{by} \,\; \eqref{f0:sound:7}\notag\\
            \subst{\cons{\meta{x}}{p}}{\sigma}
          &\sqsubseteq \cons{t}{f} \label{f0:sound:9}
        \end{align}
        As a conclusion, the induction hypothesis and
        \eqref{f0:sound:n} imply \eqref{f0:sound:BIND},
        \eqref{f0:sound:incl2} and \eqref{f0:sound:9}, which,
        together, imply 
        \(\ind{P}(\match{\cons{t}{f}}%
                        {\cons{\meta{x}}{p}}%
                        {\sigma_1},%
                  n+1, \sigma)\). \hfill \(\Box\)

    \end{enumerate}

\end{enumerate}


