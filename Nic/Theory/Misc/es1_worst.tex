%%-*-latex-*-

\textbf{TEMPORARY: THIS SECTION IS OUT-DATED.........}

\medskip

Let us build a configuration such that the proof tree for the pattern
matching contains the highest possible number of judgements: since
each judgement corresponds to the application of an inference rule,
i.e., one step of the matching algorithm, their total number
corresponds to the total number of steps\,---\,which we choose here to
be the cost (complexity) measure.

An inductive examination of every rule, i.e., read from conclusion to
premises, reveals that the pattern's length always strictly decreases
by one or two lexemes, except for the rule \textsf{UNPAR}\(_1\), which
keeps it invariant. So, if we want to have a proof tree strictly
longer than the pattern, we must repeat the application of this
rule. The longer the repetition, the longer the tree.

First, let us consider a forest made of a single tree and a pattern
made of a single element \(m\). Since the proof tree is actually a
list (there is only one pattern\hyp{}matching premise at most for each
inference rule), its length is exactly the cost. Therefore let us
enumerate all the cases for \(m\) and find out which tree in the
forest maximises the proof length.
\begin{enumerate}

  \item \(m = \mlp\).\\ Only rule \textsf{UNPAR}\(_2\) can be applied,
    but it should be followed by \textsf{END}, since the pattern is
    shortened by one (and we have only one element in it). So this
    case is not the worst.

  \item \label{worst:mrp} \(m = \mrp\).\\ Only rule \textsf{ELIM} can
    be applied, but it should be followed by \textsf{END}, since the
    pattern is shortened by one (and we have only one element in
    it). So this case is not the worst.

  \item \(m = \meta{x}\).\\ Only rules \textsf{BIND}\(_3\) and
    \textsf{UNPAR}\(_1\) can be applied. The former is an axiom, so it
    does not lead to the worst case (just one step). The latter cannot
    be used because our system is ordered in such a way that binding
    rules have priority over unparsing rules. So, this case is not the
    worst.

  \item \label{worst:l} \(m \in {\cal L}\).\\ Only rules \textsf{ELIM}
    and \textsf{UNPAR}\(_1\) can be applied. The former leads to the
    case \ref{worst:mrp}. The latter is worth trying. Each application
    of \textsf{UNPAR}\(_1\) cuts the root of the first tree in the
    forest, so, since we want to maximise the repetition of
    \textsf{UNPAR}\(_1\), we should choose a tree which is actually a
    list. In the end, the unique leaf (i.e., a lexeme) is eliminated
    by \textsf{ELIM} if it is the element in the pattern. If the tree
    is a list of \(n\) nodes (including the leaf), we need \(n-1\)
    applications of \textsf{UNPAR}\(_1\) and \(1\) of \textsf{ELIM}.

\end{enumerate}
If we consider now having more elements in the pattern, the case
\ref{worst:l} shows that the worst case happens when there are only
lexemes in the pattern, and these lexemes are the leaves of the
tree. The cost then is, again, exactly the number of nodes in the
tree.
\begin{theorem}[Worst case]
For a tree of \(n\) nodes, the maximum number of steps of the matching
algorithm ES(1) is \(n\) (and the pattern is its fringe).
\end{theorem}
\noindent In other words, the ES(\(1\)) matching algorithm requires at
most exactly one complete traversal of the initial abstract syntax
tree.
