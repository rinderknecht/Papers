%%-*- latex -*-

The completeness of our algorithm means that every time a substitution
on a pattern matches a tree, our algorithm computes a substitution
which is included in the first one (which may contain useless
meta\hyp{}variable bindings). In other words, all minimal
substitutions that enable a tree\hyp{}matching are computed by our
pattern matching. Formally, this can be expressed as the following
theorem.
\begin{theorem}[Completeness]\hfill
\begin{center}
If   \(\subst{{p}}{\sigma} \sqsubseteq h\)
then \(\smat{h}{{p}} \twoheadrightarrow \sigma'\) 
and  \(\sigma' \subseteq \sigma\).
\end{center}
\end{theorem}


\paragraph{Proof.}

Let \(\ind{Z}({p}, f, \sigma, \sigma')\) be the proposition
\begin{center}
\emph{If}   \(\subst{{p}}{\sigma} \sqsubseteq f\)
\emph{then} \(\smj{f}{{p}}{\sigma'}\)
\emph{and} \(\sigma' \subseteq \sigma\).
\end{center}
The completeness is equivalent to \(\ind{Z}({p}, [h], \sigma,
\sigma')\). First, let us assume that
\begin{align}
  \subst{{p}}{\sigma} \sqsubseteq f \label{es1:compl:1}
\end{align}
(otherwise the theorem would be trivially true). This means that there
exists a derivation \(\Delta\) in the inference system defining
\((\sqsubseteq)\), at figure~\ref{es1_tree_matching_def}, whose
conclusion is \(\subst{{p}}{\sigma} \sqsubseteq f\). This
derivation is a tree, which makes it possible to reckon by general
induction on its structure, i.e., we assume that \(\ind{Z}\) holds for
the conclusion of any sub\hyp{}derivation of \(\Delta\) (the
\emph{induction hypothesis}) and then prove that \(\ind{Z}\) holds for
\(\subst{{p}}{\sigma} \sqsubseteq f\). We proceed case by
case on the kind of rule that can end the derivation.
\begin{enumerate}

  \item Case where \(\Delta\) ends with \textsf{EMP}.\\ We have \(f =
    \el\) and also \(\subst{{p}}{\sigma} = \el \eqn{1}
    \subst{\el}{\sigma}\). Therefore \({p} = \el\) (the
    substitution is injective by construction), which leads to
    \(\smat{f}{{p}} = \smj{\el}{\el}{\sigma_\varnothing}\) by
    means of the axiom \textsf{END} (figure~\ref{es1_match_def}). And
    we trivially have \(\sigma_\varnothing \subseteq \sigma\). We
    conclude that \(\ind{W}(\el, \el, \sigma, \sigma_\varnothing)\)
    holds.

  \item Case where \(\Delta\) ends by \textsf{EQ}.\\ Then, for all
    abstract syntax tree \(h_1\) (perhaps reduced to a single lexeme),
    there exists a closed pattern \(\close{p}\) and a forest \(f'\) such
    that
    \begin{align}
      \subst{{p}}{\sigma} &\triangleq
      \cons{h_1}{\close{p}} \label{es1:compl:2}\\ 
      f &\triangleq \cons{h_1}{f'} \label{es1:compl:3}
    \end{align}
    and \(\Delta\) has the shape
    \begin{mathpar}
      \inferrule*[right={\text{\sf EQ}}]
        {\inferrule*
           {\inferrule*[vdots=1.5em]{}{ }}
           {\close{p} \sqsubseteq f'}}
        {\cons{h_1}{\close{p}} \sqsubseteq \cons{h_1}{f'}}
    \end{mathpar}
    By examining the definition of substitutions at
    figure~\ref{es1_subst_def}, we deduce that we have two exclusive
    options for \(h_1\):
    \begin{enumerate}

      \item Case where \(h_1 = l \in {\cal L}\).\\ Definition
        \eqref{es1:compl:2} and definition \((\eqn{3})\) of the
        substitutions imply that there exists a meta\hyp{}parsed
        pattern \({p}'\) such that
        \begin{align}
          {p} &= \cons{l}{{p}'}\label{es1:compl:4}\\
          \close{p} &= \subst{{p}'}{\sigma}\notag
        \end{align}
        So \(\Delta\) has the refined shape
        \begin{mathpar}
          \inferrule*[right={\text{\sf EQ}}]
            {\inferrule*
               {\inferrule*[vdots=1.5em]{}{ }}
               {\subst{{p}'}{\sigma} \sqsubseteq f'}}
            {\subst{{p}}{\sigma} \sqsubseteq f}
        \end{mathpar}
        Let us assume that the induction hypothesis holds for the
        premise of \textsf{EQ}, i.e.,
        \begin{align}
          \smat{f'}{{p}'} &\twoheadrightarrow
          \sigma' \label{es1:compl:6}\\
          \sigma' &\subseteq \sigma \label{es1:compl:7}
        \end{align}
        Configuration \eqref{es1:compl:6} can be the premise of
        pattern\hyp{}matching rule \textsf{ELIM}
        (figure~\ref{es1_match_def}), whose conclusion is then
        \begin{gather}
          \smj{\cons{l}{f'}}{\cons{l}{{p}'}}{\sigma'} 
          \label{es1:compl:8}  
        \end{gather}
        By \eqref{es1:compl:3} and \eqref{es1:compl:4},
        \eqref{es1:compl:8} is equivalent to
        \begin{gather}
          \smj{f}{{p}}{\sigma'} \label{es1:compl:9}
        \end{gather}
        As a conclusion, the induction hypothesis and
        \eqref{es1:compl:1} imply \eqref{es1:compl:9} and
        \eqref{es1:compl:7}, i.e., \(\ind{W}(\cons{l}{{p}'},
        \cons{l}{f'}, \sigma, \sigma')\) holds.

      \item Case where \(h_1 = t_1 \in {\cal T}\).\\
        Definition \eqref{es1:compl:2} and definition \((\eqn{2})\) of
        the substitutions imply that there exists a meta\hyp{}parsed
        pattern \({p}'\) and a meta\hyp{}variable \(x\) such
        that
        \begin{align}
          {p} &=
          \cons{\meta{x}}{{p}'} \label{es1:compl:10}\\
          \close{p} &= \subst{{p}'}{\sigma}\notag\\
          t_1 &= \sigma(x) \label{es1:compl:12}
        \end{align}
        So \(\Delta\) has the refined shape
        \begin{mathpar}
          \inferrule*[right={\text{\sf EQ}}]
            {\inferrule*
               {\inferrule*[vdots=1.5em]{}{ }}
               {\subst{{p}'}{\sigma} \sqsubseteq f'}}
            {\subst{\cons{\meta{x}}{{p}'}}{\sigma} \sqsubseteq
              \cons{t_1}{f'}}
        \end{mathpar}
        Let us now consider all the sub\hyp{}cases for \(f'\).
        \begin{enumerate}

          \item Case \(f' = \el\).\\ In this case, the premise of
            \textsf{EQ} is simplified into
            \(\subst{{p}'}{\sigma} \sqsubseteq \el\), which
            can only be an instance of the axiom \textsf{EMP} (see
            figure~\ref{es1_tree_matching_def}), therefore
            \(\subst{{p}'}{\sigma} = \el\). Equation
            \((\eqn{1})\) at figure~\ref{es1_subst_def} implies then
            that
            \begin{gather}
              {p}' = \el \label{es1:compl:16}
            \end{gather}
            (the substitutions are injective by definition). Equations
            \eqref{es1:compl:10} and \eqref{es1:compl:16} imply
            \begin{gather}
              {p} = [\meta{x}] \label{es1:compl:17}
            \end{gather}
            Equations \(f' = \el\), \eqref{es1:compl:3},
            \eqref{es1:compl:17} and axiom \textsf{BIND}\(_3\) at
            figure~\ref{es1_match_def} imply
            \begin{gather}
              \smj{f}{{p}}{\{x \mapsto t_1\}} \label{es1:compl:18}
            \end{gather}
            The definition \eqref{model:incl} of substitution
            inclusions trivially establishes that
            \begin{gather}
              \{x \mapsto t_1\} \subseteq \sigma \label{es1:compl:19}
            \end{gather}
            because of \eqref{es1:compl:12}. As conclusion, the
            induction hypothesis and \eqref{es1:compl:1} imply
            \eqref{es1:compl:18} and \eqref{es1:compl:19}, i.e.,
            \(\ind{W}([\meta{x}], [t_1], \sigma, \{x \mapsto t_1\})\)
            holds.

          \item \label{es1:compl:26} Case \(f' = \cons{l}{f''}\), with
            \(l \in {\cal L}\).\\ Then the premise of \textsf{EQ} is
            refined into \(\subst{{p}'}{\sigma} \sqsubseteq
            \cons{l}{f''}\). It can only be the conclusion of another
            instance of the same rule \textsf{EQ}, so there exists a
            meta\hyp{}parsed pattern \({p}''\) such that
            \({p}' = \cons{l}{{p}''}\) and the
            derivation \(\Delta\) is actually
            \begin{mathpar}
              \inferrule*[right={\text{\sf EQ}}]
                {\inferrule*[right={\text{\sf EQ}},leftskip=1em]
                   {\inferrule*[leftskip=1em]
                     {\inferrule*[vdots=1.5em]{}{(\Delta')}}
                     {\subst{{p}''}{\sigma} \sqsubseteq f''}}
                   {\subst{\cons{l}{{p}''}}{\sigma}
                     \sqsubseteq \cons{l}{f''}}}
                {\subst{\cons{\meta{x},l}{{p}''}}{\sigma}
                  \sqsubseteq \cons{t_1,l}{f''}}
            \end{mathpar}
            and
            \begin{gather}
              {p} = \cons{\meta{x},l}{{p}''}
              \label{es1:compl:20}
            \end{gather}
            Since \(\Delta'\) is a sub\hyp{}derivation of \(\Delta\),
            the induction hypothesis applies to its conclusion, i.e.,
            there exists a substitution \(\sigma'\) such that
            \begin{align}
              \smat{f''}{{p}''} &\twoheadrightarrow
              \sigma' \label{es1:compl:21}\\
              \sigma' &\subseteq \sigma \label{es1:compl:22}
            \end{align}
            We need a short lemma about the substitution \(\sigma'\)
            before we analyse \(\Delta\) further. Let us consider
            \begin{itemize}

              \item \(x \not\in \dom{\sigma'}\).\\ Then, by definition
                of inclusions of substitutions \eqref{model:incl} 
                implies
                \begin{gather*}
                  \sigma' \subseteq \sigma' \oplus x \mapsto t_1
                \end{gather*}

              \item \(x \in \dom{\sigma'}\).\\ The definition of
                inclusions \eqref{model:incl} applied to
                \eqref{es1:compl:22} imply that \(\sigma'(x) =
                \sigma(x)\), thus, by \eqref{es1:compl:12}, we have
                \(\sigma'(x) = t_1\). So \(\sigma' \subseteq \sigma'
                \oplus x \mapsto t_1\), since, by definition
                \eqref{model:oplus}, \((\sigma' \oplus x \mapsto t)(x)
                \triangleq t_1 = \sigma'(x)\).

            \end{itemize}
            In both cases, we proved that
            \begin{gather}
              \sigma' \subseteq \sigma' \oplus x \mapsto t_1
              \label{es1:compl:23}
            \end{gather}
            Relations \eqref{es1:compl:21} and \eqref{es1:compl:23}
            can be the premise of an instance of rule
            \textsf{BIND}\(_1\):
            \begin{mathpar}
              \inferrule
                {\smj{f''}{{p}''\!}{\!\sigma'}
                 \and
                 \sigma' \subseteq \sigma' \oplus x \mapsto t_1}
                {\smj{\cons{t_1, l\!}{\!f''}}%
                     {\cons{\meta{x}, l\!}{\!{p}''}}%
                     {\sigma' \!\oplus\! x \!\mapsto\! t_1}}
            \end{mathpar}
            In other words, by \eqref{es1:compl:20} and the definition
            of \(f\), the following holds:
            \begin{gather}
              \smj{f}{{p}}{\sigma' \oplus x \mapsto
                t_1} \label{es1:compl:24}
            \end{gather}
            For all \(y \in \dom{\sigma' \oplus x \mapsto t_1}\),
            there are two cases:
            \begin{itemize}

              \item \(y = x\).\\ By definition \eqref{model:oplus} and
                \eqref{es1:compl:12} we have \((\sigma' \oplus x
                \mapsto t_1)(y) \triangleq t_1 = \sigma(x) =
                \sigma(y)\).

              \item \(y \neq x\).\\Then, by definition
                \eqref{model:oplus}, \((\sigma' \oplus x \mapsto
                t_1)(y) = \sigma'(y)\).  Besides, \eqref{es1:compl:22}
                is equivalent, by definition \eqref{model:incl} to
                \begin{gather*}
                  \forall y \in \dom{\sigma'}.(\sigma'(y) =
                  \sigma(y))
                \end{gather*}
                So \((\sigma' \oplus x \mapsto t_1)(y) = \sigma(y)\).

            \end{itemize}
            In both cases, we established that
            \begin{gather}
              \sigma' \oplus x \mapsto t_1 \subseteq
              \sigma \label{es1:compl:25}
            \end{gather}
            due to definition of inclusion \eqref{model:incl}. As a
            conclusion, the induction hypothesis and
            \eqref{es1:compl:1} imply \eqref{es1:compl:24} and
            \eqref{es1:compl:25}, i.e.,
            \(\ind{W}(\cons{\meta{x},l}{{p}''},
            \cons{t_1,l}{f''}, \sigma, \sigma' \oplus x \mapsto t_1)\)
            holds.

          \item \(f' = \cons{t_2}{f''}\).\\ Then the premise of
            \textsf{EQ} is refined into
            \(\subst{{p}'}{\sigma} \sqsubseteq
            \cons{t_2}{f''}\). Let us assume that the induction
            hypothesis holds for it, i.e., there exists a substitution
            \(\sigma'\) such that
            \begin{align}
              \smat{\cons{t_2}{f''}}{{p}'} &\twoheadrightarrow
              \sigma' \label{es1:compl:13}\\
              \sigma' &\subseteq \sigma\notag
            \end{align}
            Just as in the case~\ref{es1:compl:26}, we can prove
            \begin{align}
              \sigma' &\subseteq \sigma' \oplus x \mapsto t_1
              \label{es1:compl:15}\\
              \sigma' \oplus x \mapsto t_1 &\subseteq \sigma
              \label{es1:compl:27}
            \end{align}
            Relations~\eqref{es1:compl:13} and \eqref{es1:compl:15}
            can be the premises of an instance of the rule
            \textsf{BIND}\(_2\):
            \begin{mathpar}
              \inferrule
                {\smj{\cons{t_2}{f''}}{{p}'}{\sigma'}
                 \and
                 \sigma' \subseteq \sigma' \oplus x \mapsto t_1}
                {\smj{\cons{t_1, t_2\!}{\!f''}}%
                     {\cons{\meta{x}\!}{\!{p}'}}%
                     {\sigma' \oplus x \mapsto t_1}}
            \end{mathpar}
            In other words
            \begin{gather}
              \smj{f}{{p}}{\sigma' \oplus x \mapsto
                t_1} \label{es1:compl:28}
            \end{gather}
            As a conclusion, the induction hypothesis and
            \eqref{es1:compl:1} imply \eqref{es1:compl:28} and
            \eqref{es1:compl:27}, i.e.,
            \(\ind{W}(\cons{\meta{x}}{{p}'},
            \cons{t_1,t_2}{f''}, \sigma, \sigma' \oplus x \mapsto
            t_1)\) holds.
        \end{enumerate}

    \end{enumerate}

  \item Case where \(\Delta\) ends by \textsf{PAT}.\\ Then there
    exists two closed patterns \(\close{p}_1\) and \(\close{p}_2\) as
    well as a constructor \(c\) and a two forests \(f_1\) and \(f_2\)
    such that
    \begin{align}
         \subst{{p}}{\sigma} 
      &= \cons{\pat{\close{p}_1}}{\close{p}_2} \label{es1:compl:29}\\
         f 
      &= \cons{c(f_1)}{f_2} \label{es1:compl:30}
    \end{align}
    Equations \eqref{es1:compl:29} and \((\eqn{4})\) at
    figure~\ref{es1_subst_def} imply that there exists two unparsed
    patterns \({p}_1\) and \({p}_2\) such that
    \begin{align*}
         \subst{{p}_1}{\sigma}
      &= \close{p}_1\\
         \subst{{p}_2}{\sigma}
      &= \close{p}_2\\
         {p}
      &= \cons{\pat{{p}_1}}{{p}_2}
    \end{align*}
    Thus the premises of \textsf{PAT} are refined into
    \begin{align*}
      \subst{{p}_1}{\sigma} &\sqsubseteq f_1\\
      \subst{{p}_2}{\sigma} &\sqsubseteq f_2
    \end{align*}
    Let us assume that the induction hypothesis holds for them, i.e.,
    there exists two substitutions \(\sigma_1\) and \(\sigma_2\) such
    that
    \begin{align}
       \smat{f_1}{{p}_1} &\twoheadrightarrow
       \sigma_1 \label{es1:compl:31}\\
       \sigma_1 &\subseteq \sigma \label{es1:compl:32}\\
       \smat{f_2}{{p}_2} &\twoheadrightarrow
       \sigma_2 \label{es1:compl:33}\\
       \sigma_2 &\subseteq \sigma \label{es1:compl:34}
    \end{align}
    By definition \eqref{model:incl} of inclusion,
    \eqref{es1:compl:32} and \eqref{es1:compl:34} imply
    \begin{gather}
      \forall x \in \dom{\sigma_1} \cap \dom{\sigma_2}.(\sigma_1(x) =
      \sigma_2(x))
    \end{gather}
    which, in turn, implies
    \begin{gather}
      \sigma_1 \subseteq \sigma_1 \oplus \sigma_2 \label{es1:compl:35}
    \end{gather}
    Judgements \eqref{es1:compl:31}, \eqref{es1:compl:33} and
    \eqref{es1:compl:35} can serve as premises to an instance of rule
    \textsf{UNPAR}\(_1\), whose conclusion is then
    \begin{gather}
      \smj{\cons{c(f_1)\!}{\!f_2}}%
          {\cons{\pat{{p}_1}}{\!{p}_2}}%
          {\sigma_1 \oplus \sigma_2} \label{es1:compl:36}
    \end{gather}
    Besides, by definition \eqref{model:incl} of inclusion, definition
    \eqref{model:oplus} of updates, \eqref{es1:compl:32} and
    \eqref{es1:compl:34} imply
    \begin{gather}
      \forall x \in \dom{\sigma_1} \cup \dom{\sigma_2}.((\sigma_1
      \oplus \sigma_2)(x) = \sigma(x)) \label{es1:compl:37}
    \end{gather}
    Since we trivially have
    \begin{gather*}
      \dom{\sigma_1 \oplus \sigma_2} = \dom{\sigma_1} \cup
      \dom{\sigma_2}
    \end{gather*}
    then \eqref{es1:compl:37} is equivalent to
    \begin{gather}
      \sigma_1 \oplus \sigma_2 \subseteq \sigma \label{es1:compl:38}
    \end{gather}
    As a conclusion, the induction hypothesis and \eqref{es1:compl:1}
    imply \eqref{es1:compl:36} and \eqref{es1:compl:38}, i.e.,
    \(\ind{W}(\cons{\pat{{p}_1}}{\!{p}_2},
    \cons{c(f_1)\!}{\!f_2}, \sigma, \sigma_1 \oplus \sigma_2)\) holds.

  \item Case where \(\Delta\) ends by \textsf{SUB}.\\ Then there
    exists a constructor \(c\) and two forests \(f_1\) and \(f_2\)
    such that
    \begin{gather}
      f = \cons{c(f_1)}{f_2} \label{es1:compl:39}
    \end{gather}
    and, because the system defining \((\sqsubseteq)\) is partially
    ordered and \textsf{SUB} is the last rule, it is implied that
    there are no closed patterns \(\close{p}_1\) and \(\close{p}_2\) such
    that \(\subst{{p}}{\sigma} = \cons{c(f_1)}{\close{p}_1}\)
    or \(\subst{{p}}{\sigma} =
    \cons{\pat{\close{p}_1}}{\close{p}_2}\) (which would lead to the
    conclusions of \textsf{EQ} or \textsf{PAT}). In other words, for
    all \(\close{p}_1\) and \(\close{p}_2\),
    \begin{align*}
      \subst{{p}}{\sigma}
      &\neq\cons{c(f_1)}{\close{p}_1}\\
      \subst{{p}}{\sigma} &\neq
    \cons{\pat{\close{p}_1}}{\close{p}_2}
    \end{align*}
    Equivalently, for all \(x\), \({p}_1\) and \({p}_2\),
    \begin{align}
        {p} &\neq \cons{\meta{x}}{{p}_1}
      & \text{with} \,\; \sigma(x) = c(f_1) \label{es1:compl:43}\\
        {p} &\neq
        \cons{\pat{{p}_1}}{{p}_2} \label{es1:compl:44}
    \end{align}
    The derivation \(\Delta\) has thus the shape
    \begin{mathpar}
      \inferrule*[right=\text{\sf SUB}]
        {\inferrule*
           {\inferrule*[vdots=1.5em]{}{ }}
           {\subst{{p}}{\sigma} \sqsubseteq f_1 \cdot f_2}}
         {\subst{{p}}{\sigma} \sqsubseteq \cons{c(f_1)}{f_2}}
    \end{mathpar}
    Let us assume that the induction hypothesis holds for the premise
    of \textsf{SUB}, i.e.,
    \begin{align}
        \smat{f_1 \cdot f_2}{{p}} 
      &\twoheadrightarrow \sigma' \label{es1:compl:40}\\
      \sigma' &\subseteq \sigma \label{es1:compl:41}
    \end{align}
    Let us turn over to the system defining the pattern matching at
    figure~\ref{es1_match_def} and check whether \eqref{es1:compl:40}
    can either be an instance of a premise or an instance of a
    conclusion.
    \begin{itemize}

      \item Conditions \eqref{es1:compl:39}, \eqref{es1:compl:43} and
        \eqref{es1:compl:44} imply that \eqref{es1:compl:40} cannot be
        a conclusion of any rule except \textsf{UNPAR}\(_2\). Since
        the rules are partially ordered, \eqref{es1:compl:40} cannot
        be the conclusion of rules before \textsf{UNPAR}\(_2\) and
        \(f_1\) must start with a tree not reduced to a lexeme. Even
        so, no judgement in the derivation ending with
        \eqref{es1:compl:40} can contain the configuration
        \(\smat{f}{{p}}\) because the forest of the premises
        are always smaller than the forest in the conclusion, in terms
        of number of nodes (constructors). So this case is impossible.

      \item The judgement \eqref{es1:compl:40} can be the instance of
        the premise of rule \textsf{UNPAR}\(_2\), which leads to the
        conclusion
        \begin{align}
          \smat{\cons{c(f_1)}{f_2}}{{p}} &\twoheadrightarrow
          \sigma'\notag
          \intertext{which, by \eqref{es1:compl:39}, is simply}
          \smat{f}{{p}} &\twoheadrightarrow
          \sigma'\label{es1:compl:42}
        \end{align}
        under the conditions of rule \textsf{UNPAR}\(_2\) being
          ordered last. More precisely, if a derivation ends with
          \textsf{UNPAR}\(_2\), then its conclusion has the shape
          \(\smj{\cons{c(f_1)\!}{\!f_2}}{{p}}{\sigma}\) \emph{and it
          is implied that all the following holds:}
        \begin{itemize}

        \item there exists no variable \(x\), no lexeme \(l\), no
          meta\-parsed pattern \({p}'\) and no forest \(f\) such that
          \({p} = \cons{\meta{x},l}{{p}'}\) and \(f_2 = \cons{l}{f}\)
          (conclusion of \textsf{BIND}\(_1\));

        \item there exists no variable \(x\), no meta\-parsed pattern
          \({p}'\), no tree \(t\) and no forest \(f\) such that \({p}
          = \cons{\meta{x}}{{p}'}\) and \(f_2 = \cons{t}{f}\)
          (conclusion of \textsf{BIND}\(_2\));

        \item \(f_2 \neq \el\) and there exists no variable \(x\) such
          that \({p} = [\meta{x}]\) (conclusion of
          \textsf{BIND}\(_3\));

        \item there exists no meta\-parsed patterns \({p}_1\) and
          \({p}_2\) such that \({p} = \cons{\pat{{p}_1}}{{p}_2}\)
          (conclusion of \textsf{UNPAR}\(_1\)).

      \end{itemize}
          These conditions express that the conclusion
          \eqref{es1:compl:42} cannot be the conclusion of rules
          before \textsf{UNPAR}\(_2\). Therefore, assuming this
          constraints hold, we can conclude that the induction
          hypothesis and \eqref{es1:compl:1} imply
          \eqref{es1:compl:42} and \eqref{es1:compl:41}, i.e.,
          \(\ind{W}({p}, f, \sigma, \sigma')\) holds. Otherwise, we
          must envisage that \eqref{es1:compl:40} is the premise of
          \textsf{BIND}\(_1\) or \textsf{BIND}\(_2\), or is an
          instance of axiom \textsf{BIND}\(_3\). None of these cases
          allow us to deduce the configuration \(\smat{f}{{p}}\)
          because all these rules increase the length of the pattern,
          from premise to conclusion. In this case, the theorem is
          false.

    \end{itemize}


\end{enumerate}
