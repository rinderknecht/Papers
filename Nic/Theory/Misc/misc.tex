%% Rewrite Systems

Let us consider the ordered system
\begin{align*}
n     &\rightarrow (n,1)\\
(0,m) &\rightarrow m\\
(n,m) &\rightarrow (n-1,n \times m)
\end{align*}
where \(n\) and \(m\) are positive integers and arithmetic operators
\((-)\) and \((\times)\) are defined out of the system. It can be
proved that all the compositions of rewrite steps starting with a
positive integer \(n\) end with an integer which is the the factorial
of \(n\). That is to say
\[
n \rightarrow (n,1) \rightarrow \dots \rightarrow (0,n!) \rightarrow n!
\]
The composition of a binary relation \((\rightarrow)\) at least once
is called the \emph{transitive closure} and noted
\((\xrightarrow{+})\). So, in the present case, the factorial function
over the integers belongs to the transitive closure of
\((\rightarrow)\), i.e., \(n \xrightarrow{+} n!\). 
When the rewrite relation is actually a function, it is convenient to
use a function name by means of a constructor. 

Thus we have \(\textsf{fact}(n) \xrightarrow{+} n!\). 


The reason to prefer rewrite rules over recursive equations is that
the former force the order in which the computing steps have to be
taken (following the arrow), whilst equations can be used both
ways.

Let us note \((\xrightarrow{n})\) the composition of \((\rightarrow)\)
\(n\) times:
\begin{align*}
  (\xrightarrow{1}) &\triangleq (\rightarrow)\\
  (\xrightarrow{n+1}) &\triangleq (\rightarrow) \circ (\xrightarrow{n})
& \text{with} \,\; n > 0
\intertext{and by \((\xrightarrow{+})\) the transitive closure of
  \((\rightarrow)\), i.e.,}
(\xrightarrow{+}) &\triangleq \bigcup_{i > 0}{(\xrightarrow{i})} 
& \text{with} \,\; i \in \mathbb{N}
\intertext{and by \((\xrightarrow{*})\) the reflexive\hyp{}transitive
  closure of \((\rightarrow)\), i.e.,}
(\xrightarrow{*}) &\triangleq (=) \cup (\xrightarrow{+})
\end{align*}
In other words, \((\xrightarrow{*})\) is the composition of
  \((\rightarrow)\) any number of times, including zero. For example,
\begin{align*}
 \textsf{append}(\textsf{push}(x, \textsf{nil}()), &\textsf{nil}())\\
&\rightarrow \textsf{push}(x, \textsf{append}(\textsf{nil}(),
 \textsf{nil}()))\\
&\rightarrow \textsf{push}(x, \textsf{nil}())
\end{align*}
which means that
\[
\textsf{append}(\textsf{push}(x, \textsf{nil}()), \textsf{nil}())
\xrightarrow{2} \textsf{push}(x, \textsf{nil}())
\]
or, more generally,
\[
\textsf{append}(\textsf{push}(x, \textsf{nil}()), \textsf{nil}())
\xrightarrow{*} \textsf{push}(x, \textsf{nil}())
\]



For instance, consider the toggle
\begin{align*}
  0 &\rightarrow 1 & 1 &\rightarrow 0
\end{align*}
The previous relation \((\rightarrow)\) is actually a \emph{partial
  function} over the integers, since each integer can be rewritten
into exactly one integer or none (e.g., \(2\)). It can be composed:
\begin{align*}
0 &\rightarrow 1 \rightarrow 0 \rightarrow 1 \rightarrow \dots\\
1 &\rightarrow 0 \rightarrow 1 \rightarrow 0 \rightarrow \dots
\end{align*}
Note that all the rewrites can be composed as long as wished (in
functional terms, this is because the domain and the codomain of the
function \((\rightarrow)\) are equal). Some systems only allow one
rewrite step at most, as
\begin{align*}
(0,0) &\rightarrow 0 & (0,1) &\rightarrow 1 &
(1,0) &\rightarrow 1 & (1,1) &\rightarrow 0
\end{align*}
Rewrite systems can contain variables to merge several rules into one
or allowing an infinite number of possible cases if the variables
denote terms of an infinite kind. For instance, the above system can
be equivalently rewritten as
\begin{align*}
 (1,1) &\rightarrow 0 & (x,0) &\rightarrow x & (0,x) &\rightarrow x
\end{align*}
assuming that \(x\) is either \(0\) or \(1\). (Note that this
constraint is not expressed now by means of rewrite rules: it is
defined out of the system.) Usually, rewrite rules are not ordered,
but they can be, in order to simplify the specification. In this case,
rules have to be considered in the order they are written (left to
right, top to bottom). For example, the previous system can be
simplied as

For example
\begin{align*}
(x,x) &\rightarrow 0 & (x,y) &\rightarrow 1
\end{align*}
where the rules are ordered and \(x\) and \(y\) are supposed to be
either \(0\) or \(1\). Note that a variable may occur several times in
a left\hyp{}hand side (like \(x\) in \((x, x)\)): this implies that
the terms denoted must be syntactically identical. 


We can use the
transitive closure of a rewrite relation to define the process by
which a function call is evaluated. If \(\textsf{fact}(n)
\xrightarrow{+} v\) and \(v\) cannot be rewritten, then \(v\) is the
\emph{value} of \(\textsf{fact}(n)\). With this point of view, the
first system we presented, the toggle, loops for ever on every
input. 

, instead of a symbol, and to use it as in
mathematics, e.g., \(f(n)\). The solution consists using a constructor
(see \emph{Constructors}) as the function name, for example \(f\) with
the expected arity, and define rewrite rules of the shape `\(f(\dots)
\rightarrow \dots\)' 


We also have
\[
\textsf{append}(\textsf{nil}(), \textsf{push}(x, \textsf{nil}()))
\xrightarrow{*} \textsf{push}(x, \textsf{nil}())
\]
Two terms which rewrite to the same identical term are said to be
\emph{congruent}, or \emph{equal} when there is no confusion with
identity. This is formally written as
\begin{center}
\emph{If}   \(x \xrightarrow{*} r\) 
\emph{and}  \(y \xrightarrow{*} r\)
\emph{then} \(x = y\).
\end{center}
For instance, this is how we prove that
\begin{align*}
\textsf{append}(&\textsf{push}(x, \textsf{nil}()), \textsf{nil}())\\
&= \textsf{append}(\textsf{nil}(), \textsf{push}(x, \textsf{nil}()))
\end{align*}
In other words: \([x] \cdot \el = \el \cdot [x]\).


%% Inference systems

For example, the axiom
\(A(x)\) actually means \(\forall x.A(x)\), and the rule
\[
\inferrule
  {P_1(x) \\ P_2(y)}
  {P(x,y)}
\]
signifies \(\forall x,y.(P_1(x) \, \wedge \, P_2(y) \Rightarrow P(x,
y))\). Given an inference system involving in the conclusions one or
more relations, it must be constrained so that these relations are the
\emph{smallest relations} satisfying the rules. To understand why,

Here, the relations involved are the unary relations (i.e., with only
one argument) \textsf{even} and \textsf{odd}. The axiom states that
\(0\) is even (by definition). For all \(n\), if \(n\) is odd, then
\(n+1\) is even and if \(n\) is even, then \(n+1\) is odd. The problem
is that many predicates (i.e., unary relations) satisfy these
conditions. Take for example \(\textsf{even}(n)\) and
\(\textsf{odd}(n)\) being always true for all \(n\) (thus
independently of \(n\)): true implies true, but nothing interesting is
defined here (the predicates are mere tautologies). But the smallest
predicates, in other words, the ones which are true `the less often,'
satisfying the rules are \(\textsf{even}(n) \triangleq (n \mod{2} =
0)\) and \(\textsf{odd}(n) \triangleq (n \mod{2} = 1)\). Let us make
use now of the notation with an horizontal line and show that it is
intuitive when composing several rules.

 For
instance, the value of \(\textsf{odd}(3)\) is `true' because the
indirectly\hyp{}recursive call \(\textsf{even}(2)\) returns `true.'


%% Trees

, e.g., by means of a
breadth\hyp{}first traversal (layer by layer, from root to
leaves)

Constructors allow to interpret a tuple as a finite \emph{tree}.

     \emph{Trees} are graphical interpretations of terms
     hierarchical layouts of information made of \emph{nodes},
     associated with data, which must be connected to other nodes by
     means of segments. The hierarchy imposes that there are
     horizontal layers of nodes. The upper layer always contains a
     single node, called the \emph{root}, and it is depicted at the
     top of the figure. Every node in a layer (1)~may connect to
     several nodes in the next lower layer, (2)~must be connected to
     exactly one node of the immediate upper layer, if
     any. Connections inside a layer or between non\hyp{}consecutive
     layers are not allowed. The nodes which are not connected to a
     lower layer are called the \emph{leaves}.

Note that the tuples of the constructors are not represented in the
tree, since they encode the structure of the subtrees, which is
already visible. 


\section{Algorithm \emph{A(0)}}

\noindent\textbf{Pattern matching.} Pattern matching is defined here
by the following rewrite rules:
\begin{align*}
\match{[]}{[]}{\sigma} 
\rightarrow & \, \sigma
& \!\!\!\textsf{END}\\
\match{\cons{l}{f}}{\cons{l}{p}}{\sigma} 
\rightarrow & \, \match{f}{p}{\sigma}
& \!\!\!\textsf{ELIM}\\
\match{\cons{c(l,t_2,\dots,t_{A(c)})}{\!f}}{\cons{l}{p}}{\sigma}
\rightarrow & \,
\match{\cons{t_2,\dots,t_{A(c)}}{\!f}}{p}{\sigma}
& \!\!\!\textsf{UNPAR}_1\\
\match{\cons{c(t_1,\dots,t_{A(c)})}{\!f}}{\cons{\cstr{lpar}{}}{p}}{\sigma}
\rightarrow & \,
\match{\cons{t_1,\dots,t_{A(c)},\cstr{rpar}{}}{\!f}}{p}{\sigma}
& \!\!\!\textsf{UNPAR}_2\\
& \; \text{if} \, t_1 \in {\cal T}\\
\match{\cons{t}{f}}{\cons{\meta{x}}{p}}{\sigma}
\rightarrow & \,
\match{f}{p}{\sigma \oplus x \mapsto t}
& \!\!\!\textsf{BIND}
\end{align*}
if \(t \in {\cal T}\) and (\(x \not\in \dom{\sigma}\) or \(\sigma(x) =
t\)).

\section{Algorithme \emph{N(1)}}

\noindent\textbf{Filtrage.} Ici la relation de filtrage est diff�rente
de celles pr�c�dentes car elle ne relie que des configurations. Pour
tout \(l \in {\cal L}\) on d�finit
\begin{mathpar}
\inferrule{}{\match{\cons{l}{f}}{\cons{l}{p}}{\sigma} 
\rightarrow \match{f}{p}{\sigma}}
\quad\TirName{\text{\sf ELIM}}

\inferrule
{}
{\match{\cons{c(t_1, \dots, t_{A(c)})\!}{\!f}}{\cons{l\!}{\!p}}{\sigma} 
\rightarrow 
\match{\cons{t_1, \dots, t_{A(c)}\!}{\!f}}{\cons{l\!}{\!p}}{\sigma}}
\quad\TirName{\text{\sf UNPAR}}

\inferrule
{t \in {\cal T}\\
\match{f}{p}{\sigma} \rightarrow \match{f'}{p'}{\sigma'}\\
(x \not\in \dom{\sigma'} \vee \sigma'(x) = t)}
{\match{\cons{t}{f}}{\cons{\meta{x}}{p}}{\sigma} 
\rightarrow \match{f'}{p'}{\sigma'\oplus x \mapsto t}}
\quad\TirName{\text{\sf BIND}}
\end{mathpar}
Ce syst�me est dirig� par la syntaxe: pour chaque configuration soit
aucune r��criture ne s'applique ou une seule.


\subsection{Environment inclusion}
 
\begin{lemma}[Inclusion]\hfill
\label{incl}
\begin{center}
If \(x \not\in \dom{\sigma}\) or \(\sigma(x) = t\) then \(\sigma \subseteq
\sigma \oplus x \mapsto t\).
\end{center}
\end{lemma}
\begin{enumerate}

  \item Case \(x \not\in \dom{\sigma}\).\\ By definition of
    \(\oplus\), we have 
    \[\forall y \in \dom{\sigma}.((\sigma \oplus x
      \mapsto t)(y) = \sigma(y))
    \]
    because \(y \neq x\). Futhermore, the definition of \(\subseteq\)
    implies that \(\sigma \subseteq \sigma \oplus x \mapsto t\).

  \item Case \(\sigma(x) = t\).\\ Let \(y \in \dom{\sigma}\). There are
    two possibilities:
    \begin{enumerate}

      \item Case \(x = y\).\\ \((\sigma \oplus x \mapsto t)(y)
        \triangleq t = \sigma(x) = \sigma(y)\).

      \item Case \(x \neq y\).\\ \((\sigma \oplus x \mapsto t)(y)
        \triangleq \sigma(y)\).

    \end{enumerate}
    In both cases, \(\forall y \in \dom{\sigma}.(\sigma(y) = (\sigma \oplus
    x \mapsto t)(y))\), which, by definition of the inclusion, is
    simply \(\sigma \subseteq \sigma \oplus x \mapsto t\).

\end{enumerate}
In all cases, we established that \(\sigma \subseteq \sigma \oplus x
\mapsto t\).\hfill\(\Box\)


\subsection{Right neuter forest}

\begin{lemma}[Right neuter forest]
\label{neuter}
\[
f \cdot \el = f
\]
\end{lemma}

\noindent\textbf{Proof.} Let \(\ind{B}(f)\) be the proposition ``\(f
\cdot \el = f\).'' We shall prove \(\forall f.\ind{B}(f)\) by means of
induction on the structure of \(f\), i.e., we shall prove
\[
\ind{B}([]) \wedge \forall f.(\ind{B}(f) \Rightarrow \forall
t.(\ind{B}(\cons{t}{f})))
\]
The induction principle implies then \(\forall f.\ind{B}(f)\).
\begin{enumerate}

  \item Case \(f = \el\).\\ In this case, we must prove
    \(\ind{B}(\el)\) directly:
    \begin{align*}
      \el \cdot \el &= \el & \text{by} \,\; \eqref{model:conc:1}
    \end{align*}
    This is simply \(\ind{B}(\el)\).
    
  \item Let us assume the induction hypothesis \(\ind{B}(f)\). We have
    \begin{align*}
      \cons{t}{f} \cdot \el &= \cons{t}{f \cdot \el}
      & \text{by} \,\; \eqref{model:conc:2}\\
      &= \cons{t}{f}
      & \text{by} \,\; \ind{B}(f)
    \end{align*}
    This is equivalent to \(\ind{B}(\cons{t}{f})\).
 
\end{enumerate}
The induction principle implies then that \(\forall
f.\ind{B}(f)\).\hfill \(\Box\)


%% ES(1) unparsed patterns
%
\begin{itemize}

  \item \(\el \in \overline{\cal P}\);

  \item if \(l \in {\cal L}\) and \(\overline{p} \in \overline{\cal
    P}\) then \(\cons{l}{\overline{p}} \in \overline{P}\);

  \item if \(x \in {\cal V}\) and \(\overline{p} \in \overline{\cal
    P}\) then \(\cons{\meta{x}}{\overline{p}} \in
    \overline{\cal P}\);

  \item if \(\overline{p} \in \overline{\cal P}\) then
    \(\{\cons{\mlp}{\overline{p}}, \cons{\mrp}{\overline{p}}\}
    \subseteq \overline{\cal P}\).

\end{itemize}

A node in a pattern tree is created
every time a pair of opening and closing parentheses are found. Each
subtree corresponds to the elements of the (recursively)
meta\hyp{}parsed pattern between the meta\hyp{}parentheses. Just as
for abstract syntax trees, we prefer, for technical reasons, to encode
these \emph{pattern trees} with a list of subtrees (see
section~\ref{model}). 


%% ES(1) pattern matching

\begin{figure}
\begin{mathpar}
\inferrule*{}{\mj{\el}{\el}{\sigma}{\sigma}}
\quad\TirName{\text{\sf END}}

\inferrule
  {m \in {\cal L} \cup \{\mrp\}\\
    \mj{f}{p}{\sigma}{\sigma'}}
  {\mj{\cons{m}{f}}{\cons{m}{p}}{\sigma}{\sigma'}}
\,\TirName{\text{\sf ELIM}}

\inferrule*[right=\text{\sf BIND}\(_1\)]
  {t \in {\cal T} \and
   m \in {\cal L} \cup \{\mrp\}\\
    \mj{f}{p}{\sigma}{\sigma'} \and
   \sigma' \subseteq \sigma' \oplus x \mapsto t}
  {\mj{\cons{t, m}{f}}%
      {\cons{\meta{x}, m}{p}}%
      {\sigma}{\sigma' \oplus x \mapsto t}}

\inferrule*[right=\text{\sf BIND}\(_2\)]
  {t_1 \in {\cal T} \and
    t_2 \in {\cal T}\\
    \mj{\cons{t_2}{f}}{p}{\sigma}{\sigma'} \and
    \sigma'\subseteq \sigma' \oplus x \mapsto t_1}
  {\mj{\cons{t_1, t_2}{f}}%
      {\cons{\meta{x}}{p}}%
      {\sigma}{\sigma' \oplus x \mapsto t_1}}

\inferrule*[right=\text{\sf BIND}\(_3\)]
  {t \in {\cal T} \and
    \sigma \subseteq \sigma \oplus x \mapsto t}
  {\mj{[t]}{[\meta{x}]}{\sigma}{\sigma \oplus x \mapsto t}}

\inferrule*[right=\text{\sf UNPAR}\(_1\)]
{m \in {\cal L} \cup \{\meta{x} \mid x \in {\cal V}\}\\\\
 \mj{\subterms{c} \cdot f}{\cons{m\!}{\!p}}{\sigma}{\sigma'}}
{\mj{\cons{c(\subterms{c})\!}{\!f}}{\cons{m\!}{\!p}}{\sigma}{\sigma'}}

\inferrule*[right=\text{\sf UNPAR}\(_2\)]
{\mj{\subterms{c} \cdot \cons{\mrp\!}{\!f}}%
    {p}%
    {\sigma}{\sigma'}}
{\mj{\cons{c(\subterms{c})\!}{\!f}}%
    {\cons{\mlp\!}{\!p}}%
    {\sigma}{\sigma'}
}
\end{mathpar}
where we impose that the rules \textsf{BIND}\(_1\),
\textsf{BIND}\(_2\) and \textsf{BIND}\(_3\) have priority over the
rule \textsf{UNPAR}\(_1\).
\caption{Pattern matching \textit{ES}(1).\label{es1_match_def}}
\end{figure}



\subsection{Fringe catenation}

\paragraph{Fringe.}

Let \(\phi\) be the function defined as
\begin{align*}
\fringe{\el} &\eqn{1} \el\\
\fringe{\cons{t}{f}} &\eqn{2} \fringe{t} \cdot \fringe{f}
\end{align*}

\begin{lemma}[Fringe Catenation]
\label{catenation}
\[\fringe{f_1 \cdot f_2} = \fringe{f_1} \cdot \fringe{f_2}\]
\end{lemma}
\noindent In other words, the fringe of the catenation is the
catenation of the fringes.

\paragraph{Proof.} 

Let \(\ind{Q}(f_1, f_2)\) be the proposition ``\(\fringe{f_1 \cdot
  f_2} = \fringe{f_1} \cdot \fringe{f_2}\).'' We shall prove \(\forall
f_1, f_2.\ind{Q}(f_1, f_2)\) by means of induction on the structure of
\(f_1\), i.e., we shall prove
\[
\forall f_2.(\ind{Q}(\el, f_2) \wedge \forall f_1.(\ind{Q}(f_1, f_2)
\Rightarrow \forall t.\ind{Q}(\cons{t}{f_1}, f_2)))
\]
The induction principle implies then \(\forall f_1, f_2.\ind{Q}(f_1,
f_2)\).
\begin{enumerate}

  \item Case \(f_1 = \el\)\\ In this case, we must prove
    \(\ind{Q}(\el, f_2)\) directly:
    \begin{align*}
      \fringe{\el \cdot f_2} 
      &= \fringe{f_2} & \text{by} \,\; \eqref{model:conc:1}\\
      &= \el \cdot \fringe{f_2} & \text{by} \,\;
      \eqref{model:conc:1}\\
      &\eqn{1} \fringe{\el} \cdot \fringe{f_2}
    \end{align*}
    This means that \(\ind{Q}(\el, f_2)\) holds.

  \item Let us assume the induction hypothesis \(\ind{Q}(f_1,
    f_2)\). We have, for all \(t\),
    \begin{align*}
      \fringe{\cons{t}{f_1} \cdot f_2}
     &\eqn{2} \fringe{t} \cdot \fringe{f_1 \cdot f_2}\\
     &= \fringe{t} \cdot (\fringe{f_1} \cdot \fringe{f_2})
     & \text{by} \,\; \ind{Q}(f_1, f_2)\\
     &= (\fringe{t} \cdot \fringe{f_1}) \cdot \fringe{f_2}
     & \text{by} \,\; \text{\ref{assoc}}\\
     &\eqn{2} \fringe{\cons{t}{f_1}} \cdot \fringe{f_2}
    \end{align*}
    which is equivalent to \(\ind{Q}(\cons{t}{f_1}, f_2)\). 
\end{enumerate}
The induction principle implies \(\forall f_1, f_2.\ind{Q}(f_1,
f_2)\). \hfill \(\Box\)


\subsection{Catenation associativity}

\begin{lemma}[Catenation associativity]
\label{assoc}
\[
f_1 \cdot (f_2 \cdot f_3) = (f_1 \cdot f_2) \cdot f_3
\]
\end{lemma}

\paragraph{Proof.}

Let \(\ind{A}(f_1, f_2, f_3)\) denote the proposition ``\(f_1 \cdot
(f_2 \cdot f_3) = (f_1 \cdot f_2) \cdot f_3\).'' We shall prove
\(\forall f_1, f_2, f_3.\ind{a}(f_1, f_2, f_3)\) by means of induction
on the structure of \(f_1\), i.e., we shall prove, for all \(f_2\),
\(f_3\),
\[
\ind{A}(\el, f_2, f_3) \wedge \forall f_1.(\ind{A}(f_1, f_2, f_3)
\Rightarrow \forall t.(\ind{A}(\cons{t}{f_1}, f_2, f_3)))
\]
The induction principle implies then 
\[
\forall f_1, f_2, f_3.\ind{A}(f_1, f_2, f_3)
\]
\begin{enumerate}

  \item Case \(f_1 = \el\).\\ In this case, we must prove
    \(\ind{A}(\el, f_2, f_3)\) directly:
    \begin{align*}
      \el \cdot (f_2 \cdot f_3)
      &= f_2 \cdot f_3
      & \text{by} \,\; \eqref{model:conc:1}\\
      &= (\el \cdot f_2) \cdot f_3
      & \text{by} \,\; \eqref{model:conc:1}
    \end{align*}
    This is none other than \(\ind{A}(\el, f_2, f_3)\).

  \item Let us now assume the induction hypothesis \(\ind{A}(f_1, f_2,
    f_3)\). We have
    \begin{align*}
      \cons{t}{f_1} \cdot (f_2 \cdot f_3)
      &= \cons{t}{f_1 \cdot (f_2 \cdot f_3)}
      & \text{by} \,\; \eqref{model:conc:2}\\
      &= \cons{t}{(f_1 \cdot f_2) \cdot f_3}
      & \text{by} \,\; \ind{A}(f_1, f_2, f_3)\\
      &= \cons{t}{f_1 \cdot f_2} \cdot f_3
      & \text{by} \,\; \eqref{model:conc:2}\\
      &= (\cons{t}{f_1} \cdot f_2) \cdot f_3
      & \text{by} \,\; \eqref{model:conc:2}
    \end{align*}
    This is none other than \(\ind{A}(f_1, f_2, f_3)\).
\end{enumerate}
The induction principle implies \(\forall f_1, f_2, f_3.\ind{A}(f_1,
f_2, f_3)\). \hfill \(\Box\)


%% F(0)
%
\paragraph{Meta\hyp{}parsed patterns.}

Given a pattern \(\overline{p}\), the first task is to check whether
\(\overline{p} \in \overline{\cal P}\). It is achieved by the
meta\hyp{}parsing function \textsf{parse}, which either fails due to
mismatched meta\hyp{}parentheses or returns the initial pattern where
all patterns enclosed in meta\hyp{}parentheses (included) have been
replaced by a tree of patterns. These trees are not abstract syntax
trees because they contain patterns, not code (thus the new patterns
are still unparsed with respect to the programming language
syntax). Let \(\unary{pat}\) be their unique constructor. Thus we have
\(\unary{pat} \not\in {\cal C}\). Let us define the set of
\emph{meta\hyp{}parsed patterns} as the smallest set \({\cal P}\) such
that
\begin{itemize}

  \item \(\el \in {\cal P}\);

  \item if \(l \in {\cal L}\) and \(p \in {\cal P}\) then
    \(\cons{l}{p} \in {\cal P}\);

  \item if \(x \in {\cal V}\) and \(p \in {\cal P}\) then
    \(\cons{\meta{x}}{p} \in {\cal P}\);

  \item if \(p_1, p_2 \in {\cal P}\) and \(p_1 \neq \el\) then
    \(\cons{\pat{p_1}}{p_2} \in {\cal P}\).

\end{itemize}
\piccaption{\label{f0_pattern_tree}}
\parpic[l]{\includegraphics[bb=71 671 128 721]{pattern_tree}} The
meta\hyp{}parsing function \textsf{parse} is formally defined by means
of an ordered rewrite system defining jointly two other functions,
\textsf{shift} and \textsf{reduce}. See
figure~\ref{f0_metaparsing}. Rule \((\xrightarrow{a})\) handles the
empty pattern. Rule \((\xrightarrow{c})\) says that lexemes and
meta\hyp{}variables are directly copied to the result in the same
order as in the input. Rule \((\xrightarrow{b})\) handles an opening
meta\hyp{}parenthesis: it is pushed into an auxiliary analysis stack
passed as second argument to the call of function \textsf{shift}. The
purpose of this function is to parse the input with the additional
knowledge that an opening meta\hyp{}parenthesis was found when it was
first called ---~in other words, the analysis is performed inside
meta\hyp{}parentheses. Thus the goal is reached when this
supplementary stack contains only one pattern tree, since it must
correspond to a pair of meta\hyp{}parentheses at the top\hyp{}level in
the input pattern. In this case, i.e., rule \((\xrightarrow{d})\), the
pattern tree is copied to the output and the meta\hyp{}parsing is
resumed at the top\hyp{}level with \textsf{parse}. Rule
\((\xrightarrow{f})\) copies anything from the input pattern to the
auxiliary stack, except closing meta\hyp{}parentheses because this
case triggers a reduction in rule \((\xrightarrow{e})\). Indeed,
finding \mrp means that the shortest prefix in the auxiliary stack
ending with an opening meta\hyp{}parenthesis must be replaced by the
corresponding pattern tree and then shifting is resumed (i.e.,
meta\hyp{}parsing not at top\hyp{}level). Rule \((\xrightarrow{h})\)
pops any item, except \mlp, from the auxiliary stack and pushes it to
a temporary stack which contains the first\hyp{}level subtrees of the
pattern tree under construction. If an opening meta\hyp{}parenthesis
is reached, in rule \((\xrightarrow{g})\), it means that the shortest
prefix to be reduced has been found, therefore a pattern tree of root
\textsf{pat} is produced on top of the remaining auxiliary stack. Note
that rule \((\xrightarrow{g})\) do not build a pattern tree without
subtrees: the sequence \verb|%(%)| is a meta\hyp{}syntax error. As
usual with rewrite systems, errors are not specified: a
meta\hyp{}parsing error occurs if a term is reached and it cannot be
rewritten further. The reader can check that the unparsed pattern
\[
[l_1, \mlp, l_2, l_3, \mlp, \meta{x}, l_4, \mrp, \mrp]
\]
is meta\hyp{}parsed into the pattern
\[
[l_1, \pat{[l_2, l_3, \pat{[\meta{x}, l_4]}]}]
\]
which contains the pattern tree shown in
figure~\ref{f0_pattern_tree}.
\begin{figure}
\begin{align*}
  \cstr{parse}{\el, \el}
&\xrightarrow{a} \el\\
  \cstr{parse}{\cons{\lpar}{\overline{p}}, \cons{c(f_1)}{f_2}}
&\xrightarrow{b} \cstr{shift}{\overline{p}, [\rpar], f_1 \cdot
    \cons{\lpar}{f_2}}\\
  \cstr{parse}{\cons{e}{\overline{p}}}
&\xrightarrow{c} \cons{e}{\cstr{parse}{\overline{p}}}\\
  \cstr{shift}{\overline{p}, [\pat{p}]}
&\xrightarrow{d} \cons{\pat{p}}{\cstr{parse}{\overline{p}}}\\
  \cstr{shift}{\cons{\mrp}{\overline{p}}, p}
&\xrightarrow{e} \cstr{shift}{\overline{p}, \cstr{reduce}{p, \el}}\\
  \cstr{shift}{\cons{e}{\overline{p}}, p}
&\xrightarrow{f} \cstr{shift}{\overline{p}, \cons{e}{p}}\\
  \cstr{reduce}{\cons{\mlp}{p_1}, \cons{e}{p_2}}
&\xrightarrow{g} \cons{\pat{\cons{e}{p_2}}}{p_1}\\
  \cstr{reduce}{\cons{e}{p_1}, p_2}
&\xrightarrow{h} \cstr{reduce}{p_1, \cons{e}{p_2}}
\end{align*}
\caption{Meta\hyp{}parsing of patterns in
  \textit{F}(0).\label{f0_metaparsing}}
\end{figure}


Let us prove \(\ind{W}\) by lexicographic induction on the
pair made of the forest and the pattern. Let us first define the order
on \((p,f)\) as follows. The \emph{size} of a tree is the number of
nodes it contains, and the size of a forest is the number of nodes all
its trees contain:
\begin{align*}\label{size_def}
N(\el) &= 0\\
N(\cons{t}{f}) &= N(t) + N(f)
  && \text{if} \;\, t \in {\cal T}\\
N(c(s)) &= 1 + N(s)
 && \text{if} \;\, c \in {\cal C}\\
N(l) &= 0 
 && \text{if} \;\, l \in {\cal L}
\end{align*}
The \emph{length} \(\len{p}\) of a pattern \(p\) is defined simply as
\begin{align*}
   \len{\el} &= 0\\
   \len{\cons{e}{p}} &= 1 + \len{p}
\end{align*}
These functions allow us to partially order our pairs with
\((\prec)\). By definition, \((p_1, f_1) \prec (p_2, f_2)\) if and
only if \(N(f_1) < N(f_2)\) or \(N(f_1) = N(f_2)\) and \(\len{p_1} <
\len{p_2}\). This order is actually noetherian, i.e., for any pair
\((p, f)\), there are no infinite descending chains. Informally:
\[
\neg\exists p, f.(\dots \prec \dots \prec \dots \prec (p, f)).
\]
We even have a stronger property: there exists a minimum element for
the partial order, which is \((\el, \el)\). So \(\langle{\cal F}
\times {\cal P}, \prec\rangle\) is a meet\hyp{}semilattice, that is,
all the descending chains are finite and can be extended to reach the
same minimal element. Informally:
\[
\forall p, f.((\el, \el) \prec \dots \prec \dots \prec (p, f)).
\]
Our strategy is then to prove that, for all \(\pi\), \(\sigma\),
\(\sigma'\),
\begin{gather*}
(\forall \pi'.(\pi' \prec \pi \Rightarrow \ind{W}(\pi', \sigma,
  \sigma'))) \Rightarrow \ind{W}(\pi, \sigma, \sigma')
\end{gather*}
In other words, let us prove directly that \(\ind{W}\) holds for the
minimum forest, i.e., \(\ind{W}((\el, \el), \sigma, \sigma')\) is
true, then, assuming that \(\ind{W}(\pi', \sigma, \sigma')\) holds for
any \(\pi'\) such that \(\pi' \prec \pi\) (the \emph{induction
  hypothesis}), we shall establish that \(\ind{W}(\pi, \sigma,
\sigma')\) holds too. The induction principle implies then that
\(\forall \pi, \sigma, \sigma'.\ind{W}(\pi, \sigma, \sigma')\) holds.
\begin{enumerate}

  \item \(\pi = (\el, \el)\).\\ We have
    \begin{align}
      \subst{\el}{\sigma}
      &\eqn{1} \el\notag\\
      &\sqsubseteq \el
      &\text{by} \,\; \textsf{EMP}\notag\\
      &= f \label{x:compl:0}
    \end{align}
    Let us prove that there exists a substitution \(\sigma'\) such
    that \(\smj{\el}{\el}{\sigma'}\) and \(\sigma' \subseteq
    \sigma\). An examination of the inference rules shows that the
    only rule whose conclusion matches the previous configuration is
    \textsf{END}. Therefore
    \begin{equation}
      \smj{\el}{\el}{\sigma_\varnothing} \label{x:compl:1}
    \end{equation}
    that is to say, \(\sigma' = \sigma_\varnothing\). The definition
    \eqref{model:incl} of the inclusion of substitutions implies that,
    for all \(\sigma\),
    \begin{equation}
      \sigma_\varnothing \subseteq \sigma \quad \text{since}\,\;
      \dom{\sigma_\varnothing} = \varnothing \label{x:compl:2}
    \end{equation}
    As a conclusion, \eqref{x:compl:0}, \eqref{x:compl:1} and
    \eqref{x:compl:2} imply \(\ind{W}((\el, \el), \sigma, \sigma')\).

  \item \(\pi \neq (\el, \el)\).\\ The induction hypothesis being
    \begin{equation}
      \forall \pi'.(\pi' \prec \pi \Rightarrow \ind{W}(\pi', \sigma,
      \sigma')) \label{x:compl:ind}
    \end{equation}
    we must prove that \(\ind{Q}(\pi, \sigma, \sigma')\) holds. A
    quick examination of the configurations in the inference rules
    defining \((\twoheadrightarrow)\) shows that both the forest and
    the pattern must be non-empty --- otherwise, \(\ind{Q}(\pi,
    \sigma, \sigma')\) cannot hold. Let us assume now that the
    following relation holds:
    \begin{equation}
      \subst{p}{\sigma} \sqsubseteq f \label{x:compl:hypo}
    \end{equation}
    (otherwise the theorem is trivially true) and prove
    \begin{gather*}
    \smj{f}{p}{\sigma'} \;\, \text{and} \;\, \sigma' \subseteq \sigma.
    \end{gather*}
    Since \(f\) is not empty, it is of the form
    \begin{gather}
      f = \cons{t}{f'} \label{x:compl:8}
    \end{gather}
    We proceed case by case on the kinds of \(p\) and \(t\).
    \begin{enumerate}

      %\(t = l_1 \in {\cal L}\)
      \item \label{x:compl:case:1} \(p = \cons{l}{p'}\) where \(l \in
        {\cal L}\).\\ Since \(N(f) = N(t) + N(f') < N(f')\), we have
        \((p', f') \prec (p, f)\), therefore the induction hypothesis
        \eqref{x:compl:ind} applies to \((p', f')\) and implies that
        \(\ind{Q}((p', f'), \sigma, \sigma')\) holds, i.e.,
        \begin{gather}
          \smj{f'}{p'}{\sigma'} \label{x:compl:4}\\
          \sigma' \subseteq \sigma \label{x:compl:5}
        \end{gather}
        Furthermore,
        \begin{align}
              \fringe{f} 
           &= \fringe{l_1} \cdot \fringe{f'}\notag\\ 
           &\eqn{3} [l_1] \cdot \fringe{f'}\notag\\
           &= \cons{l_1}{\el \cdot \fringe{f'}}
           & \text{by} \,\; \eqref{model:conc:2}\notag\\
           &= \cons{l_1}{\fringe{f'}}
           & \text{by} \,\; \eqref{model:conc:1}\notag
           \intertext{Furthermore,}
             \subst{p}{\sigma}
           &= \subst{\cons{l_2}{p'}}{\sigma}\notag\\
           &\eqn{7} \cons{l_2}{\subst{p'}{\sigma}}\notag
           \intertext{So, equality \eqref{x:compl:hypo} implies then}
           l_1 &= l_2 = l \label{x:compl:6}\\
             \fringe{f'}
           &= \subst{p'}{\sigma} \label{x:compl:3}
        \end{align}
        Then, \eqref{x:compl:6}, \eqref{x:compl:4} and inference
        rule \textsf{ELIM} imply
        \begin{gather}
          \mj{\cons{l}{f'}}{\cons{l}{p'}}{\sigma_0}{\sigma'}\notag
          \intertext{which is equivalent to}
          \mj{f}{p}{\sigma_0}{\sigma'} \label{x:compl:7}
        \end{gather}
        Finally, \eqref{x:compl:hypo}, \eqref{x:compl:7} and
        \eqref{x:compl:5} imply that \(\ind{Q}((f, p), \sigma,
        \sigma_0, \sigma')\) holds.

      \item \(t = l \in {\cal L}\) and \(p =
        \cons{\meta{x}}{p'}\).\\ Following the same deductions
        as in case \ref{x:compl:case:1}, we draw
        \begin{align*}
          \fringe{f} &= \cons{l}{\fringe{f'}}
          \intertext{Furthermore,}
          \subst{p}{\sigma} &= \subst{\cons{\meta{x}}{p'}}{\sigma}\\
            &\eqn{6} \fringe{\sigma(x)} \cdot \subst{p'}{\sigma}
        \end{align*}


      \item \(t = c(\subterms{})\).

    \end{enumerate}

\end{enumerate}

Let us prove this proposition by general
induction on the size of the forest \(f\). The \emph{size} of a tree
is the number of nodes it contains, and the size of a forest is the
number of nodes all its trees contain:
\begin{align*}\label{size_def}
N(\el) &= 0\\
N(\cons{t}{f}) &= N(t) + N(f)
  && \text{if} \;\, t \in {\cal T}\\
N(c(s)) &= 1 + N(s)
 && \text{if} \;\, c \in {\cal C}\\
N(l) &= 0 
 && \text{if} \;\, l \in {\cal L}
\end{align*}
The size allows us to partially order forests with `\(\prec\)':
\[
f_1 \prec f_2 \Longleftrightarrow N(f_1) < N(f_2)
\]
This order is actually noetherian, i.e., for all forest \(f\), there
are no infinite descending chains
\[
\dots \prec \dots \prec \dots \prec f
\]
We even have a stronger property: there exists a minimum element for
the partial order: the empty forest. So \(\langle{\cal F},
\prec\rangle\) is a meet-semilattice (that is, all the descending
chains are finite and can be extended to reach the same minimal
element). Our strategy is then to prove that, for all \(f\), \(p\),
\(\sigma\), \(\sigma'\),
\[
(\forall f'.(f' \prec f \Rightarrow \ind{W}(f', p,\sigma, \sigma')))
\Rightarrow \ind{W}(f, p, \sigma, \sigma')
\]
In other words, let us prove directly that \(\ind{W}\) holds for the
minimum forest, i.e., \(\ind{W}(\el, p, \sigma, \sigma')\) is true, then,
assuming that \(\ind{W}(f', p, \sigma, \sigma')\) holds for any \(f'\)
such that \(f' \prec f\) (the \emph{induction hypothesis}), we shall
establish that \(\ind{W}(f, p, \sigma, \sigma')\) holds too. The induction
principle implies then that \(\forall f, p, \sigma, \sigma'.\ind{W}(f, p,
\sigma, \sigma')\) holds, which, in turn, implies the completeness
theorem.
  \item \(f = \el\).\\ Let us assume
    \begin{equation}
      \fringe{\el} = \subst{p}{\sigma} \label{x:compl:0}
    \end{equation}
    and prove that there exists a substitution \(\sigma'\) such that
    \(\mj{\el}{p}{\sigma_0}{\sigma'}\) and \(\sigma' \subseteq \sigma\). We
    have \(\el \eqn{1} \fringe{\el} = \subst{p}{\sigma}\). Since the
    empty pattern can only be the result of a substitution on the
    empty pattern (cf. \(\eqn{5}\)), we deduce that \(p =
    \el\). Therefore \(\match{\el}{p}{\sigma_0} =
    \match{\el}{\el}{\sigma_0}\). Besides, an examination of the
    inference rules shows that the only rule whose conclusion matches
    the previous configuration is \textsf{END}. Therefore
    \begin{equation}
      \mj{\el}{\el}{\sigma_0}{\sigma_0} \label{x:compl:1}
    \end{equation}
    that is to say, \(\sigma' = \sigma_0\). The definition
    \eqref{model:incl} of the inclusion of  substitutions implies,
    for all \(\sigma\), 
    \begin{equation}
      \sigma_0 \subseteq \sigma \quad \text{since}\,\;
      \dom{\sigma_0} = \varnothing \label{x:compl:2}
    \end{equation}
    As a conclusion, \eqref{x:compl:0}, \eqref{x:compl:1} and
    \eqref{x:compl:2} imply \(\ind{W}(\el, p, \sigma, \sigma')\).

  \item \(f \neq \el\).\\ The induction hypothesis being
   \begin{equation}
     \forall f'.(f'\prec f \Rightarrow \ind{W}(f', p, \sigma, \sigma'))
   \end{equation}
   we must prove that \(\ind{Q}(f, p, \sigma, \sigma')\) holds. Let us
   then assume further that the following equality holds:
   \begin{equation}
     \fringe{f} = \subst{p}{\sigma} \label{x:compl:hypo}
   \end{equation}
   (otherwise the theorem is trivially true) and prove
   \[
     \mj{f}{p}{\sigma_0}{\sigma'} \;\, \text{and} \;\, \sigma' \subseteq
     \sigma.
   \]
   Since \(f\) is not empty, it is of the form \(f =
   \cons{t}{f'}\). We shall proceed case by case on the kind of \(t\).
   \begin{align*}
        \fringe{f} 
     &= \fringe{\cons{t}{f'}}\\
     &\eqn{2} \fringe{t} \cdot \fringe{f'}
   \end{align*}


% x_determinacy.tex

We proceed firstly case by case on the shape of the
pattern and, secondly, on the kind of rule that may end the
derivations.

  \item Case \(p = \el\).\\ Then, compulsorily, the derivation
    \(\Delta\) simply corresponds to an instance of the rule
    \textsf{END}, thus \(f = \el\) and \(\sigma = \sigma_0\). The same
    is true about \(\Delta'\), and \(\sigma' = \sigma_0\). As a
    consequence, \(\sigma = \sigma'\), that is to say, \(\ind{S}(\el,
    f, \sigma, \sigma')\) holds.

  \item Case \(p = \cons{l}{p'}\), where \(l \in {\cal L}\).\\ Only
    two inference rules can contain \(\smat{f}{\cons{l}{p'}}\), where
    \(l \in {\cal L}\), in the conclusion: \textsf{ELIM} and
    \textsf{UNPAR}. Consequently, let us consider the four cases that
    this induces for the derivations \(\Delta\) and \(\Delta'\).
    \begin{enumerate}

      \item Case where \(\Delta\) and \(\Delta'\) end by
        \textsf{ELIM}.
        \begin{mathpar}
          \inferrule[\text{\sf ELIM}]
            {\inferrule
               {\inferrule*[vdots=1.5em]{}{(\Delta_1)}}
               {\smj{f'}{p'}{\sigma}}}
            {\smj{\cons{l}{\!f'}}{\cons{l}{p'}}{\sigma}}
          \,
          \inferrule[\text{\sf ELIM}]
            {\inferrule
               {\inferrule*[vdots=1.5em]{}{(\Delta'_1)}}
               {\smj{f'}{p'}{\sigma'}}}
            {\smj{\cons{l}{\!f'}}{\cons{l}{p'}}{\sigma'}}
        \end{mathpar}
        where, since we assumed \eqref{x:deter:delta} and
        \eqref{x:deter:delta'},
        \begin{enumerate}

          \item \label{x:deter:1} \(l \in {\cal L}\),

          \item \label{x:deter:3} \(p \triangleq \cons{l}{p'}\),

          \item \label{x:deter:2} \(f \triangleq \cons{l}{f'}\).

        \end{enumerate}
        Since \(\Delta_1\) is a sub\hyp{}derivation of \(\Delta\) and
        \(\Delta'_1\) is a sub\hyp{}derivation of \(\Delta'\), let us
        assume that the induction hypothesis holds for their
        conclusions, i.e., \(\ind{S}(p', f', \sigma, \sigma')\)
        holds. Therefore
        \begin{equation}
          \sigma = \sigma' \label{x:deter:4}
        \end{equation}
        Finally, the induction hypothesis, \eqref{x:deter:delta} and
        \eqref{x:deter:delta'} imply \eqref{x:deter:4}, i.e.,
        \(\ind{S}(\cons{l}{p'}, \cons{l}{f'}, \sigma, \sigma')\).

      \item \label{elim_unpar_bis} Case where \(\Delta\) ends by
        \textsf{ELIM} and \(\Delta'\) ends by \textsf{UNPAR}.\\ This
        case is impossible because this implies
        \begin{enumerate}

          \item \(f = \cons{l}{f_2}\) where \(l \in {\cal L}\) (from
            \textsf{ELIM}),

          \item \(f = \cons{c(f_1)}{f_2}\) (from \textsf{UNPAR}).

        \end{enumerate}
        and \(l \neq c(f_1)\) by definition of the lexemes as trees of
        height \(0\) (\(c(f_1)\) is at least of height \(1\)).

      \item Case where \(\Delta\) ends by \textsf{UNPAR} and
        \(\Delta'\) ends by \textsf{ELIM}.\\ This case is the
        symmetric of~(\ref{elim_unpar_bis}).

      \item \label{unpar2_bis} Case where both \(\Delta\) and
        \(\Delta'\) end by \textsf{UNPAR}.
        \begin{mathpar}
          \inferrule[\text{\sf UNPAR}]        
            {\inferrule
              {\inferrule*[vdots=1.5em]{}{(\Delta_1)}}
              {\smj{f_1 \!\cdot\! f_2}{p}{\sigma}}}
            {\smj{\cons{c(f_1)\!}{\!f_2}}{p}{\sigma}}
          \;
          \inferrule[\text{\sf UNPAR}]        
            {\inferrule
              {\inferrule*[vdots=1.5em]{}{(\Delta'_1)}}
              {\smj{f_1 \!\cdot\! f_2}{p}{\sigma'}}}
            {\smj{\cons{c(f_1)\!}{\!f_2}}{p}{\sigma'}}
        \end{mathpar}
        where, since we assumed \eqref{x:deter:delta} and
        \eqref{x:deter:delta'},
        \begin{enumerate}

          \item \label{x:deter:5} \(f \triangleq
            \cons{c(f_1)}{f_2}\).

        \end{enumerate}
        Since \(\Delta_1\) is a sub\hyp{}derivation of \(\Delta\) and
        \(\Delta'_1\) is a sub\hyp{}derivation of \(\Delta'\), let us
        assume that the induction hypothesis holds for their
        conclusions, that is, \(\ind{S}(p, f_1 \cdot f_2, \sigma,
        \sigma')\) holds:
        \begin{equation}
          \sigma = \sigma' \label{x:deter:6}
        \end{equation}
        Finally, the induction hypothesis, \eqref{x:deter:delta} and
        \eqref{x:deter:delta'} imply \eqref{x:deter:6}, i.e.,
        \(\ind{S}(p, \cons{c(f_1)}{f_2}, \sigma, \sigma')\).

   \end{enumerate}   

  \item Case where \(p = \cons{\meta{x}}{p'}\).\\ Only two inference
    rules can contain the configuration
    \(\smat{f}{\cons{\meta{x}}{p'}}\) as a conclusion: \textsf{UNPAR}
    and \textsf{BIND}. Let us consider all the four possibilities.
    \begin{enumerate}

      \item Case where both \(\Delta\) and \(\Delta'\) end by
        \textsf{UNPAR}.\\ This case is equivalent
        to~(\ref{unpar2_bis}), because the kind of pattern is
        irrelevant here.

      \item Case where \(\Delta\) ends by \textsf{BIND} and
        \(\Delta'\) ends by \textsf{UNPAR}.\\ This case is the
        symmetric of~(\ref{bind_unpar_bis}). 

      \item Case where \(\Delta\) and \(\Delta'\) both end by
        \textsf{BIND}.
        \begin{mathpar}
          \inferrule[\text{\sf BIND}]
            {\inferrule*
               {\inferrule*[vdots=1.5em]{}{(\Delta_1)}}
               {\smj{f'}{p'}{\sigma_1}}\\
             \sigma_1 \subseteq \sigma_1 \oplus x \mapsto t
            }
            {\smj{\cons{t}{f'}}%
                {\cons{\meta{x}}{p'}}%
                {\sigma_1 \oplus x \mapsto t}}

          \inferrule[\text{\sf BIND}]
            {\inferrule*
                {\inferrule*[vdots=1.5em]{}{(\Delta'_1)}}
                {\smj{f'}{p'}{\sigma_2}}\\
             \sigma_2 \subseteq \sigma_2 \oplus x \mapsto t}
            {\smj{\cons{t}{f'}}%
                {\cons{\meta{x}}{p'}}%
                {\sigma_2 \oplus x \mapsto t}}
        \end{mathpar}
        where, since we assumed \eqref{x:deter:delta} and
        \eqref{x:deter:delta'},
        \begin{enumerate}

          \item \label{x:deter:8} \(p \triangleq
            \cons{\meta{x}}{p'}\),

          \item \label{x:deter:7} \(f \triangleq \cons{t}{f'}\),
            with \(t \in {\cal T}\),

          \item \label{x:deter:9} \(\sigma \triangleq \sigma_1 \oplus x
            \mapsto t\),

          \item \label{x:deter:10} \(\sigma' \triangleq \sigma_2 \oplus
            x \mapsto t\).

        \end{enumerate}
        Since \(\Delta_1\) is a sub\hyp{}derivation of \(\Delta\) and
        \(\Delta'_1\) is a sub\hyp{}derivation of \(\Delta'\), let us
        assume that the induction hypothesis holds for their
        conclusions, i.e., \(\ind{S}(p', f', \sigma_1,
        \sigma_2)\). Therefore
        \begin{align}
          \sigma_1 &= \sigma_2\notag\\
          \sigma_1 \oplus x \mapsto t 
          &= \sigma_2 \oplus x \mapsto t\notag\\
          \sigma
          &= \sigma'
          &\text{by} \,\; \text{\ref{x:deter:9}} \,\; \text{and} \,\;
          \text{\ref{x:deter:10}} \label{x:deter:12}
        \end{align}
        Finally, the induction hypothesis, \eqref{x:deter:delta} and
        \eqref{x:deter:delta'} imply \eqref{x:deter:12}, i.e.,
        \(\ind{S}(\cons{\meta{x}}{p'}, f, \sigma, \sigma')\).



% x_soundness.tex

We proceed firstly
case by case on the shape of the pattern and, secondly, on the kind of
rule that may end the derivation.

\begin{enumerate}

  \item Case \(p = \el\).\\ Because the inference system is ordered,
    the derivation \(\Delta\) must correspond to an instance of the
    axiom \textsf{END}, and not of the rule \textsf{UNPAR}. Therefore
    we deduce that \(f = \el\), which leads to
    \begin{gather*}
    \subst{p}{\sigma} = \subst{\el}{\sigma} \eqn{1} \el \sqsubseteq
    \el = f
    \end{gather*} 
    We conclude that \(\ind{R}(\el, f, \sigma)\) holds.

 \item Case \(p = \cons{l}{p'}\), where \(l \in {\cal L}\).\\ Only two
   inference rules can have \(\smj{f}{\cons{l}{p'}}{\sigma}\), with
   \(l \in {\cal L}\), in the conclusion: \textsf{ELIM} and
   \textsf{UNPAR}. Let us consider all the possible derivations
   \(\Delta\).
   \begin{enumerate}

     \item Case where \(\Delta\) ends with \textsf{ELIM}.
        \begin{mathpar}
          \inferrule*[right=\text{\sf ELIM}]
            {l \in {\cal L}\\
             \inferrule*
               {\inferrule*[vdots=1.5em]{}{ }}
               {\smj{f'}{p'}{\sigma}}
            }
            {\smj{\cons{l}{f'}}{\cons{l}{p'}}{\sigma}}
       \end{mathpar}
       where, since we assumed \eqref{x:sound:head},
       \begin{enumerate}
          
         \item \label{x:sound:13} \(f \triangleq \cons{l}{f'}\),

         \item \label{x:sound:14} \(p \triangleq \cons{l}{p'}\).

       \end{enumerate}
       Let us assume that the induction hypothesis holds for the
       premise of \textsf{ELIM}, i.e., \(\ind{R}(p', f', \sigma)\)
       holds. Therefore
       \begin{gather}
         \subst{p'}{\sigma} \sqsubseteq f' \label{x:sound:2}
       \end{gather}
        Besides, we have
        \begin{align}
          \subst{p}{\sigma}
          &= \subst{\cons{l}{p'}}\notag
          & \text{by} \,\; \text{\ref{x:sound:14}}\\
          &\eqn{3} \cons{l}{\subst{p'}{\sigma}}\notag\\
          &\sqsubseteq \cons{l}{f'}
          &\text{by} \,\; \eqref{x:sound:2} \,\; \text{and} \,\;
          \textsf{EQ}\notag\\
          &= f
          & \text{by} \,\; \text{\ref{x:sound:13}}\notag\\
            \subst{p}{\sigma}
          &\sqsubseteq f \label{x:sound:5}
        \end{align}
        As a conclusion, the induction hypothesis and
        \eqref{x:sound:head} imply \eqref{x:sound:5}, so
        \(\ind{R}(\cons{l}{p'}, f, \sigma)\) holds.

     \item \label{unpar} Case where \(\Delta\) ends with
       \textsf{UNPAR}.
       \begin{mathpar}
         \inferrule*[right=\text{\sf UNPAR}]
           {\inferrule*
              {\inferrule*[vdots=1.5em]{}{ }}
              {\smj{f_1 \cdot f_2}{p}{\sigma}}}
           {\smj{\cons{c(f_1)}{f_2}}{p}{\sigma}}
       \end{mathpar}
       where, since we assumed \eqref{x:sound:head},
       \begin{enumerate}

         \item \label{x:sound:15} \(f = \cons{c(f_1)}{f_2}\).
       
       \end{enumerate}
       Let us assume that the induction hypothesis holds for the
       premise of \textsf{UNPAR}, i.e., \(\ind{R}(p, f_1 \cdot f_2,
       \sigma)\) holds. Then
       \begin{gather}
         \subst{p}{\sigma} \sqsubseteq f_1 \cdot f_2 \label{x:sound:4}
       \end{gather}
        Matching \eqref{x:sound:4} and inference rule \textsf{SUB}
        imply
        \begin{align}
          \subst{p}{\sigma} &\sqsubseteq
          \cons{c(f_1)}{f_2}\notag\\
           &= f
           &\text{by} \,\; \text{\ref{x:sound:15}}\notag\\
            \subst{p}{\sigma}
           & \sqsubseteq f \label{x:sound:I}
        \end{align}
        As a conclusion, the induction hypothesis and
        \eqref{x:sound:head} imply \eqref{x:sound:I}, so
        \(\ind{R}(\cons{l}{p'}, f, \sigma)\) holds in this case
        too. (Note that the structure of the pattern \(p\) is
        irrelevant here.)

   \end{enumerate}

   \item Case \(p = \cons{\meta{x}}{p'}\).\\ Only two inference rules
     can have the configuration \(\smat{f}{\cons{\meta{x}}{p'}}\) in
     the conclusion: \textsf{UNPAR} and \textsf{BIND}. Let us consider
     all the possible derivations.
     \begin{enumerate}

        \item Case where \(\Delta\) ends with \textsf{UNPAR}.\\ This
          case is the same as case~(\ref{unpar}). 

        \item Case where \(\Delta\) ends with \textsf{BIND}.
        \begin{mathpar}
          \inferrule*[right=\text{\sf BIND}]
            {t \in {\cal T}\\
              \inferrule*
                {\inferrule*[vdots=1.5em]{}{ }}
                {\smj{f'}{p'}{\sigma'}}}
            {\smj{\cons{t}{\!f'}}%
                {\cons{\meta{x}\!}{\!p'}}%
                {\sigma' \oplus x \mapsto t}}
        \end{mathpar}
        where, because we assumed \eqref{x:sound:head},
        \begin{enumerate}

          \item \label{x:sound:17} \(f \triangleq \cons{t}{f'}\), where
            \(t \in {\cal T}\),

          \item \label{x:sound:18} \(p \triangleq
            \cons{\meta{x}}{p'}\),

          \item \label{bind_x} \(\sigma' \subseteq \sigma' \oplus x
            \mapsto t\),

            \item \label{bind_rho} \(\sigma \triangleq \sigma' \oplus
              x \mapsto t\).

        \end{enumerate}
        Let us assume that the induction hypothesis holds for the
        premise of \textsf{BIND}, i.e., \(\ind{R}(p', f', \sigma')\)
        holds. Therefore
        \begin{gather}
          \subst{p'}{\sigma'} \sqsubseteq f' \label{x:sound:7}
        \end{gather}
        We also have
        \begin{align}
           \sigma(x) 
          &\triangleq (\sigma' \oplus x \mapsto t)(x)
          &\text{by} \,\; \text{\ref{bind_rho}}\notag\\
          &= t
          &\text{by} \,\; \eqref{model:oplus}\notag\\
           \sigma(x)
          &= t \label{x:sound:8}
           \intertext{and}
           \subst{p'}{\sigma'}
           &= \subst{p'}{(\sigma' \oplus x \mapsto t)}
           &\text{by} \,\; \text{\ref{bind_x}} \,\; \text{and} \,\;
           \text{\ref{minimality}}\notag\\
           \subst{p'}{\sigma'}
           &= \subst{p'}{\sigma}
           &\text{by} \,\; \text{\ref{bind_rho}} \label{x:sound:19}
        \end{align}
        Besides, we have
        \begin{align}
           \subst{p}{\sigma}
        &\triangleq \subst{\cons{\meta{x}}{p'}}{\sigma}
        &\text{by} \,\; \text{\ref{x:sound:18}}\notag\\
        &\eqn{2} \cons{\sigma(x)}{\subst{p'}{\sigma}}\notag\\
        &= \cons{t}{\subst{p'}{\sigma}}
        &\text{by} \,\; \eqref{x:sound:8}\notag\\
        &= \cons{t}{\subst{p'}{\sigma'}}
        &\text{by} \,\; \eqref{x:sound:19}\notag\\
        &\sqsubseteq \cons{t}{f'}
        &\text{by} \,\; \eqref{x:sound:7} \,\; \text{and} \,\;
           \textsf{EQ}\notag\\
        &\triangleq f
        &\text{by} \,\; \text{\ref{x:sound:17}}\notag\\
           \subst{p}{\sigma}
        &\sqsubseteq f \label{x:sound:9}
        \end{align}
        In the end, the induction hypothesis and \eqref{x:sound:head}
        imply \eqref{x:sound:9}, so \(\ind{R}(\cons{\meta{x}}{p'}, f,
        \sigma)\) holds. \hfill \(\Box\)
     \end{enumerate}

\end{enumerate}


