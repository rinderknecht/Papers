%%-*-latex-*-

\section{Conclusion}
\label{concl}

We have shown how concrete syntax pattern matching can be integrated
with minimal effort in any parser\hyp{}based tool, written in any host
language and manipulating any subject language, including
multi\hyp{}language tools such as legacy systems analysers or
intentional programming systems. Matching concrete syntax can make the
code of such tools more concise and more readable.  The purely
syntactic matching can be easily complemented with any other checks in
the host language, due to a natural embedding of patterns as strings
in the host language.

Furthermore, unparsed patterns give a very simple means to make such
tools extensible with user\hyp{}defined behaviour. In particular, most
existing tools can be made extensible with minimal effort. Extensible
compilers are a particular application in which users may add their
own program checks. Other possible applications may involve model
checkers, program inspectors, etc.

The formal model we introduced allowed us to precisely define our
matching algorithms. The first makes use of backtracking and is thus
of theoretical interest. The same formalism allowed us to prove its
soundness and completeness with respect to the classic pattern
matching. We also proved that it defines a function. The second
algorithm, \textit{ES}(1), is of practical interest because it runs in
worst\hyp{}case linear\hyp{}time with a lookahead of one lexeme. This
lookahead is naturally an approximation of the lexical context, its
consequence is to lose completeness with respect to the classic
pattern matching, despite being proven correct. This incompleteness
can be overcome in all cases by adding meta\-parentheses to the
pattern.

We illustrated \textit{ES}(1) on realistic, albeit small, examples
procesed by the prototype \Matchbox. We also described rather in
detail how the unparsed patterns have been successfully integrated
into \GCC.

Unparsed patterns can be improved in several respects. For instance,
it would be interesting, both from a practical and theoretical point
of view, to reduce even further the amount of meta\-parentheses
required to obtain a complete matching algorithm. Also, as far as
expressiveness is concerned, pattern variables could be typed and
could also match tokens. Moreover, a single pattern variable could
match a list of elements such as found in some other matchers, even
when that list of elements is not grouped as a distinct subtree. Also,
it would be useful to allow for empty trees in the definition of
unparsing (yielding an empty list). In terms of execution time, it is
an open question whether the subtree matching problem, when using
unparsed patterns, can be solved more efficiently than in quadratic
time. Finally, concrete uses of dynamic patterns remain to be
investigated.
