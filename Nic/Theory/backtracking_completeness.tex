%%-*-latex-*-

\subsection{Completeness of the backtracking algorithm}
\label{backtracking:completeness_proof}

\begin{theorem}[Completeness]\hfill
\begin{center}
If   \(\subst{\overline{p}}{\sigma} \sqsubseteq h\)
then \(\smat{h}{\overline{p}} \twoheadrightarrow \sigma'\) 
and  \(\sigma' \subseteq \sigma\).
\end{center}
\end{theorem}

\begin{proof}[Theorem~\ref{backtracking:completeness_proof}] Let
  \(\ind{W}(\overline{p}, f, \sigma, \sigma')\) be the proposition:
\begin{center}
\emph{If}   \(\subst{\overline{p}}{\sigma} \sqsubseteq f\)
\emph{then} \(\smj{f}{\overline{p}}{\sigma'}\)
\emph{and} \(\sigma' \subseteq \sigma\).
\end{center}
The completeness is equivalent to \(\ind{W}(\overline{p}, [h], \sigma,
\sigma')\). First, let us assume that
\begin{align}
  \subst{\overline{p}}{\sigma} \sqsubseteq f \label{x:compl:1}
\end{align}
(otherwise the theorem would be trivially true). This means that there
exists a derivation \(\Delta\) in the inference system
(Figure~\ref{x_tree_matching_def}) defining \((\sqsubseteq)\) whose
conclusion is \(\subst{\overline{p}}{\sigma} \sqsubseteq f\). This
derivation is a list, which makes it possible to reckon by induction
on its structure, i.e., we assume that \(\ind{W}\) holds for the
premise of the last rule in \(\Delta\) (the \emph{induction
hypothesis}) and then prove that \(\ind{W}\) holds for
\(\subst{\overline{p}}{\sigma} \sqsubseteq f\). We proceed case by
case on the kind of rule that can end the derivation.
\begin{enumerate}

  \item Case where \(\Delta\) ends with \textsf{EMP}.\\ We have \(f =
    \el\) and \(\subst{\overline{p}}{\sigma} = \el \eqn{1}
    \subst{\el}{\sigma}\). Therefore \(\overline{p} = \el\) (the
    substitution is injective by construction), which leads to
    \(\smat{f}{\overline{p}} = \smj{\el}{\el}{\sigma_\varnothing}\) by
    means of the axiom \textsf{END} (Figure~\ref{x_match_def}). And we
    trivially have \(\sigma_\varnothing \subseteq \sigma\). We
    conclude that \(\ind{W}(\el, \el, \sigma, \sigma_\varnothing)\)
    holds.

  \item Case where \(\Delta\) ends by \textsf{EQ}.\\ Then, for all
    parse tree \(h\) (perhaps reduced to a single lexeme), there
    exist two forests \(f_1\) and \(f_2\) such that
    \begin{align}
      \subst{\overline{p}}{\sigma}
        &\triangleq \cons{h}{f_1} \label{x:compl:5}\\
      f &\triangleq \cons{h}{f_2} \label{x:compl:6}
    \end{align}
    and \(\Delta\) has the shape
    \begin{mathpar}
      \inferrule*[right={\text{\sf EQ}}]
        {\inferrule*
           {\inferrule*[vdots=1.5em]{}{ }}
           {f_1 \sqsubseteq f_2}}
        {\cons{h}{f_1} \sqsubseteq \cons{h}{f_2}}
    \end{mathpar}
    By examining the definition of substitutions in
    Figure~\ref{x_subst_def}, we deduce that we have two exclusive
    options for \(h\):
    \begin{enumerate}
 
      \item Case where \(h = l \in {\cal L}\).\\ Definition
        \eqref{x:compl:5} and definition \((\eqn{3})\) of the
        substitutions imply that there exists an unparsed pattern
        \(\overline{p}'\) such that
        \begin{align}
             \overline{p}
          &= \cons{l}{\overline{p}'}\label{x:compl:8}\\
             f_1
          &= \subst{\overline{p}'}{\sigma}.\notag
        \end{align}
        So \(\Delta\) has the refined shape
        \begin{mathpar}
          \inferrule*[right={\text{\sf EQ}}]
            {\inferrule*
               {\inferrule*[vdots=1.5em]{}{ }}
               {\subst{\overline{p}'}{\sigma} \sqsubseteq f_2}}
            {\subst{\overline{p}}{\sigma} \sqsubseteq f}
        \end{mathpar}
        Let us assume that the induction hypothesis holds for the
        premise of \textsf{EQ}, i.e.,
        \begin{align}
            \smat{f_2}{\overline{p}'}
          &\twoheadrightarrow \sigma' \label{x:compl:2}\\
            \sigma'
          &\subseteq \sigma. \label{x:compl:3}
        \end{align}
        Configuration \eqref{x:compl:2} can be the premise of
        pattern\hyp{}matching rule \textsf{ELIM}
        (Figure~\ref{x_match_def}), whose conclusion is then
        \begin{gather}
          \smj{\cons{l}{f_2}}{\cons{l}{\overline{p}'}}{\sigma'}.
          \label{x:compl:10}
        \end{gather}
        By \eqref{x:compl:6} and \eqref{x:compl:8}, \eqref{x:compl:10}
        is equivalent to
        \begin{gather}
          \smj{f}{\overline{p}}{\sigma'}. \label{x:compl:7}        
        \end{gather}
        As a conclusion, the induction hypothesis and
        \eqref{x:compl:1} imply \eqref{x:compl:7} and
        \eqref{x:compl:3}, i.e., \(\ind{W}(\cons{l}{\overline{p}'},
        \cons{l}{f'}, \sigma, \sigma')\) holds.

      \item Case where \(h = t \in {\cal T}\).\\ Definition
        \eqref{x:compl:5} and definition \((\eqn{2})\) of the
        substitutions (Figure~\ref{x_subst_def}) imply that there
        exists an unparsed pattern \(\overline{p}'\) and a
        meta\-variable \(x\) such that
        \begin{align}
          \overline{p} &= \cons{\meta{x}}{\overline{p}'} \label{x:compl:11}\\
          f_1 &= \subst{\overline{p}'}{\sigma}\notag\\
          t &= \sigma(x). \label{x:compl:12}
        \end{align}
        So \(\Delta\) has the refined shape
        \begin{mathpar}
          \inferrule*[right={\text{\sf EQ}}]
            {\inferrule*
               {\inferrule*[vdots=1.5em]{}{ }}
               {\subst{\overline{p}'}{\sigma} \sqsubseteq f'}}
            {\subst{\overline{p}}{\sigma} \sqsubseteq f}
        \end{mathpar}
        Let us assume that the induction hypothesis holds for the
        premise of \textsf{EQ}, i.e.,
        \(\subst{\overline{p}'}{\sigma} \sqsubseteq f'\) of
        \textsf{EQ}, i.e.,
        \begin{align}
          \smat{f'}{\overline{p}'} &\twoheadrightarrow \sigma'
          \label{x:compl:14}\\
          \sigma' &\subseteq \sigma. \label{x:compl:15}
        \end{align}
        Let us consider two cases now:
        \begin{enumerate}

          \item \(x \not\in \dom{\rho'}\).\\ Then, by definition of
            inclusions of substitutions \eqref{model:incl} implies
            \begin{gather*}
              \sigma' \subseteq \sigma' \oplus x \mapsto t.
            \end{gather*}

          \item \(x \in \dom{\rho'}\).\\ The definition of inclusions
            \eqref{model:incl} applied to \eqref{x:compl:15} implies
            that \(\sigma'(x) = \sigma(x)\), thus, by
            \eqref{x:compl:12}, we have \(\sigma'(x) = t\). So
            \(\sigma' \subseteq \sigma' \oplus x \mapsto t\), since,
            by definition \eqref{model:oplus}, \((\sigma' \oplus x
            \mapsto t)(x) \triangleq t = \sigma'(x)\).

        \end{enumerate}
        In both cases, we proved that
        \begin{gather}
          \sigma' \subseteq \sigma' \oplus x \mapsto t.
          \label{x:compl:16}
        \end{gather}
        Configuration \eqref{x:compl:14} and property
        \eqref{x:compl:16} can be the premises to the
        pattern\hyp{}matching rule \textsf{BIND}, which leads to the
        conclusion
        \begin{gather}
          \smj{\cons{t}{f'}}{\cons{\meta{x}}{\overline{p}'}}{\sigma'
            \oplus x \mapsto t}. \label{x:compl:17}
        \end{gather}
        By \eqref{x:compl:11} and \eqref{x:compl:6},
        \eqref{x:compl:17} is equivalent to
        \begin{gather}
          \smj{f}{\overline{p}}{\sigma' \oplus x \mapsto
          t}. \label{x:compl:18}
        \end{gather}
        As a conclusion, the induction hypothesis and
        \eqref{x:compl:1} imply \eqref{x:compl:18} and
        \eqref{x:compl:16}, i.e.,
        \(\ind{W}(\cons{\meta{x}}{\overline{p}'}, \cons{t}{f'},
        \sigma' \oplus x \mapsto t, \sigma')\).

    \end{enumerate}

  \item Case where \(\Delta\) ends by \textsf{SUB}.\\ Then there
    exist two forests \(f_1\) and \(f_2\) and a constructor \(c \in
    {\cal C}\) such that
    \begin{align}
      f \triangleq \cons{c(f_1)}{f_2}. \label{x:compl:19}
    \end{align}
    The derivation \(\Delta\) has thus the shape
    \begin{mathpar}
      \inferrule*[right=\text{\sf SUB}]
        {\inferrule*
           {\inferrule*[vdots=1.5em]{}{ }}
           {\subst{\overline{p}}{\sigma} \sqsubseteq f_1 \cdot f_2}}
        {\subst{\overline{p}}{\sigma} \sqsubseteq \cons{c(f_1)}{f_2}}
    \end{mathpar}
    Let us assume that the induction hypothesis holds for the premise
    of \textsf{SUB}, i.e.,
    \begin{align}
      \smat{f_1 \cdot f_2}{\overline{p}} &\twoheadrightarrow \sigma'
      \label{x:compl:20}\\ 
      \sigma' &\subseteq \sigma. \label{x:compl:21}
    \end{align}
    The configuration \eqref{x:compl:20} can be the premise of the
    pattern\hyp{}matching rule \textsf{UNPAR}, which leads to the
    conclusion
    \begin{gather}
      \smj{\cons{c(f_1)}{f_2}}{\overline{p}}{\sigma'}. \label{x:compl:22}
    \end{gather}
    As a conclusion, the induction hypothesis and \eqref{x:compl:1}
    imply \eqref{x:compl:22} and \eqref{x:compl:21}, i.e.,
    \(\ind{W}(\overline{p}, \cons{c(f_1)}{f_2}, \sigma, \sigma')\).

\end{enumerate}
\end{proof}
