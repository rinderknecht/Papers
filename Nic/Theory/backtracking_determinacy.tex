%%-*-latex-*-

\subsection{Determinacy of the backtracking algorithm}
\label{backtracking:determinacy}

\begin{theorem}[Determinacy]\hfill
\begin{center}
If \(\smj{h}{\overline{p}}{\sigma}\)
and \(\smj{h}{\overline{p}}{\sigma'}\)
then \(\sigma = \sigma'\).
\end{center}
\end{theorem}
\noindent In other words, the fact that the system is not
syntax\hyp{}directed, hence, that the implementation needs a
backtracking strategy, does not impede that there is none the less
only one result to the pattern matching of a forest.

\begin{proof}[Theorem~\ref{backtracking:determinacy}] Let
  \(\ind{S}(\overline{p}, f, \sigma, \sigma')\) be the proposition
\begin{center}
\emph{If} \(\smj{f}{\overline{p}}{\sigma}\) 
\emph{and} \(\smj{f}{\overline{p}}{\sigma'}\)
\emph{then} \(\sigma = \sigma'\).
\end{center}
In particular, \(\ind{S}(\overline{p}, [h], \sigma, \sigma')\) is
equivalent to the determinacy. Firstly, let us assume that
\begin{gather}
\smj{f}{\overline{p}}{\sigma} \label{x:deter:delta}\\
\smj{f}{\overline{p}}{\sigma'} \label{x:deter:delta'}
\end{gather}
are true (otherwise the theorem is trivially true). This means that
there exists a pattern\hyp{}matching derivation \(\Delta\) whose
conclusion is \(\smj{f}{\overline{p}}{\sigma}\) and another derivation
\(\Delta'\) whose conclusion is
\(\smj{f}{\overline{p}}{\sigma'}\). These derivations are trees
(actually, lists), so we can reason by general induction on them,
i.e., we assume that \(\ind{S}\) holds for (the conclusions of) any
pair of sub\hyp{}derivations, one from \(\Delta\) and the other from
\(\Delta'\), and then prove that \(\ind{S}\) holds for the conclusions
of \(\Delta\) and \(\Delta'\) as well ---~that is to say,
\(\ind{S}(\overline{p}, f, \sigma, \sigma')\) is true. (This induction
schema implies that the configurations must be the same in the
judgements of the sub\hyp{}derivations considered as an hypothesis.)
We proceed case by case on the kinds of rule that may end the
derivations. Since \(\overline{p}\) and \(f\) must be the same in the
conclusions of \(\Delta\) and \(\Delta'\), some pairs of ending rules
for the derivations are impossible. For example, if \textsf{ELIM} ends
one derivation, then \textsf{BIND} cannot end the other because the
former mandates that \(\overline{p} = \cons{l}{\overline{p}'}\), with
\(l \in {\cal L}\), whilst the latter requires that \(\overline{p} =
\cons{\meta{x}}{\overline{p}'}\) and \(l \neq \meta{x}\). In the
following, we only give the possible combinations. Also, if \(\Delta\)
can end with a rule \(X\) and \(\Delta'\) with another rule \(Y\), we
do not give the case where \(\Delta\) ends with rule \(Y\) and
\(\Delta'\) with rule \(X\) (symmetric case).
\begin{enumerate}

  \item Case where \(\Delta\) and \(\Delta'\) end by \textsf{END}.\\
    Then \(\overline{p} = \el\), \(f = \el\), \(\sigma =
    \sigma_\varnothing\) and \(\sigma' = \sigma_\varnothing\). As a
    consequence, \(\sigma = \sigma'\), that is, \(\ind{S}(\el, \el,
    \sigma_\varnothing, \sigma_\varnothing)\) holds.

  \item Case where \(\Delta\) and \(\Delta'\) end by \textsf{ELIM}.
    \begin{mathpar}
      \inferrule*[right=\text{\sf ELIM}]
         {\inferrule
            {\inferrule*[vdots=1.5em]{}{(\Delta_1)}}
            {\smj{f'}{\overline{p}'}{\sigma}}}
         {\smj{\cons{l}{\!f'}}{\cons{l}{\overline{p}'}}{\sigma}}

       \inferrule*[right=\text{\sf ELIM}]
         {\inferrule
            {\inferrule*[vdots=1.5em]{}{(\Delta'_1)}}
            {\smj{f'}{\overline{p}'}{\sigma'}}}
         {\smj{\cons{l}{\!f'}}{\cons{l}{\overline{p}'}}{\sigma'}}
    \end{mathpar}
    where, since we assumed \eqref{x:deter:delta} and
    \eqref{x:deter:delta'},
    \begin{enumerate}

      \item \label{x:deter:1} \(l \in {\cal L}\),

      \item \label{x:deter:3} \(\overline{p} \triangleq
        \cons{l}{\overline{p}'}\),

      \item \label{x:deter:2} \(f \triangleq \cons{l}{f'}\).

    \end{enumerate}
    Since \(\Delta_1\) is a sub\hyp{}derivation of \(\Delta\) and
    \(\Delta'_1\) is a sub\hyp{}derivation of \(\Delta'\), let us
    assume that the induction hypothesis holds for their conclusions,
    i.e., \(\ind{S}(\overline{p}', f', \sigma, \sigma')\)
    holds. Therefore
    \begin{equation}
      \sigma = \sigma' \label{x:deter:4}.
    \end{equation}
    Finally, the induction hypothesis, \eqref{x:deter:delta} and
    \eqref{x:deter:delta'} imply \eqref{x:deter:4}, that is to say,
    \(\ind{S}(\cons{l}{\overline{p}'}, \cons{l}{f'}, \sigma,
    \sigma')\).

  \item Case where \(\Delta\) and \(\Delta'\) both end by
  \textsf{BIND}.
    \begin{mathpar}
      \inferrule[\text{\sf BIND}]
        {\inferrule*
           {\inferrule*[vdots=1.5em]{}{(\Delta_1)}}
           {\smj{f'}{\overline{p}'}{\sigma_1}}\\
         \sigma_1 \subseteq \sigma_1 \oplus x \mapsto t}
        {\smj{\cons{t}{f'}}%
             {\cons{\meta{x}}{\overline{p}'}}%
             {\sigma_1 \oplus x \mapsto t}}

      \inferrule[\text{\sf BIND}]
        {\inferrule*
           {\inferrule*[vdots=1.5em]{}{(\Delta'_1)}}
              {\smj{f'}{\overline{p}'}{\sigma_2}}\\
            \sigma_2 \subseteq \sigma_2 \oplus x \mapsto t}
           {\smj{\cons{t}{f'}}%
                {\cons{\meta{x}}{\overline{p}'}}%
                {\sigma_2 \oplus x \mapsto t}}
    \end{mathpar}
    where, since we assumed \eqref{x:deter:delta} and
    \eqref{x:deter:delta'},
    \begin{enumerate}

       \item \label{x:deter:8} \(\overline{p} \triangleq
         \cons{\meta{x}}{\overline{p}'}\),

       \item \label{x:deter:7} \(f \triangleq \cons{t}{f'}\), with \(t
         \in {\cal T}\),

       \item \label{x:deter:9} \(\sigma \triangleq \sigma_1 \oplus x
         \mapsto t\),

       \item \label{x:deter:10} \(\sigma' \triangleq \sigma_2 \oplus x
         \mapsto t\).

     \end{enumerate}
     Since \(\Delta_1\) is a sub\hyp{}derivation of \(\Delta\) and
     \(\Delta'_1\) is a sub\hyp{}derivation of \(\Delta'\), let us
     assume that the induction hypothesis holds for their conclusions,
     i.e., \(\ind{S}(\overline{p}', f', \sigma_1,
     \sigma_2)\). Therefore
     \begin{align}
           \sigma_1 
        &= \sigma_2\notag\\
           \sigma_1 \oplus x \mapsto t 
        &= \sigma_2 \oplus x \mapsto t\notag\\
           \sigma
        &= \sigma'
        &\text{by} \,\; \text{\ref{x:deter:9}} \,\; \text{and} \,\;
          \text{\ref{x:deter:10}}. \label{x:deter:12}
     \end{align}
     Finally, the induction hypothesis, \eqref{x:deter:delta} and
     \eqref{x:deter:delta'} imply \eqref{x:deter:12}, that is to say,
     \(\ind{S}(\cons{\meta{x}}{\overline{p}'}, f, \sigma, \sigma')\).

  \item \label{unpar2_bis} Case where both \(\Delta\) and \(\Delta'\)
    end by \textsf{UNPAR}.
    \begin{mathpar}
      \inferrule*[right=\text{\sf UNPAR}]        
        {\inferrule
           {\inferrule*[vdots=1.5em]{}{(\Delta_1)}}
           {\smj{f_1 \!\cdot\! f_2}{\overline{p}}{\sigma}}}
        {\smj{\cons{c(f_1)\!}{\!f_2}}{\overline{p}}{\sigma}}

      \inferrule*[right=\text{\sf UNPAR}]        
        {\inferrule
           {\inferrule*[vdots=1.5em]{}{(\Delta'_1)}}
              {\smj{f_1 \!\cdot\! f_2}{\overline{p}}{\sigma'}}}
        {\smj{\cons{c(f_1)\!}{\!f_2}}{\overline{p}}{\sigma'}}
    \end{mathpar}
    where, since we assumed \eqref{x:deter:delta} and
    \eqref{x:deter:delta'},
    \begin{enumerate}

       \item \label{x:deter:5} \(f \triangleq \cons{c(f_1)}{f_2}\).

    \end{enumerate}
    Since \(\Delta_1\) is a sub\hyp{}derivation of \(\Delta\) and
    \(\Delta'_1\) is a sub\hyp{}derivation of \(\Delta'\), let us
    assume that the induction hypothesis holds for their conclusions,
    that is, \(\ind{S}(\overline{p}, f_1 \cdot f_2, \sigma, \sigma')\)
    holds:
    \begin{equation}
      \sigma = \sigma'. \label{x:deter:6}
    \end{equation}
    Finally, the induction hypothesis, \eqref{x:deter:delta} and
    \eqref{x:deter:delta'} imply \eqref{x:deter:6}, that is to say,
    \(\ind{S}(\overline{p}, \cons{c(f_1)}{f_2}, \sigma, \sigma')\).

  \item \label{bind_unpar_bis} Case where \(\Delta\) ends by
    \textsf{UNPAR} and \(\Delta'\) ends by \textsf{BIND}.
    \begin{mathpar}
      \inferrule[\text{\sf UNPAR}]
        {\inferrule*
           {\inferrule*[vdots=1.5em]{}{(\Delta_1)}}
           {\smj{f_1 \cdot f'}%
                {\cons{\meta{x}}{\overline{p}'}}%
                {\sigma}}}
        {\smj{\cons{c(f_1)}{f'}}%
             {\cons{\meta{x}}{\overline{p}'}}%
             {\sigma}}

      \inferrule[\text{\sf BIND}]
        {\inferrule*
            {\inferrule*[vdots=1.5em]{}{(\Delta'_1)}}
            {\smj{f'}{\overline{p}'}{\sigma''}}\\
            \sigma'' \subseteq \sigma'' \oplus x \mapsto t}
        {\smj{\cons{t}{f'}}%
             {\cons{\meta{x}}{\overline{p}'}}%
             {\sigma'' \oplus x \mapsto t}}
     \end{mathpar}
     where, since we assumed \eqref{x:deter:delta} and
     \eqref{x:deter:delta'},
     \begin{enumerate}

        \item \(\overline{p} \triangleq
        \cons{\meta{x}}{\overline{p}'}\), \label{x:deter:14}

        \item \(t \triangleq c(f_1)\), \label{x:deter:13}

        \item \(\sigma' \triangleq \sigma'' \oplus x \mapsto
          t\). \label{x:deter:15}

     \end{enumerate}
     The premise of \textsf{UNPAR} can only be itself the conclusion
     of the inference rule \textsf{UNPAR} or \textsf{BIND}. If
     \textsf{UNPAR} again, the same is true about its premise. So, the
     two cases can be merged into one, i.e., the shape of \(\Delta\)
     is (by writing only the pattern\hyp{}matching judgements)
     \begin{mathpar}
       \inferrule*[right=\text{\sf UNPAR}]
         {\inferrule*[right=\text{\sf UNPAR},rightskip=3em]
            {\inferrule*[right=\text{\sf UNPAR},rightskip=3em]
               {\inferrule*[right=\text{\sf BIND},vdots=1.5em,rightskip=3em]
                 {\inferrule*
                    {\inferrule*[vdots=1.5em]{}{(\Delta_2)}}
                    {\smj{g \cdot f'}{\overline{p}'}{\sigma_1}}}
                 {\smj{\cons{t}{g \cdot f'}}%
                      {\cons{\meta{x}}{\overline{p}'}}%
                      {\sigma_1 \oplus x \mapsto t}}}
               {\vdots}}
            {\smj{f_1 \cdot f_2}%
                 {\cons{\meta{x}}{\overline{p}'}}%
                 {\sigma_1 \oplus x \mapsto t}}}
         {\smj{\cons{c(f_1)}{f_2}}%
              {\cons{\meta{x}}{\overline{p}'}}%
              {\sigma_1 \oplus x \mapsto t}}
     \end{mathpar}
     where all the rules, if any, below \textsf{BIND} are
     \textsf{UNPAR} rules and
     \begin{equation}
       \sigma = \sigma_1 \oplus x \mapsto t. \label{x:deter:16}
     \end{equation}
     (Because the derivation is finite, the rule \textsf{BIND} must
     appear at one point up.) The number of unparsing steps can be
     \(0\), in which case \textsf{UNPAR} is followed inductively by
     \textsf{BIND}. Note that the list \(g\) may be empty if \(A(c) =
     1\) for each tree \(c(\dots)\) unparsed. Let us consider the two
     possible cases for \(g\).
     \begin{enumerate}

       \item Case \(g = \el\).\\ In this case, \(\smj{g \cdot
         f'}{\overline{p}'}{\sigma_1}\) is actually
         \(\smj{f'}{\overline{p}'}{\sigma_1}\). Since \(\Delta_2\) is
         a sub\hyp{}derivation of \(\Delta_1\), which is, in turn, a
         sub\hyp{}derivation of \(\Delta\), then \(\Delta_2\) is a
         sub\hyp{}derivation of \(\Delta\). Since \(\Delta'_1\) is a
         sub\hyp{}derivation of \(\Delta'\), the induction hypothesis
         holds for both the conclusions of \(\Delta_2\) and
         \(\Delta'_1\), so
         \begin{align}
           \sigma_1 &= \sigma''\notag\\
           \sigma_1 \oplus x \mapsto t &= \sigma'' \oplus x \mapsto
           t\notag\\
           \sigma &= \sigma'' \oplus \mapsto t
           & \text{by} \,\; \eqref{x:deter:16}\notag\\
           \sigma &= \sigma'
           & \text{by} \,\; \text{\ref{x:deter:15}}\notag\\
           \sigma &= \sigma' \label{x:deter:17}
         \end{align}
         Finally, the induction hypothesis, \eqref{x:deter:delta} and
         \eqref{x:deter:delta'} imply \eqref{x:deter:17} in this case,
         i.e., \(\ind{S}(\cons{\meta{x}}{\overline{p}'}, f, \sigma,
         \sigma')\) holds.

       \item Case \(g \neq \el\).\\ Now let us establish that the
         conjunction of the following judgements is contradictory,
         i.e., the derivation of one of them implies that the other
         cannot be derived.
         \begin{align}
             \smat{g \cdot f'}{\overline{p}'} 
           &\twoheadrightarrow \sigma_1 \label{x:deter:18} 
           & \text{where} \; g \neq \el\\
             \smat{f'}{\overline{p}'}
           &\twoheadrightarrow \sigma'' \label{x:deter:19}
         \end{align}
         The rules \textsf{ELIM} and \textsf{BIND}, read deductively
         (i.e., the premises before the conclusion), increment the
         length of the pattern whilst rule \textsf{UNPAR} keeps it
         unchanged. The derivations are lists starting with the same
         axiom \textsf{END}. Because of that and since
         \eqref{x:deter:18} and \eqref{x:deter:19} share the same
         pattern \(\overline{p}'\), it is sufficient to prove that
         \eqref{x:deter:18} cannot derive \eqref{x:deter:19} and,
         reciprocally, that \eqref{x:deter:19} cannot derive
         \eqref{x:deter:18}. Moreover, we only need to consider
         derivations by means of \textsf{UNPAR}, since it is the sole
         rule which keeps the pattern invariant.
         \begin{enumerate}

           \item \label{short_long} The \emph{length} \(\len{f}\) of a
             forest \(f\) is defined as
             \begin{align*}
                \len{\el} &= 0\\
                \len{\cons{t}{f}} &= 1 + \len{f}
             \end{align*}
             The rule \textsf{UNPAR}, read deductively (i.e, premise
             first, conclusion next), shortens the length of the
             forest or keeps it invariant (if \(A(c) = 1\)), since we
             have
             \begin{align*}
                   \len{f_1 \cdot f_2} &= \len{f_1} +
                    \len{f_2} \triangleq A(c) + \len{f_2}\\
                   \len{\cons{c(f_1)}{f_2}} &= 1 + \len{f_2}
             \end{align*}
             Therefore \eqref{x:deter:19} cannot derive
             \eqref{x:deter:18}, because \(g \neq \el\), so \(g \cdot
             f_2\) is strictly longer than \(f_2\), i.e., \(\len{f_2}
             \, < \len{g \cdot f_2}\).

           \item Let us consider now the derivation from
             \eqref{x:deter:18} by applying, perhaps repeatedly, the
             inference rule \textsf{UNPAR}. The effect of the
             deduction on the forest consists in replacing a prefix of
             at least one tree of \(f_1 \cdot f_2\) by a single tree
             of root \(c\), i.e., \(\cons{c(f_1)}{f_2}\). In
             particular, one tree can be replaced by one tree, but at
             one point at least two trees must be rewritten into one,
             because the derivation must try to reach
             \eqref{x:deter:19}, which contains a forest strictly
             shorter. Thus, the start of the derivation has the
             following shape
             \begin{mathpar}
               \inferrule*[Right=\text{\sf UNPAR}]
                  {\inferrule*[Right=\text{\sf UNPAR}]
                     {\smj{g \cdot f_2}{\overline{p}'}{\sigma_1}}
                     {\vdots}}
                  {\smj{\cons{t_1}{f_2}}{\overline{p}'}{\sigma_1}}
             \end{mathpar}
             The prefix \(g\) has been rewritten into a single tree
             \(t_1\) (by the way, this process is called
             \emph{parsing}). Let us assume that one more usage of the
             rule \textsf{UNPAR} rewrites at least \emph{two} more
             trees from \(\cons{t_1}{f_2}\) into one. (If not, it
             means that the size of the forest would remain constant,
             i.e., \(A(c) = 1\), and strictly greater than \(f_2\), as
             mentioned above.) But this shortening means that the
             resulting forest differs from \(f_2\) by, at least, the
             first tree. Therefore \eqref{x:deter:18} cannot derive
             \eqref{x:deter:19}.

         \end{enumerate}
         This achieves to prove that the sub\hyp{}case \(g \neq \el\)
         is impossible.

    \end{enumerate}

\end{enumerate}
\end{proof}
