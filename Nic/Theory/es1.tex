%%-*-latex-*-

\section{Algorithm \textit{ES}(1)}
\label{es1}

\subsection{Definitions}
\label{es1:def}

\paragraph{Parse Trees and Unparsed Patterns.}

Let \mlp and \mrp be meta\-lexemes corresponding respectively to the
opening and closing meta\-parentheses, whose concrete syntax is
\texttt{\%(} and \texttt{\%)}. These meta\-parentheses are inserted by
the user in order to guide matching the pattern against the parse tree
by forcing the enclosed pattern to match one
subtree. Meta\-parentheses can be present anywhere in the pattern as
long as they are properly paired. Unparsed patterns are formally
defined as the smallest set \(\overline{\cal P}\) such that
\begin{itemize}

  \item \(\el \in \overline{\cal P}\);

  \item if \(l \in {\cal L}\) and \(\overline{p} \in \overline{\cal
    P}\), then \(\cons{l}{\overline{p}} \in \overline{\cal P}\);

  \item if \(x \in {\cal V}\) and \(\overline{p} \in \overline{\cal
    P}\), then \(\cons{\meta{x}}{\overline{p}} \in \overline{\cal
    P}\);

  \item if \(\overline{p}_1, \overline{p}_2 \in \overline{\cal P}\)
    and \(\overline{p}_1 \neq \el\) then \([\mlp] \cdot \overline{p}_1
    \cdot [\mrp] \cdot \overline{p}_2 \in \overline{\cal P}\). 

\end{itemize}

\paragraph{Pattern Trees and Meta\-parsed Patterns.}

Given \(\overline{p}\), the first task is to check whether
\(\overline{p} \in \overline{\cal P}\). This is done by means of the
meta\-parsing function \textsf{metaparse}, which either fails due to
mismatched meta\-parentheses or returns the initial pattern where all
patterns enclosed in meta\-parentheses (these included) have been
replaced by a \emph{pattern tree}. These trees are not parse trees
because they contain patterns (thus the new patterns are still
unparsed with respect to the programming language syntax). Let
\(\unary{pat}\) be their unique constructor. Thus we have
\(\unary{pat} \not\in {\cal C}\). Let us define the set of
\emph{meta\-parsed patterns} \(p\) as the smallest set \({\cal P}\)
such that
\begin{itemize}

  \item \(\el \in {\cal P}\);

  \item if \(l \in {\cal L}\) and \(p \in {\cal P}\), then
    \(\cons{l}{p} \in {\cal P}\);

  \item if \(x \in {\cal V}\) and \(p \in {\cal P}\), then
    \(\cons{\meta{x}}{p} \in {\cal P}\);

  \item if \(p_1, p_2 \in {\cal P}\) and \(p_1 \neq \el\), then
    \(\cons{\pat{p_1}}{p_2} \in {\cal P}\).

\end{itemize}
The meta\-parsing function \textsf{metaparse} is formally defined by
means of an ordered rewrite system defining jointly three other
functions, \textsf{shift}, \textsf{reduce} and \textsf{check}. See
figure~\ref{es1_metaparsing}, where \(\overline{p}\) is an unparsed
pattern, \(p\) is a meta\-parsed pattern, \(\tilde{p}\) is, in
general, a temporary mix of both, \(l\) is a lexeme and \(e\) is a
lexeme or a meta\-lexeme.\footnote{The reader puzzled by the apparent
complexity of the definition must remember that it uses stacks as the
only data structure. It is recommended to skip it upon first
reading. Note also that the given rewrite system is straightforward to
implement in a functional language.}

Rule \((\xrightarrow{a})\) handles the empty pattern. Rules
\((\xrightarrow{b})\) and \((\xrightarrow{c})\) express that
meta\-variables and lexemes are directly copied to the result in the
same order as in the input.

Rule \((\xrightarrow{d})\) handles an opening meta\-parenthesis: it is
pushed into an auxiliary analysis stack passed as second argument to
the call of function \textsf{shift}. The purpose of this function is
to metaparse the input with the additional knowledge that an opening
meta\-parenthesis was found when it was first called ---~in other
words, the analysis is performed inside meta\-parentheses. Thus the
goal is reached when this supplementary stack contains only one
pattern tree, since it must correspond to a pair of meta\-parentheses
at the top\hyp{}level in the input pattern. In this case, i.e., rule
\((\xrightarrow{e})\), the pattern tree is copied to the output and
the meta\-parsing is resumed at the top\hyp{}level with
\textsf{metaparse}.

Rule \((\xrightarrow{g})\) copies anything from the input pattern to
the auxiliary stack, except closing meta\-parentheses because this
case triggers a reduction in rule \((\xrightarrow{f})\). Indeed,
finding \mrp means that the shortest prefix of the auxiliary stack
ending with an opening meta\-parenthesis must be replaced by the
corresponding pattern tree and then shifting is resumed (i.e.,
meta\-parsing but not at the top\hyp{}level). Rule
\((\xrightarrow{i})\) pops any item, except \mlp, from the auxiliary
stack and pushes it to a temporary stack which contains the
first\hyp{}level subtrees of the pattern tree under construction. If
an opening meta\-parenthesis is reached, in rule
\((\xrightarrow{h})\), it means that the shortest prefix to be reduced
has been found, therefore a pattern tree (always of root \textsf{pat})
is produced on top of the remaining auxiliary stack. Since the
sequence \texttt{\%(\%)} is a meta\-syntax error, we must check that
rule \((\xrightarrow{h})\) does not build a pattern tree without
subtrees: this is the purpose of function \textsf{check} and the
unique rule \((\xrightarrow{j})\).

As usual with rewrite systems, errors are not specified: a
meta\-parsing error occurs if a term is reached and it cannot be
rewritten furthermore. A tiny example of meta\-parsing is given next.

\begin{figure}[t]
\framebox[\columnwidth]{%
\(
\begin{aligned}
  \cstr{metaparse}{\el}
&\xrightarrow{a} \el\\
  \cstr{metaparse}{\cons{\meta{x}}{\overline{p}}}
&\xrightarrow{b} \cons{\meta{x}}{\cstr{metaparse}{\overline{p}}}\\
  \cstr{metaparse}{\cons{l}{\overline{p}}}
&\xrightarrow{c} \cons{l}{\cstr{metaparse}{\overline{p}}}\\
  \cstr{metaparse}{\cons{\mlp}{\overline{p}}}
&\xrightarrow{d} \cstr{shift}{\overline{p}, [\mlp]}\\
  \cstr{shift}{\overline{p}, [\pat{{p}}]}
&\xrightarrow{e} \cons{\pat{{p}}}{\cstr{metaparse}{\overline{p}}}\\
  \cstr{shift}{\cons{\mrp}{\overline{p}}, \tilde{p}}
&\xrightarrow{f} \cstr{shift}{\overline{p}, \cstr{reduce}{\tilde{p}, \el}}\\
  \cstr{shift}{\cons{e}{\overline{p}}, \tilde{p}}
&\xrightarrow{g} \cstr{shift}{\overline{p}, \cons{e}{\tilde{p}}}\\
  \cstr{reduce}{\cons{\mlp}{\tilde{p}}, p}
&\xrightarrow{h} \cstr{check}{\cons{\pat{p}}{\tilde{p}}}\\
  \cstr{reduce}{\cons{e}{\tilde{p}}, p}
&\xrightarrow{i} \cstr{reduce}{\tilde{p}, \cons{e}{p}}\\
  \cstr{check}{\cons{\pat{\cons{e}{p}}}{\tilde{p}}}
&\xrightarrow{j} \cons{\pat{\cons{e}{p}}}{\tilde{p}}
\end{aligned}
\)
}
\caption{Meta\-parsing of patterns in
  \textit{ES}(1) (ordered rules)\label{es1_metaparsing}}
\end{figure}

\smallskip

\piccaption{A pattern tree\label{es1_pattern_tree}}
\parpic[r]{\fbox{\includegraphics[bb=71 671 128 721]{pattern_tree}}}
\noindent Pattern \(l_1 \; \texttt{\%(} \; l_2 \; l_3 \; \texttt{\%(}
\; \texttt{\%x} \; l_4 \; \texttt{\%)} \; \texttt{\%)}\) is the
simplified form of unparsed pattern \([l_1, \!\mlp, l_2, l_3, \!\mlp,
\meta{\!x\!}, l_4, \mrp, \mrp].\) Its meta\-parsing by
\textsf{meta\-parse} yields \([l_1, \!\pat{[l_2, \!l_3,
\!\pat{[\meta{\!x\!}, l_4]}]}]\), which contains the pattern tree of
figure~\ref{es1_pattern_tree}.
\begin{remark}
Thereupon we write `pattern' instead of `meta\-parsed pattern'.
\end{remark}


\paragraph{Substitutions and Closed Patterns.}

Let us extend substitutions to cope with patterns, not just
meta\-variables, as we did about the back\-tracking algorithm in
sub\-section~\ref{backtracking:definitions}. Closed patterns are
defined inductively as the smallest set \(\close{\cal P}\) such
that\footnote{The circles above are meant to suggest
self\hyp{}containment.}
\begin{itemize}

  \item \(\el \in \close{\cal P}\);

  \item if \(h \in {\cal H}\) and \(\close{p} \in \close{\cal P}\),
        then \(\cons{h}{\close{p}} \in \close{\cal P}\);

  \item if \(\close{p}_1, \close{p}_2 \in \close{\cal P}\) and
    \(\close{p}_1 \neq \el\), then
    \(\cons{\pat{\close{p}_1}}{\close{p}_2} \in \close{\cal P}\).

\end{itemize}
Note the absence of meta\-variables in \(\close{\cal P}\) and,
instead, the presence of trees (\(h \in {\cal H}\)). The formal
definition of substitutions can be found in
figure~\ref{es1_subst_def}. It is clear that the result of such
substitutions is always a closed pattern.
\begin{figure}[H]
\framebox[\columnwidth]{%
\(
\begin{aligned}
  \subst{\el}{\sigma} 
&\eqn{1} \el\\
  \subst{\cons{\meta{x}}{p}}{\sigma}
&\eqn{2} \cons{\sigma(x)}{\subst{p}{\sigma}}\\
  \subst{\cons{l}{p}}{\sigma}
&\eqn{3} \cons{l}{\subst{p}{\sigma}}\\
  \subst{\cons{\pat{p_1}}{p_2}}{\sigma}
&\eqn{4}
  \cons{\pat{\subst{p_1}{\sigma}}}{\subst{p_2}{\sigma}}
\end{aligned}
\)
}
\caption{Substitutions on meta\-parsed patterns%
\label{es1_subst_def}}
\end{figure}


\paragraph{Closed\hyp{}Tree Matching.}

A relationship \((\sqsubseteq)\) is needed to capture the concept of a
closed\hyp{}tree matching a parse tree, just as we did about the
backtracking algorithm in
sub\-section~\ref{backtracking:definitions}. The difference here is
that closed trees may contain special nodes \textsf{pat} that
distinguish them from parse trees. The formal definition we propose is
based on ordered inference rules and given in
figure~\ref{es1_tree_matching_def}.

Rule \textsf{ONE} states that a pattern matches a tree if the same
pattern matches the corresponding singleton forest. This allows to
relate a pattern and a tree.

Rule \textsf{EMP} expresses that the empty pattern always matches the
empty forest.

Rule \textsf{EQ} specifies that if the head of the pattern and the
forest is the same closed tree, then the pattern matches the forest if
the remaining pattern \(\close{p}\) and forest \(f\) match.

Rule \textsf{PAT} handles the case of a pattern tree: there is a match
if the subtrees \(\close{p}_1\) of the pattern tree match the subtrees
\(f_1\) of the forest head and if the remaining pattern
\(\close{p}_2\) matches the remaining forest \(f_2\).

Rule \textsf{SUB} states that if the two heads are different trees
(this is implicit due to partial ordering and \textsf{SUB} being the
last), there is a match if the pattern matches the catenation of the
subtrees \(f_1\) of the tree in the forest and the remaining forest
\(f_2\). This rule must always be considered last, i.e., if a
derivation ends with \textsf{SUB}, the last conclusion has the shape
\(\close{p} \sqsubseteq \cons{c(f_1)}{f_2}\) \emph{and it is implied
that there are no closed patterns} \(\close{p}_1\) \emph{and}
\(\close{p}_2\) \emph{such that} \(\close{p} =
\cons{c(f_1)}{\close{p}_1}\) or \(\close{p} =
\cons{\pat{\close{p}_1}}{\close{p}_2}\) (which would lead to the
conclusions of \textsf{EQ} or \textsf{PAT}).
\begin{figure}[H]
\framebox[\columnwidth]{
\begin{mathpar}
\inferrule*{}{\el \sqsubseteq \el}
\;\TirName{\text{\sf EMP}}

\inferrule
  {\close{p} \sqsubseteq f}
  {\cons{h}{\close{p}} \sqsubseteq \cons{h}{f}}
\,\TirName{\text{\sf EQ}}

\inferrule
  {\close{p} \sqsubseteq [h]}
  {\close{p} \sqsubseteq h}
\,\TirName{\text{\sf ONE}}

\inferrule*[right=\text{\sf PAT}]
  {\close{p}_1 \sqsubseteq f_1 
   \and
   \close{p}_2 \sqsubseteq f_2}
  {\cons{\pat{\close{p}_1}}{\close{p}_2} \sqsubseteq 
   \cons{c(f_1)}{f_2}}
\quad
\inferrule*[right=\text{\sf SUB}]
  {\close{p} \sqsubseteq f_1 \cdot f_2}
  {\close{p} \sqsubseteq \cons{c(f_1)}{f_2}}
\end{mathpar}
}
\caption{Closed\hyp{}pattern matching for \textit{ES}(1)%
(\textsf{SUB} is last)
\label{es1_tree_matching_def}}
\end{figure}

\paragraph{Pattern Matching.}

Compared to the informal description of the backtracking pattern
matching given in section~\ref{backtracking}, the binding and the
unparsing steps are here specialised so to exclude each other based on
the shape of the configuration at hand (this is
syntax\hyp{}direction). The critical case is when the current pattern
starts with a meta\-variable: a step may be chosen that leads to a
match failure whilst the other step would have not. We will come back
later on this point. For now, let us describe the new binding and
unparsing operations.
\begin{itemize}

  \item \textbf{Binding.} The first element of the pattern is a
    meta\-variable \(\meta{x}\) and the first tree \(t\) of the parse
    forest is not a leaf. The meta\-variable is bound to the tree,
    i.e., \(x \mapsto t\) in the following cases.
    \begin{itemize}

      \item When the parse forest is reduced to one tree. An unparsing
      is possible if the tree root has only one subtree (so cutting
      out the root leads to another single tree), but the programmers
      usually want to bind the biggest tree. This step is the rule
      \textsf{BIND}\(_3\) in figure~\ref{es1_match_def}.

      \item When the second element of the pattern and the second tree
      of the forest are the same lexeme \(l\). Hence this decision is
      based on a lookahead of one lexeme after the first tree. This
      case corresponds to the rule \textsf{BIND}\(_1\) in
      figure~\ref{es1_match_def}. Note that there is a slight
      optimisation here, since \(l\) is eliminated in the same rule
      (otherwise rule \textsf{ELIM} would always have been used after
      this one).

      \item When the second tree of the forest is not a leaf. This is
      the rule \textsf{BIND}\(_2\) in figure~\ref{es1_match_def}.

    \end{itemize}
    The substitution \(\sigma\) resulting from the recursive call
    (among the premises) must be compatible with the binding \(x
    \mapsto t\), i.e., \(\sigma \subseteq \sigma \oplus x \mapsto t\).

  \item \textbf{Unparsing.} The root of the first tree in the parse
  forest is cut out and the tree is replaced by its direct
  subtrees.
    \begin{itemize}

      \item If the first element of the pattern is a tree pattern,
      i.e., corresponds to a meta\-parenthesis in the original
      unparsed pattern, then this tree pattern is matched against the
      first tree and the remaining pattern is matched against the
      remaining forest. The two resulting substitutions, \(\sigma_1\)
      and \(\sigma_2\), must be compatible, i.e., no binding in one is
      present in the other with the same meta\-variable, unless the
      tree is the same (in short: \(\sigma_1 \subseteq \sigma_1 \oplus
      \sigma_2\)). This is rule \textsf{UNPAR}\(_1\) in
      figure~\ref{es1_match_def}.

      \item Otherwise, the new configuration is rewritten. This is
      rule \textsf{UNPAR}\(_2\) in figure~\ref{es1_match_def}, which
      must be considered last, because its conclusion can overlap with
      the one of \textsf{BIND} rules or \textsf{UNPAR}\(_1\).

    \end{itemize}

\end{itemize}

\begin{figure}[H]
\framebox[\columnwidth]{
\begin{mathpar}
\inferrule*{}{\smj{\el}{\el}{\sigma_\varnothing}}
\;\TirName{\text{\sf END}}

\inferrule
  {\smj{f}{{p}}{\sigma}}
  {\smj{\cons{l}{f}}{\cons{l}{{p}}}{\sigma}}
\,\TirName{\text{\sf ELIM}}

\inferrule*[right=\text{\sf BIND}\(_1\)]
  {\smj{f}{{p}}{\sigma}
   \and
   \sigma \subseteq \sigma \oplus x \mapsto t}
  {\smj{\cons{t, l}{f}}%
      {\cons{\meta{x}, l}{{p}}}%
      {\sigma \oplus x \mapsto t}}

\inferrule*[right=\text{\sf BIND}\(_2\)]
  {\smj{\cons{t_2}{f}}{{p}}{\sigma}
   \and
   \sigma \subseteq \sigma \oplus x \mapsto t_1}
  {\smj{\cons{t_1, t_2}{f}}%
      {\cons{\meta{x}}{{p}}}%
      {\sigma \oplus x \mapsto t_1}}

\inferrule*[right=\text{\sf BIND}\(_3\)]
  {}
  {\smj{[t]}{[\meta{x}]}{\{x \mapsto t\}}}

\inferrule*[right=\text{\sf UNPAR}\(_1\)]
{\sigma_1 \subseteq \sigma_1 \oplus \sigma_2\\
 \smj{f_1}{{p}_1}{\sigma_1}
 \and
 \smj{f_2}{{p}_2}{\sigma_2}}
{\smj{\cons{c(f_1)\!}{\!f_2}}%
    {\cons{\pat{{p}_1}}{\!{p}_2}}%
    {\sigma_1 \oplus \sigma_2}}

\inferrule*[right=\text{\sf UNPAR}\(_2\)]
{\smj{f_1 \cdot f_2}{{p}}{\sigma}}
{\smj{\cons{c(f_1)\!}{\!f_2}}{{p}}{\sigma}}
\end{mathpar}
}
\caption{Pattern matching \textit{ES}(1)
(\(\textsf{UNPAR}_2\) is last)
\label{es1_match_def}}
\end{figure}

\subsection{Example}

Consider the source tree in figure~\ref{fig:src_tree} and the unparsed
pattern \texttt{\%x = \%y - \%z}, as well as the tree of all possible
choices with the backtracking algorithm in
figure~\ref{fig:backtracking_example}. Here, this tree is reduced to a
list because there is no choice between rules (they are ordered). So
we have the trace \textsf{UNPAR}\(_2\), \textsf{BIND}\(_1\),
\textsf{UNPAR}\(_2\), \textsf{BIND}\(_1\) and \textsf{BIND}\(_3\).

\subsection{Properties}

\paragraph{Termination and Worst\hyp{}Case Complexity.} The system
terminates because each rule either strictly decreases the number of
elements in the pattern or the number of nodes in the parse forest. A
successful run happens when a configuration cannot be rewritten
furthermore and it contains an empty pattern and an empty parse
forest. Let us assume that the cost of applying an inference rule is
constant. Then the cost of a run is upper\hyp{}bounded by a constant
times the number of rules in the proof tree (which can be considered
as a trace of the execution). The worst\hyp{}case complexity is thus
obtained by an initial configuration which leads to the maximum number
of rules being used. The observation made about the termination of the
system gives us the clue as how to proceed in finding such
input. Because a meta\-variable cannot be bound to a lexeme (i.e., a
tree leaf), the nodes of the bounded subtree are not considered by the
algorithm. Thus, the worst case contains no meta\-variables, so no
\textsf{BIND} rule is used. Moreover, the size of the pattern has no
impact on the cost, since a meta\-parenthesis or a lexeme are
eliminated at the same time a node or a leaf is removed from the parse
forest (see rules \textsf{ELIM} and \textsf{UNPAR}\(_1\)). The
worst\hyp{}case complexity follows: it is linear in the number of
nodes in the parse forest.

\paragraph{Soundness.}
%%-*-latex-*-

\begin{theorem}[Soundness]\hfill
\label{es1:soundness}
\begin{center}
If \(\smat{h}{p} \twoheadrightarrow \sigma\) 
then \(\subst{p}{\sigma} \sqsubseteq h\).
\end{center}
\end{theorem}

\medskip

\noindent\textsc{Proof~\ref{es1:soundness}} (Soundness). 

\noindent Let \(\ind{U}(p, f, \sigma)\) be the proposition
\begin{center}
\emph{If} \(\smj{f}{p}{\sigma}\)
\emph{then} \(\subst{p}{\sigma} \sqsubseteq f\).
\end{center}
Then \(\ind{U}(p, [h], \sigma)\) is equivalent to the
soundness. Let
\begin{equation}
  \smj{f}{p}{\sigma} \label{es1:sound:head}
\end{equation}
(otherwise the theorem is trivially true). This means that there
exists a pattern\hyp{}matching derivation \(\Delta\) whose conclusion
is \(\smj{f}{p}{\sigma}\). This derivation is a tree; we
hence can reason by structural induction on it, i.e., we assume that
\(\ind{U}\) holds for the premises of the last rule in \(\Delta\)
(this is the \emph{induction hypothesis}) and then prove that
\(\ind{U}\) holds for \(\smj{f}{p}{\sigma}\). We proceed
case by case on the kind of rule that can end \(\Delta\).
\begin{enumerate}

  \item Case where \(\Delta\) ends with \textsf{END}.\\ We have
    \(p = \el\), \(f = \el\) and \(\sigma =
    \sigma_\varnothing\). Therefore \(\subst{p}{\sigma} =
    \subst{\el}{\sigma} \eqn{1} \el \sqsubseteq \el = f\).  Thus
    \(\ind{U}(\el, \el, \sigma_\varnothing)\) holds.

  \item Case where \textsf{ELIM} ends \(\Delta\).
    \begin{mathpar}
      \inferrule*[right=\text{\sf ELIM}]
         {\inferrule*
            {\inferrule*[vdots=1.5em]{}{ }}
            {\smj{f'}{p'}{\sigma}}}
         {\smj{\cons{l}{f'}}{\cons{l}{p'}}{\sigma}}
    \end{mathpar}
    where, since we assumed \eqref{es1:sound:head},
    \begin{enumerate}

      \item \label{es1:sound:24} \(p \triangleq
        \cons{l}{p'}\),

      \item \label{es1:sound:23} \(f \triangleq \cons{l}{f'}\).

    \end{enumerate}
    Let us assume that the induction hypothesis holds for the premise
    of \textsf{ELIM}, that is to say, \(\ind{U}(p', f',
    \sigma)\) holds. Thus
    \begin{gather}
      \subst{p'}{\sigma} \sqsubseteq f'. \label{es1:sound:02}
    \end{gather}
    Besides, we have
    \begin{align}
      \subst{p}{\sigma}
      &= \subst{\cons{l}{p'}}{\sigma}\notag
      & \text{by} \,\; \text{\ref{es1:sound:24}}\\
      &\eqn{3} \cons{l}{\subst{p'}{\sigma}}
      &\text{cf. Fig.~\ref{es1_subst_def}}\notag\\
      &\sqsubseteq \cons{l}{f'}
      &\text{by} \,\; \eqref{es1:sound:02} \,\; \text{and} \,\;
      \textsf{EQ} \, (\text{Fig.~\ref{es1_tree_matching_def}})\notag\\
      &= f
      & \text{by} \,\; \text{\ref{es1:sound:23}}\notag\\
        \subst{p}{\sigma}
      &\sqsubseteq f. \label{es1:sound:09}
    \end{align}
    As a conclusion, the induction hypothesis and
    \eqref{es1:sound:head} imply \eqref{es1:sound:09} in this case,
    i.e., \(\ind{U}(p, f, \sigma)\) holds.

  \item Case where \textsf{BIND}\(_1\) ends \(\Delta\).
    \begin{mathpar}
        \inferrule
          {\inferrule*
            {\inferrule*[vdots=1.5em]{}{ }}
            {\smj{f''}{p''}{\sigma'}}}
          {\smj{\cons{t, l}{\!f''}}%
               {\cons{\meta{x}, l}{p''}}%
               {\sigma' \oplus x \mapsto t}}
      \end{mathpar}
      where, since we assumed \eqref{es1:sound:head},
      \begin{enumerate}

        \item \label{es1:sound:sigma'} \(\sigma' \subseteq \sigma'
          \oplus x \mapsto t\),

        \item \label{bind2_p} \(p \triangleq
          \cons{\meta{x}, l}{p''}\),

        \item \label{es1:sound:1} \(f \triangleq \cons{t, l}{f''}\),


         \item \label{es1:sound:sigma} \(\sigma \triangleq \sigma'
           \oplus x \mapsto t\).

      \end{enumerate}
      Let us assume that the induction hypothesis holds for the
      premise of \textsf{BIND}\(_1\), i.e., 
      \(\ind{U}(p'', f'', \sigma')\) holds:
      \begin{gather}
        \subst{p''}{\sigma'} \sqsubseteq f''. \label{es1:sound:M}
      \end{gather}
      From \ref{es1:sound:sigma'} and lemma \ref{minimality}, we draw
      \begin{align}
           \subst{p''}{\sigma'} 
        &= \subst{p''}{(\sigma' \oplus x \mapsto
             t)}\notag\\
        &= \subst{p''}{\sigma}
        &\text{by} \,\; \text{\ref{es1:sound:sigma}}\notag\\
           \subst{p''}{\sigma}
        &= \subst{p''}{\sigma'}\notag\\
        &\sqsubseteq f''
        &\text{by} \,\; \eqref{es1:sound:M}\notag\\
           \subst{p''}{\sigma}
        &\sqsubseteq f''. \label{es1:sound:00}
      \end{align}
      We have
      \begin{align}
           \sigma(x) 
        &= (\sigma' \oplus x \mapsto t)(x)
        &\text{by} \,\; \text{\ref{es1:sound:sigma}}\notag\\
        &= t
        &\text{by} \,\; \eqref{model:oplus}\notag\\
           \sigma(x) 
        &= t. \label{es1:sound:N}
      \end{align}
      Besides, we have the following equalities:
      \begin{align}
          \subst{p}{\sigma} 
        &= \subst{\cons{\meta{x},l}{p''}}{\sigma}
        &\text{by} \,\; \text{\ref{bind2_p}}\notag\\
        &\eqn{2}
          \cons{\sigma(x)}{\subst{\cons{l}{p''}}{\sigma}}
        &\text{cf. Fig.~\ref{es1_subst_def}}\notag\\
        &\eqn{3}
          \cons{\sigma(x),l}{\subst{p''}{\sigma}}
        &\text{cf. Fig.~\ref{es1_subst_def}}\notag\\
        &= \cons{t,l}{\subst{p''}{\sigma}}
        &\text{by} \,\; \eqref{es1:sound:N}\notag\\
          \subst{p}{\sigma} 
        &=
          \cons{t,l}{\subst{p''}{\sigma}}. \label{es1:sound:13}
      \end{align}
      Closed\hyp{}tree matching \eqref{es1:sound:00} and the derivation
      \begin{mathpar}
        \inferrule*[right=\text{\sf{EQ}}]
          {t \in {\cal H}
           \and
           \inferrule*[right=\text{\sf{EQ}}]
              {l \in {\cal H}
               \and
               \subst{p''}{\sigma} \sqsubseteq f''}
              {\cons{l}{\subst{p''}{\sigma}}
               \sqsubseteq \cons{l}{f''}}}
          {\cons{t, l}{\subst{p''}{\sigma}}
           \sqsubseteq \cons{t, l}{f''}}
      \end{mathpar}
      imply
      \begin{align}
           \cons{t, l}{\subst{p''}{\sigma}} 
        &\sqsubseteq
           \cons{t, l}{f''}\notag\\
           \subst{p}{\sigma}
        &\sqsubseteq
           \cons{t, l}{f''}
        &\text{by} \,\; \eqref{es1:sound:13}\notag\\
        &= f
        &\text{by} \,\; \text{\ref{es1:sound:1}}\notag\\
           \subst{p}{\sigma}
        &\sqsubseteq f. \label{es1:sound:O}
      \end{align}
      As a conclusion, the induction hypothesis and
      \eqref{es1:sound:head} imply \eqref{es1:sound:O} in this case,
      i.e., \(\ind{U}(\cons{\meta{x}}{p'}, f, \sigma)\).

  \item Case where \textsf{BIND}\(_2\) ends \(\Delta\).
    \begin{mathpar}
      \inferrule
        {\inferrule*
           {\inferrule*[vdots=1.5em]{}{ }}
           {\smj{\cons{t_2}{f'}}{p'}{\sigma'}}}
        {\smj{\cons{t_1, t_2}{f'}}%
           {\cons{\meta{x}}{p'}}%
           {\sigma' \oplus x \mapsto t_1}}
    \end{mathpar}
    where, since we assumed \eqref{es1:sound:head}, 
    \begin{enumerate}

      \item \label{bind1_def_p} \(p \triangleq
        \cons{\meta{x}}{p'}\),

      \item \label{bind1_def_f} \(f \triangleq \cons{t_1, t_2}{f'}\),

      \item \label{es1:sound:BIND2:sigma} \(\sigma \triangleq \sigma'
        \oplus x \mapsto t_1\),

      \item \label{bind1_def_x} \(\sigma' \subseteq \sigma' \oplus x
        \mapsto t_1\).

    \end{enumerate}
    Let us assume that the induction hypothesis holds for the
    premise of \textsf{BIND}\(_2\), i.e.,
    \(\ind{U}(p', \cons{t_2}{f'}, \sigma')\):
    \begin{gather}
      \subst{p'}{\sigma'} \sqsubseteq \cons{t_2}{f'}. \label{es1:sound:05}
    \end{gather}
    From \ref{bind1_def_x} and lemma \ref{minimality}, we draw
    \begin{align}
         \subst{p'}{\sigma'}
      &= \subst{p'}{(\sigma' \oplus x \mapsto t_1)}\notag\\
      &= \subst{p'}{\sigma}
      &\text{by} \,\;
         \text{\ref{es1:sound:BIND2:sigma}}\notag\\
         \subst{p'}{\sigma}
      &= \subst{p'}{\sigma'}\notag\\
      &\sqsubseteq \cons{t_2}{f'}
      &\text{by} \,\; \eqref{es1:sound:05}\notag\\
         \subst{p'}{\sigma}
      &\sqsubseteq \cons{t_2}{f'}. \label{es1:sound:19}
    \end{align}
    We have
    \begin{align}
          \sigma(x) 
      &= (\sigma' \oplus x \mapsto t_1)(x)
      &\text{by} \,\;
       \text{\ref{es1:sound:BIND2:sigma}}\notag\\
      &= t_1
      &\text{by} \,\; \eqref{model:oplus}\notag\\
         \sigma(x) 
      &= t_1. \label{es1:sound:P}
    \end{align}
    Furthermore,
    \begin{align}
         \subst{p}{\sigma} 
      &= \subst{\cons{\meta{x}}{p'}}{\sigma}
      &\text{by} \,\; \text{\ref{bind1_def_p}}\notag\\
      &\eqn{2} \cons{\sigma(x)}{\subst{p'}{\sigma}}
      &\text{cf. Fig.~\ref{es1_subst_def}}\notag\\
      &= \cons{t_1}{\subst{p'}{\sigma}}
      &\text{by} \,\; \eqref{es1:sound:P}\notag\\
      &\sqsubseteq \cons{t_1, t_2}{f'}
      &\text{by} \,\; \eqref{es1:sound:19}, \textsf{EQ}
       \, (\text{Fig.~\ref{es1_tree_matching_def}})\notag\\
      &=f 
      &\text{by} \,\; \text{\ref{bind1_def_f}}\notag\\
         \subst{p}{\sigma}
      &\sqsubseteq f. \label{es1:sound:Q}
    \end{align}
    As a conclusion, the induction hypothesis and
    \eqref{es1:sound:head} imply \eqref{es1:sound:Q} in this case,
    i.e., \(\ind{U}(\cons{\meta{x}}{p'}, f, \sigma)\).

  \item Case where \textsf{BIND}\(_3\) ends \(\Delta\).
    \begin{mathpar}
      \inferrule*[right=\;\text{\sf BIND}\(_3\)]
         {}
         {\smj{[t]}{[\meta{x}]}{\{x \mapsto t\}}}
    \end{mathpar}
    where, since we assumed \eqref{es1:sound:head},
    \begin{enumerate}

      \item \label{es1:sound:15} \(p \triangleq [\meta{x}]\),

      \item \label{es1:sound:16} \(f \triangleq [t]\),

      \item \label{es1:sound:14} \(\sigma \triangleq \{x \mapsto t\}\).

    \end{enumerate}
    Because \textsf{BIND}\(_3\) is an axiom, we must prove 
    \(\ind{U}([\meta{x}], [t], \{x \mapsto t\})\) without relying on
    the induction principle:
    \begin{align}
         \subst{p}{\sigma}
      &= \subst{[\meta{x}]}{\sigma}
      & \text{by} \,\; \text{\ref{es1:sound:15}}\notag\\
      &\eqn{2} \cons{\sigma(x)}{\subst{\el}{\sigma}}
      &\text{cf. Fig.~\ref{es1_subst_def}}\notag\\
      &\eqn{1} \cons{\sigma(x)}{\el}
      &\text{cf. Fig.~\ref{es1_subst_def}}\notag\\
      &\triangleq [\sigma(x)]\notag\\
      &= [t]
      & \text{by} \,\; \ref{es1:sound:14} \,\; \text{and} \,\;
        \eqref{model:oplus}\notag\\
        \subst{p}{\sigma}
      &= [t]. \label{es1:sound:R}
    \end{align}
    Since we also have the derivation
    \begin{mathpar}
      \inferrule*[right=\text{\sf EQ}]
        {\inferrule*[right=\text{\sf EMP}]
          { }
          {\el \sqsubseteq \el}}
        {\cons{t}{\el} \sqsubseteq \cons{t}{\el}}
    \end{mathpar}
    we know that \([t] \sqsubseteq [t]\), which, in conjunction with
    \eqref{es1:sound:R}, implies
    \begin{align}
        \subst{p}{\sigma}
      &\sqsubseteq [t]\notag\\
      &= f
      & \text{by} \,\; \text{\ref{es1:sound:16}}\notag\\
        \subst{p}{\sigma}
      &\sqsubseteq f. \label{es1:sound:17}
    \end{align}
    As a conclusion, the induction hypothesis and
    \eqref{es1:sound:head} imply \eqref{es1:sound:17}, i.e.,
    \(\ind{U}([\meta{x}], [t], \{x \mapsto t\})\) holds.

  \item Case where \(\Delta\) ends with \textsf{UNPAR}\(_1\).
    \begin{mathpar}
      \inferrule*[right=\text{\sf UNPAR}\(_1\)]
        {\inferrule*
          {\inferrule*[vdots=1.5em]{}{(\Delta_1)}}
          {\smj{f_1}{p_1}{\sigma_1}}
         \and
         \inferrule*
          {\inferrule*[vdots=1.5em]{}{(\Delta_2)}}
          {\smj{f_2}{p_2}{\sigma_2}}}
        {\smj{\cons{c(f_1)\!}{\!f_2}}%
            {\cons{\pat{p_1}}{\!p_2}}%
            {\sigma_1 \oplus \sigma_2}
        }
    \end{mathpar}
    where, since we assumed \eqref{es1:sound:head},
    \begin{enumerate}
      
      \item \label{es1:sound:22} \(p \triangleq
        \cons{\pat{p_1}}{p_2}\),

      \item \label{es1:sound:21} \(f \triangleq \cons{c(f_1)}{f_2}\),

      \item \label{es1:sound:26} \(\sigma_1 \subseteq \sigma_1 \oplus
        \sigma_2\).
          
    \end{enumerate}
    The derivations \(\Delta_1\) and \(\Delta_2\) are
    sub\hyp{}derivations of \(\Delta\), therefore the induction
    hypothesis holds for their conclusions, i.e.,
    \(\ind{U}(p_1, f_1, \sigma_1)\) is true:
    \begin{gather}
      \subst{p_1}{\sigma_1} \sqsubseteq f_1 \label{es1:sound:01}
    \end{gather}
    and \(\ind{U}(p_2, f_2, \sigma_2)\) is true as well:
    \begin{gather}
      \subst{p_2}{\sigma_2} \sqsubseteq f_2. \label{es1:sound:F}
    \end{gather}
    Directly from the definition \eqref{model:oplus:ext}, it comes
    \begin{gather}
      \sigma_2 \subseteq \sigma_1 \oplus \sigma_2. \label{es1:sound:27}
    \end{gather}
    It follows
    \begin{align}
       \subst{p_1}{\sigma_1}
      &= \subst{p_1}{(\sigma_1 \oplus \sigma_2)}
      &\text{by} \,\; \text{\ref{es1:sound:26}} \,\; \text{and} \,\;
        \text{\ref{minimality}}\label{es1:sound:28}\\
        \subst{p_2}{\sigma_2}
      &= \subst{p_2}{(\sigma_1 \oplus \sigma_2)}
      &\text{by} \,\; \eqref{es1:sound:27} \,\; \text{and} \,\;
        \text{\ref{minimality}}.\label{es1:sound:29}
    \end{align}
    Let \(\sigma \triangleq \sigma_1 \oplus \sigma_2\). Besides, we
    have
    \begin{align}
         \subst{p}{\sigma}
      &= \subst{\cons{\pat{p_1}}{p_2}}{\sigma}
      &\text{by} \,\; \text{\ref{es1:sound:22}}\notag\\
      &\eqn{4}
       \cons{\pat{\subst{p_1}{\sigma}}}%
            {\subst{p_2}{\sigma}}
      &\text{cf. Fig.~\ref{es1_subst_def}}\notag\\
      &= \cons{\pat{\subst{p_1}{\sigma_1}}}%
              {\subst{p_2}{\sigma}}
      &\text{by} \,\; \eqref{es1:sound:28}\notag\\
      &= \cons{\pat{\subst{p_1}{\sigma_1}}}%
              {\subst{p_2}{\sigma_2}}
      &\text{by} \,\; \eqref{es1:sound:29}\notag\\
      &\sqsubseteq \cons{c(f_1)}{f_2}
      &\text{by} \,\; \eqref{es1:sound:01}, \eqref{es1:sound:F},
       \textsf{PAT}\notag\\
      &= f
      &\text{by} \,\; \text{\ref{es1:sound:21}}\notag\\
       \subst{p}{\sigma}
      &\sqsubseteq f. \label{es1:sound:G}
    \end{align}
    As a conclusion, the induction hypothesis and
    \eqref{es1:sound:head} imply \eqref{es1:sound:G},
    i.e., \(\ind{U}(\cons{\pat{p_1}}{p_2}, f,
    \sigma)\) holds.

  \item \label{case_l_UNPAR2} Case where \textsf{UNPAR}\(_2\) ends
    \(\Delta\).
    \begin{mathpar}
      \inferrule*[right=\text{\sf UNPAR}\(_2\)]
         {\inferrule*
            {\inferrule*[vdots=1.5em]{}{ }}
            {\smj{f_1 \cdot f_2}{p}{\sigma}}}
         {\smj{\cons{c(f_2)\!}{\!f_2}}{p}{\sigma}}
    \end{mathpar}
    where, since we assumed \eqref{es1:sound:head},
    \begin{enumerate}

      \item \label{es1:sound:25} \(f \triangleq \cons{c(f_1)}{f_2}\).

    \end{enumerate}
    Let us assume that the induction hypothesis holds for the premise
    of \textsf{UNPAR}\(_2\), i.e., \(\ind{U}(p, f_1 \cdot
    f_2, \sigma)\) is true. Therefore
    \begin{align}
         \subst{p}{\sigma} 
       &\sqsubseteq f_1 \cdot f_2\notag\\
       &\sqsubseteq \cons{c(f_1)}{f_2}
       &\text{by} \,\; \textsf{SUB} \,
       (\text{Fig.~\ref{es1_tree_matching_def}})\notag\\
       &= f
       &\text{by} \,\; \text{\ref{es1:sound:25}}\notag\\
         \subst{p}{\sigma}
       & \sqsubseteq f. \label{es1:sound:I}
    \end{align}
    As a conclusion, the induction hypothesis and
    \eqref{es1:sound:head} imply \eqref{es1:sound:I} in this case,
    i.e., \(\ind{U}(p, f, \sigma)\). (Note that the
    structure of the pattern \(p\) is irrelevant
    here.)\hfill \(\Box\)

\end{enumerate}


\paragraph{Completeness.} This algorithm is not complete in the sense
the backtracking algorithm was in
section~\ref{backtracking:completeness}. Consider the unparsed pattern
\texttt{\%x = \%y - \%z - \%t} and the same parse tree in
figure~\ref{fig:src_tree}. The execution trace is
\textsf{UNPAR}\(_2\), \textsf{BIND}\(_1\), \textsf{UNPAR}\(_2\),
\textsf{BIND}\(_1\) and then failure, that is, no rule apply. But
there exists a successful closed\hyp{}tree matching if the
substitution \(\{\textsf{x} \mapsto \text{var}(\texttt{a}), \textsf{y}
\mapsto \text{var}(\texttt{a}), \textsf{z} \mapsto
\text{mul}(\text{var}(\texttt{b}), \texttt{*},
\text{var}(\texttt{c})), \textsf{t} \mapsto \text{var}(\texttt{d})\}\)
is applied to the pattern first. Therefore, a match failure with
\textit{ES}(1) can either mean that there is no matching or that one
exists but was not found. In the latter case, meta\-parentheses must
be added to the pattern in order to force an unparsing step instead of
a binding.
\begin{figure}[b]
\framebox[\columnwidth]{%
  \begin{tabular}{ccc}
    \includegraphics[bb=71 669 109 721]{meta_x} &
    \includegraphics[bb=71 673  78 721]{eq} &
    \includegraphics[bb=71 671 180 721]{metaparsed}
  \end{tabular}
}
\caption{\texttt{\%x = \%(\%(\%y - \%z\%) - \%t\%)}\label{fig:meta}}
\end{figure}
In the previous example, the successful substitution is found by
\textit{ES}(1) if the unparsed pattern is transformed into \texttt{\%x
= \%(\%(\%y - \%z\%) - \%t\%)}. It is meta\-parsed into the pattern
shown in figure~\ref{fig:meta}. The proof tree of the matching of this
meta\-parsed pattern against the parse tree is given in
figure~\ref{fig:trace_es1}. In figure~\ref{fig:backtracking_example},
it is compulsory that (exactly) one branch from the root leads to an
axiom (each branch is a logical disjunction, that is, algorithmically,
a backtracking point), whereas here all branches must lead to an axiom
(each branch is a logical conjunction of premises).
\begin{figure}[t]
\framebox[\columnwidth]{
  \includegraphics[bb=71 605 156 721]{proof_es1}
}
\caption{A simplified proof tree in \textit{ES}(1)\label{fig:trace_es1}}
\end{figure}

Of course, as a worst\hyp{}case, if the pattern is fully
meta\-parenthesised \emph{with respect to the grammar}, \textit{ES}(1)
becomes complete in the sense above, e.g., \texttt{\%(\%x =
\%(\%(\%(\%y\%) - \%z\%) - \%t\%)\%)}. Indeed, such parenthesising
leads to a meta\-parsed pattern\hyp{}tree which is isomorphic to a
tree pattern, as in classic pattern matching. In other words, there is
always a way to meta\-parenthesise an unparsed pattern to make it as
expressive as the classic, tree\hyp{}based, pattern
matching. Moreover, in practice, only a few meta\-parentheses may be
needed for a given pattern, as shown above.

By looking closely at the rewrite rule, we can figure out necessary
conditions for an unparsed pattern to lead to a loss of completeness
in matching. These conditions can serve as empirical guidelines to
prevent the problem from appearing. As mentionned above, the problem
occurs when an instance of rule \textsf{BIND}\(_1\) or
\textsf{BIND}\(_2\) is used and later leads to a failure. This failure
could have been avoided, had \textsf{UNPAR}\(_2\) been selected,
perhaps several times, followed by \textsf{BIND}\(_1\) or
\textsf{BIND}\(_2\). Assuming that it is rule \textsf{BIND}\(_1\) that
should be delayed after some \textsf{UNPAR}\(_2\) steps, it means that
the lexeme \(l\) is repeated in the parse forest and occurs each time
after a tree (not a lexeme). Left\hyp{}associative operators in
programming languages usually occur in such kind of grammatical
constructs. This is why the previous pattern had to be written
\texttt{\%x = \%(\%y - \%z\%) - \%t\%)}.

\begin{figure}[H]
\framebox[\columnwidth]{%
  \includegraphics{const_static}
}%
\caption{A simplified parse tree (no empty words)\label{fig:const_static}}
\end{figure}

Let us consider now that it is rule \textsf{BIND}\(_2\) that should
have been delayed. This leads unparsed patterns as ``\texttt{\%q \%t
\%x ;}'' to fail to match the parse tree given in
figure~\ref{fig:const_static}. An unparsing step should be taken
instead of a premature binding, so that \texttt{\%q} matches the
qualifications and \texttt{\%t} the type. The way out of this is to
use the unparsed pattern \texttt{\%(\%(\%q \%t\%) \%x;\%)}
instead. This situation, where two subtrees are not separated by a
lexeme, is rare in practice.
