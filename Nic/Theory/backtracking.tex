%%-*-latex-*-

\begin{figure}[b]
\framebox[\columnwidth]{
  \begin{tabular}{cc}
    \subfloat[Tree\hyp{}like pattern\label{fig:pat_tree}]{
      \includegraphics[bb=71 601 147 721]{pat_tree_ex}
    }
    &
    \subfloat[Source parse tree\label{fig:src_tree}]{
      \includegraphics[bb=71 601 156 721]{match_ex}
    }
  \end{tabular}
}
\caption{Tree pattern matching \texttt{a = a - b*c - d}}
\label{fig:match_ex}
\end{figure}


\section{A Backtracking Algorithm}
\label{backtracking}

Before presenting a precise algorithm for unparsed\hyp{}pattern
matching, let us discuss informally an example. Consider the problem
of matching the pattern \texttt{\%x = \%y - \%z} against the \Clang
expression \texttt{a = a - b * c - d}. Pattern\hyp{}tree matching
would first parse the expression into the parse tree in
figure~\ref{fig:src_tree}, then parse the pattern using an extended
parser into the parse tree in figure~\ref{fig:pat_tree}, and then
match the latter against the former. As a result, all the
meta\-variables are correctly bound with respect to the grammar in the
substitution \(\{\textsf{x} \mapsto \texttt{"a"}, \textsf{y} \mapsto
\texttt{"a-b*c"}, \textsf{z} \mapsto \texttt{"d"}\}\).

The key idea of unparsed patterns is to avoid parsing the pattern by
going the other way around, i.e., by unparsing the source parse tree
and comparing the result with a textual pattern. However, if the parse
tree is simply unparsed into a string, matching would fall back to the
case of matching between two strings, which is very imprecise, because
it would yield both the substitution \(\{\textsf{x} \mapsto
\texttt{"a"}, \textsf{y} \mapsto \texttt{"a-b*c"}, \textsf{z} \mapsto
\texttt{"d"}\}\), which is correct, and \(\{\textsf{x} \mapsto
\texttt{"a"}, \textsf{y} \mapsto \texttt{"a"}, \textsf{z} \mapsto
\texttt{"b*c-d"}\}\), which is incorrect, since the subtraction
operator is left\hyp{}associative.

Moreover, meta\-variables are bound to strings (i.e., concrete
syntax), rather than being bound to subtrees of the parse tree. This
is not suitable when using pattern matching to manipulate the matched
subtrees, which is a quite common case within parsing\hyp{}based
tools.

The technical issue is that the whole parse tree is fully unparsed
(i.e., de\-structured) at once, dropping the references to all the
subtrees. In order to avoid that, the parse tree should be unparsed
level by level (in a breadth\hyp{}first traversal), and the unparsed
pattern (which is a list of lexemes and meta\-variables here) should
either be partially matched against the current unparsed forest or the
latter should be further unparsed.

These two alternatives are sometimes possible on the same
configuration. The first one, i.e., partial matching, can be tried
first and if it leads to a failure, the second one, i.e., unparsing,
is tried instead. If both options lead to a failure, then the whole
matching is deemed a failure. This technique is called
\emph{backtracking} and does not lead to a linear\hyp{}time algorithm
in the worst\hyp{}case (in the size of the pattern plus the size of
the source tree). Also, the order in which matching or unparsing are
tried is not significant as there is no way to guess which would me
more likely to be successful \emph{a priori}.

More precisely, firstly, the source parse tree is pushed on an empty
analysis stack. This stack is, in general, a parse forest. We shall
speak of the ``left of the forest'' instead of the ``top of the
stack.'' Secondly, the textual pattern, here the string \texttt{\%x =
\%y - \%z}, is transformed into a list of lexemes and
meta\-variables.\footnote{In practice, this is not necessary because
the leaves of the source parse tree contain all the lexical
information needed to realise a match step (read further). But, as
much as the theory of unparsed patterns is concerned, it is simpler to
assume some tokenisation of the pattern, hence avoiding the use of
regular expressions in the formal model or the algorithm itself.}
Thirdly, given an initial empty substitution \(\sigma\), the algorithm
non\hyp{}deterministically chooses one of the two following actions,
and backtrack in case of failure.
\begin{enumerate}

  \item \textbf{Matching.} The first element \(e\) of the pattern is
    matched against the leftmost tree \(h\) of the forest. This can be
    achieved in two different situations:
    \begin{enumerate}
    
      \item \textbf{Elimination.} If \(h\) and \(e\) are the same
        lexeme, then the remaining pattern is matched against the
        remaining forest, with the same substitution \(\sigma\).

      \item \textbf{Binding.} If \(h\) is not a lexeme (i.e., it is
        not a leaf) and \(e\) is the meta\-variable \(x\), which is
        either already bound to a subtree equal to
        \(h\)\footnote{Unparsed patterns are not linear, i.e., a
        meta\-variable can occur more than once.} or unbound in
        \(\sigma\), then the remaining pattern is matched against the
        remaining forest, with \(\sigma\) updated with \(x\) bound to
        \(h\).

  \end{enumerate}

  \item \textbf{Unparsing.} If the forest starts with a tree \(t\),
    unparsing consists in replacing \(t\) by the forest of its direct
    subtrees (in other words, the root of \(t\) is cut out) and trying
    again with the same pattern and the same substitution.

\end{enumerate}
The algorithm always stops because either the pattern length or the
forest size strictly decreases at each step. If the final pattern is
empty, then it is a success; otherwise it is a failure. In case of
success, the final substitution is the result (it contains all the
bindings of the meta\-variables to subtrees of the source parse tree).


\subsection{Definitions}
\label{backtracking:definitions}

\paragraph{Parse Trees and Unparsed Patterns.}

Trees have been formally defined in section~\ref{model}. Unparsed
patterns will be noted \(\overline{p}\), and the set of unparsed
patterns is inductively defined as the smallest set \(\overline{\cal
  P}\) such that
\begin{itemize}

  \item \(\el \in \overline{\cal P}\);

  \item if \(l \in {\cal L}\) and \(\overline{p} \in \overline{\cal
    P}\), then \(\cons{l}{\overline{p}} \in \overline{\cal P}\);

  \item if \(x \in {\cal V}\) and \(\overline{p} \in \overline{\cal
    P}\), then \(\cons{\meta{x}}{\overline{p}} \in \overline{\cal P}\).

\end{itemize}

\paragraph{Substitutions and Closed Patterns.} 

Let us extend the substitutions defined in section~\ref{model}, in
order to cope with unparsed patterns, not just meta\-variables. The
effect of a substitution on a pattern will be to replace every
occurrence in the pattern of the meta\-variables in its domain by the
corresponding parse trees. The substitutions computed by any of our
matching algorithms are total, i.e., they replace \emph{all} the
meta\-variables of the pattern. It is handy to distinguish the forests
which contain no meta\-variables by calling them \emph{closed forests}
or \emph{closed patterns}, and their contents \emph{closed trees}. In
order to distinguish a substitution applied to a meta\-variable \(x\)
from a substitution on an unparsed pattern \(\overline{p}\), we shall
note \(\sigma(x)\) the former and \(\subst{\overline{p}}{\sigma}\) the
latter. Consider the formal definition of substitutions in
figure~\ref{x_subst_def}.  The first equation \((\eqn{1})\) means that
the substitution on the empty pattern is always the empty forest. The
second equation \((\eqn{2})\) defines the substitution of a
meta\-variable by its associated tree: the tree is added to the left
of the resulting forest and the substitution proceeds recursively over
the remaining unparsed pattern. The third equation \((\eqn{3})\)
specifies that the substitutions always leave lexemes unchanged.

\begin{figure}[t]
\framebox[\columnwidth]{%
\(
\begin{aligned}
\subst{\el}{\sigma} 
&\eqn{1} \el\\
\subst{\cons{\meta{x}}{\overline{p}}}{\sigma}
&\eqn{2} \cons{\sigma(x)}{\subst{\overline{p}}{\sigma}}\\
\subst{\cons{l}{\overline{p}}}{\sigma}
&\eqn{3} \cons{l}{\subst{\overline{p}}{\sigma}} 
\end{aligned}
\)
}
\caption{Substitutions on unparsed patterns\label{x_subst_def}}
\end{figure}

\begin{figure}[H]
\framebox[\columnwidth]{%
\begin{mathpar}
\inferrule{}{\smj{\el}{\el}{\sigma_\varnothing}}
\quad\TirName{\text{\sf END}}

\inferrule
{\smj{f}{\overline{p}}{\sigma}}
{\smj{\cons{l}{f}}{\cons{l}{\overline{p}}}{\sigma}}
\,\TirName{\text{\sf ELIM}}

\inferrule*[right=\text{\sf BIND}]
{\smj{f}{\overline{p}}{\sigma}\\
 \sigma \subseteq \sigma \oplus x \mapsto t}
{\smj{\cons{t}{f}}%
     {\cons{\meta{x}}{\overline{p}}}%
     {\sigma \oplus x \mapsto t}}

\inferrule*[right=\text{\sf UNPAR}]
  {\smj{f_1 \cdot f_2}{\overline{p}}{\sigma}}
  {\smj{\cons{c(f_1)}{f_2}}{\overline{p}}{\sigma}}
\end{mathpar}
}
\caption{A backtracking pattern matching\label{x_match_def}}
\end{figure}

\paragraph{Pattern Matching.}

Pattern matching is defined by the inference system given in
figure~\ref{x_match_def}, where the rules are unordered. Let us call a
\emph{configuration} the pair \(\smat{f}{\overline{p}}\). In case the
forest contains only one tree \(h\), let us write
\(\smat{h}{\overline{p}}\) instead of
\(\smat{[h]}{\overline{p}}\). The pattern matching associates a
configuration to a substitution. This system of inference rules is not
syntax\hyp{}directed, because the conclusions of rules \textsf{BIND}
and \textsf{UNPAR} overlap: a non\hyp{}deterministic choice between
binding and unparsing must be done. This dilemma cannot be decided
solely based on the shape of the configuration and thus the
implementation must rely on a backtracking mechanism, as we said
before. Note that no rule has more than one premise involving the
\((\twoheadrightarrow)\) relation, hence the proof trees (i.e.,
derivations, when read deductively) are actually lists. Rule
\textsf{END} rewrites the empty configuration to the empty
substitution; this happens as the last rewrite step ---~from whence
its name. Let us read the rules inductively, since this reading
corresponds to an algorithm.

Rule \textsf{ELIM}: if the pattern and the tree start with the same
lexeme, then remove the lexemes and try to rewrite the remaining
configuration.

Rule \textsf{BIND}: a meta\-variable \(x\) is bound to a tree \(t\),
i.e., \(x \mapsto t\), if the remaining configuration rewrites to a
substitution \(\sigma\) which either already contains the binding or
whose domain does not contain \(x\) (i.e., \(\sigma \subseteq \sigma
\oplus x \mapsto t\)); the resulting substitution is the updating of
\(\sigma\) with the new binding.

Rule \textsf{UNPAR}: to match the same pattern against the same tree
\(t = c(f_1)\) whose root has been cut off (i.e., \(f_1\) remains);
this is \emph{unparsing}. Note that the configuration
\(\smat{\cons{c(f_1)}{f_2}}{\cons{\meta{x}}{\overline{p}}}\) can lead
both to an unparsing or a binding.

\subsection{Example}

Consider the source tree in figure~\ref{fig:src_tree} and the unparsed
pattern \texttt{\%x = \%y - \%z}. The tree of all possible choices for
the rules of the backtracking system in figure~\ref{x_match_def} is
given in figure~\ref{fig:backtracking_example}. The unique successful
trace is in bold.
\begin{figure}[H]
\framebox[\columnwidth]{
  \includegraphics[bb=71 523 221 721]{backtracking_example}
}
\caption{Matching \texttt{\%x = \%y - \%z} (in bold)}
\label{fig:backtracking_example}
\end{figure}


\subsection{Properties}

Some important questions raised by this formal definition are: Is it
an algorithm? (i.e., does it terminate?) With respect to what is it
sound and/or complete?

\paragraph{Termination.} 

As said before, termination is entailed by the observation that either
the pattern length (i.e., the number of lexemes and meta\-lexemes), or
the forest size (i.e., the number of nodes) strictly decreases after
any rule is applied to any given configuration.

%% \paragraph{Determinacy.}

%% It is worth wondering whether the inference system used for defining
%% the pattern matching is a function, i.e., if it computes, for a given
%% configuration, a unique substitution (in case of success) or none (in
%% case of failure). This property is called \emph{determinacy} and the
%% concern arises whenever the system is not syntax\hyp{}directed. Here,
%% it is possible to prove that our system is \emph{not} deterministic by
%% means of an example.

\begin{figure}[H]
\framebox[\columnwidth]{%
  \includegraphics[scale=0.5]{sqsubseteq}
}
\caption{Closed\hyp{}tree matching \([a,b,c] \sqsubseteq h\)}
\label{fig:sqsubseteq}
\end{figure}

\paragraph{Closed\hyp{}Tree Matching.}

Soundness and completeness are properties which suppose another level
of specification, usually more abstract, or at the same conceptual
level. In any case, this other description should provide a concept of
matching which is more `natural' or obvious than the one under
consideration. Thus, by proving that the latter is sound with respect
to the former, it is proved that it is included in it. In other words,
when an algorithm is proved sound with respect to a specification, it
means that all outputs satisfy the specification ---~thus are correct.

The intuitive notion we need here is what it means for a closed tree
to match a parse tree, i.e., when a parse tree (thinking of a pattern
in which all meta\-variables have been substituted) is contained in
another parse tree (the source program). This is the
\emph{closed\hyp{}tree matching}, which is a special case of the
classic tree matching (in the latter, the pattern tree may embed
meta\-variables and is called a pattern). This way, it becomes
possible to compare the expressive power of the backtracking pattern
matching with respect to the more familiar tree matching, used by many
existing tools.

Informally, let us say that a tree \(h_1\) matches a tree \(h_2\) if
and only if \(h_1\) is included somewhere at the bottom of \(h_2\), as
shown in figure~\ref{fig:sqsubseteq}. Let us note \(h_1 \sqsubseteq
h_2\) the relation `The tree \(h_1\) matches the tree
\(h_2\).'\footnote{The symbol \(\sqsubseteq\) hints at the fact that
this relation is a special case of tree inclusion, where the leaves of
the trees must coincide.} Technically, we only need to define
\((\sqsubseteq)\) between a forest \(f\) and a tree \(h\) in such a
manner that \(f \sqsubseteq h\) implies that there exists a
constructor \(c\) such that \(c(f) \sqsubseteq h\). But we shall not
precise this further.

The formal definition we propose here is based on partially ordered
inference rules (see section~\ref{model}), displayed in
figure~\ref{x_tree_matching_def}. We give now a logical (or deductive)
reading of the rules.

\begin{figure}[H]
\framebox[\columnwidth]{
\begin{mathpar}
\inferrule*{}{\el \sqsubseteq \el}
\quad\TirName{\text{\sf EMP}}

\inferrule
  {f_1 \sqsubseteq f_2}
  {\cons{h}{f_1} \sqsubseteq \cons{h}{f_2}}
\,\TirName{\text{\sf EQ}}

\inferrule*[right=\text{\sf SUB}]
  {f \sqsubseteq f_1 \cdot f_2}
  {f \sqsubseteq \cons{c(f_1)}{f_2}}

\inferrule*[right=\text{\sf ONE}]
  {f \sqsubseteq [h]}
  {f \sqsubseteq h}
\end{mathpar}
}
\caption{Closed\hyp{}forest matching\label{x_tree_matching_def}}
\end{figure}

Rule \textsf{ONE} states that if the forest \(f\) matches a forest
made of a single tree \(h\), then it matches \(h\). (This allows to
relate a closed forest and a single tree.)

Axiom \textsf{EMP} says that the empty forest always matches the empty
forest.

Rule \textsf{EQ} specifies that if a non\hyp{}empty forest \(f_1\)
matches another non\hyp{}empty forest (possibly the same), then the
forest \(\cons{h}{f_1}\) matches the forest \(\cons{h}{f_2}\), where
\(h\) is a tree (possibly a lexeme). 

Rule \textsf{SUB} states that if a forest \(f\) matches the catenation
of forests \(f_1\) and \(f_2\), such that the constructor \(c\) can
have \(f_1\) as direct subtrees, then \(f\) matches
\(\cons{c(f_1)}{f_2}\) (i.e., grouping some trees into a new tree
changes nothing). When reading the rules inductively, i.e.,
bottom\hyp{}up, or, in other words, algorithmically, we must add the
constraint that rule \textsf{SUB} must always be considered last,
i.e., if a derivation (i.e., a proof tree) ends with \textsf{SUB}, its
conclusion has the shape \(f \sqsubseteq \cons{c(f_1)}{f_2}\)
\emph{and it is implied that there is no forest} \(f'\) \emph{such
that} \(f = \cons{c(f_1)}{f'}\) (which would be the conclusion of
\textsf{EQ}).
\begin{remark}
There are other ways to define closed\hyp{}tree matching, such as
asserting the existence of a certain morphism between two trees, but
our approach is more direct. It does not matter that it is very
inefficient (for instance, consider that, in rule \textsf{EQ}, the
equality of two trees must be checked), since it does not serve the
purpose of an implementation but of a reference for the backtracking
algorithm, in order to allow the definition of its soundness and
completeness.
\end{remark}

\paragraph{Soundness.}

The soundness of the pattern matching means that all computed
substitutions, once applied to the original pattern, yield a closed
forest which matches the original tree (i.e., is included in it in the
sense above). Informally: all successful pattern matchings lead to
successful closed\hyp{}tree matchings. Formally:
\begin{theorem}[Soundness]\hfill
\label{backtracking:soundness}
\begin{center}
If \(\smat{h}{\overline{p}} \twoheadrightarrow \sigma\),
then \(\subst{\overline{p}}{\sigma} \sqsubseteq h\).
\end{center}
\end{theorem}

\medskip

%%-*-latex-*-

\noindent\textsc{Proof~\ref{backtracking:soundness}} (Soundness).\\
\noindent Let \(\ind{R}(\overline{p}, f, \sigma)\) be the proposition
`\emph{If} \(\smj{f}{\overline{p}}{\sigma}\),
\emph{then} \(\subst{\overline{p}}{\sigma} \sqsubseteq f\).'
Then \(\ind{R}(p, [h], \sigma)\) is equivalent to the soundness
property (by rule \textsf{ONE}). Firstly, let us assume that
\begin{equation}
  \smj{f}{\overline{p}}{\sigma} \label{x:sound:head}
\end{equation}
is true (otherwise the theorem is trivially true). This means that
there exists a pattern matching derivation \(\Delta\) whose conclusion
is \(\smj{f}{\overline{p}}{\sigma}\). This derivation is a list, which
makes it possible to reckon by induction on its structure, i.e., one
assumes that \(\ind{R}\) holds for the premise of the last rule in
\(\Delta\) (this is the \emph{induction hypothesis}) and then proves
that \(\ind{R}\) holds for \(\smj{f}{\overline{p}}{\sigma}\). A case
by case analysis on the kind of rule that can end the derivation
guides the proof.
\begin{enumerate}

   \item Case where \(\Delta\) ends with \textsf{END}.\\ In this case,
     \(\overline{p} = \el\) and \(f = \el\). Therefore
     \(\subst{\overline{p}}{\sigma} = \subst{\el}{\sigma} \eqn{1} \el
       \sqsubseteq \el = f.\)
     We conclude that \(\ind{R}(\el, \el, \sigma)\) holds.
 
%     \medskip

   \item Case where \(\Delta\) ends with \textsf{ELIM}.
        \begin{mathpar}
          \inferrule*[right=\text{\sf ELIM}]
            {\inferrule*
               {\inferrule*[vdots=1.5em]{}{ }}
               {\smj{f'}{\overline{p}'}{\sigma}}
            }
            {\smj{\cons{l}{f'}}{\cons{l}{\overline{p}'}}{\sigma}}
       \end{mathpar}
       where, since we assumed \eqref{x:sound:head},
       \begin{enumerate}
          
         \item \label{x:sound:13} \(f \triangleq \cons{l}{f'}\),

         \item \label{x:sound:14} \(\overline{p} \triangleq
           \cons{l}{\overline{p}'}\).

       \end{enumerate}
       Let us assume that the induction hypothesis holds for the
       premise of \textsf{ELIM}, i.e., \(\ind{R}(\overline{p}', f',
       \sigma)\) holds:
       \begin{gather}
         \subst{\overline{p}'}{\sigma} \sqsubseteq f'.
         \label{x:sound:2}
       \end{gather}
        Besides, we have
        \begin{align}
          \subst{\overline{p}}{\sigma}
          &= \subst{\cons{l}{\overline{p}'}}{\sigma}
          \sqsubseteq \cons{l}{f'}
          \eqn{3} \cons{l}{\subst{\overline{p}'}{\sigma}} = f.
          &\text{by~\ref{x:sound:14}, \eqref{x:sound:2}, \textsf{EQ},
          \ref{x:sound:13}}\label{x:sound:5}
        \end{align}
        As a conclusion, the induction hypothesis and
        \eqref{x:sound:head} imply \eqref{x:sound:5}, so
        \(\ind{R}(\cons{l}{\overline{p}'}, \cons{l}{f'}, \sigma)\)
        holds.

%     \medskip

   \item Case where \(\Delta\) ends with \textsf{BIND}.
        \begin{mathpar}
          \inferrule*[right=\text{\sf BIND}]
            {\inferrule*
                {\inferrule*[vdots=1.5em]{}{ }}
                {\smj{f'}{\overline{p}'}{\sigma'}}}
            {\smj{\cons{t}{\!f'}}%
                {\cons{\meta{x}\!}{\!\overline{p}'}}%
                {\sigma' \oplus x \mapsto t}}
        \end{mathpar}
        where, because we assumed \eqref{x:sound:head},
        \begin{enumerate}

          \item \label{x:sound:17} \(f \triangleq \cons{t}{f'}\),

          \item \label{x:sound:18} \(\overline{p} \triangleq
            \cons{\meta{x}}{\overline{p}'}\),

          \item \label{bind_x} \(\sigma' \subseteq \sigma' \oplus x
            \mapsto t\),

            \item \label{bind_rho} \(\sigma \triangleq \sigma' \oplus
              x \mapsto t\).

        \end{enumerate}
        Let us assume that the induction hypothesis holds for the
        premise of \textsf{BIND}, i.e., \(\ind{R}(\overline{p}', f',
        \sigma')\) holds:
        \begin{gather}
          \subst{\overline{p}'}{\sigma'} \sqsubseteq f'.
          \label{x:sound:7}
        \end{gather}
        We also have
        \begin{align}
           \sigma(x) 
          &\triangleq (\sigma' \oplus x \mapsto t)(x) = t.
          &\text{by} \,\; \text{\ref{bind_rho}} \,\; 
          \text{and} \,\; \eqref{model:oplus}\label{x:sound:8}\\
           \subst{\overline{p}'}{\sigma'}
           &= \subst{\overline{p}'}{(\sigma' \oplus x \mapsto t)} =
           \subst{\overline{p}'}{\sigma}.
           &\text{by} \,\; \text{\ref{bind_x}},
           \text{Lem.~\ref{minimality}}  \,\; \text{and} \,\;
           \text{\ref{bind_rho}} \label{x:sound:19}
        \end{align}
        Besides, we have
        \begin{align}
           \subst{\overline{p}}{\sigma}
        &\triangleq \subst{\cons{\meta{x}}{\overline{p}'}}{\sigma}\notag
         \eqn{2} \cons{\sigma(x)}{\subst{\overline{p}'}{\sigma}}
        &\text{by} \,\; \text{\ref{x:sound:18}} \,\;\text{and} \,\; 
        \text{Fig.~\ref{x_subst_def}}\notag\\
        &= \cons{t}{\subst{\overline{p}'}{\sigma}}
         = \cons{t}{\subst{\overline{p}'}{\sigma'}}
        &\text{by} \,\; \eqref{x:sound:8} \,\; \text{and} \,\;
        \eqref{x:sound:19}\notag\\
        &\sqsubseteq \cons{t}{f'} \triangleq f.
        &\text{by} \,\; \eqref{x:sound:7},
           \textsf{EQ} \,\; \text{and} \,\;
           \text{\ref{x:sound:17}}\label{x:sound:9}
        \end{align}
        In the end, the induction hypothesis and \eqref{x:sound:head}
        imply \eqref{x:sound:9}, so
        \(\ind{R}(\cons{\meta{x}}{\overline{p}'}, \cons{t}{f'},
        \sigma)\) holds.

%     \medskip

   \item \label{unpar} Case where \(\Delta\) ends with
       \textsf{UNPAR}.
       \begin{mathpar}
         \inferrule*[right=\text{\sf UNPAR}]
           {\inferrule*
              {\inferrule*[vdots=1.5em]{}{ }}
              {\smj{f_1 \cdot f_2}{\overline{p}}{\sigma}}}
           {\smj{\cons{c(f_1)}{f_2}}{\overline{p}}{\sigma}}
       \end{mathpar}
       where, since we assumed \eqref{x:sound:head},
       \begin{enumerate}

         \item \label{x:sound:15} \(f = \cons{c(f_1)}{f_2}\).
       
       \end{enumerate}
       Let us assume that the induction hypothesis holds for the
       premise of \textsf{UNPAR}, i.e., \(\ind{R}(\overline{p}, f_1
       \cdot f_2, \sigma)\) holds:
       \begin{align}
         \subst{\overline{p}}{\sigma} &\sqsubseteq f_1 \cdot f_2
          \sqsubseteq \cons{c(f_1)}{f_2} = f.
          &\text{by} \,\; \textsf{SUB} \,
         \text{(Fig.~\ref{x_tree_matching_def})} \,\;
           \text{and} \,\; \text{\ref{x:sound:15}}\label{x:sound:I}
       \end{align}
        As a conclusion, the induction hypothesis and
        \eqref{x:sound:head} imply \eqref{x:sound:I}, so
        \(\ind{R}(\cons{l}{\overline{p}'}, \cons{c(f_1)}{f_2},
        \sigma)\) holds.\hfill \(\Box\)

\end{enumerate}


\paragraph{Completeness.}

The completeness of our algorithm means that every time a complete
substitution on a pattern matches a tree, our algorithm computes a
substitution which is included in the first one. Indeed, the computed
substitution never contains useless bindings, but the other one
may. Therefore, the completeness property is perhaps better stated by
referring to \emph{minimal substitutions}: all minimal substitutions
that enable a closed\hyp{}tree matching are computed by our pattern
matching. Formally, this can be expressed as follows.
\begin{theorem}[Completeness]\hfill
\label{backtracking:completeness}
\begin{center}
If   \(\subst{\overline{p}}{\sigma} \sqsubseteq h\),
then \(\smat{h}{\overline{p}} \twoheadrightarrow \sigma'\) 
and  \(\sigma' \subseteq \sigma\).
\end{center}
\end{theorem}

\medskip

\noindent\textsc{Proof~\ref{backtracking:completeness}}
(Completeness). By structural induction. See
appendix~\ref{backtracking:completeness_proof}.


\paragraph{Compliance.}

As a corollary, the algorithm is sound and complete with respect to
the closed\hyp{}tree matching:
\begin{corollary}[Compliance]\hfill
\label{backtracking:compliance}
\begin{center}
\(\subst{\overline{p}}{\sigma} \sqsubseteq h\) if and only if
\(\smat{h}{\overline{p}} \twoheadrightarrow \sigma'\) 
and  \(\sigma' \subseteq \sigma\).
\end{center}
\end{corollary}
\noindent In other words, the concept of closed\hyp{}tree matching
coincides exactly with the backtracking algorithm.

\medskip

\noindent\textsc{Proof~\ref{backtracking:compliance}}
(Compliance). The way from left to right is the completeness. From the
soundness, it comes that
\begin{center}\it
If \(\smat{h}{\overline{p}} \twoheadrightarrow \sigma'\), then
\(\subst{\overline{p}}{\sigma'} \sqsubseteq h\).
\end{center}
The minimality lemma~\ref{minimality} implies then
\(\subst{\overline{p}}{\sigma} \sqsubseteq h\). \(\Box\)

\paragraph{Further Discussion.}

One solution to overcome the inefficiency of non\hyp{}determinacy is
to make explicit the tree structure in the textual pattern by adding
some kind of parentheses, so each step becomes uniquely
determined. For example, to match the pattern \texttt{\%x = \%y - \%z}
against the parse tree in figure~\ref{fig:src_tree}, we would add to
the pattern meta\-lexemes called \emph{meta\-parentheses} (i.e.,
parentheses that do not belong to the object language), which are here
represented as escaped parentheses: \texttt{\%(} and \texttt{\%)}. For
example, we may use the pattern \texttt{\%(\%x = \%(\%(\%y\%) -
\%z\%)\%)}. Note that in general we cannot use plain parentheses to
unveil the structure because the language may already contain
parentheses which may have a completely different meaning other than
grouping. For instance, in the following \textsf{Korn shell}
(\textsf{ksh}) pattern:
\begin{verbatim} 
case %x in [yY]) echo yes;; *) echo no;; esac
\end{verbatim}
we must use meta\-parentheses to explicit the tree structure, because
parentheses would not pair the way we want ---~in fact, they would not
pair at all.

Fully meta\-parenthesised patterns enable linear\hyp{}time pattern
matching. However, the explicit structure comes at the price of
seriously obfuscating the pattern. Clearly, if we may say, fully
meta\-parenthesised patterns are quite difficult to read and
write. The legibility of the pattern can be improved if the matching
algorithm allows some of the meta\-parentheses to be dropped. This is
what is shown in the next section, where the system is furthermore
syntax\hyp{}directed.


%% there are two obvious problems with this
%% approach.

%% Firstly, meta\-variables are bound to strings in concrete syntax,
%% rather than being bound to subtrees of the parse tree. This is not
%% suitable when using pattern matching to manipulate the matched
%% subtrees, which is a quite common case within parsing\hyp{}based
%% tools.

%% Secondly, 

