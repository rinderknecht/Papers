\documentclass[a4paper,11pt,twoside]{article}

% Language and fonts
%
\usepackage[british]{babel}    % British English
\usepackage[T1]{fontenc}       % Required for hyphenation and \DJ
\usepackage[utf8]{inputenc}    % UTF-8 encoding
\usepackage{xspace}            % Spacing after macros
\usepackage{hyphenat}          % \hyp{} is a breakable dash
\usepackage[style=british]{csquotes} % British quotes
\usepackage{url}               % To typeset URLs
\usepackage{mdwlist}           % Tight vertical spacing of list items
\usepackage{microtype}         % Microtypographic enhancements
\usepackage[backend=bibtexu,   % BibTeX with Unicode,
            natbib=true,       % natbib aliases
            citestyle=authoryear]{biblatex}
\usepackage{amsmath,amssymb}

%\apptocmd{\thebibliography}{\raggedright}{}{} % No Underfull \hbox

% Hyperlinks
%
% Note: figure references do not seem to work.
% Note: Set "USE_PS := yes" in Makefile.cfg
%
\usepackage[breaklinks]{hyperref}
%\usepackage[ps2pdf,breaklinks]{hyperref}
%\usepackage[ps2pdf]{hyperref}
\hypersetup{colorlinks=true,urlcolor=blue,citecolor=blue,linkcolor=blue}

\newcommand\SPIN{\textsf{SPIN}\xspace}
\newcommand\Promela{\textsf{Promela}\xspace}
\newcommand\UPPAAL{\textsf{UPPAAL}\xspace}
\newcommand\UPPAALTRON{\textsf{UPPAAL-TRON}\xspace}
\newcommand\UPPAALSMC{\textsf{UPPAAL-SMC}\xspace}
\newcommand\McErlang{\textsf{McErlang}\xspace}
\newcommand\QuickCheck{\textsf{QuickCheck}\xspace}
\newcommand\UML{\textsf{UML}\xspace}
\newcommand\Erlang{\textsf{Erlang}\xspace}
\newcommand\Clang{\textsf{C}\xspace}
\newcommand\Esterel{\textsf{Esterel}\xspace}
\newcommand\Lustre{\textsf{Lustre}\xspace}
\newcommand\Simulink{\textsf{Simulink}\xspace}
\newcommand\RefactorErl{\textsf{RefactorErl}\xspace}
\newcommand\Clean{\textsf{Clean}\xspace}
\newcommand\Spoofax{\textsf{Spoofax}\xspace}
\newcommand\TTCN{\textsf{TTCN-3}\xspace}
\newcommand\BPEL{\textsf{BPEL}\xspace}
\newcommand\YAWL{\textsf{YAWL}\xspace}
\newcommand\Agda{\textsf{Agda}\xspace}
\newcommand\JavaScript{\textsf{JavaScript}\xspace}
\newcommand\Scheme{\textsf{Scheme}\xspace}

\hyphenation{now-a-days}

\title{Model-based Design of Cyber-Physical Systems}
\author{Christian Rinderknecht}
\date{\today}

\addbibresource{verif2013.bib}

\begin{document}

\maketitle

\begin{abstract}
We survey existing tools and methods for the simulation, testing and
verification of concurrent communicating systems, and their pertinence
for model-driven design and programming of cyber\hyp{}physical
systems. We also cover workflows and tests and conclude with two
proposals.
\end{abstract}

\section{Formal methods}

In the context of our project, we define a \emph{cyber-physical
  system} (CPS) as a wireless mobile network of embedded systems
featuring environmental sensors and computational capability. Usually,
an embedded system carries on specific control functions with
real\hyp{}time constraints, e.g., related to safety, on top of a
real\hyp{}time operating system (RTOS), but we will extend the notion
to encompass general\hyp{}purpose computers, for example, smartphones
and \href{http://www.raspberrypi.org/}{Raspberry Pis} running
GNU/Linux.

Depending on the scale and the nature of the software applications,
different network topologies, routing and synchronisation protocols
may be used, for example centralised architectures
(client\hyp{}server) or decentralised ones, like
peer\hyp{}to\hyp{}peer (P2P) or mobile ad hoc networks (MANET), where
nodes are also routers. At the application layer, real\hyp{}time
constraints might be expected, as well as code (a.k.a.\@ location)
migration and security guarantees.

The \emph{model-driven design} of CPS uses formal models, possibly
including real\hyp{}time features, which are checked with regards to
some desired properties and tested through simulation, before
implementations are derived.

\emph{Model checking} is a formal method whereby the truth of a
logical formula (e.g., of a temporal nature) is checked on a finite
model of a system. It is different from \emph{testing}, which consists
in interacting with an implementation, whose state space may be
infinite, and observing whether its responses conform to its model (if
no stimulation of the system is performed, we have an instance of
\emph{passive testing}). Testing hence applies to implementations and
can, at best, only prove the presence of errors, whilst model checking
applies to models and conclusively proves the absence of design errors
or finds a counterexample (with an execution trace, called
\emph{scenario} in telecommunications), thanks to the finiteness
imposed by construction. Clearly, these two approaches are
complementary, and it even makes sense to ``test'' a model by
\emph{simulation}, that is, by animating runs of the model itself and
thusly derive \emph{test sequences} (of inputs and outputs), ranging
from purely random walks to guided explorations, for example to cover
all the transitions of a component. Model\hyp{}based test generation
is usually a heuristics which avoids the potential combinatorial
explosion of the state space at the cost of incompleteness.

Another approach is \emph{theorem proving}, which delivers the same
guarantees as model checking (except in presence of a counterexample)
but does not require the model to be finite, because it relies on
induction. Some proof assistants, like \textsf{Coq}, can even generate
implementations from proofs, thanks to the use of a constructive
logic. Theorem proving can fruitfully be used in conjunction with
model checking \citep{vanGasterLensingSmetsersvanEekelen:2011}.

Finally, \emph{static analysis} consists in approximating sets of
values and behaviours of implementations as they would occur at
run-time, without instrumenting the source code and actually running
it. A very general theoretical framework for static analysis is
\emph{abstract interpretation}, which is based on a sound abstraction
of the semantics of a programming language and a symbolic execution of
the program. More syntactical algorithms, like static type checking or
inference, also work well in practice and both approaches are
useful. Static analysis can be thought of as abstracting a model from
a program, and is thus complementary to the techniques we have
mentioned earlier.

In the following sections, we examine the models, logics and tools
most widely used for model checking, and we assess whether they are
suitable for designing and programming CPS and, more precisely, which
features of CPS they support best. We finally outline different
approaches to harness them to our project.

\section{Model checking}

\citet{Kaliappan:2008} offers a short survey on verification
techniques from the software engineering perspective, with an emphasis
on systems specified with \UML. \citet{Zheng:YYYY} provides a good
tutorial.

In this section, we review and compare two verification systems,
namely \SPIN and \UPPAAL, which are mature, maintained and used both
in academy and industry. We proceed by considering translations
between models, from model to implementation (``top\hyp{}down'') and
from implementation to model (``bottom\hyp{}up'', or \emph{model
  extraction}), in particular \McErlang, a model checker for \Erlang
written in \Erlang. But first, we need some notion about
\emph{temporal logics}.

\subsection{Temporal logics}

The properties or, put more properly, the claims to be checked on the
models are expressed in some temporal logic, that is, a modal logic
interpreted over time \citep{Zheng:XXXX}. The most common properties
expected from a system are
\begin{itemize*}

  \item reachability, i.e., ``Something good may happen;''

  \item liveness, i.e., ``Something good eventually happens;''

  \item safety, i.e., ``Something good always happens.''

\end{itemize*}
Reachability is a weak property, which is nonetheless useful to debug
a model under construction, for example, by asserting that a process
may access a given resource (a scenario can usually be automatically
constructed). Safety is a strong property reminiscent of a global
invariant in non\hyp{}temporal logics, and it can be utilised for
example to state mutual exclusion, that is, that, at any time, there
will be no more than one process accessing a common resource, or the
absence of deadlocks, that is, that, at any time, there will be no
processes waiting transitively on each other for the release of a lock
on a shared resource. Other interesting safety properties are the
non\hyp{}reception of unspecified messages and the absence of invalid
end states.

In a temporal logic, doing nothing obviously realises safety, and this
is where liveness comes into play, to require some progress. For
example, liveness is useful for expressing the absence of starvation,
in other words, the fact that a process will gain access to a
resource. In the theory of programming languages, partial correctness,
by which, if a program terminates, then its output satisfies some
condition, would be considered a safety property in temporal logics,
whilst termination would be considered a liveness property. A
criterion to distinguish between safety and liveness is the following:
safety can, in theory, be \emph{disproved} in finite time, by simple
observation of the states of the system execution, whereas liveness
can be \emph{proved} in finite time.

Let us note that temporal logics assume infinite futures, and since
the system is finite, its behaviour must then be cyclic, as this
happens to be the case of many communication systems implementing
sessions or interactive loops, like servers and command shells. This
aspect is perhaps addressed more obviously by properties like
\emph{fairness}, which states that ``something good will happen an
arbitrary number of times,'' and can be conceived as a repeated
liveness property. For example: ``If access to the critical section is
infinitely often requested, then access will be granted infinitely
often.'' Fairness is most useful when dealing with the scheduling of
processes.

\subsection{SPIN}

\SPIN~\citep{Holzmann:1997,Holzmann:2003,Benari:2008} is a system
supporting the design, simulation and model checking of asynchronous
systems. Accordingly, sequential computations tend to be abstracted so
the state of the system reflects mainly process interactions. These
can be classified into several kinds: rendezvous primitives,
asynchronous message passing through buffered channels, shared
variables, and any combination of these.

The language for defining the models is \Promela (process
meta\hyp{}language), from which the name \SPIN is actually derived:
``Simple \Promela Interpreter.'' It is an imperative specification
language with a \Clang look and feel, except for the lack of pointers,
dynamic memory allocation and functions returning values. The purpose
of those restrictions, and others, is mainly to constrain the number
of states in the model to be finite. In theory, this enables all
correctness properties to become decidable, although, in practice,
some engineering methods and techniques are needed to overcome
resource limitations in the presence of very large state spaces.

The language used in \SPIN to express temporal formulas is based upon
the standard
\href{http://en.wikipedia.org/wiki/Linear_temporal_logic}{Linear
  Temporal Logic} (LTL), where the future is denoted by the set of all
possible paths that the system can take, and quantification in
formulas is always universal, that is, it is not possible to isolate a
strict subset of paths with respect to a property. The typical
temporal operators used are
\begin{itemize*}

  \item $\bigcirc \varphi$, meaning that $\varphi$ is true in the
    \emph{next} moment in time;

  \item $\Box \varphi$, meaning that $\varphi$ is true in \emph{all}
    future moments;

  \item $\lozenge \varphi$, meaning that $\varphi$ is true in
    \emph{some} future moment;

  \item $\varphi \mathrel{\textsf{U}} \psi$, meaning that $\varphi$ is
    true \emph{until} $\psi$ is true.
 
\end{itemize*}
The usual connectors of classical logics are present as well in the
LTL, like conjunction, negation etc. but time is \emph{not} explicitly
present, for example in the guise of clocks. As a consequence, time
constraints, like timeouts, can not be modelled, but partial orderings
of process interactions can.

Once a \Promela model and temporal formulas in LTL have been written,
\SPIN generates a dedicated \Clang~program, whose run checks the
claims against the specification. In case the formulas are found to be
satisfied, there is no automatic generation of a skeleton of the
implementation, which leaves a gap where errors can be introduced. In
case the state space is too large, \Promela allows for the embedding
of \Clang code as atomic sections whose scope is local and invisible
to other processes (note: there is no hierarchy of processes). Of
course, this is only feasible if the code in question does not deal
directly with the control aspect of the protocol, otherwise its
effects have to be observable in the global state space. Purely
computational tasks can be abstracted this way,
though. \citet{PajicJiangLeeSokolskyMangharam:2012} remark: \blockquote{Due to the limitations of the verification
  process (e.g., restricted model size), some parts of the models used
  for verification represent over\hyp{}approximations of the more
  ``realistic'' models. For example, for verification of
  Cyber\hyp{}Physical Systems (CPS) that feature a tight coupling
  between the real\hyp{}time controller and (usually) continuous
  physical environment, it is necessary to model the closed-loop
  system as a whole.}

By contrast, note that in the control system industry (e.g.,
automotive industry, aerospace, robotics, logic circuit design etc.)
\emph{synchronous languages} like \Esterel, \Lustre, \Simulink
etc. are used to express the models, and these allow automatic
generation of the implementation. By contrast, the application domain
of \SPIN is that of dynamic asynchronous software systems, which allow
the instanciation and scheduling of new processes at run\hyp{}time,
and where processes may perform complex computational tasks in
addition to protocol related operations.

\subsection{UPPAAL}

\emph{Timed automata} are a standard modelling formalism for
real\hyp{}time systems because they enable efficient system
verification by model checking and testing
\citep{Bouyer:2003,BengtssonYi:2004}. Basically, a timed automaton is
a finite\hyp{}state machine extended with clock variables. The time
denoted by the clocks is dense or continuous, in other words, it is a
mathematical real number. Moreover, edges of the automata are
threefold, in all generality: not only do they carry \emph{send or
  receive} actions for synchronisation on unbuffered channels, but
also \emph{guards}, which are constraints on the clock values that
must hold for the edges to be executable, and \emph{clock resets} to
zero. The semantics of a network of timed automata is their
synchronous product. This kind of model is usually extended further
with bounded discrete variables which are part of the system state and
are used as if in an imperative language. In sum, the state of the
system consists in the locations (vertices) of all the automata, the
clock values and the discrete variable values.

The model checker \UPPAAL \citep{Amnell:2001,BehrmannDavidLarsen:2004}
uses an extension of timed automata, called \emph{timed safety
  automata} because the theoretical model is further extended with
features which help in proving safety and liveness properties, like
urgent and committed locations, urgent synchronisations and broadcast
channels. User functions are allowed for computational tasks and have
a \Clang~look and feel, except for the absence of
pointers. Synchronisations are realised only with rendezvous on one
action name: the sender is blocked until the receiver is in a state
with an out\hyp{}going executable edge receiving the same action; the
sender is released after the receiver has accessed and perhaps
modified the global state and reached the next local state
(vertex). In other words, there is no parameter passing with actions:
instead, shared variables and side\hyp{}effects are used to exchange
information (bidirectionally) during the rendezvous.

The query language is based on \emph{Temporal Computational Temporal
  Logic} (TCTL), a modal extension of \emph{Computational Temporal
  Logic} (CTL) with time. The concept of future in the TCTL is that of
a tree of all possible traces from the present moment. The objects of
the logical formulas are paths (in the tree) and individual states.
Quantifications are universal or existential: $\textsf{A} \, \varphi$,
which means that ``for all paths, the formula $\varphi$ holds,'' and
$\textsf{E}\, \varphi$, which means that ``there exists a path such
that $\varphi$ holds.'' Furthermore, we find the following modal
operators:
\begin{itemize*}

  \item $\textsf{X} \, \varphi$, meaning: ``In the next state,
    $\varphi$ is true;''

  \item $\textsf{F}\,\varphi$, meaning: ``In a future state, $\varphi$
    is true;''

  \item $\textsf{G} \, \varphi$, meaning: ``Globally in the future,
    $\varphi$ is true;''

  \item $\varphi \mathrel{\textsf{U}} \psi$, meaning: ``$\varphi$ is
    true \emph{until} $\psi$ is true.''

\end{itemize*}
For example \citep{Raimondi:2008}: $\textsf{AG}(\varphi \rightarrow
(\textsf{EF}\,\psi))$ means: ``It is globally the case that, if $\varphi$
holds, then there exists a path such that, at some point in the
future, $\psi$ holds as well.''

\UPPAAL is actually a toolkit made of a graphical user interface and a
built\hyp{}in model checker. The graphical interface is convenient and
helps understanding the model through simulation. There is no
automatic generation of a skeleton of the implementation, which
presents a risk of introducing errors.

There is an extension of model checking which attracts a lot of
attention nowadays: \emph{statistical model checking}
\citep{David:2012,LegayDavid:2012}. Simply put, the timed automata are
extended with probabilities to become \emph{Priced Timed Automata},
and a stochastic semantics is used. The temporal logic TCTL is also
extended conservatively to support probabilistic queries. The main
interest of these extension is to have model checking go beyond worst
case analyses, like the worst case response time of a recurrent task
following a given schedule, and assess the average case behaviour of
real\hyp{}time systems. In other words, statistical model checking
decides whether a system statisfies a property \emph{with some degree
  of confidence}. There are two extensions to \UPPAAL supporting this
new paradigm: \UPPAALTRON \citep{LarsenMikucionisNielsen:2009} and
\UPPAALSMC \citep{Bulychev_et_al:2012}.

\subsection{Comparison}

Perhaps the most obvious differences between the modelling languages
of \SPIN and \UPPAAL are the absence of real\hyp{}time (i.e., clocks)
in the former and the absence of asynchronicity (i.e., buffered
channels) in the latter. 

\Promela allows several kinds of synchronisations, whilst \UPPAAL only
offers binary rendezvous (except for broadcast) and the sharing of
variables. In other words, \Promela features synchronisation by
message passing, whilst \UPPAAL uses rendezvous and shared
variables. In particular, this means that the exchange of information
in \UPPAAL can be bidirectional. There has been a recent attempt at a
workaround in \UPPAAL by \citet{Boudjadar_et_al:2013}, who define a
new class of automata called \emph{communicating timed automata},
which are timed automata extended with the concept of ``call (of
another automaton) with parameters.'' These automata are defined by
translation into standard \UPPAAL automata. See also
\citet{ChandrasekaranMukund:2006} and their \emph{timed
  message\hyp{}passing automata}.

As far as data types are concerned, both \Promela and \UPPAAL have
primitive and structured data types, including arrays, although
\Promela does not allow arrays to be passed as a parameter during
synchronisation.

Buffered channels can be modelled explicitly in \UPPAAL as timed
automata with an array of messages as a local variable
\citep{ChandrasekaranMukund:2006}, but this may quickly overload the
state space, as all permutations of the messages are observable. In \Promela, channels are first\hyp{}class objects, that is, they are
values and they can be sent from one process to another in order to
create a private communication channel. By contrast, in \UPPAAL,
channels are not a concept and they have to be inferred on the basis
of the synchronisations, i.e., if two automata can synchronise, an
implicit channel exists between them.

There have been some proposals
\citep{TripakisCourcoubetis:1996,Bosnacki:2002,NabialekJanowskaJanowski:2008}
to extend \Promela with real\hyp{}time features, but they are not
widely used.

It is possible in \Promela to create at most~255 processes at
run\hyp{}time, whilst \UPPAAL requires a statically fixed number of
automata (so\hyp{}called template instanciations). The idea of
\citet{Boudjadar_et_al:2013} to extend \UPPAAL with dynamic automaton
creation is not entirely convincing here because it relies implicitly
on some unstated static analysis to determine an upper bound on the
maximum number of automata active simultaneously.

Some industrial applications have been carried out with the
side\hyp{}effect or main intent to compare \SPIN and \UPPAAL
\citep{JensenLarsenSkou:1996,Lingegowda:2006,NabialekJanowskaJanowski:2008,GuentherMiliusMoeller:2012,AlzahraniGeorgieva:2013}. Also
worth of notice is the extensive dual tutorial by
\citet{Kluppelholz:2012}. Therefore, to a certain extent, \Promela and
\UPPAAL can simulate each other (it is an open question as to what are
the largest subsets of these languages that are equivalent by
bisimulation), but, in practice, the models of the former often use
asynchronous channels and the models of the latter often rely
prominently on clocks, which makes the two languages incommensurable
in general.

As for the temporal logics, LTL versus TCTL, the timed nature of TCTL
makes then trivially incomparable. The untimed fragment of TCTL, which
is CTL, has \emph{not} the same expressive power as LTL. A CTL formula
and a LTL formula are said equivalent if, for all models expressed as
Labelled Transition Systems, they have the same truth value. It can be
shown that there are formulas in CTL which have no equivalent in LTL,
and vice versa, although there exists equivalent formulas. For a
discussion of their respective merits, on practical and theoretical
grounds, see \citet{Vardi:2001} and the tutorials by
\citet{Katoen:2007} and \citet{Bellarmine_Krug:2010}.

\subsection{Translations}

We briefly consider here three sorts of translations involving the
models of \SPIN and \UPPAAL, either as source or target: from one
model to another, from a model to an implementation and from an
implementation to a model.

An example of translation from \UPPAAL to another modelling language
is given by \citet{Abel:2009}, who does not build the product
automaton before translation but uses a stepwise algorithm based on
the semantics of timed automata. Another example of intermodel
translation is that of \citet{NabialekJanowskaJanowski:2008}, whom we
mentioned earlier because their source is a discrete\hyp{}time
extension of \Promela. Their target is a general language of timed
automata, compatible with that of \UPPAAL.

There have been a few studies on the automatic generation of
implementations from models, for example from \Promela and \UPPAAL to
\Clang or \Clang-like languages
\citep{KristensenMejlholmPedersen:XXXX,Loefller:1996,Hendriks:2001}.

As far as model extraction is concerned, we would mention a recent
translation from \Clang to \Promela by \citet{Jiang:2009}. Another way
to proceed is that of \McErlang
\citep{Benac_EarleFredlund:2010,Benac_EarleFredlund:2012} with a timed
semantics of \Erlang programs. With this system, \Erlang programs need
to use a library providing time\hyp{}related functions, for example,
to obtain or record the current time, and a dedicated run\hyp{}time
must be used instead of the standard virtual machine. More precisely,
the run\hyp{}time is an interpreter for \Erlang in \Erlang. Time is
discrete and, by default, there is no global clock ticks: the
difference between two successive timeouts defines a time slice. The
vertices of the state graph consist of the virtual time elapsed since
the start, and the edges are annotated by send, receive or timeout
statements. Some restrictions ensure that the graph is always finite,
so model checking is possible. \McErlang can also be integrated with
\QuickCheck, a popular tool for unit testing
\citep{Anonymous:XXXX,Svensson:XXX}.

\section{Testing}

Statistical model checking is closely related to the use of model
checkers for testing \citep{LarsenMikucionisNielsen:2004}, because
randomness is often needed by the heuristics deriving test sequences
according to a given objective, like the coverage of an embedded
component (that is, a part of a system under test which has no direct
connection with the tester).

The specificity of testing (an implementation derived from) an \UPPAAL
model is that real\hyp{}time constraints must be taken into account
when deriving the test suite; in fact, the test system becomes itself
a real\hyp{}time system. One interesting technique proposed by
\citet{Hessel_et_al:2008} consists in adding a timed automata (an
\emph{observer automaton}) to a given \UPPAAL model, as well as
annotations to assign test objectives, coverage and verdicts. These
annotations are then translated into pure \UPPAAL in such a way that
model checking itself will generate the test sequences. More
precisely, a formula $\psi$ is devised so that it specifies the
desired structural criterion, like coverage of some part of the
control flow graph, and we check for ``always not $\psi$'': the
counterexample given by the model checker is the test case. The
caveat, as usual with model checking, is that the state space must be
finite.

Another closely related issue is the matching of scenarios with time
constraints against a given model. These scenarios can be conceived as
test sequences, with a positive or a negative expected verdict.
\citet{ChandrasekaranMukund:2006} propose an extension of the classic
\UPPAAL timed automata, called \emph{timed message\hyp{}passing
  automata}, and Timed Message Sequence Charts, a timed extension of
MSC. They show how to translate these into \UPPAAL, at the cost of
some restrictions on expressiveness and efficiency.


\section{Proposals}

This section outlines two research proposals for the programming of
CPS.

\subsection{Model-based design for CPS}

We assume that we have a library of \Erlang combinators for tasks and
workflows, inspired by the \textsf{iTasks} framework in \Clean. Its
usage is eased by an extension of \Erlang with infix
operators. Moreover, we suppose that we have at our disposal some
static analyses applicable to \Erlang programs, either dedicated by
means of \RefactorErl, or generic by means of \McErlang.

Different model\hyp{}driven designs should be supported by our
framework.
\begin{enumerate}

  \item \emph{Design with combinators.} If the workflow of the CPS
    application fits the scope of the combinators' library, we may
    simply want to directly and only use combinators.

  \item \emph{Model checking with \SPIN.} If model checking is
    required or deemed valuable, we may want to model the CPS in
    \Promela. The limitation is that the model must be finite, in
    particular, a maximum of~255 processes can be active
    simultaneously, and no real\hyp{}time constraints are possible.
    \begin{enumerate}

      \item \label{lab:gen_erl} \emph{Generation of \Erlang.} If this
        is not an issue, model checking can be done and a mapping from
        processes to network (\Erlang) nodes shoud be provided. Some
        research would be then necessary to translate \Promela to
        \Erlang, perhaps targeting the combinators' library or the
        \textsf{OTP} behaviour \texttt{gen\_fsm}.

      \item \emph{Model refinement.} If \Promela proves to be too
        restrictive, or if we had to hide some behaviours from the
        state space due to memory limitations, we may consider a model
        refinement whereby the system is divided into one global model
        and several local (sub)models, each being the refinement of a
        \Promela process considered as a stub or abstract, and being
        mapped to one network node. When creating submodels which,
        collectively, define the behaviour of one \Promela process
        (one network node), we must make sure that we indeed make a
        refinement, so we do not invalidate any assumption at the
        global level (the observable timed traces of the submodels
        must be included in the timed traces of the corresponding
        abstract \Promela process). From there, different routes open.
        \begin{enumerate}

          \item \emph{Generation of \Erlang.} As in step
            \ref{lab:gen_erl}, \Erlang could be produced from the
            local \Promela models.

          \item \emph{Translation to \UPPAAL.} If real\hyp{}time
            constraints are important, we could opt for another way
            and translate our model such that the global model is
            asynchronous and written in \Promela, and the local models
            are synchronous and written in \UPPAAL. (In control
            systems, this kind of design is called ``Globally
            Asynchronous, Locally Synchronous'' or GALS.) The simpler
            way to achieve that is to rely on \Promela declarations of
            unbuffered channels and translate the process involved
            into \UPPAAL. Each remaining local asynchronous
            communication, for example with physical sensors, would be
            simulated by one timed automata (one per \Promela channel)
            holding a bounded array of messages as a local variable.

           The translation from \Promela to \UPPAAL should introduce
           real\hyp{}time constraints. One solution would perhaps
           define a Domain Specific Language (DSL) as a superset of
           \Promela, which includes real\hyp{}time annotations. These
           annotations would be stripped every time standard \Promela
           is required. Another way is to revive some of the
           real\hyp{}time extensions to \Promela proposed by some
           researchers. It would be interesting also to consider
           whether \UPPAAL textual models could be directly embedded
           in (deterministic) atomic sections in an extension of
           \Promela, differing to \UPPAAL all analyses about
           these. Yet another angle would be to consider an existing
           DSL for specifying program transformations, like in the
           \Spoofax workbench, and both inject and remove
           real\hyp{}time constraints, akin to parsing and
           pretty\hyp{}printing, the latter being useful when the
           untimed design has to be modified. After the model checking
           and simulation of the \UPPAAL submodels, some research is
           needed to automatically translate these into \Erlang, as we
           envisaged it in step \ref{lab:gen_erl} for \Promela models.

        \end{enumerate}

    \end{enumerate}

\end{enumerate}


\subsection{Model-based testing of CPS}

\subsubsection{TTCN-3}

\TTCN (\emph{Testing and Test Control Notation version~3}) is a
programming language strongly and statically typed, specialised for
the conformance testing of telecommunication systems, but it is also a
language used to specify interfaces for the test infrastructures. It
is used for large projects in telecommunication, for instance mobile
telephony, rather than software engineering in general. (By the way,
it would be interesting to determine if this difference is cultural or
technical in nature.)

From the point of view of programming theory, this language presents
many interesting features which definitely merit attention. For
instance, \TTCN features
\begin{itemize}

  \item a standardised operational semantic,

  \item an expressive type system,

  \item pattern matching on expressions,

  \item dynamic creation of processes,

  \item multiple paradigms of communication, as asynchronous exchange
    of messages, RPC (synchronous) etc.

\end{itemize}
Some of these characteristics are also found in the functional
programming language \Erlang, used for implementing distributed, fault
tolerant systems. Others aspects, not mentioned here (like
\emph{snapshot semantic}), are very specific to the domain of
application of \TTCN. Nevertheless, the resemblance suggests that a
semantic of \TTCN could be defined by translation to
\Erlang. Moreover, since \TTCN can be used to test distributed systems
and embedded systems, it makes sense to use it on CPS.

\subsubsection{Testable workflows}

A different approach would be to generate a testable workflow instead
of a workflow for operational deployment. This would entail the
creation of additional tasks whose purpose would be to \emph{trace},
via pipes and taps, the communication of the other tasks. Beyond pure
tracing, we could envisage also \emph{passive testing}, whereby the
testers would carry test objectives and stop when a verdict is
reached, without disturbing the whole system while it is running. Some
test tasks would also define \emph{Message Sequence Charts} (MSC), or
some equivalent formalism, to be matched against the exchanges of a
set of communicating tasks, distributed in general on different
network nodes.

In other words, we propose to create tasks, using the same framework
of combinators, whose sole purpose would be to support tracing,
passive testing and conformance testing (with respect to MSC). The
specification of a testing workflow would be provided at the same time
as the operational workflow, hence meeting the requirement of ``Task
and Test Driven Development''.

\subsection{Workflows for business processes}

Many papers have been written about the design of business processes
by means of patterns and workflows and recent years have seen the
emergence of standard catalogs which are used as benchmarks by
researchers and software companies alike. The combinator approach of
the CPS project seems to share some common tasks with the business
workflows, although the languages (\BPEL, \YAWL) appear very specific
to the business framework. The checking of these models has attracted
an increasing attention, in particular with the model checker
\SPIN. It would be interesting to assess how far these business
workflows overlap with our task-oriented worflows for CPS and how the
published patterns of translation to \Promela can be reused for model
checking.

\subsection{Functional Reactive Programming}

\emph{Reactive Programming} is a paradigm for programming
communicating processes in the manner of synchonous data\hyp{}flow
languages. In particular, this type of programming, or programming
languages, is seen as an attractive alternative to the pervasive use
of callback functions, which tend to obscure the control flow. When
reactive programming is supported by means of a DSL embedded in a
functional language, it is said \emph{Functional Reactive Programming}
(FRP). The other approach consists in designing a full\hyp{}fledged
FRP, but it requires more work, although it offers potentially more
options for static analysis of the reactive behaviour. Moreover, with
the DSL, the features of the embedding language can be used, allowing
more expressivity for the logic part.

Roughly speaking, in FRP, certain variables, called \emph{signals} or
\emph{behaviours}, have time\hyp{}varying values. The difference with
side\hyp{}effects is that the mutation is implicit and all expressions
depending on behaviours are automatically updated. (In case of a FRP
standalone language, the run\hyp{}time must analyse the data flow to
minimally propagate the changes.) The other distinctive feature of FRP
is \emph{events}, which are timed values denoting physical events,
like a mouse button being pressed, or predicates. Depending on the
semantics, signals and events can be first\hyp{}class values or
not. Many behaviours can be naturally expressed in terms of reactions
to events, giving rise to reactive behaviours. Since time is not
present in the semantic of functional languages, signals and events
are defined as abstract data types which are created and destructured
by functions, e.g., events may be specified in terms of boolean
functions of continuous, implicit time or by, in non\hyp{}strict
functional languages, signals can be modelled by infinite (lazy)
streams of values. A library is provided to \emph{lift} (convert)
static values (constants) into behaviours and it is convenient if the
language offers syntactical to implicit lift static values.

FRP is a good theoretical framework for the CPS project, because it
can account for concurrency, migration (that is, code mobility) and
dynamic creation of processes. We would have to add features like
fault tolerance, either by extending a FRP library or by implementing
some abstract types with \Erlang during code generation. Recently,
several backends have been developed for different functional
languages, which produce \JavaScript in order to implement web
services. There have been some work to translate \JavaScript to
\Scheme, hopefully it would help to translated it to \Erlang as well.

\Agda is a system providing a programming language with dependent
types which are leveraged by a theorem prover. Recent works have shown
the equivalence of Linear-time Temporal Logic (LTL), used in the model
checker \SPIN, and FRP. A reactive library has been written in \Agda,
using the type checker to verify LTL properties on the reactive
(client) programs. The possibility to have an infinite space state,
thanks to the availability of induction, makes \Agda an appealing
option, although the learning curve is steep. There is also a backend
to the \Agda compiler, which produces \JavaScript and therefore would
allow us to translate automatically the \Agda programs (not the
theorems) to \Erlang.

%\bibliographystyle{unsrt}
%\bibliography{verif}
\printbibliography
\nocite{*}

\end{document}
