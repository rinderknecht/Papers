%% -*- ispell-dictionary: "british" -*-

\documentclass[11pt,a4paper]{article}

\usepackage[british]{babel}    % British English
\usepackage[T1]{fontenc}
\usepackage[utf8]{inputenc}    % UTF-8 encoding
\usepackage{xspace}
\usepackage{hyphenat}
\usepackage{url}
\usepackage{alltt}
\usepackage[style=authoryear,backend=bibtex,natbib=true]{biblatex}
\addbibresource{ie2014.bib}
%\usepackage[comma]{natbib}
%\bibliographystyle{plainnat}
\usepackage{amsmath,amssymb}
\usepackage[scaled]{beramono}

\newcommand\plang[1]{\textsf{#1}\xspace}
\newcommand{\cpp}{\textsf{C} \hspace*{-2.5mm} \raise 0.7mm \hbox {${\scriptscriptstyle ++}$}}
\newcommand\exc[1]{\texttt{\small #1}}

\hyphenation{va-gue-ly}

\title{A Survey on Teaching and Learning Recursive Programming}
%\author{}
\date{}
\author{Christian Rinderknecht\\
\small \url{rinderkn@caesar.elte.hu}\\
\small Department of Programming Languages and Compilers\\
\small E\"otv\"os Lor\'and University\\
\small Budapest, Hungary}

% Main
%           
\begin{document}

\maketitle

%%-*-latex-*-

\begin{abstract}
Recursion is a powerful programming technique which is notoriously
difficult to master, especially in functional languages because they
prominently feature structural recursion as the main control\hyp{}flow
mechanism. We propose several hypotheses to understand the issue and
put some to the test by designing an open\hyp{}source interactive
interface based on a tangible block\hyp{}world with augmented reality
and software feedback. Stacks of blocks are used as an analogy for the
list data structure, which enables the simplest form of structural
recursion. After using this application, students are expected to
transfer their training to directly write recursive programs in
sequential \erlang, a purely functional language.
\end{abstract}


\bigskip

\noindent\textbf{Keywords:} computer science education, didactics of
programming, recursion, tail recursion, embedded recursion, iteration,
loop, mental models.

\section*{Foreword}

In this article, we survey how practitioners and educators have been
teaching recursion, both as a concept and a programming technique, and
how pupils have been learning it. After a brief historical account, we
opt for a thematic presentation with cross\hyp{}references, and we
append an extensive bibliography which was very carefully
collated. The bibliography is the foundation of our contribution in
the sense that we started from it, instead of gathering it in support
of our own ideas, as is usual in research papers.

In writing this survey, we committed ourself to several guidelines
which the reader is advised to keep in mind while reading.
\begin{enumerate}

  \item We restricted ourself exclusively to the published literature
    on teaching and learning recursive programming, not computer
    programming in general. While it may be argued that, for example,
    articles and books about functional programming almost constantly
    make use of recursion, we preferred to focus on the papers
    presenting didactical issues explicitly and exclusively related to
    recursion.

  \item We did not review the emergence of the concept of recursion
    from its mathematical roots, and we paint a historical account
    with our fingers, just enough to address the main issues of this
    survey with a minimal background information.

  \item We did not want to mix our personal opinion and assessment of
    the literature with its description, because we wanted this
    article to be in effect a \emph{thematic index to the literature},
    even though it is not possible to cite all references in the text,
    for room's sake. The only places where we explicitly express our
    own ideas are in our own publications, in the definitions found in
    the introduction (in the absence of bibliographic reference) and
    in the conclusion.

  \item We did not cover topics like the teaching of recursion and
    co\hyp{}recursion in the context of lazy evaluation, or
    programming languages based on process algebras or dataflow,
    because they have not been addressed specifically in didactics
    publications, perhaps because they are advanced topics usually
    best suited for postgraduate students, who are expected to master
    recursion, and most publications deal with undergraduates or
    younger learners.

  \item Despite our best efforts in structuring the ideas found in the
    literature, the following presentation contains some measure of
    repetition because papers often cover mutually related topics, so
    a printed survey cannot capture exactly what is actually a
    semantic graph, and such a graph would better support a
    meta\hyp{}analysis of the literature (where
    cross\hyp{}referencing, publication timelines, experimental
    protocols and statistics would be in scope) rather than a survey.

\end{enumerate}

\section*{Introduction}

In abstract terms, a definition is \emph{recursive} if it is
self\hyp{}referential. For instance, in programming languages,
function definitions may be recursive, and type definitions as
well. \textcite{Giveon:1990} provided an insightful discussion of the
didactical issues involved in the different meanings ascribed to the
word, which appeared first in print by Robert Boyle in
\oldstylenums{1660} (\emph{New Experiments Physico\hyp{}Mechanicall},
chap.~XXVI, p.~203) to qualify the movement of a moving pendulum,
which returns or ``runs back''. (Beware the incorrect and ominous ``to
recurse''.)

\paragraph{A short history}

Formal definitions based on recursion played an important role in the
foundation of arithmetic \citep{Peano:1889} and constructive
mathematics \citep{Skolem:1923,Robinson:1947,Robinson:1948}, as well
as in the nascent theories of computability
\citep{Soare:1996,vanOudheusden:2009,Daylight:2010,Lobina:2012}, with
the caveat that recursion theory is only named so for historical
reasons. The first computers were programmed in assembly languages and
machine codes \citep{Knuth:1996}, but the first step towards recursion
is the advent of labelled subroutines and hardware stacks, by the end
of the \oldstylenums{1950}s. \plang{BASIC} epitomises an early attempt
at lifting these features into a language more abstract than assembly:
recursion is simulated by explicitly pushing (\exc{GOSUB}) program
pointers (line numbers) on the (implicit) control stack, and by
popping (\exc{RETURN}) them. According to the definition above,
this is not recursion, which is to be understood as being purely
syntactic (a function definition), not semantic (the evaluation of an
expression). Moreover, the lack of local scoping precludes passing
parameters recursively. Nevertheless, ``recursion'' has been taught
with \plang{BASIC} by \textcite{Daykin:1974}.

With even more abstract programming languages, fully\hyp{}fledged
recursion became a design option, first advocated in print by
\textcite{Dijkstra:1960} and \textcite{McCarthy:1960}, and implemented in
\plang{LISP}, \plang{ALGOL}, \plang{PL/I} and \plang{Logo}
\citep{Martin:1985,Lavallade:1985}, with the notable exceptions of
\plang{Fortran} and \plang{COBOL}. Formal logic was then used to
ascribe meanings to programs, some semantics relying on recursion,
like rewrite systems, some others not, like set theory or
\(\lambda\)-calculus (where recursion is simulated with
fixed\hyp{}point combinators). Even though the opinion of
\textcite{Dijkstra:1974} \citeyearpar{Dijkstra:1975} varied, recursion
proved a powerful means for expressing algorithms
\citep{Dijkstra:1999} \citep{Reingold:2012}, especially on recursive
data structures like lists, \emph{i.e.}, stacks, and trees. With the
legacy of \plang{LISP} and \plang{ALGOL}, together with the rise and
spread of personal computers, recursion became a common feature of
modern programming languages, and arguably an essential one
\citep{Papert:1980,Ford:1982,Astrachan:1994}.

\section{Recursion, iteration and loops}

Recursion is often not clearly understood, as demonstrated by the
frequent heated or misguided discussions on internet forums, in
particular about the optimisation of tail calls. Moreover, some
researchers implicitly equate loop and iteration, use the expression
``iterative loop'', or call recursion a process implemented by means
of a control stack, whilst others use a syntactic criterion. We must
define recursion in order to relate it to other concepts, like loops,
iteration, tail recursion, embedded recursion, structural recursion
etc.

\paragraph{Definitions}
\label{definitions}

\textcite{Giveon:1990} remarked: ``the concept of recursion is being
vaguely and inconsistently constructed from some syntactical
properties of the program, from its associated semantics and from
features borrowed from models of execution of programs.'' Indeed,
broadly speaking, there are two angles to approach the question: the
static (syntactic) approach and the dynamic approach to
recursion. Sometimes the dichotomy is put in terms of \emph{programs}
(structured texts or abstract syntax trees) versus \emph{processes}
(autonomous agents or stateful actors). In fact, to address the vast
literature about the teaching and learning of recursion, it is
essential to understand both views and their relationship.

In the static comprehension, recursion is restricted to the general
definition we gave at the start of this introduction: the occurrence
of the symbol being defined inside its definition (what is called
impredicativity in logic). Implicitly, this means of course that it
must be clear what the denotation of the occurring symbol is, in order
to determine whether it is an instance of the definition. For example,
in the language \plang{OCaml}, the function definition
{\small
\begin{alltt}
\textbf{let rec} f x = (\textbf{fun} _ -> x) f
\end{alltt}
}
\noindent is recursive because the name~\exc{f} in the right\hyp{}hand
side refers to the~\exc{f} in the left\hyp{}hand side. Note that there
is no call to~\exc{f} in the definition of~\exc{f}, so the concept of
\emph{recursive call} is actually irrelevant: recursion in this
context is a property of definitions based on lexical scoping rules,
not of the objects potentially computed (values), nor the way they are
computed (semantics). In particular, the recursive definition of a
function does not necessarily entails that it is total, hence
terminates for all inputs. (A type system may enforce that property,
as in \plang{Coq} or \plang{Agda}.)

Sometimes, an examination of the program cannot determine
whether the occurrence of a symbol refers to the definition at
hand. For instance, let us consider the following fragment of
\plang{Java}:
{\small
\begin{alltt}
\textbf{public class} T \{ \textbf{public void} g(T t) \{ ... t.g(t); ... \} \}
\end{alltt}}
\noindent The occurrence of~\exc{g} in the expression \exc{t.g(t)}
does not necessarily refers to the current definition of the
method~\exc{g}, because that method may be overridden in
subclasses. Therefore, here, we cannot conclude that~\exc{g} is
recursive according to the static criterion. (It can be argued,
though, that the class~\exc{T} is recursive because its definition
includes the declaration (type) of its method~\exc{g}, where
\exc{T}~occurs---Interfaces perhaps illustrate this better.)

The dynamic comprehension of recursive functions can be expressed
abstractly as a property about \emph{dynamic call graphs}: recursion
is a reachable cycle, which means, in operational terms, that the
control flow of calls returns to a vertex (a function) which was
previously called. Here, the notion of \emph{recursive definition} is
not central, and it makes sense to speak of \emph{recursive call} (a
back edge closing a path). Another, less general, approach to a
dynamic definition of recursion relies on a particular execution
model, often based on stack frames allocated to function calls and
their lexical context. Anyway, as a property about the control flow,
recursion in that sense becomes undecidable in general for
Turing\hyp{}complete languages.

It should be noted that the static and dynamic definitions of
recursion may overlap, but are different in general, that is, if a
function is recursive according to the syntactic criterion, it may not
be recursive according to the dynamic criterion (as the above
\plang{OCaml} function~\exc{f} illustrates), and vice versa. Consider
the following \plang{OCaml} program implementing the factorial
function:
{\small
\begin{alltt}
# \textbf{let} pre self n = \textbf{if} n = 0 \textbf{then} 1 \textbf{else} n * self(n-1);;
\emph{val pre : (int -> int) -> int -> int = <fun>}
# \textbf{let rec} fact n = pre fact n;;
\emph{val fact : int -> int = <fun>}
# fact 5;;
\emph{- : int = 120}
\end{alltt}
}
\noindent The syntactic criterion decides that~\exc{pre} is not
recursive and~\exc{fact} is; the dynamic criterion sees these two
functions as \emph{mutually recursive}, that is, the control flow goes
from one to the other, and vice versa. Furthermore, there are
different techniques to achieve dynamic recursion without static
recursion at all. For example, using fixed\hyp{}point combinators in
\plang{OCaml} with the command\hyp{}line option \exc{-rectypes}:
{\small
\begin{alltt}
# \textbf{let} pre self n = \textbf{if} n = 0 \textbf{then} 1 \textbf{else} n * self(n-1);;
\emph{val pre : (int -> int) -> int -> int = <fun>}
# \textbf{let} y f = (\textbf{fun} x a -> f (x x) a) (\textbf{fun} x a -> f (x x) a);;
\emph{val y : (('a -> 'b) -> 'a -> 'b) -> 'a -> 'b = <fun>}
# \textbf{let} fact = y pre;;
\emph{val fact : int -> int = <fun>}
# fact 5;;
\emph{- : int = 120}
\end{alltt}
}
\noindent Here, neither the higher\hyp{}order function~\exc{y} (called
the \emph{call\hyp{}by\hyp{}value Y combinator}), nor the function
\exc{pre} are statically recursive (as the absence of the keyword
\exc{rec} shows well), but they are mutually recursive in the dynamic
sense. (The rationale behind the definition of~\exc{y} is obscure, but
relies on the fact that \exc{(y f) x} yields the computation of
\exc{(f(y f)) x}, showing that \exc{y f} is the fixed point
of~\exc{f}.) It is even possible to define the factorial function
without recursion, loops or jumps (\exc{goto}) in~\plang{C}, but the
program is cryptic: {\small
\begin{alltt}
\textbf{#include}<stdio.h>
\textbf{#include}<stdlib.h>

\textbf{typedef int} (*fp)();

\textbf{int} fact(fp f, \textbf{int} n) \{
  \textbf{return} n? n * ((\textbf{int} (*)(fp,\textbf{int}))f)(f,n-1) : 1; \}

\textbf{int} read(\textbf{int} dec, \textbf{char} arg[]) \{
  \textbf{return} ('0' <= *arg && *arg <= '9')? 
         read(10*dec+(*arg - '0'),arg+1) : dec; \}

\textbf{int main}(\textbf{int} argc, \textbf{char}** argv) \{
  \textbf{if} (argc == 2) printf("%u\textbackslash{n}",fact(&fact,read(0,argv[1])));
  \textbf{else} printf("Only one integer allowed.\textbackslash{n}");
  \textbf{return} 0; \}
\end{alltt}}
\noindent (See \textcite{GoldbergWiener:2009} for a practical use of such
a simulated recursion in \plang{Erlang}.) References can also be used
to define the factorial function without static recursion, with a
technique called \emph{Landin's knot}: {\small
\begin{alltt}
# \textbf{let} g = ref (\textbf{fun} n -> 42);;
\emph{val g : ('_a -> int) ref = \{contents = <fun>\}}
# \textbf{let} f n = \textbf{if} n = 0 \textbf{then} 1 \textbf{else} n * !g(n-1);;
\emph{val f : int -> int = <fun>}
# \textbf{let} fact = g := f; \textbf{fun} n -> !g(n);;
\emph{val fact : int -> int = <fun>}
# fact 5;;
\emph{- : int = 120}
\end{alltt}
}
\noindent Here, none of the definitions are statically recursive,
although~\exc{f} is dynamically recursive.

Finally, it is perhaps worth insisting on the case where there are
more than one definition, like \(f(x) := g(x-1)\) and \(g(x) :=
h(x+1,f(x-1))\). Neither definition is statically recursive, although
they are mutually recursive according to the dynamic
interpretation. Furthermore, it is clear that these definitions are
equivalent to \(f(x) := h(x,f(x-2))\), which is statically
recursive. This shows that the concept of mutual recursion is dynamic,
but the static criterion could be extended to apply transitively to
the \emph{static call graph}, which is an over\hyp{}approximation of
the dynamic call graph, so we can speak of mutual recursion in a
static sense as well, but keeping in mind that there can be mutual
recursion statically when there is none dynamically.

\paragraph{Tail recursion, iteration and loops}

The concept of \emph{tail recursion} is difficult to apprehend because
it is built upon both the dynamic call graph and the \emph{data
  flow}. We have already seen that recursion can be defined as a cycle
in the dynamic call graph. Here, we define the \emph{dataflow graph}
as the dynamic call graph with an additional kind of edges oriented
according to the direction where the data flows (it is a multigraph):
if a caller passes arguments to the callee, there is a \emph{data
  edge} doubling the \emph{control edge}; if the value of a function
call is needed to further compute an expression or complete an
instruction, there is a data edge from the callee to the caller, that
is, a backward data edge with respect to the control edge. Since, in
the absence of run\hyp{}time errors, the result of a call is needed,
at the very least, to stand for the result of the caller itself, there
is always a back edge. Therefore, we could make those edges implicit
and only retain them when the value of the call is needed in a
strictly embedding expression, not just to be returned in turn. Tail
recursion is then a cycle along the control edges, which is not a
retrograde cycle following the data edges. In other words, the data
flows solely in the same direction as the control flows. (Note that,
in general, there may be no data flow between two calls.)

For instance, the value of the recursive call in \(f(x,y) :=
f(x,g(y))\) is the value of~\(f(x,y)\) being defined, so the call is
tail recursive. On the other hand, the value of the call \(f(x-1)\) in
\(f(x) := x \times f(x-1)\) is not the value of the call \(f(x)\)
being defined because a multiplication by~\(x\) is pending, so it is
not tail recursive. The same holds for \(f(x) := g(x,f(x-1))\). Note
that, within the dynamic interpretation of recursion, the concept of
tail recursion applies to function calls, not to function definitions
as a whole, so it is technically incorrect to say that a function
definition is tail recursive.

Within the static understanding of recursion, it is not possible to
define tail recursion in general because only definitions may be
recursive and only calls may be in \emph{tail position}. The latter
refers to a syntactic criterion which implies that the value of a call
is only used to become the value of the current function being
called. In practice, however, it is possible to speak of a tail
recursive call when the static and dynamic interpretations agree, that
is, when a definition includes non\hyp{}ambiguously a call to the
function defined (a special case of static recursion) and that call is
in tail position. Nevertheless, since the very reason to distinguish
tail recursive calls is that they can often be compiled as efficiently
as loops are (a technique known as \emph{tail call optimisation}), the
interaction between the control flow and the data flow must be made
explicit anyway, even within a static framework, and this proves
challenging to students and professors alike. Even more puzzling is
the fact that the optimisation applies to non\hyp{}recursive calls as
well, as long as they are in tail position.

When a recursive call is not tail recursive, it is sometimes called an
instance of \emph{embedded recursion}. In theory, it is always
possible to rewrite any embedded recursion into tail recursion, but
the result can be rather hard to understand, hence difficult to design
directly. Moreover, in programming languages featuring conditional
loops (\exc{while}), recursion can be avoided in theory, but, in
practice, many algorithms are expressed more compactly or more legibly
if recursive.  A \emph{loop} is a segment of code syntactically
distinguished and whose evaluation is repeated until a condition on
the state of the memory is met. The syntactic condition, \emph{e.g.,}
a keyword and markers for a block, is meant to differentiate loops
from source code whose control flow relies on jumps (\exc{goto}) and
could actually be an unstructured implementation of loops (using
backward jumps), but are not loops. \emph{Iteration} is none other
than the concept of repetition applied to a piece of source code,
therefore, from a theoretical standpoint, it should include recursion
and loops, but, in practice, iteration is often used as a synonym for
the execution of a loop in an imperative language (\emph{looping}); in
a purely functional language, iteration is tail recursion. Conditional
loops (\exc{while}) and recursion have the same expressive power, so
using one form or the other is a matter of style as long as
side\hyp{}effects are allowed, because loops require a model of
computation where data is mutable.

As we mentioned earlier, some researchers prefer to define recursion
not on programs, but on \emph{processes}, that is, on the dynamic
interpretation of programs. For instance, \textcite{Kahney:1983}
defines recursion as a process ``that is capable of triggering new
instantiations of itself, with control passing forward to successive
instantiations and back from terminated ones.'' Of course, one data
structure suitable for implementing this mechanism is the
\emph{control stack}, which we already mentioned about ``recursion in
\plang{BASIC}'' \citep{Daykin:1974}. It is perhaps interesting to
notice the use of the ``forward'' and ``backward'' terminology about
the control flow on the call graph, although that graph is oriented
from callers to callees and there are no back edges because these
would not denote calls but returns. (Our own definition of dynamic
recursion is a cycle in the dynamic call graph, where ``backward''
qualifies the data flow superimposed on the call graph.) We will see
in a forthcoming section that this operational interpretation of
recursion can be suitably exploited by kinesthetic teaching. The
sections on analogies and mental models also revisit this
choice. Finally, when contrasting the static (syntactic) and dynamic
(control stack) definitions, it is worth keeping in mind that it is
possible to compile recursive definitions of functions in such a way
that the size of the control stack remains statically bounded; in
other words, recursion can always be transformed into iteration.

\paragraph{Teaching}

Clearly, recursion and loops are not mutually exclusive and may serve
the same purpose, which often bewilders the beginner. Consequently, a
simple attempt at a remedy consists in clearly separating the
different concepts at stake in the evaluation of a program
\citep{Velazquez:2000}, so that side\hyp{}effects, for instance, do
not get in the way of learning recursion declaratively. To teach the
difference between iteration and embedded recursion, some researchers
have proposed to teach how to translate an embedded recursive
definition into an iteration, while remaining in the same programming
language
\citep{AugensteinTenenbaum:1976,RubioVelazquez:2009,Rubio:2010,Rinderknecht:2012}. \textcite{Foltynowicz:2007}
went even further by deriving loops from embedded recursion, and vice
versa, which is of great theoretical and practical interest, in
particular for understanding compilers and interpreters. By exhibiting
a systematic way to move back and forth from recursion to loops, while
maintaining the meaning invariant, these didactic approaches aim at
demystifying recursion without resorting to a low\hyp{}level view of
evaluation with the control stack.

Finite iteration is unidirectional in the sense that the control flow
does not return to a previous program location where the environment,
\emph{i.e.}, the bindings of the variables to their values, is the
same. Embedded recursion is often called bidirectional when it is
based on the strict interpretation of the composition of functions, as
opposed to a non\hyp{}strict semantics, like lazy evaluation, which is
perhaps better explained by graph rewriting. Consider for instance
\(f(g(x))\), where~\(x\) is a value. First, the value of \(g(x)\) is
computed (control and data flow forward), that value is bound to an
implicit variable~\(y\) (control and data flow back) and then the call
\(f(y)\) is evaluated (control and data flow forward).

Finally, let us take note of a radical and contrarian view: to avoid
recursion as much as possible
\citep{Anonymous:1977,BunemanLevy:1980}. For instance,
\textcite{Harvey:1992} advocates the use of a functional style where
recursion is hidden inside higher\hyp{}order functions like maps and
folds. This is indeed the approach often taken when teaching purely
functional programming languages, especially those with a non-strict
semantics like \plang{Miranda} or \plang{Haskell}.

\section{Functional programming}

\textcite{Segal:1994} notes that, in the context of the functional
programming language \plang{Miranda}, ``by using the library of
functions as a toolbox, recursion, the underlying structure of many of
the functions and the only repetitive construct provided by the
language, can remain largely hidden.'' \textcite{Er:1984} argued that
recursion is made difficult by block\hyp{}structured programming
languages, which suggests that one way of encouraging the use of
desirable constructs, like recursion, would be to employ or develop
domain\hyp{}specific languages \citep{SinhaVessey:1992}; \emph{cf.}\@
\textcite{BrooksMillerRoperWood:1992}. It would then make sense to
teach recursion with functional languages, because these feature
prominently mathematical functions and immutable data, forcing the
programmer to think recursively
\citep{HendersonRomero:1989,Howland:1998}.

Because it is possible, for the purpose of teaching, to define a
semantics for functional languages based on term or graph rewriting,
\textcite{Velazquez:1999}, \textcite{ParejaUrquizaVelazquez:2007} and
\textcite{Rinderknecht:2012} can ask learners to trace by hand the
evaluation of their small programs. \textcite{Segal:1994} remarks that
``we would argue [...] that the ability to be able to evaluate a
recursive function mentally or `by hand' (that is, independent of a
machine), is an essential component of recursive knowledge for both
learners and experts.'' In the case of teaching higher\hyp{}order
functions, using manual reductions is also a recommendation of
\textcite{ClackMyers:1995}, who also list a long series of typical
errors and their analysis. Furthermore, \textcite{Burton:1995}
observes that
\begin{quote}
\it perhaps students are puzzled, unnecessarily, by the the language
(I refer to natural language here) with which we talk to them about
recursion. Peter Landin is fond of pointing out the numerous
inconsistencies with which such language is riddled (the phrase
``calls itself'', for instance, probably elides all kinds of different
semantic levels). An advantage of teaching via reduction sequences is
that it enables us to take the (natural) language out---just reduce,
reduce, reduce (perhaps with the aid of a machine).
\end{quote}
He also recommends what he calls a ``separation of concerns'' in
teaching at first list processing, pattern matching and recursion in
isolation: this avoids the issue for the students to assimilate
recursion at the same time as other imperfectly understood
concepts. \textcite{Velazquez:1999} also relies on term rewriting to
teach recursion before moving to recursion in an imperative language
with recursive data types.

By writing down the rewrite system in the exact order of a
top\hyp{}down design, students become accustomed to laying out calls
to functions yet to be defined; by also asking them to write down all
the left\hyp{}hand sides of the rules (patterns) before proceeding to
the right\hyp{}hand sides in random order, not only completeness is
improved, but also the conception of a program as a text written in
one pass is undermined, and the model of a form or a blueprint is
proposed instead. This twofold method seems to defuse a bit the
typical question of a recursive call (right\hyp{}hand side) to the
current function ``still under construction'', because at least all
the configurations of the input (left\hyp{}hand side) have been
already laid out and it is also normal to call yet undefined
functions, just like it is normal to have pending references in a map
being drawn to other parts yet to be filled. This view seems to be one
of the conclusions of \textcite{Vitale:1989}, when he writes, in
abstract terms:
\begin{quote}
\it It is proposed that a restricted notion of ``recursion'' could be
usefully defined, entailing:
\begin{enumerate}

  \item that the attitude of the subject, with respect to the
    definition of a notion, the solution of a problem, the answer to a
    question, etc., should contain a measure of \emph{suspended
      attention}, deferring in a way the final restructuring of the
    definition, solution, answer, etc., to the completion of a
    downward and then upward spiralling path;

  \item that the spiralling path should be describable by the
    dialectical coexistence of \emph{permanence} (the path, global
    because relying on the various steps) and \emph{change} (the pitch
    of the spiral, local because defined---and possibly changing---at
    every turn).

\end{enumerate}
\end{quote}
(For some technical corrections on the article of
\textcite{Vitale:1989} and some context on the relevance of recursion
in the cognitive sciences and artificial intelligence, see the
follow\hyp{}ups by \textcite{Trautteur:1989} and
\textcite{Apostel:1991}, as well as \textcite{Kieren:1989} in the
context of \plang{Logo}.) Furthermore, by using directed acyclic
graphs to represent programs and data, instead of abstract syntax
trees, \emph{aliasing} (data sharing) becomes visible and the control
stack and heap can arise from this model without resorting to
low\hyp{}level descriptions \citep{Rinderknecht:2012}.

Another approach, advocated by
\textcite{FelleisenFindlerFlattKrishnamurthi:2001}, consists in
systematically starting with the definition of recursive data types,
because such types already suggest the recursive structure of the
function definition to process their values. We will revisit this
method when presenting \emph{structural
  recursion}. \textcite{Pirolli:1986} showed that focusing the
teaching of recursion on the structure of the function definition is
more effective than insisting on the evaluation process, with traces
of the control and data flows.

When loops are taught after recursion in a functional language, no
\emph{transfer of skills} seems to be observed, undermining the idea
that iteration is inherently simpler than recursion
\citep{Mirolo:2011}. For an equivalent study with logic programming in
\plang{Prolog}, see \textcite{Haberman:2004}. Moreover, simple
functional programs on lists can be translated systematically in
\plang{Java} \citep{Rinderknecht:2012}, following design patterns
similar to those by \textcite{FelleisenFriedman:1997},
\textcite{Bloch:2003} and \textcite{Sher:2004}. The programs which are
derived are in static single assignment form and eschew the \exc{null}
from Pandora's vase \citep{Cobbe:2008,Hoare:2009}. However,
\textcite{Segal:1994}, \textcite{ClackMyers:1995} noted that inducing
students to think recursively with functional languages may yield some
of the problems encountered with imperative languages, and
\textcite{PazLapidot:2004} showed how prior experience with imperative
programming influences the learning of functional programming. This
brings us to examine when is recursion taught.


\section{Curricular approaches}

The scheduling of the teaching of recursion in school curriculums has
long been debated
\citep{Olson:1987,BarfurthRetschitzki:1987,Greer:1989}. For example,
\textcite{ZmudaHatch:2007} compare two approaches: the scheduling of
consecutive units of teaching on recursion versus the intermittent
teaching of recursion, whereby two units about recursion are separated
by a different topic.

\paragraph{Secondary schools}
\label{secondary_to_college}

In many countries, programming literacy, as opposed to vocational
training on software products (\emph{e.g.}, ICT in the United Kingdom
since the \oldstylenums{1990}s), is still absent in the secondary
schools curriculums. For instance, the French government officially
introduced it only in July \oldstylenums{2011}, as an option for
science majors, and recursion is not even mentioned in the new
regulation, whose implementation started in September
\oldstylenums{2012}. (The mathematics curriculum contains only one
paragraph about algorithms, which must be explicitly iterative
\citep{Modeste:2012}.) Wherever programming is featured in
introductory courses, recursion is usually avoided, even though it is
present in mathematics courses, usually in the guise of numerical
progressions, Euclid's algorithm, Newton\hyp{}Raphson approximation
method, and proofs by mathematical induction
\citep{Buck:1963}. Therefore, because university students often
experience significant difficulties in grasping recursive programming
\citep{Sooriamurthi:2001,Ginat:2004}, some educators have insisted on
a better articulation between secondary and post\hyp{}secondary
curriculums. For instance, some researchers have been promoting a
greater presence of discrete mathematics and proof techniques in
secondary schools
\citep{AbramovichPieper:1996,daRosa:2002,RosensteinFranzblauRoberts:1997,ZDM:A,ZDM:B},
as well as the creation of computing clubs with activities about
recursion \citep{GunionMilfordStege:2009a}. Others have emphasised the
duality between recursive programming and mathematical induction
\citep{Peelle:1976,Ford:1984,LeronZazkis:1986,Anderson:1992,BrandtRichey:2004,Polycarpou:2006},
which may be used as means to a transfer of skills from secondary
mathematics, as is, into college informatics. Even a reverse transfer
of skills, from recursive programming to problem solving in
mathematics, has been envisaged by \textcite{Hausmann:1985}.

The teachers gleaning recursive definitions in the fields of secondary
mathematics often come up with numerical progressions, including the
versatile Fibonacci numbers
\citep{RubioPajak:2006,RubioHernan:2007,Rubio:2008}, combinatorial
identities from Pascal's triangle, the pervasive factorial or the game
known as ``The Tower(s) of Hanoi (or Brahma).\@''
\citep{BunemanLevy:1980,Anderson:1992,BenanderBenander:2008}
Unfortunately, the pertinence of such examples is undermined by the
fact that they frequently enjoy closed forms (like \(1 + 2 + \dots + n
= n(n+1)/2\)) or they are computationally inefficient
\citep{Er:1984,Knight:1988,Costello:1990,Robertson:1999,Stojmenovic:2000,Manolopoulos:2005},
which may not be an issue for a mathematician. Furthermore, to
university students interested in programming or professional
training, these contrived exercises may appear useless and fail to
match their expectations, tainting recursion by association. The same
reaction is likely when outbidding with functions defined by more
complex recurrent equations, like McCarthy's ``91 function,''
Takeuchi's function \citep{Knuth:2000} or the simplified form of
Ackermann's function \citep{Robinson:1947,Robinson:1948}.

Fortunately, most textbooks avoid these pitfalls.

\paragraph{Textbooks}
\label{textbooks}

Since the aim of a textbook is to cover a given curriculum, it should
not come as a surprise that there are no textbook exclusively devoted
to recursive programming, but there have been some companion books, at
least up to the \oldstylenums{1990}s, when computer programming
entered mainstream education with the spread of personal
computers. (As mentioned before, during the same period, hardware
architectures and programming languages widely enabled recursion.)

In university education, from about \oldstylenums{1965} to
\oldstylenums{1975}, computer science emerged as a discipline
independent from mathematics, which explains the rigorous approach of
the books and the interest in theoretical explanations, as well as
low\hyp{}level implementations of recursion. This didactic choice was
enabled by the mathematical savvy of the students and the few
abstraction layers between programming languages and the hardware of
the time. For example, \textcite{Barron:1968} is concerned with the
pragmatics of recursion, its implementation in run\hyp{}time
environments, the comparison with iteration, the natural application
to sorting, the mechanisms for recursion in compilers and numerical
algorithms---all this with \plang{ALGOL}. \textcite{Burge:1975} starts
with \(\lambda\)\hyp{}calculus and combinatory logic, and proceeds
with the evaluation of mathematical expressions, the definition and
traversal of recursive data structures (lists and trees), parsing,
sorting algorithms---also in \plang{ALGOL}.

The following period, from about \oldstylenums{1975} to
\oldstylenums{1985}, saw recursion uprooted from theoretical grounds
and presented both as a method and a programming technique for solving
problems whose data structures are recursive (structural recursion),
making plain the benefit because the program structure itself then
matches that of the data it processes. For instance,
\textcite{Rohl:1984} begins with linked lists and binary trees,
explains the solving strategy ``divide and conquer'' (The input is
split, each non\hyp{}atomic part is recursively processed and the
partial solutions are finally combined to form the complete solution.)
and widens the scope to include mutual recursion
\citep{RubioUrquizaPareja:2008} and recursion on graphs---all with
\plang{Pascal}. \textcite{Roberts:1986} \citeyearpar{Roberts:2006}
wrote the most enduring book, first using \plang{Pascal} and now
\plang{Java}, where the main difference with previous volumes lies in
recursion being illustrated by drawing fractals and backtracking when
stuck in a labyrinth, whereas implementation issues make up the last
chapter only.


\paragraph{Methodology}
\label{methodology}

To tackle the understanding of the control flow, it is useful to work
on design methodology \citep{KesslerAnderson:1986}. Indeed, embedded
recursion is wrongly conceived as an expression of the familiar
counting or accumulation technique within loops, not the consequence
of the analysis of the original problem. As a remedy, students could
be taught to think declaratively when programming recursively in
imperative languages \citep{Giveon:1989,GinatShifroni:1999}, that is,
to distinguish specification (what) from evaluation (how)
\citep{Ford:1984}. Equivalently, this means that recursion could be
taught first as a method for solving problems (analysis and synthesis,
familiar to mathematicians since Antiquity), before showing it to be
also a programming technique
\citep{McKavanagh:1992,McKavanagh:2004}. In the same vein,
\textcite{Ginat:2005}, \textcite{GinatArmoni:2006} follow a principle
of P\'olya distinguishing \emph{working forwards}, which is a
heuristics consisting in approaching the solution by stepwise
deductions, and \emph{working backwards}, which supposes the goal
attained and concentrating on the inductive chain, back to the
problem. In the context of functional programming,
\textcite{Rinderknecht:2012} calls the first method
\emph{small\hyp{}step design} because the programmer focuses on the
least that can be done in one evaluation step towards the solution,
and the second \emph{big\hyp{}step design} because they assume that
the final value is obtained in one step and it has to be (recursively)
decomposed in terms of the input. In general, the first way leads to
iteration, whereas the second yields embedded recursion. These two
methods should be taught as complementary heuristics, because, for the
same problem, they may not bear definitions of commensurable
efficiency.

\paragraph{Curriculum}
\label{curriculum}

To overcome students' reluctance to use recursion within a course on
procedural or object\hyp{}oriented programming, it has been proposed
to teach singly\hyp{}linked lists before arrays and loops
\citep{TurbakRoydenStephanHerbst:1999,BruceDanylukMurtagh:2005,GoldwasserLetscher:2007},
which makes recursion appear as a rather natural way to move to and
fro inside a unidirectional data structure. It is not surprising that
this proposal, where recursion in data types comes before recursion in
functions, often originates from the context of object\hyp{}oriented
programming languages
\citep{FelleisenFriedman:1997,Levine:2000,Bloch:2003,Sher:2004}, but
is also prominent in statically typed functional languages---refer to
the book by
\textcite{FelleisenFindlerFlattKrishnamurthi:2001}. Indeed, when
generalised to other recursive data types, like trees, this kind of
recursion is called \emph{structural recursion} and, as mentioned
earlier, it yields programs reflecting the structure of the data type,
which is helpful since the latter is designed first. For instance, a
binary tree is either empty or made of a root and two subtrees, thus
the complete traversal of such a tree is expected to require a test
for the tree emptiness and two recursive calls.

\paragraph{Didactics}
\label{didactics}

Some researchers have been tackling the issue of teaching and learning
recursion through the lenses of cognitive sciences and psychology,
inferring the \emph{mental models} of recursion
\citep{SandersGalpinGotschi:2006,Mirolo:2009}, in particular the
faulty ones that novices construct by interacting with experts and the
problem to solve. As explained by \textcite{BhuiyanGreerGordon:1994},
a mental model is twofold: ``(1)~a knowledge structure in a person's
mind that incorporates \emph{descriptive knowledge} and
\emph{functional knowledge} about a concept or device; (2)~a control
mechanism that determines how this knowledge is used in problem
solving.'' Many of the references we gave in previous sections already
contain significant discussions and analyses of mental models, as they
are used as a rationale for guiding the design, for example, of a
tutoring system or a curriculum. In the introduction, we also have
mentioned \textcite{Giveon:1990}, who discusses some pedagogical
issues with the different meanings of the word \emph{recursion}, and
it is fitting now to cite as well \textcite{LobinaGarcia:2009} and
\textcite{Lobina:2011,Lobina:2012}, who bring forth a thoughtful
analysis of the usages of the same word in the cognitive sciences,
with an emphasis on linguistics and psychology. Indeed, these
disciplines are essential to the didactics of
programming. \textcite{LobinaGarcia:2009} write: ``In the 1950's,
linguists correctly employed recursion in reference to specific
rewrite rules, but ever since their elimination from linguistic
theory, most linguists have used recursion, rather puzzlingly, to
refer to those structures that recursive rewrite rules were used to
generate. This may well be the unfortunate legacy of employing rewrite
rules.'' Consequently, they recommend to reserve the term recursion
for processes, not the products of these, because not all hierarchy
(self\hyp{}embedding) is generated by recursive processes. With these
distinctions in mind, which will be touched upon again in the section
about analogies, it is further worth reading
\textcite{Kilpatrick:1985}, who discusses the analogical use of the
words \emph{reflection} and \emph{recursion} in the didactics of
mathematics.

According to the \emph{constructivist theory of learning}
\citep{WuDaleBethel:1998,Ben-Ari:2001} promoted by Jean Piaget,
learners construct mental models to understand the world and act
proactively, instead of passively reproducing a series of facts and
being enjoined belief in a theory, as happens with too many
traditional lectures. \textcite{InhelderPiaget:1963} write: ``the
source of thinking making possible to design recursive solutions to
problems lies in elemental forms of reasoning arising from students'
comprehension of the relations between the elements to which his/her
\emph{actions} are applied when attempting to solve instances of
problems.'' (The emphasis is ours.) The study of
\textcite{daRosa:2005} argues that the role of the teacher is to help
the student to transform this instrumental knowledge into a conceptual
knowledge, and finally into formalisation, that is, program
writing. Some researchers speak of ``misconceptions'', others do not
because they consider that these are simply transient stages,
non-viable conceptions---a \emph{viable} conception allowing to
predict the outcome of new
experiments. \textcite{GotschiSandersGalpin:2003} explain: ``Teachers
should generate perturbations in the students' existing conceptual
structures and hence foster new combinations of concepts. This means
that lecturers should present students with problems and examples that
challenge their current understanding and reveal non-viable
constructions.''

In the same vein, a \emph{constructionist theory}, developed by
\textcite{Papert:1980}, goes further by insisting that learning is
best or truly achieved by making tangible objects in interaction with
the environment, which includes the educator. These approaches do not
diminish in any way the role of the teacher, who is simply encouraged
to engage constructively with the pupils, and not to act as an oracle
or a judge. It is assumed that the learners build their knowledge
themselves, based upon previous idiosyncratic conceptions, which they
reassess by means of interactive experiments under the benevolent
supervision of an expert. Within this framework, where reassessment
entails either reinforcement or refutation, the self\hyp{}referential
nature of recursive definitions may seem a priori a cognitive
challenge, which \textcite{Papert:1960b} expresses as ``the property
of recursion being not the repetition of the same act as such, but the
repetition of an act that is at the same time the same act and a
different one.'' In fact, the interest in mental models of recursion
did not wait for the personal computers to reach homes and classrooms,
as it can be traced (in the context of the psychology of mathematics
first, and then computing) back to \textcite{Papert:1960a}
\citeyearpar{Papert:1960b} and Piaget
\citep{InhelderPiaget:1963,PiagetStratz:1974}. (See
\textcite{Matalon:1963,EliotLovellDaytonMcGrady:1979} also for early
research on children.) Children and adolescents were at the centre of
pedagogical investigations with the programming language
\plang{Logo}. One hypothesis of Papert is that the \emph{syntonicity}
enabled by \plang{Logo} helps the children to learn: ``Turtle geometry
is learnable because it is syntonic.\@'' \cite[p.~68]{Papert:1980}
Roughly speaking, syntonicity is a psychological feeling of
identification with a putative external agent, in this case the cursor
on the screen, called the turtle. This feeling, supported by the fact
that the movements of the turtle are relative to its current position
(\emph{cf.}\@ \plang{PostScript} below, in the section about
\plang{Logo}), entices the children to engage and enjoy what they
make, which is more than a drawing since it involves a (projected)
whole body experience.

\paragraph{Mental models}

According to \textcite{Kahney:1983}, \textcite{KahneyEisenstadt:1982},
the mental model of experts, called \emph{copies model}, is based on
dynamic instances of procedures, \emph{i.e.}, processes, either
passing (``forwards'') the control to newly created instances, or, if
terminated, returning it (``backwards'') to the instance who passed
it---\textcite{George:2000a} called the former \emph{active flow}, and
the latter \emph{passive flow}. The copies model is the only one
viable, that is, consistent with the operational semantics of
recursive definitions. Students, on the other hand, seem to often
build the \emph{looping model} of recursion, whereby embedded
recursion is wrongly understood as a kind of iteration and, typically
will consider the base cases as halting conditions
\citep{HabermanAverbuch:2002}. To reduce the risk of confusion,
\textcite{McDougall:1985} recommended that, when teaching
\plang{Logo}, the ``use of tail recursion for iterative situations be
deliberately avoided. [...] Avoidance of early use in programming of
tail recursion for repetition might avoid confusion with iteration in
children's mental models of recursion.'' Indeed, according to
\textcite{Tempel:1985}, ``other flavors of recursion may not be
encountered at all'' by the learners.

Experiments with experts and novices were set up by Kahney to validate
or refute the hypothesis that students had a looping mental
model. With high probability, it appeared that most of the students
held the looping model instead of the copies model, and some of them
had idiosyncratic models in mind, like the \emph{null model} (when
recursion is rejected), the \emph{syntactic model} (when the structure
of the program is used to predict its outcome, or, when writing it,
the necessity of base and recursive cases is understood, but not the
derivation of the actions), and the \emph{odd model} (when the meaning
of some keywords, \emph{e.g.,} \exc{EXIT} and \exc{CONTINUE}, is taken
from their English usage).

To explain the odd model, \textcite{PazLapidot:2004} suggest to
consider the interference of natural language in learning recursion,
in the context of learning \plang{DrScheme}:
\begin{quote}
\it It may be that some students attribute to the function the ability
to change the parameter's value, because of the association they
create between the programming language and natural language. It is
possible that students [...]  interpret the expression \exc{(first L)}
as, for example, `take the first element'. The meaning of taking the
first element, for them, is to extract it and drop the remaining
elements, so that \exc{L}~is left only with the first element.
\end{quote}
In the same vein, some researchers insist on bringing to the fore and
qualifying the linguistic aspect of the relationship between learners
and teachers. They set up experiments, record all interactions with
software and video, then analyse the transcripts to pinpoint the
misunderstandings, trace them back to plausible causes and try to
capture the mental model at work
\citep{AndersonPirolliFarrell:1984,LevyLapidot:2000,LevyLapidotPaz:2001,Levy:2001,MurnaneWarner:2001,LevyLapidot:2002}.
Furthermore, these verbal exchanges can be conducted not solely to
infer a mental model of the student and reach a diagnostic and remedy,
but even to become a maieutic process on its own right
\citep{ChangWangDaiSung:1999,ChangLinSungChen:2000}.

G\"otschi and some collaborators
\citep{Gotschi:2003,GotschiSandersGalpin:2003,SandersGalpinGotschi:2006}
refined and extended Kahney's classification of mental models; for
instance, they identified amongst their university students an
\emph{active model}, when they understand the instantiations of
recursive calls with smaller arguments and the reaching of the base
cases, but they nevertheless fail to grasp the backward, or passive,
flow of control from the completed instances to the current, pending
one. They also proposed the \emph{step model}, whereby students have
not a complete concept of recursive flow of control and execute only
one recursive call yielding a base case. There is also the
\emph{return value model}, which stems from misconceptions about when
the values of function calls are constructed. The two last models are
linked to some confusion about the evaluation of function calls in
general, like parameter passing and making a function's return value.

\textcite{BhuiyanGreerGordon:1989}
\citeyearpar{BhuiyanGreerMcCalla:1991} prefer the expression
\emph{mental method} instead of \emph{mental model} and proposed a
more detailed classification where \emph{generative methods} comprise
the \emph{loop method}, the \emph{syntactic method}, the
\emph{analytic method}, and the \emph{analysis/synthesis method};
moreover, \emph{trace methods}
\citep{Bhuiyan:1992,ScholtzSanders:2010} are used by students to
verify the correctness of their solutions.
(\textcite{GotschiSandersGalpin:2003} define a trace as ``a student's
representation of the flow of control and the calculation of the
solution of a recursive program.'') The loop method is the obvious
consequence of the flawed loop mental model. The syntactic method is
frequently used by novices who have little understanding of recursion
as a problem\hyp{}solving method, but a good declarative knowledge
about it. They know how to lie out a recursive template with base
cases and recursive cases fitting into simple categories, and it works
well for a wide variety of simple problems, but they have difficulties
for more complex ones, for example when \emph{generative recursion}
\citep{FelleisenFindlerFlattKrishnamurthi:2004} is needed, that is,
when a recursive call does not apply directly to a substructure of the
input, but to a transformed substructure. This issue is linked to a
lack of understanding of recursion as a design method, therefore, the
next method, \emph{i.e.,} the analytic method, applies to slightly
complex problems and proceeds from input and output requirements to an
intermediary solution, before writing the code. The analysis/synthesis
method goes further by dividing the problem into subproblems whose
solutions must be combined: this is the most general method. (See
earlier paragraph on methodology.) \textcite{dichevaClose:1996}
\citeyearpar{CloseDicheva:1997} focused on
misconceptions. \textcite{Wu:1993} and \textcite{WuDaleBethel:1998}
explored the learning of recursion in the framework of David Kolb's
model (\emph{experiential learning theory}), which we cannot detail
here. For yet other angles, like programming competences, concrete
vs. abstract models, static vs. dynamic copies model, classes of
recursive functions etc.\@ see \textcite{Er:1995},
\textcite{Burton:1995}, \textcite{Chen:1998} and
\textcite{Mirolo:2010}.

\textcite{AnzaiUesato:1982} found that children's understanding of a
recursive definition in the context of mathematics is eased by prior
experience with iteration, although they added that it may be the case
that \emph{writing} recursive definitions in a programming language
requires different, additional skills. \textcite{KesslerAnderson:1986}
worked in the context of programming languages and searched for
transfer of skills between tail recursion and iteration for novices
and they confirmed the conclusion of \textcite{AnzaiUesato:1982}: both
studies found a positive transfer from writing loops to writing
recursive definitions, but not vice versa (although tail recursion is
arguably too similar to iteration). Moreover, it seems that the
incorrect looping model of recursion, previously acquired on loops, is
more helpful than learning recursion directly. By contrast,
\textcite{Wiedenbeck:1988} found that previous knowledge of iterative
examples does not seem to facilitate subsequent learning on similar
recursive problems, although \emph{comprehension} was slightly
improved. Furthermore, \textcite{KurlandPea:1985} studied how
\oldstylenums{12}~year old subjects understood recursive definitions
and iterations in \plang{Logo}. They found that previous familiarity
with iteration helps understanding tail recursion but hampers the
correct grasping of embedded recursion, in accord with later work by
\textcite{Murnane:1992}. Note that this is not a direct contradiction
of \textcite{KesslerAnderson:1986}, because the latter used tail
recursion, and, for \textcite{Wiedenbeck:1988}, the transfer of skills
is about comprehension, not design.

The role of examples in learning recursion has been investigated by
\textcite{PirolliAnderson:1985}, \textcite{Wiedenbeck:1989},
\textcite{Pirolli:1991} and
\textcite{TasconRinderknechtKimKim:2010}. Examples should be used to
develop \emph{analogical} problem\hyp{}solving mechanisms, but care
must be taken not to rely too much on them too early, lest the
learners get stuck in the syntactic model of Kahney, and
\emph{knowledge compilation} mechanisms should also be built from past
experiences.

\section{Visualisation and animation}

Many educators try to capitalise on the fact that vision plays an
important role in acquiring concepts and informing their composition
to build new ones. This opens different lines of inquiry: visual
analogies of recursion, animating the evaluation of programs, visual
programming languages, integrated development environments, virtual
worlds and games.

\subsection{Analogies}
\label{analogies}

\paragraph{Objects}

It is often claimed that everyday life lacks analogies for the concept
of recursion \citep{PirolliAnderson:1985}, so it is no surprise that
most authors come up with the same objects, such as cauliflowers,
including the healthy broccoli, ringed targets, tree branches,
reflections on facing mirrors, tilings \citep{ChuJohnsonbaugh:1987},
ladders \citep{LevyLapidot:2002} and Russian dolls
\citep{BrewerSeagraves:1985}. Typical geometric figures are fractals
\citep{Riordon:1984c,ElenbogenOKennon:1988,Wakin:1989,BruceDanylukMurtagh:2005,Ammari:2005,Stephenson:2009a,Gordon:2006}
and certain kinds of artwork, most notably by the Dutch graphic artist
M.~C.~Escher \citep{GunionMilfordStege:2009b}. Their structures are
characterised by self\hyp{}replication with self\hyp{}embedding (also
called \emph{nesting}), but, unfortunately, these examples are perhaps
more likely to suggest infinity than recursion (whose evaluation must
terminate to be useful and, in the case of embedded recursion, may
require backtracking), and this involuntary association of infinity
and recursion may explain the avoidance of the latter by novices
\citep{Wiedenbeck:1989}. By contrast, and with a more optimistic tone,
\textcite{Papert:1980} (p.~71) wrote the following about an exercise
with \plang{Logo} aimed at demonstrating recursion:
\begin{quote}
\it Thus we have a trick called ``recursion'' for setting up a
never\hyp{}ending process whose initial steps are shown [...]. Of all
ideas I have introduced to children, recursion stands out as the one
idea that is particularly able to evoke an excited response. I think
this is partly because the idea of going on forever touches on every
child's fantasies and partly because recursion itself has roots in
popular culture. For example, there is the recursion riddle: If you
have two wishes what is the second? (Two more wishes.) And there is
the evocative picture of a label with a picture of itself. By opening
the rich opportunities of playing with infinity the cluster of ideas
represented by the [...] procedure puts a child in touch with
something of what it is like to be a mathematician.
\end{quote}
But it seems difficult to generalise this observation, as
\textcite{Turkle:1984} reports that ``Matthew, a good\hyp{}natured and
precocious child of five, was eagerly learning to write computer
programs to make graphic designs on the screen. His mood changed
abruptly and he left the computer in tears when he understood how to
make a recursive program: a program whose action includes setting in
motion an exactly similar program whose action includes setting in
motion an exactly similar program, and so on.''
\textcite{McDougall:1991}, also using \plang{Logo}, reported that
recursion in objects (figures produced by \plang{Logo} processes) is
firmly conceived by her daughter as different from recursion in
processes or programs, but claimed that this conception is
nevertheless useful because her daughter, who did not confuse
iteration and embedded recursion, used it for teaching a
peer. Earlier, \textcite{Thompson:1985}, also observing the dichotomy,
asked the students to describe verbally the recursive structure and
move towards the recursive \plang{Logo}
program. \textcite{Murnane:1991} discussed the demerits and merits of
various models of recursion, and also uses
\plang{Logo}. (Section~\ref{logo} will be devoted to \plang{Logo}.)

\paragraph{Processes}

Researchers have also looked at everyday examples of recursive
processes, instead of objects, for example, the fall of dominoes
aligned in a row, which seems to suggest recurrent reasoning to
children who are \oldstylenums{12}~years old or so, in the sense that
they almost express the fall of \emph{any} domino by the fall of the
previous one (a local property), but descriptions by younger children
are of the iterative type: the first domino falls and lets the second
fall, and so the third will fall etc.\@ \citep{PiagetStratz:1974} Of
course, the recursion suggested here by the experiment is tail
recursion, because it ends with the fall of the last
domino. Nevertheless, \textcite{Yang:2004} \citeyearpar{Yang:2008}
went further and claimed that such series of dominoes are an analogy
for \emph{linear recursion}, which is an embedded recursion with one
recursive call. More accurate are the processes which use
backtracking, a distinctive feature of embedded recursion, to model
the behaviour of an avatar or robot stuck in a labyrinth
\citep{LissMcMillan:1988,Dorf:1992,Roberts:2006}.

\textcite{Wirth:2008} provided an entertaining recursive method to
randomly park cars in a street, and \textcite{Brown:1972} tried to
familiarise social scientists with recursion through examples in
\plang{Logo}.

\textcite{Schiemenz:2002} came up with an application of recursion to
business management, with more examples of recursive objects and
recursive problem\hyp{}solving. \textcite{Kimura:1977} used businesses
too as a framework for explaining the notions of program, processor
and process. Embedded recursion is then expressed as the delegation of
a task to a group of assistants working on complementary sets of the
input. The same analogy is found in a paper by
\textcite{Edgington:2007}, and, if enacted by the students as
theatrical roles, it becomes \emph{kinesthesis} and a
multi\hyp{}sensory experience for learning recursion in the classroom
\citep{Dorf:1992,Levine:2000,BegelGarciaWolfman:2004,Katai:2009}. In
particular, \textcite{Ben-AriReich:1996} proposed to dramatise
recursive algorithms, that is, to associate the solution to a
real\hyp{}world task with an algorithm having the same recursive
structure, \emph{e.g.}, eating a chocolate bar and searching an
array. The students enact the solution and later write the
program. These playful activities can be considered as kinds of
parlour games, which leads us now to review computer games dedicated
to recursion.

\subsection{Computer games} 

There is an increasing interest for games, or game\hyp{}like features
(\emph{e.g.}, achievement badges), for supporting educational purposes
(so-called \emph{gamification} of education), even in higher
education. Although it was mentioned in
section~\ref{secondary_to_college} that ``The Tower(s) of Hanoi'' has
long been quite popular, very few studies have been carried out
specifically about teaching recursion with video games. Amongst them,
the setting of \textcite{RossiouPapadakis:2007} is a virtual
classroom, and \textcite{ChaffinDoranHicksBarnes:2009} designed a game
to facilitate the transfer of skills to writing recursive
programs. While very limited in time and number of participants, both
studies support the use of computer games for teaching and learning
recursion as a concept. As an alternative to games, recursive
processes can also be merely illustrated by a series of snapshots or
by an animation. The simplest form of visualisation consists in
augmenting the text of a program with semantic annotations and
pictures.

\subsection{Augmented text}
\label{augmented_text}

Since the \oldstylenums{1970}s, many graphical notations for inputs
and \emph{activation trees}, sometimes called \emph{recursion graphs},
have been proposed, allowing novices to record and follow the
evaluation of function calls
\citep{Jackson:1976,Kruse:1982,Haynes:1995,Hsin:2008}. For example, if
the input is a binary tree to be traversed, the activation tree is
also a binary tree because each node is a call to the same function
but with different subtrees as arguments. The teacher can show on the
blackboard the location in the data diagram at the same moment that
the activation tree is extended. \textcite{WeiMurray:2008} draw
activation trees within a hyperbolic geometry. Moreover, memory
allocation and variable assignments decorate the corresponding
\plang{Java} program. \textcite{KurtzJohnson:1985} animated the data
diagram only.

The syntax of many imperative languages, like \plang{Pascal}, is based
on blocks, which makes it hard to trace the execution of function
calls, particularly in the presence of recursion \citep{Er:1984}. This
leads some instructors to recur to a low\hyp{}level simulation of the
execution, reifying the otherwise invisible control stack
\citep{LeeMitchell:1985,Dupuis:1989}, but, according to
\textcite{GinatShifroni:1999}, this puts too much emphasis on the
computing model (see also the paragraph about methodology). See also
\textcite{Pirolli:1986}, whom we mentioned earlier in the section
about functional programming.

\textcite{BellGilbert:1974} proposed to use Wirth's syntax diagrams,
designed for specifying grammars of programming languages like
\plang{Pascal}. The usefulness of Backus\hyp{}Naur Forms (defining
context\hyp{}free languages) and Lindenmayer systems (L-systems) to
teach recursion has also been noted by \textcite{Er:1984},
\textcite{Proulx:1997} and \textcite{Velazquez:1999}
\citeyearpar{Velazquez:2000}. \textcite{Zelenski:1999} proposed the
generation of random sentences to experiment recursion and
\textcite{Levine:2000} then commented that students have no trouble at
all, perhaps because the textual expansion of a non\hyp{}terminal is
hardly seen as a procedure call, let alone calling itself.

For other closely related approaches, also based on annotations and
pictures, see \textcite{Er:1995}, \textcite{HuiIverson:1995},
\textcite{JehngTungChang:1999}, \textcite{George:1995}
\citeyearpar{George:1996} \citeyearpar{George:2000a}
\citeyearpar{George:2000b} and \textcite{TungChangWongJehng:2001},
some of whom we mentioned earlier about rewrite systems and functional
languages.

\subsection{Multimedia environments}
\label{multimedia}

Animation has been more widely implemented by means of dedicated
multimedia environments \citep{Rosenthal:2005}, either in isolation
(for didactic purposes only), or in connection with programming
environments \citep{WilcoksSanders:1994}. Here, we will only review
briefly those systems designed specifically to teach recursion.

\textcite{SternNaish:2002a} \citeyearpar{SternNaish:2002b} proposed a
classification based on an analysis of recursive algorithms for
sorting arrays and updating dictionaries: the first category groups
searches, the second sorts and the last insertions. They claim that
such distinctions enable the tailoring of better animations, aimed at
reinforcing the understanding of
recursion. \textcite{FernandezPerezVelazquezUrquiza:2007} proposed and
implemented an automated classification based on source code
inspection from which a dedicated animation is generated. That system
was developed extensively \citep{VelazquezPerezUrquiza:2008}
\citeyearpar{VelazquezPerezUrquiza:2009b}
\citeyearpar{VelazquezPerezUrquiza:2009a}
\citep{VelazquezPerez:2010}. The approach is practical and eclectic,
with animations not only of the activation tree, but also of the data
structure, the trace and the control stack.

Another direction is open by \emph{intelligent tutoring systems}
\citep{Pirolli:1986} (or \emph{interactive learning environments}),
which are multimedia environments that provide interactive feedback
and advice to the programmer. The system contains typical beginners'
strategies so it can comment upon the code being written
\citep{McCallaGreer:1993}. These strategies are based on mental models
of the learners. It seems that the system designed by
\textcite{Greer:1987} became a reference for
\textcite{BhuiyanGreerGordon:1989} \citeyearpar{BhuiyanGreerGordon:1992}
\citeyearpar{BhuiyanGreerGordon:1994}, \textcite{Bhuiyan:1992} and
\textcite{GreerMcCallaPriceHolt:1994}.

As miscellanea, see
\textcite{Moreno:1992,WuLinChiou:1996,WuLeeMei:1998}. For a structured
text editor guaranteeing the termination of recursive predicates in
\plang{Prolog}, see \textcite{BundyGrosseBrna:1991}.

\paragraph{Visual programming}

Visual programming languages enable the composition of program
constructs by manipulating graphical representations instead of
writing text. \textcite{GoodBrna:1996} were the first to investigate
whether these languages provided a better support for learning
recursion than textual languages, and concluded negatively.
Spreadsheet languages are sometimes considered as visual programming
languages or even functional languages, and
\textcite{BurnettRenKoCookRothermel:2001} focused on testing recursive
programs with them. \textcite{Kim:2003} proposed a string of classroom
exercises to learn recursion with \plang{Excel}.

\paragraph{Virtual worlds}

\textcite{TasconRinderknechtKimKim:2010} designed an interactive
interface based on a tangible block\hyp{}world with augmented reality
to learn iteration on lists and aiming at the transfer of skills to
directly write tail recursive definitions in \plang{Erlang}. An
earlier, three\hyp{}dimensional virtual world was designed by
\textcite{DannCooperPausch:2001}. For two\hyp{}dimensional geometry
considered as a virtual world, we have \plang{Logo}.

\subsection{The \plang{Logo} years}
\label{logo}

We would be remiss not to devote a whole section to \plang{Logo}. The
first thing that strikes the reader of the abundant literature about
\plang{Logo} is the enthusiasm that blows, page after page.
Microcomputers were arriving in the classrooms and everyone was
excited and deeply interested in their programming: teachers, of
course, but also psychologists, didacticians, mathematicians, computer
scientists, software companies, and even the children themselves,
whose education was the focal point of attention. The geometric
figures produced by the execution of \plang{Logo} programs, the design
underpinnings of the language, like its recursive, functional
programming style and its grounding in developmental psychology, all
this put \plang{Logo} at the confluence of almost all the streams
surveyed here: dynamic and geometric analogies for recursion, virtual
worlds (in a more abstract sense, they are called \emph{microworlds}
in the context of \plang{Logo}, like the microworld of the turtle, the
microworld of words, of lists etc.\@), integrated environments,
functional programming, games and theory of learning.
\textcite{Papert:1980} was a pioneer of this movement, taking part to
the design of \plang{Logo} at the end of the \oldstylenums{1960}s, and
we quoted him in section~\ref{analogies} about recursion.

\textcite{McDougall:1985} \citeyearpar{McDougall:1988}
\citeyearpar{McDougall:1989} \citeyearpar{McDougall:1990a}
\citeyearpar{McDougall:1990b} \citeyearpar{McDougall:1991} has used
\plang{Logo} to teach her nine\hyp{}year\hyp{}old daughter, who ended
mastering embedded recursion by age eleven. According to her, this
result confirmed what Papert conjectured, namely that young children
in a computer\hyp{}rich environment can learn abstract or formal
thinking---In passing, Papert never attributed this capability to
\plang{Logo} alone. Unfortunately, the size of the study group makes
it hard to generalise the findings. \textcite{Rouchier:1986a}
\citeyearpar{Rouchier:1986b} \citeyearpar{Rouchier:1987} observed
adolescents' difficulties in learning embedded recursion after
understanding loops and iteration, and proposed to start teaching
embedded recursion first. See also the articles by
\textcite{Barfurth:1987}, \textcite{BarfurthRetschitzki:1987}.

Following in the same footsteps, others
\citep{GobetNunezRetschitzki:1989,RetschitzkiGobetNunez:1989,GurtnerGexGobetNunez:1990,RetschitzkiGexGobetGurtnerNunez:1991}
noted that, in the geometric microworld of \plang{Logo}, it is
difficult to come up with exercises which show the superiority of
embedded recursion over iteration, whereas the microworld of lists is
more pertinent. Perhaps the reason is that drawing is inherently a
side\hyp{}effect, thereby it empowers loops. In the case of
\plang{PostScript}, a concatenative programming language dedicated to
graphics, the implicit evaluation stack is used for all computations,
including delaying the side\hyp{}effect of drawing, which is triggered
by an explicit \exc{showpage} instruction, so programming remains
declarative. Unfortunately, once embedded recursion has been wrongly
understood as an iteration in the turtle microworld, the
misunderstanding is carried over to the microworlds of words and
lists. Moreover, these researchers observed the same difficulties in
recognising the base cases (\exc{STOP} rule) as with any other
programming language. \textcite{Giveon:1991} presented a variant of
\plang{Logo} with multiple turtles that can move concurrently, and
advocated that this paradigm yields simpler recursive programs than
traditional (singly threaded) dialects of \plang{Logo}.

\section*{Conclusion}

The teaching and learning of recursion in computer programming courses
has long been a subject of inquiry, attracting a wide range of
researchers from many fields of knowledge. It is not possible to
isolate a current trend of investigation, as the hallmark of the
newest papers can already be found in the early \oldstylenums{1990}s,
although there seems to be a recent decline in the number of
publications and a concentration around a few researchers. Here are a
few points that may deserve some attention.
\begin{itemize}

\item Perhaps the common weakness of many experimental protocols lies
  in the small number of students (usually, one class), the short span
  of time (usually, one semester) and the difficulty to define a
  control group. Consequently, it may help to bring on board
  statisticians in order to design larger and longer experiments (at
  least a three\hyp{}year period).

\item Many studies lump all novices, whereas it seems useful to
  distinguish different profiles and cater them with different
  learning strategies, as some have proposed. But since the
  identification of the student mental model can only be achieved by
  teaching, this begets the question of \emph{adaptive teaching
    strategies}, once the student has been classified.

\item The approaches based on text rewriting (grammars, L-systems,
  rewrite systems) do not seem to raise issues with learners as far as
  recursion is concerned. It would be interesting to confirm this and
  explore whether the purported skills can be transferred to
  block\hyp{}structured programming languages.

\item Mutual recursion has been studied by Rubio\hyp{}S\'anchez and
  his colleagues \citep{RubioPajak:2006,RubioUrquizaPareja:2008}, who
  deemed it sometimes easier to teach than direct recursion. If
  confirmed, this would open a new way to teach direct recursion by
  program transformation (inlining)
  \citep{KaserRamakrishnanPawagi:1993}. Examples of mutual recursion
  arise naturally in parsers, which were a favourite example in early
  textbooks, and it was noted above that the derivation of sentences
  from formal grammars (that is, the reverse function of parsing)
  usually does not raise problems with recursion. Another use case is
  finite automata, as found in telecommunication protocols, vending
  machines, automatic teller machines etc. (One state is implemented
  by one function whose argument is any of the labels on the
  outgoing transitions.)

\item Kinesthesis and syntonicity seem to be helpful and should be
  compared with animation, as it may be that watching or imagining the
  execution of a recursive function (in other words, tracing) is
  cognitively different from involving one's own body, or a
  psychological representation of it. Perhaps augmented reality may
  help too, by creating an immersion
  \citep{TasconRinderknechtKimKim:2010}.

\item It should be impressed upon students that the control flow of
  recursion, which many authors qualify as being ``bidirectional'', is
  actually not specific to recursion by explaining the evaluation of
  arithmetic expressions with function compositions
  \citep{Burge:1975}. (In imperative languages where instructions are
  separated by semi\hyp{}colons, an instruction can be shown to be an
  implicit function---an assignment is indeed an operator in the
  \plang{C}~family---and a semi\hyp{}colon denotes composition.)

\item Many educators teaching recursion focus on the control flow,
  except perhaps if the language is object\hyp{}oriented, because, in
  that case, the \emph{data flow} becomes more relevant, and the
  design is more likely to be bottom\hyp{}up. (An algorithm ends up
  being scattered amongst several methods in different classes, so
  recursion is obscured by the amount of code to be read and mutual
  recursion is more likely.) That difference may explain why the
  professors teaching structural recursion on lists before arrays and
  loops are using some object\hyp{}oriented language or a functional
  language. Those teaching a top\hyp{}down design may end up
  reordering the definitions in the program to have them compiled
  incrementally for testing purposes, and also because this
  corresponds to the order of synthesis. (See the analysis/synthesis
  method.) By strictly lying down the top\hyp{}down design in the
  code, which requires, for example, to use prototypes in~\plang{C},
  or \exc{forward} declarations in \plang{Pascal}, the students get
  used to read incomplete programs. (The same can be said about using
  modules, of course.) Perhaps that skill is correlated with a better
  understanding of recursion.

\item Tail call optimisation should be explained without resorting to
  low\hyp{}level concepts~\citep{Rinderknecht:2012}.

\end{itemize}

\bigskip

\paragraph{Acknowledgements} The author thanks the following
researchers for providing preprints, hard copies and otherwise helpful
information: Nell Dale, C. Mitchell Dayton, Carlisle George, David
Ginat, Shafee Give'on, Fernand Gobet, Jean\hyp{}Luc Gurtner, K\'atai
Zolt\'an, Anne McDougall, John Murnane, Peter Pirolli, Fran\c{c}ois
Pottier, Ian Sanders, Manuel Rubio S\'anchez, Guiseppe Trautteur,
\'Angel Vel\'azquez and Michael Zmuda. The anonymous reviewers helped
improve the introduction.

% Bibliographic style
%
%\bibliography{ie2014}
\raggedright\printbibliography
\nocite{*}

\bigskip

\noindent\textbf{Christian Rinderknecht} received his M.Sc.~from
Universit\'e Pierre et Marie Curie (Paris, France). His doctoral
research at INRIA (French National Institute for Research in Computer
Science and Control) dealt with the application of formal methods to
telecommunication protocols and was defended in
\oldstylenums{1998}, at Universit\'e Pierre et Marie Curie. Since
\oldstylenums{2005}, he is an Assistant Professor at Konkuk University
(Seoul, Republic of Korea), in the Department of Internet and
Multimedia Engineering of the College of Information and
Telecommunication. He teaches programming languages and the analysis
of algorithms. His current research is about didactics of informatics.


\end{document}

on the \emph{control\hyp{}flow graph}, which is a a graph whose nodes
are labelled blocks of streamlined instructions instead of calls, and
edges are conditionals and jumps. This is a more detailed explanation
than our own definition of dynamic recursion as a cycle in the dynamic
call graph, where ``backward'' qualifies the data flow superimposed on
the call graph, and its purpose is to teach students not to treat
calls, in particular recursive ones, as if they were always in tail
position, without the need to explicitly define what tail calls are
