%%-*-latex-*-

\begin{abstract}
  We survey the literature about the teaching and learning of
  recursive programming. After a short history of the advent of
  recursion in programming languages and its adoption by programmers,
  we present curricular approaches to recursion, including a review of
  textbooks and some programming methodology, as well as the
  functional and imperative paradigms and the distinction between
  control flow vs.\@ data flow. We follow the researchers in stating
  the problem with base cases, noting the similarity with induction in
  mathematics, making concrete analogies for recursion, using games,
  visualizations, animations, multimedia environments, intelligent
  tutoring systems and visual programming. We cover the usage in
  schools of the \plang{Logo} programming language and the associated
  theoretical didactics, including a brief overview of the
  constructivist and constructionist theories of learning; we also
  sketch the learners' mental models which have been identified so
  far, and non\hyp{}classical remedial strategies, such as kinesthesis
  and syntonicity. We append an extensive and carefully collated
  bibliography, which we hope will facilitate new research.
\end{abstract}
