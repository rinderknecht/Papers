\section{Sorting $2^n$ keys}

Merging can be used to sort \emph{one} series of keys as follows. The
initial series of keys is split in two, then the two pieces are split
again etc. until singletons remain. These are then merged pairwise
etc. until only one series remains, which is inductively sorted, since
a singleton is a sorted series on its own and the merger of two series
is sorted if they are both sorted. The previous scheme leaves open the
choice of a splitting strategy and, perhaps, the most intuitive way is
to cut in two halves, which works well in the case of \(2^p\)~keys. We
will see in the next section how to deal with the general case.

Here, let us consider in \fig~\ref{fig:bot_up1}
\begin{figure}
\centering
\includegraphics{bot_up1}
\caption{Sorting \((7,3,5,1,6,8,4,2)\)
\label{fig:bot_up1}}
\end{figure}
all the mergers and their relative order to sort the series
\((7,3,5,1,6,8,4,2)\). We name this structure a \emph{merge tree},
because the nodes of the tree are sorted series, either singletons or
resulting from the merging of its two children. The root logically
holds the result. The merge tree is best understood from a
bottom\hyp{}up, level by level examination. Let us note \(\C{}{2^p}\)
the number of comparisons to sort \(2^p\)~keys and consider a merge
tree with \(2^{p+1}\)~leaves. It is made of two immediate subtrees
with \(2^p\)~leaves and the root holds \(2^{p+1}\)~keys. Therefore
\begin{equation*}
\C{}{1} = 0,
\quad
\C{}{2^{p+1}} = 2 \cdot \C{}{2^p} + \C{\Join}{2^p,2^p}.
\end{equation*}
Unrolling the recursion, we arrive at
\begin{equation}
\C{}{2^{p+1}} = 2^p\sum_{k=0}^p{\frac{1}{2^k}\C{\Join}{2^k,2^k}}.
\label{eq:cost_power_2}
\end{equation}

\paragraph{Minimum cost}

When the given series is already sorted, either in increasing or
decreasing order, the number of comparisons is minimum. In fact, given
a minimum\hyp{}comparison merge tree, any exchange of two subtrees
whose roots are merged leaves the number of comparisons
invariant. This happens because the merge tree is built bottom\hyp{}up
and the number of comparisons is a symmetric function. Let us note
\(\B{}{2^p}\) the minimum number of comparisons to sort
\(2^p\)~keys. From equations \eqref{eq:best_merge} and
\eqref{eq:cost_power_2}, we draw
\begin{equation}
%\abovedisplayskip=0pt
\belowdisplayskip=0pt
\B{}{2^p}
  = 2^{p-1}\!\sum_{k=0}^{p-1}{\frac{1}{2^k}\B{\Join}{2^k,2^k}}
  = p2^{p-1}.\label{eq:best_power}
\end{equation}

\paragraph{Maximum cost}

Just as with the best case, constructing a maximum\hyp{}comparison
merge sort is achieved by making worst cases for all the subtrees, for
example, \((7,3,5,1,4,8,6,2)\). Let~\(\W{}{2^p}\) be the maximum
number of comparisons for sorting \(2^p\)~keys. From equations
\eqref{eq:worst_merge} and \eqref{eq:cost_power_2}, we deduce
\begin{equation}
%\abovedisplayskip=2pt
\belowdisplayskip=0pt
\W{}{2^p}
  = 2^{p-1}\!\sum_{k=0}^{p-1}{\frac{1}{2^k}\W{\Join}{2^k,2^k}}
  = (p-1)2^p + 1.
\label{eq:worst_power}
\end{equation}

\paragraph{Average cost}

For a given series, all permutations of which are equally likely, the
average cost of sorting it by merging is obtained by considering the
average costs of all the subtrees of the merge tree: all the
permutations of the keys are considered for a given length. Therefore,
equation~\eqref{eq:cost_power_2} is satisfied by the average cost
\(\M{}{2^p}\), that is, the average number of comparisons for sorting
\(2^p\)~keys. Besides, equation~\eqref{eq:mean_merge} yields
\begin{equation*}
%\abovedisplayskip=2pt
%\belowdisplayskip=2pt
\M{\Join}{n,n} = 2n - 2 + \frac{2}{n+1}.
\end{equation*}
Together with equation \eqref{eq:cost_power_2}, we further draw, for
\(p > 0\),
\begin{align*}
\abovedisplayskip=0pt
\belowdisplayskip=0pt
\M{}{2^p}
  &= 2^{p-1}\sum_{k=0}^{p-1}{\frac{1}{2^k}\M{\Join}{2^k,2^k}}
  = 2^p\sum_{k=0}^{p-1}{\frac{1}{2^k}\left(2^k - 1 + \frac{1}{2^k +
      1}\right)}\notag\\
  &= 2^{p}\left(p - \sum_{k=0}^{p-1}{\frac{1}{2^k}}
     + \sum_{k=0}^{p-1}{\frac{1}{2^k(2^k+1)}}\right),\notag\\
\M{}{2^p}
  &= 2^{p}\left(p - \sum_{k=0}^{p-1}{\frac{1}{2^k}}
     + \sum_{k=0}^{p-1}\left(\frac{1}{2^k}
     - \frac{1}{2^k+1}\right)\!\!\right)
   = p2^p - 2^p \sum_{k=0}^{p-1}\frac{1}{2^k+1}\notag\\
  &= p2^p - 2^p \sum_{k \geqslant 0}\frac{1}{2^k+1}
     + 2^p \sum_{k \geqslant p}\frac{1}{2^k+1}
   = p2^p - \alpha 2^p + \sum_{k \geqslant 0}\frac{1}{2^{k}+2^{-p}},
\end{align*}
where \(\alpha := \sum_{k \geqslant 0}\frac{1}{2^k+1} \simeq
1.264500\) is irrational~\cite{Borwein:1992}. Since \(0 < 2^{-p} <
1\), we have \(1/(2^k + 1) < 1/(2^k+2^{-p}) < 1/2^k\) and we conclude
\begin{equation}
(p - \alpha)2^p + \alpha < \M{}{2^p} < (p-\alpha)2^p + 2.
\label{ineq:M_join}
\end{equation}
The uniform convergence of the series \(\sum_{k \geqslant
  0}\frac{1}{2^{k}+2^{-p}}\) allows us to interchange the limits
on~\(k\) and~\(p\) and deduce that \(\M{}{2^p} - (p-\alpha)2^p - 2 \to
0^{-}\), as \(p \to \infty\). In other words, \(\M{}{2^p}\) is best
approximated by its upper bound, for large values of~\(p\).
