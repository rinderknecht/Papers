\section{Conclusion}

We have shown that it is possible to obtain good bounds on the
minimum, maximum and average number of comparisons of merge sort
without resorting to advanced real analysis, like Fourier analysis,
nor complex analysis, like Mellin transforms. Whilst these powerful
techniques indeed bring the best results, they are not suitable for
postgraduate students in informatics who are introduced to sorting
algorithms.

Bachmann's \(O\)~notation is often misused for lower bounds and it is
deceptively simple because it must be abused to be really useful and
it seems to conflict with algebra over numbers. Using the
\(\Theta\)~notation when possible is better, but the extra work of
doing so then may become on par with determining asymptotic
equivalences, where multiplicative constants are explicit. It is
probably an overkill for most students to find out the exact linear
coefficients, but, for the most motivated, this paper shows how to do
so at the cost of slightly less precision sometimes.
