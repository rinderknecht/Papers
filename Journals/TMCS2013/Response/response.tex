\documentclass[10pt]{article}

\usepackage{hyphenat}
\usepackage{amsmath,amssymb,amsthm,stmaryrd}
\usepackage{url}

\title{Answers to the referee's report}

\author{Christian Rinderknecht}
\date{}

\newcommand\fig{\textsc{figure}}
\newcommand\Fig{\textsc{Figure}}
\newcommand\figs{\textsc{figures}}

\newcommand\Cost{\mathcal{C}}
\newcommand\Best{\mathcal{B}}
\newcommand\Worst{\mathcal{W}}
\newcommand\Mean{\mathcal{A}}

\newcommand\C[2]{\Cost_{#2}^{#1}}
\newcommand\B[2]{\Best_{#2}^{#1}}
\newcommand\W[2]{\Worst_{#2}^{#1}}
\newcommand\M[2]{\Mean_{#2}^{#1}}

\newcommand\floor[1]{\lfloor{#1}\rfloor}
\newcommand\ceiling[1]{\lceil{#1}\rceil}

%\newcommand\totaleq{\mathrel{\preccurlyeq^t\!}}
%\newcommand\total{\mathrel{\prec^t}}

%\newcommand\eqn[1]{\mathrel{\stackrel{\smash{#1}}{=}}}

\newcommand\mypar[1]{\fontsize{10}{12}\noindent\textbf{\emph{#1}.}}


% Main document
%
\begin{document}

\maketitle

\allowdisplaybreaks

\paragraph{About the Executive summary} In the introduction I try to
make the case, citing Knuth, that Bachmann's \(O\)~notation is often misused for lower bounds and it is
deceptively simple, as it must be abused to be really useful and it
seems to conflict with algebra over numbers. Using the
\(\Theta\)~notation when possible is better, but the extra work of
doing so then may become on par with determining asymptotic
equivalences, where multiplicative constants are explicit. As for the
relevance of finding out the exact linear coefficients, it is probably
an overkill for most undergraduate students. The average cost and its
linear terms may be of interest to postgraduate students with
theoretical leanings and can be carried out as a series of
homeworks. The point of this paper is to make this analysis easier, in
any case, at the cost of slightly less precision sometimes.

\paragraph{About the detailed comments}

\begin{itemize}

  \item ``Two minor points in the “Minimum cost” paragraph on page~2.
     [...]''\\
     You are right.

  \item ``Figure 1 on page 3 has errors in the first three
    rows. [...]''\\
    You are right.

  \item ``Another minor point [...]''\\
    Agreed.

  \item ``A very minor point: [...]''\\
    Yes.

  \item ``Equation (12) on page 6 needs to be better
    motivated. [...]''\\
    I clarified in the text. It comes from observing that, in the case
    \(n=2^p\), we have \(\B{}{n} = \frac{1}{2}n\lg n\), so we would
    like to know any additional linear correction in the general
    case. As for the strange notation \(\pred{L}{\cdot}\) for
    predicates, it is indeed not usual but, as a computer scientist, I
    like to distinguish the different levels of discourse more than
    mathematicians, who are more comfortable with slight notational
    ambiguities. (By the way, if you magnify the text, you will see
    that it has nothing to do with the notation for the absolute
    value.)
    
  \item ``I cannot derive the displayed inequality that follows
    equation~(13) on page~7. [...]''\\
    Please note that, for didactical reasons, I switch to a backward
    mode of reasoning here, as indicated by ``is implied by'' and
    ``the following inequality implies the result''. Hence, for
    instance,
\begin{equation*}
\tfrac{1}{2}(m+i)\lg(m+i) + a(m+i) + b
\leqslant
\tfrac{1}{2}m\lg m + (\tfrac{1}{2}i\lg i + ai + b) + i
\end{equation*}
and the inductive hypothesis 
\begin{equation*}
\tfrac{1}{2}m\lg m + (\tfrac{1}{2}i\lg i + ai + b) + i
\leqslant 
\B{}{m} + \B{}{i} + i = \B{}{m+i}
\end{equation*}
imply the inductive step we are looking for:
\begin{equation*}
\tfrac{1}{2}(m+i)\lg(m+i) + a(m+i) +
b \leqslant \B{}{m+i}.
\end{equation*}
In other words, this step is about finding a necessary condition so
that, together with the inductive hypothesis, the result follows.

Your point about the inexistence of a constant~\(a\) is mistaken: your
conditions are satisfiable by having \(a < 0\), which is exactly what
happens to be the case.

  \item ``I do not see where the next displayed inequality on
    page~7. [...]''\\
    If you gather the definitions in the text, namely \(m=2^p\) and
    \(i=x2^p\), you have \(i=mx\), which you substitute in 
\begin{equation*}
\tfrac{1}{2}(m+i)\lg(m+i) + a(m+i) + b
\leqslant
\tfrac{1}{2}m\lg m + (\tfrac{1}{2}i\lg i + ai + b) + i.
\end{equation*}
to obtain the equivalent inequality
\begin{equation*}
a \leqslant \Phi(x), \;\text{where \(\Phi(x) := \tfrac{1}{2}x\lg x - \tfrac{1}{2}(1+x)\lg(1+x) + x\)}.
\end{equation*}

\item ``On page~7, where determining \(a_{\text{max}}\) in line~6 from
  the bottom, [...]''\\
  Good catch!

  \item ``On page~7, the second line from the bottom, [...]''\\
    By definition of~\(x\) such that \(i=x2^p\).

  \item ``Similarly, where does the expression in the first line of
    page~8 [...]''\\
    We have \(\B{}{n} - \tfrac{1}{2}n\lg(\tfrac{3}{4}n) = \B{}{n} -
    \tfrac{1}{2}n\lg n + \left(\tfrac{1}{2}\lg\tfrac{4}{3}\right)n
    \geqslant \lg\tfrac{4}{3}\). We want to minimise the lefthand side
    \emph{for \(n\)~integer} to see how close it can get to the tight
    lower bound \(\lg\tfrac{4}{3}\) and determine the shape of
    these values for~\(n\).

  \item ``On page~9, it might be worth mentioning [...]''\\
    Indeed.

  \item ``I cannot derive equation~(19). [...]''\\
    What you do is assume equation~(19) and find another expression
    for \(\W{}{n}\). It does not match equation~(19), but both are
    nonetheless true, which would qualify as a theorem, I suppose. My
    expression for \(\W{}{n}\) is simpler, that is all.

  \item ``In the second bullet point on page~9, the notation \(\{x\}\)
    is used before it is defined.''\\
    Good catch!

  \item ``On page~10, why should inequalities~(9) suggest to look for
    bounds [...]''\\
    I added an explanation to the text. It reads as follows now:
    ``Inequalities~(9) are equivalent to \(n\lg n - \alpha n + \alpha
    < \M{}{n} < n\lg n - \alpha n + 2\) where \(n = 2^p\), and this
    suggests us to also look for bounds of the form \(n\lg n + an +
    b\) when \(n \neq 2^p\).''

  \item ``On page~10, the displayed equation [...]''\\
    Agreed.

  \item ``In the middle of page~11, the claim [...]''\\
    You are right, I explained how to obtain the limit.

\end{itemize}

\emph{Thank you very much for having taken the time to carefully read
  my preprint.}

\end{document}
