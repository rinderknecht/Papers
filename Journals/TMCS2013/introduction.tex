Knuth~\cite{Knuth:1996} reports that the first computer program,
designed in~\oldstylenums{1945} by the mathematician John von~Neumann,
was a sorting algorithm, nowadays called \emph{merge sort}. It is
amongst the most widely taught sorting algorithms because it
epitomises the important solving strategy known as \emph{divide and
conquer}: the input is split, each non\hyp{}trivial part is
recursively processed and the partial solutions are finally combined
to form the complete solution. Whilst merge sort is not difficult to
program, determining its efficiency, by means of a \emph{cost}
function, requires advanced mathematical knowledge. Therefore, most
textbooks \cite{GrahamKnuthPatashnik:1994,CLRS:2009} only show how to
find the order of growth of an upper bound of the cost from
recurrences it satisfies when \(n\)~is a power of~\(2\), but the
general case is often not presented in the main chapters, or not at
all, and precise analytic solutions are extremely difficult, making
use of complex analysis~\cite{FlajoletGolin:1994, Hwang:1998,
ChenHwangChen:1999}. Amongst the several variants of merge
sort~\cite{Knuth:1998, GolinSedgewick:1993}, we will be dealing here
with the most popular variant, called \emph{top-down merge sort}.

We derive lower bounds $\alpha n\log_2 n + \beta n + \gamma$ and upper
bounds $\alpha n\log_2 n + \beta' n + \gamma'$ for the extremum and
mean costs, relying only on basic discrete mathematics, intuitive
figures, elementary real analysis and mathematical induction. This is
much more precise than using Bachmann's notation and state \(O(n\log
n)\) in all cases. Worse, this notation is often misconstrued by
students, and occasionally by professionals, when it is used on the
main term of the asymptotic expansion, as noted by Knuth already
in~\oldstylenums{1976}: ``Unfortunately, people have occasionally been
using the \(O\)-notation for lower bounds, for example when they
reject a particular sorting method `because its running time is
\(O(n^2)\).' I have seen instances of this in print quite often
[...]''~\cite{Knuth:2000a}.
