\section{Conclusion}

We have shown how to derive the tightest bounds \(\alpha n\lg n +
\beta n + \gamma\), where~\(\alpha\), \(\beta\)~and~\(\gamma\) are
constants, on the cost of bottom-up merge sort using only the simplest
analytical concepts (real derivatives, intermediate theorem, cubic
equations, enumeration, induction), with computer scientists as an
intended readership. We proved that the top-down variant of merge sort
is always to be preferred (same minimum cost, but smaller maximum and
average costs). When the room allowed it, we even went out of our way
to characterise as accurately as possible the inputs for which the
costs are best approached or reached by the bounds.

These results are not new, and closed forms for the costs have been
published by researchers in the \oldstylenums{1990}s, but they require
advanced mathematical skills beyond most computer scientists and
programmers, and their derivations, when their missing steps are
actually filled in, are far longer than ours.

The minimum cost and the comparaison with the average cost of
top-down merge sort had already been published, but we included
their derivation, or a personal variation, in the hope this work may
serve as a reference for anyone interested in understanding precisely
the underpinnings of merge sort, perhaps as a stepping-stone to
more challenging research literature.

