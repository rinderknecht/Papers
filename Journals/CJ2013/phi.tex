%%-*-latex-*-

\documentclass[10pt]{article}

\usepackage[T1]{fontenc}
\usepackage{pst-infixplot,pstricks-add,pst-eps}

\newcommand\fig{\textsc{figure}}
\newcommand\Fig{\textsc{Figure}}
\newcommand\figs{\textsc{figures}}

\newcommand\Cost{\mathcal{C}}
\newcommand\Best{\mathcal{B}}
\newcommand\Worst{\mathcal{W}}
\newcommand\Mean{\mathcal{A}}

\newcommand\C[2]{\Cost_{#2}^{#1}}
\newcommand\B[2]{\Best_{#2}^{#1}}
\newcommand\W[2]{\Worst_{#2}^{#1}}
\newcommand\M[2]{\Mean_{#2}^{#1}}

\newcommand\floor[1]{\lfloor{#1}\rfloor}
\newcommand\ceiling[1]{\lceil{#1}\rceil}

%\newcommand\totaleq{\mathrel{\preccurlyeq^t\!}}
%\newcommand\total{\mathrel{\prec^t}}

%\newcommand\eqn[1]{\mathrel{\stackrel{\smash{#1}}{=}}}

\newcommand\mypar[1]{\fontsize{10}{12}\noindent\textbf{\emph{#1}.}}


\begin{document}

\TeXtoEPS
\psset{yunit=40mm}
\begin{pspicture}(-1.4,0.1)(10,-1.2)
% Axes
%
\psaxes[xlabelPos=top]{->}(10,-1.2)
%\qdisk(1,-1){2pt} % Test

% Abscissa
%
\rput[C](10,-0.07){\(p\)}

% Ordinates
%
\rput[rC](-0.3,-0.2645){\(1 - \alpha\)}
\rput[rC](-0.3,-0.5145){\(\myfrac3/4-\alpha\)}
\rput[rC](-0.3,-0.5978){\(\myfrac2/3-\alpha\)}
\rput[rC](-0.3,-0.7645){\(\myfrac1/2-\alpha\)}
\rput[rC](-0.1,-1.2){\(\Phi(p,i)\)}

%% Phi(p,3)
\psPlot[linewidth=0.7pt]{-0.2}{10}{x-1.264499+1-1/4+3/(2^x+1)-((2^x+3)*log(2^x+3)/log(2)-3*log(3)/log(2)-2)/2^x}

% Phi(p,2)
\psPlot[linewidth=0.7pt]{-0.2}{10}{x-1.264499+1-1/3+2/(2^x+1)-((2^x+2)*log(2^x+2)/log(2)-4)/2^x}

%% Phi(p,1)
\psPlot[linewidth=0.7pt]{-0.2}{10}{x-1.264499+1-1/2+1/(2^x+1)-((2^x+1)*log(2^x+1)/log(2)-2)/2^x}

% Asymptote of asymptotes
%
\psline[linewidth=0.7pt,linestyle=dashed](-0.1,-0.2645)(10,-0.2645)
\rput[l](3,-0.2){limit of asymptotes}

% Asymptotes
%
\rput[l](2.5,-0.68){asymptotes}

% i=3
\psline[linewidth=0.7pt,linestyle=dashed](-0.1,-0.5145)(10,-0.5145)
\psline[linewidth=0.7pt,linestyle=dotted](2,0)(2,-1.15)
\rput[l](9,-0.45){\(i=3\)}

% i=2
\psline[linewidth=0.7pt,linestyle=dashed](-0.1,-0.5978)(10,-0.5978)
\psline[linewidth=0.7pt,linestyle=dotted](1,0)(1,-1.15)
\rput[l](9,-0.68){\(i=2\)}

% i=1
\psline[linewidth=0.7pt,linestyle=dashed](-0.1,-0.7645)(10,-0.7645)
\rput[l](9,-0.85){\(i=1\)}

% Starting points
\psdot[dotsize=4pt,dotstyle=Bo](2,-1.1386)
\psdot[dotsize=4pt,dotstyle=Bo](1,-0.9311)
\psdot[dotsize=4pt,dotstyle=Bo](0,-0.2645)

\end{pspicture}
\endTeXtoEPS

\end{document}
