\section{Merging}

Merging consists in combining two ordered series of keys into one
ordered series. Without loss of generality, we shall be only
interested in sorting keys in increasing order. One way to achieve
this consists in comparing the two smallest keys, output the smallest
and repeat the procedure until one of the series becomes empty, in
which case the other is wholly appended. By definition, the cost of
merging is the number of comparisons performed. The following
results are well known~\cite{Knuth:1998}.

\paragraph{Minimum cost}
\label{merge:best_case}

The minimum cost is achieved when the shortest series comes first in
the result, so the minimum cost \(\B{\Join}{m,n}\) when merging series
of length \(m\)~and~\(n\) is
\begin{equation}
\B{\Join}{m,n} = \min\{m,n\}.\label{eq:best_merge}
\end{equation}

\paragraph{Maximum cost}

The maximum number of comparisons occurs when the last two keys of the
result come from two series. The maximum cost \(\W{\Join}{m,n}\) when
merging series of length \(m\) and~\(n\) is
\begin{equation}
\W{\Join}{m,n} = m + n - 1.\label{eq:worst_merge}
\end{equation}

\paragraph{Average cost}

The average cost of obtaining a given sorted series by merging two
sub-series is the average of the costs of merging all pairs of
sub-series, assuming that keys are not repeated. The average cost
\(\M{\Join}{m,n}\) when merging series of size \(m\) and~\(n\) is
\begin{equation}
\M{\Join}{m,n} = m + n - \frac{m}{n+1} - \frac{n}{m+1}.
\label{eq:mean_merge}
\end{equation}
