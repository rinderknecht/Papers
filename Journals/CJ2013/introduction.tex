\section{Introduction}

Merge sort features prominently amongst the sorting algorithms most
widely taught in colleges because it epitomises the important solving
strategy known as \emph{divide and conquer}: the input is split, each
non-trivial part is recursively processed and the partial
solutions are finally combined to form the complete solution. Whilst
merge sort is not difficult to program, expressing accurately the
number of comparisons in terms of the input size, namely the
\emph{cost}, requires advanced knowledge in the analysis of
algorithms.

In particular, the bottom-up variant of merge
sort~\cite{PannyProdinger:1995,Knuth:1998} calls for a deep
understanding of Fourier analysis and complex analytic
combinatorics~\cite{Delange:1975, FlajoletGolin:1994, Hwang:1998,
  ChenHwangChen:1999, FlajoletSedgewick:2009}. This version is often
implemented when linked-lists are sorted, commonly in purely
functional languages, and stability or onlining (the sorting process
is temporally interleaved with the input process) is
required~\cite{Okasaki:1998}.

Whilst mathematicians find Fourier and complex analysis simple enough,
we aim here at making the analysis of bottom-up merge sort more
accessible to computer scientists and also at better explaining its
relationship with the top-down variant, often used when the input is
an array or stability is not required on linked-lists. More precisely,
we derive the tightest bounds of the form \(\alpha n\lg n + \beta n +
\gamma\) on the extremal and average costs of bottom-up merge sort, by
relying only on calculus, discrete mathematics and mathematical
induction (considered `simple' here), at the expense of a closed form
on the linear terms, where~\(\beta\) is not a constant, but a function
\(\beta(\lg n)\). We think that this trade-off is acceptable because
the bounds are tight.

Moreover, we propose self-contained and very detailed derivations,
which is usually not a requirement for mathematicians, but this fact
should be kept firmly in mind when comparing the actual length of
previous proofs with our own.
