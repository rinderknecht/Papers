%%-*-latex-*-

This section completes the two previous sections which present the
constraint collection from types and subtypes and it describes the
solving procedure for the collected constraints. As a result, either
the constraints have no solutions (and the corresponding \ASN
specification must be rejected) or the value sets can be finitely
represented. It is straightforward to determine whether these value
sets are empty; if they are empty then the specification is rejected.

The constraint modeling an \ASN specification is 
$$\overline{\kappa} = \CEbigand{x \in {\cal X}}{\denotC{\I}{\TRef \,
(x)}{\TSEmap{}(x)}{\TSEmap{{\cal X} \backslash \{x\}}}}$$

\noindent
where ${\cal X} \subset {\cal R}$ is the finite set of the top-level
type names of the specification and $\TSEmap{} : {\cal X} \rightarrow
V$ maps these type names to fresh variables. The construction of a
constraint modeling an entire specification is convenient during the
collection (less parenthesis involved), but it is more comfortable for
the solving procedure to use a system of constraints.

\begin{Def}[Systems of constraints]\label{systems_of_constraints}
A \emph{system of constraints} is simply a set of constraints. The
system corresponding to the \ASN specification is $\Xi \,
(\overline{\kappa})$, where

\begin{center}
\begin{tabular}{rcl}
    $\Xi \, (\kappa_0 \CEand \kappa_1)$
  & = &
    $\Xi \, (\kappa_0) \, \cup \, \Xi \, (\kappa_1)$\\
    $\Xi (\kappa)$
  & = &
    $\{\kappa\}$ otherwise.
\end{tabular}
\end{center}

\end{Def}

We introduce the solving procedure as the semantics of a system of
constraints. In this aim, we first need to define a strict subset of
the expressions \E{} containing no variables: the \emph{terms}. They
can be characterised by the variables occurring in the expressions:

\begin{Def}[Variables in expressions]\label{variables_in_expressions} 
The countable set of variables $\alpha$, $\beta$, $\gamma$ etc. is
noted $V$. The variables of an expression $e$ are given by:

\begin{center}
\begin{tabular}{rcl}
$\VAR{\alpha}$
              & = & $\{\alpha\}$\\
$\VAR{\SEneg{e}}$
              & = & $\VAR{e}$\\
$\VAR{e_0 \, \SEcup \, e_1}$
              & = & $\VAR{e_0} \, \cup \, \VAR{e_1}$\\
$\VAR{e_0 \, \SEcap \, e_1}$
              & = & $\VAR{e_0} \, \cup \, \VAR{e_1}$\\
$\VAR{e_0 \, \SEdiff \, e_1}$
              & = & $\VAR{e_0} \, \cup \, \VAR{e_1}$\\
$\VAR{\Cons \, (e_0,e_1)}$
              & = & $\VAR{e_0} \, \cup \, \VAR{e_1}$\\
$\VAR{\Bind \, (l,e_0,e_1)}$
              & = & $\VAR{e_0} \, \cup \, \VAR{e_1}$\\
$\VAR{l \pmb{:} e}$
              & = & $\VAR{e}$\\
$\VAR{e \SEdiam \varsigma}$
              & = & $\VAR{e}$\\
$\VAR{e}$      
              & = & $\varnothing$ otherwise.
\end{tabular}
\end{center}
\end{Def}

Now we can define the ground expressions, or terms.

\begin{Def}[Terms]\label{terms} 
A \emph{term} $t$ is an element of the set $\Herbrand \triangleq \{e
\in \E \mid \VAR{e} = \varnothing\}$. The set of all subsets of
$\Herbrand$ is noted $\wp \,(\Herbrand)$.
\end{Def}

Note that the terms do not only correspond to the \ASN well-typed
values, for example the term $\Cons \, (\PosInt \, (0), \Cons \,
(\String \, \texttt{""}, \Nil))$ is not a typable term, i.e. there is
no \ASN type for  the \ASN value \verb+{+\texttt{\small 0,} \verb+""}+
denoted by this term.

Another necessary step is to characterise different kind of
expressions, because the constraints in which they appear are solved
by means of a specific algorithm: one for the integer intervals, one
for the real intervals, one for the regular expressions, one for the
powersets and one for the sets.

\begin{Def}[Characterization of integer intervals]
An expression $e \in \E$ denotes an integer interval if
$\chi_{\textnormal{I}} \, (e) = \ocamlkwd{true}$,
where

\medskip

\noindent
\begin{tabular}{l}
  $\ocamlkwd{let} \,\, \ocamlkwd{rec} \,\, \chi_{\textnormal{I}}
   = \ocamlkwd{function} \,\, \SEneg{e} \rightarrow
   \chi_{\textnormal{I}} (e)$\\
  \begin{tabular}{ll}
      $\mid$
    & \hspace*{-4mm}
      $e_0 \SEcup e_1 \mid e_0 \SEcap e_1 \mid e_0 
      \SEdiff e_1 \rightarrow \chi_{\textnormal{I}} (e_0)
      \,\, \textsf{\small \&\&} \,\, \chi_{\textnormal{I}} (e_1)$\\
      $\mid$
    & \hspace*{-4mm}
      $\alpha \,\, \ocamlkwd{when} \,\, \exists e \in \E.(\alpha \CEeq
       e) \in \Xi \, (\overline{\kappa}) \rightarrow
       \chi_{\textnormal{I}} (e)$\\
      $\mid$
    & \hspace*{-4mm}
      $\SEbot \mid \Interval \, \wildcard
       \rightarrow \ocamlkwd{true} \, \mid \,
       \wildcard \rightarrow \ocamlkwd{false}$
  \end{tabular}
\end{tabular}

\end{Def}

\begin{Def}[Characterization of real intervals]
An expression $e \in \E$ denotes a real interval if
$\chi_{\textnormal{F}} \, (e) = \ocamlkwd{true}$, where

\medskip

\noindent
\begin{tabular}{l}
  $\ocamlkwd{let} \,\, \ocamlkwd{rec} \,\, \chi_{\textnormal{F}}
   = \ocamlkwd{function} \,\, \SEneg{e} \rightarrow
   \chi_{\textnormal{F}} (e)$\\
  \begin{tabular}{ll}
      $\mid$
    & \hspace*{-4mm}
      $e_0 \SEcup e_1 \mid e_0 \SEcap e_1 \mid e_0 
      \SEdiff e_1 \rightarrow \chi_{\textnormal{F}} (e_0)
      \,\, \textsf{\small \&\&} \,\, \chi_{\textnormal{F}} (e_1)$\\
      $\mid$
    & \hspace*{-4mm}
      $\alpha \,\, \ocamlkwd{when} \,\, \exists e \in \E.(\alpha \CEeq
       e) \in \Xi \, (\overline{\kappa}) \rightarrow
       \chi_{\textnormal{F}} (e)$\\
      $\mid$
    & \hspace*{-4mm}
      $\SEbot \mid \wildcard \, \wildcard \,\, 
      \asnkwdconstr{..} \,\, \wildcard \, \wildcard 
      \rightarrow \ocamlkwd{true} \, \mid \,
      \wildcard \rightarrow \ocamlkwd{false}$
  \end{tabular}
\end{tabular}

\end{Def}

\begin{Def}[Characterization of regular expressions]
An expression $e \in \E$ denotes a regular expression if
$\chi_{\textnormal{R}} \, (e) = \ocamlkwd{true}$, where

\medskip

\noindent
\begin{tabular}{l}
  $\ocamlkwd{let} \,\, \ocamlkwd{rec} \,\, \chi_{\textnormal{R}}
   = \ocamlkwd{function} \,\, \SEneg{e} \rightarrow
   \chi_{\textnormal{R}} (e)$\\
  \begin{tabular}{ll}
      $\mid$
    & \hspace*{-4mm}
      $e_0 \SEcup e_1 \mid e_0 \SEcap e_1 \mid e_0 
      \SEdiff e_1 \rightarrow \chi_{\textnormal{R}} (e_0)
      \,\, \textsf{\small \&\&} \,\, \chi_{\textnormal{R}} (e_1)$\\
      $\mid$
    & \hspace*{-4mm}
      $\alpha \,\, \ocamlkwd{when} \,\, \exists e \in \E.(\alpha \CEeq
       e) \in \Xi \, (\overline{\kappa}) \rightarrow
       \chi_{\textnormal{R}} (e)$\\
      $\mid$
    & \hspace*{-4mm}
      $\Regexp \, \wildcard \rightarrow \ocamlkwd{true} \, 
      \mid \, \wildcard \rightarrow \ocamlkwd{false}$
  \end{tabular}
\end{tabular}

\end{Def}


\begin{Def}[Characterization of powerset expressions]
An expression $e \in \E$ denotes a powerset if $\chi_{\textnormal{P}}
\, (e) = \ocamlkwd{true}$, where

\medskip

\noindent
\begin{tabular}{l}
  $\ocamlkwd{let} \,\, \ocamlkwd{rec} \,\, \chi_{\textnormal{P}}
   = \ocamlkwd{function} \,\, \SEneg{e} \rightarrow
   \chi_{\textnormal{P}} (e)$\\
  \begin{tabular}{ll}
      $\mid$
    & \hspace*{-4mm}
      $e_0 \SEcup e_1 \mid e_0 \SEcap e_1 \mid e_0 
      \SEdiff e_1 \rightarrow \chi_{\textnormal{P}} (e_0)
      \,\, \textsf{\small \&\&} \,\, \chi_{\textnormal{P}} (e_1)$\\
      $\mid$
    & \hspace*{-4mm}
      $\alpha \,\, \ocamlkwd{when} \,\, \exists e \in \E.(\alpha \CEeq
       e) \in \Xi \, (\overline{\kappa}) \rightarrow
       \chi_{\textnormal{P}} (e)$\\
      $\mid$
    & \hspace*{-4mm}
      $e \SEdiam \varsigma \rightarrow \ocamlkwd{true} \,
       \mid \wildcard \rightarrow \ocamlkwd{false}$
  \end{tabular}
\end{tabular}

\end{Def}


\begin{Def}[Characterization of set expressions]
An expression $e \in \E$ denotes a set if $\chi_{\textnormal{S}} \,
(\varnothing) \, (e) = \ocamlkwd{true}$, where

\medskip

\noindent
\begin{tabular}{l}
  $\ocamlkwd{let} \,\, \ocamlkwd{rec} \,\, \chi_{\textnormal{S}} \,
   (\Path) = \ocamlkwd{function} \,\, \SEneg{e} \rightarrow
   \chi_{\textnormal{S}} \, (\Path) \, (e)$\\
  \begin{tabular}{ll}
      $\mid$
    & \hspace*{-4mm}
      $e_0 \SEcup e_1 \mid e_0 \SEcap e_1 \mid e_0 
      \SEdiff e_1 \rightarrow \chi_{\textnormal{S}} \, (\Path) \, (e_0)
      \,\, \textsf{\small \&\&} \,\, \chi_{\textnormal{S}} \, (\Path)
      \, (e_1)$\\
      $\mid$
    & \hspace*{-4mm}
      $\alpha \,\, \ocamlkwd{when} \,\, \alpha \in \Path \rightarrow
       \ocamlkwd{true}$\\
      $\mid$
    & \hspace*{-4mm}
      $\alpha \,\, \ocamlkwd{when} \,\, 
       \exists e \in \E.(\alpha \CEeq e) \in \Xi \,
       (\overline{\kappa}) \rightarrow 
       \chi_{\textnormal{S}} \, (\{\alpha\} \cup \Path) \, (e)$\\
      $\mid$
    & \hspace*{-4mm}
      $\Interval \wildcard 
       \mid \wildcard \, \wildcard \,\, \asnkwdconstr{..} \,\,
            \wildcard \, \wildcard  
       \mid \Regexp \wildcard 
       \mid e \SEdiam \varsigma \rightarrow \ocamlkwd{false}$\\
      $\mid$
    & \hspace*{-4mm}
      $\wildcard \rightarrow \ocamlkwd{true}$
  \end{tabular}
\end{tabular}

\end{Def}

\noindent
Now we can define the semantics of expressions.

\begin{Def}[Semantics of expressions]\label{semantics_of_SE}
A \emph{substitution} $\sigma$ is a function $\sigma : V \rightarrow
\wp \,(\Herbrand)$ from vari\-ables to sets of terms. The set of
substitutions is noted $\Sigma$. The standard semantics $\SEmu$ of
expressions is a mapping $\SEmu : \E \times \Sigma \rightarrow \wp \,
(\Herbrand)$ from the Cartesian product of expressions and
substitutions into sets of terms.
\end{Def}
\begin{itemize}

  \item 
    if $\chi_{\textnormal{I}} \, (e) = \ocamlkwd{true}$ then $\SEmu \,
    (e, \sigma) = \|e\|$, where\\ $\| \wildcard \,\| \colon \E
    \rightarrow \textsf{[$>$}
      \ocamltypename{clo\-sed\_int\_in\-ter\-val} \,\, \textsf{$\mid$}
      \,\, \SEbot \textsf{]}$ is a function defined as follows:

    \begin{center}
    \begin{tabular}{lll}
        $\| e \,\| =$ 
      & \hspace*{-4mm}
        $\ocamlkwd{let}$
      & \hspace*{-4mm}
        $\alpha \,\, \emph{be a fresh variable}$\\
      & \hspace*{-4mm}
        $\ocamlkwd{in}$
      & \hspace{-4mm}
        $\ocamlvaluename{solve\_integers} \, (\alpha \CEeqI e) \,
         (\alpha)$
    \end{tabular}
    \end{center}

  \item 
    if $\chi_{\textnormal{F}} \, (e) = \ocamlkwd{true}$ then $\SEmu \,
    (e, \sigma)$ is computed by a function similar to
    $\|\wildcard\,\|$, which applies to real intervals (not presented
    for the sake of brevity);
 
    \bigskip

  \item 
    if $\chi_{\textnormal{R}} \, (e) = \ocamlkwd{true}$ then

    \begin{center}
    \begin{tabular}{rcl}
        $\SEmu \, (e, \sigma)$
      & \hspace*{-4mm} = &\\
        \multicolumn{3}{l}{
          \begin{tabular}{ll}
              $\ocamlkwd{let}$
            & \hspace*{-4mm}
              $\alpha \,\, \emph{be a fresh variable}$\\
              $\ocamlkwd{in}$
            & \hspace*{-5mm}
              $\ocamlkwd{match} \,\, \ocamlvaluename{solve\_regexp} \,
               (\alpha \CEeqR e) \, (\alpha) \,\, \ocamlkwd{with}$\\
            & 
              $s \,\, \ocamlkwd{when} \,\, s \not= \textsf{""}
               \rightarrow \{\Regexp \, (s)\}$
          \end{tabular}
        }
    \end{tabular}
    \end{center}

  \item\label{semantics_of_PSE}
   if $\chi_{\textnormal{P}} \, (e) = \ocamlkwd{true}$ then $\SEmu$ is
   defined by the equations

   \begin{center}
   \begin{supertabular}{rcl} %XXX
       $\SEmu (\alpha, \sigma)$
     & = &
       $\sigma \, (\alpha)$\\
       $\SEmu (e \SEdiam \varsigma, \sigma)$ 
     & = & 
       $\{t \SEdiam \varsigma \mid t \in \SEmu (e, \sigma)\}$\\
       $\SEmu (\pi_0 \SEcap \pi_1, \sigma)$
     & = &
       $\ocamlkwd{let} \,\, \overline{\pi} = \SEmu (\pi_0, \sigma),
        \SEmu (\pi_1, \sigma)$\\
     & &
       $\ocamlkwd{in} \,\, \varphi \, (\overline{\pi}, \SEcap,
        \SEcap)$\\  
       $\SEmu (\pi_0 \SEdiff \pi_1, \sigma)$ 
     & = &
       $\ocamlkwd{let} \,\, \overline{\pi} = \SEmu (\pi_0, \sigma),
        \SEmu (\pi_1, \sigma)$\\ 
     & &
       $\ocamlkwd{in} \,\, \varphi \, (\overline{\pi}, \SEcap,
        \SEdiff) \, \disjunion \, \varphi \, (\overline{\pi},
        \SEdiff, \ocamlvaluename{fst})$\\
       $\SEmu (\pi_0 \SEcup \pi_1, \sigma)$
     & = &\\
       \multicolumn{3}{r}{
         \begin{tabular}{ll}
             $\ocamlkwd{let}$
           & \hspace*{-4mm}
             $(\overline{\pi}_0, \overline{\pi}_1) \AS \overline{\pi}
              = \SEmu (\pi_0, \sigma), \SEmu (\pi_1, \sigma)$\\
             $\ocamlkwd{in}$
           & \hspace*{-5mm}
             $\varphi \, (\overline{\pi}, \SEcap, \SEcup) \,
              \disjunion
              \, \varphi \, (\overline{\pi}, \SEdiff,
              \ocamlvaluename{fst}) \, \disjunion \, \varphi \,
              ((\overline{\pi}_1, \overline{\pi}_0), \SEdiff,
              \ocamlvaluename{fst})$ 
         \end{tabular}
       }\\
   \end{supertabular}
   \end{center}

The second equation relies on the semantics of set expressions
presented in definition~\ref{semantics_of_SE} ($e$ ranges over the set
expressions). The second equation gives the semantics of the
intersection of two powersets. $\overline{\pi}$ is the pair of the
semantics of $\pi_0$ and $\pi_1$. The function $\varphi$ is defined as
follows:

\vspace*{-4mm}

\begin{center}
\begin{tabular}{rcl}
    $\varphi \, ((\overline{\pi}_0, \overline{\pi}_1), f, g)$
  & \hspace*{-4mm} $\triangleq$
  & \hspace*{-4mm}
    $\{e \SEdiam i \mid e_0 \SEdiam i_0 \in \overline{\pi}_0, 
       e_1 \SEdiam i_1 \in \overline{\pi}_1,$\\
    \multicolumn{3}{r}{
      $e \in \|g (e_0, e_1)\| \backslash \{\SEbot\}, i \in \|f (i_0,
       i_1)\| \backslash \{\SEbot\}\}$ 
    }
\end{tabular}
\end{center}

\noindent
The first argument of $\varphi$ is the pair $(\overline{\pi}_0,
\overline{\pi}_1)$ of semantics. The second, $f$, is the set operator
to be applied to the intervals of the elements of $\overline{\pi}_0$
and $\overline{\pi}_1$. The last, $g$, is the set operator to be
applied to the set expressions of $\overline{\pi}_0$ and
$\overline{\pi}_1$. For instance, $\varphi \, (\overline{\pi}, \SEcap,
\SEcap) = \{e \SEdiam i \mid e_0 \SEdiam i_0 \in \overline{\pi}_0, e_1
\SEdiam i_1 \in \overline{\pi}_1, e \in \|e_0 \, \SEcap \, e_1\|
\backslash \{\SEbot\}, i \in \|i_0 \, \SEcap \, i_1\| \backslash
\{\SEbot\}\}$. The intuition is that the intersection of two powersets
is the powerset whose elements are the intersection of the initial
elements with the \emph{same} cardinal (hence we compute the
intersection of the intervals).

The fourth equation defines the semantics of the difference of two
powersets: $\varphi \, (\overline{\pi}, \SEcap, \SEdiff) \, \disjunion
\, \varphi \, (\overline{\pi}, \SEdiff, \ocamlvaluename{fst})$. The
symbol $\disjunion$ is the disjunctive set union $\cup$. The elements
of the powerset $\varphi \, (\overline{\pi}, \SEcap, \SEdiff)$ are the
difference between the initial elements of same cardinal. The elements
of the powerset $\varphi \, (\overline{\pi}, \SEdiff,
\ocamlvaluename{fst})$ are the elements of $\overline{\pi}_0$ whose
cardinals are different from the cardinals of the elements of
$\overline{\pi}_1$. The rationale of the equation comes from the
trivial formula $A = (A \, \cap \, B) \, \disjunion \, (A \,
\backslash \, B)$.

The last equation defines the semantics of the union of two powersets
as $\varphi \, (\overline{\pi}, \SEcap, \SEcup) \, \disjunion \,
\varphi \, (\overline{\pi}, \SEdiff, \ocamlvaluename{fst}) \,
\disjunion \, \varphi \, ((\overline{\pi}_1, \overline{\pi}_0),
\SEdiff, \ocamlvaluename{fst})$. First, the elements of the powerset
$\varphi \, (\overline{\pi}, \SEcap, \SEcup)$ are the union of the
initial elements of same cardinal. Second, the elements of the
powerset $\varphi \, (\overline{\pi}, \SEdiff, \ocamlvaluename{fst})$
are the elements of $\overline{\pi}_0$ whose cardinals are different
from the cardinals of the elements of $\overline{\pi}_1$. Last, the
elements of the powerset $\varphi \, ((\overline{\pi}_1,
\overline{\pi}_0), \SEdiff, \ocamlvaluename{fst})$ are the elements of
$\overline{\pi}_1$ whose cardinals are different from the cardinals of
the elements of $\overline{\pi}_0$. The rationale of this equation is
the formula: $A \, \cup \, B = (A \, \cap \, B) \, \disjunion
\, (A \, \backslash \, B) \, \disjunion \, (B \, \backslash \, A)$.

  \item 
    Otherwise, if $\chi_{\textnormal{S}} \, (\varnothing) \, (e) =
    \ocamlkwd{true}$ then $\SEmu$ is defined by the following
    equations:

\begin{center}
\begin{supertabular}{rcl} %XXX
$\SEmu(\SEbot,\sigma)$ & = & $\varnothing$ \\
$\SEmu(\SEtop,\sigma)$ & = & $\Herbrand$ \\
$\SEmu(\alpha,\sigma)$    & = & $\sigma \, (\alpha)$ \\
$\SEmu(e_0 \, \SEcup \, e_1, \sigma)$
              & = & $\SEmu(e_0,\sigma) \, \cup \, \SEmu(e_1,\sigma)$ \\
$\SEmu(e_0 \, \SEcap \, e_1,\sigma)$
              & = & $\SEmu(e_0,\sigma) \, \cap \, \SEmu(e_1,\sigma)$ \\
$\SEmu(\SEneg{e},\sigma)$ 
              & = & $\Herbrand \, \backslash \, \SEmu(e,\sigma)$ \\ 
$\SEmu(e_0 \SEdiff e_1, \sigma)$
              & = & $\SEmu(e_0 \, \SEcap \, \SEneg{e_1}, \sigma)$\\
$\SEmu(\ocamlvaluename{l}\/ \pmb{:} e,\sigma)$
              & = & $\{ \ocamlvaluename{l}\/ \pmb{:} t \mid t \in
		    \SEmu(e,\sigma) \}$ \\
$\SEmu(\Cons \, (e_0, e_1),\sigma)$
              & = &\\
\multicolumn{3}{r}{$\{\Cons \, (t_0, t_1) \mid 
  t_0 \in \SEmu(e_0,\sigma), t_1 \in \SEmu(e_1,\sigma)\}$}\\
$\SEmu(\Bind \, (\ocamlvaluename{y}, e_0, e_1),\sigma)$
              & = &\\
\multicolumn{3}{r}{$\{\Bind \, (\ocamlvaluename{y}, t_0, t_1) \mid
  t_0 \in \SEmu(e_0,\sigma), t_1 \in \SEmu(e_1,\sigma)\}$}\\
$\SEmu(\ocamlvaluename{e},\sigma)$
              & = & $\{ \ocamlvaluename{e} \}$ otherwise.\\
\end{supertabular}
\end{center}

\end{itemize}

Now we can define the semantics of constraints on top of the semantics
of expressions.

\begin{Def}[Semantics of constraints]
We define the semantics $\CEmu : {\cal K} \times \Sigma \rightarrow
\ocamltypename{bool}$ of constraints as a predicate on the Cartesian
product of constraints and substitutions:

\begin{center}
\begin{tabular}{rcl}
  $\CEmu(\kappa_0 \CEand \kappa_1,\sigma)$
    & = & $\CEmu(\kappa_0,\sigma) \BOOLand
          \CEmu(\kappa_1,\sigma)$ \\ 
  $\CEmu(e_0 \, \CEsubseteq \, e_1,\sigma)$
    & = & $\SEmu(e_0,\sigma) \subseteq \SEmu(e_1,\sigma)$\\
  $\CEmu(e_0 \, \CEeq \, e_1,\sigma)$
    & = & 
  $\CEmu((e_0 \, \CEsubseteq \, e_1)
         \CEand 
         (e_1 \, \CEsubseteq \, e_0),
         \sigma)$
\end{tabular}
\end{center}

\noindent
where $\!\!\BOOLand\!\!$ is the logical boolean operator and
$\subseteq$ is the inclusion over mathematical sets (of terms).
\end{Def}

\begin{Def}[Solutions]
The \emph{set of solutions} ${\cal S}(\kappa)$ of a constraint
$\kappa$ is the set of all substitutions that satisfy the semantics of
$\kappa$: ${\cal S}(\kappa) = \{ \sigma \in \Sigma \, \mid \,
\CEmu(\kappa,\sigma) \}$.
\end{Def}

The algorithm constructing the solutions of a system of set
constraints may be the algorithm which was published by Aiken and
Wimmers in 1992~\cite{AikenWimmers:1992}. We constructed our
constraint expressions in order to almost exactly fit the input of
this algorithm (see definition~\ref{expressions}). The only thing to
do is to replace $\E_0 \, \SEdiff \, \E_1$ by $\E_0 \, \SEcap \,
\SEneg\E_1$, since these two set expressions have the same semantics
(see definition~\ref{semantics_of_SE}).

\subsection{Worst-case complexity analysis}

Let us say some words about the complexity of the solving
procedure. The collecting algorithm we have presented is obviously not
optimised, in any way. For instance, a type which is constrained by a
combination of subtyping constraints will be analysed each time a
basic constraint is analysed. The reason for this is that we want to
provide an algorithm that can also be considered as a reference formal
model, thus it must be as readable as possible. Therefore, any
optimisation is postponed until the implementation phase (as using
hash-consing or caches to solve the mentioned problem). Anyway, the
reader can convince himself that the worst case complexity of the
collecting algorithm is proportional to the size of the subtypes plus
the complexity of the computations on integer intervals and regular
expressions.

The worst case complexity of the solving procedure is the same as the
complexity of Aiken and Wimmers' algorithm. In their paper, they prove
that their algorithm is in class NEXPTIME (they also published,
together with Kozen and Vardi, a general study about the complexity of
solving set constraints~\cite{AikenKozenVardiWimmers:1993}). This
result may be very disappointing, but it is inherent to the high
expressiveness of \ASN subtyping, in particular the use of a
complement operator (hence not monotonic) with indiscriminate unions
and intersections. As a future work, it would be interesting to
implement our algorithm and to make some benchmarks. Also, there
exists constraint subclasses with polynomial complexity in the
worst case and it would be worth studying whether it is possible, in
practice, to model \ASN with them.
