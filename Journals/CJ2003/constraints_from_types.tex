%%-*-latex-*-

In this section we define the collection of constraints from types
(i.e. declarations of pattern {\small $<$\textsf{type reference}$>$
\textsf{::=} $<$\textsf{non-reference type without inner
constraints}$>$}). It is a mapping ${\cal T} \rightarrow V \rightarrow
(\R \rightarrow V) \rightarrow {\cal K}$, where ${\cal T}$ is the set
of types, $V$ is the set of variables (ranged over by $\alpha$,
$\beta$, $\gamma$ etc.), $\R$ is the countable set of type reference
names (ranged over by $x$) and ${\cal K}$ is the set of
constraints. The notation for the mapping is
$\TSE{\varsigma}{\T}{\alpha}{\TSEmap{}}$, where \T{} is the analysed
type, $\alpha$ is the variable denoting the terms corresponding to the
values of \T{} and $\TSEmap{}$ is a mapping from type reference names
into the variables, which allows the proper handling of the recursive
types. For the sake of clarity, we write sometimes $\TSEmap{\R}$ to
show that the domain of $\TSEmap{}$ is $\R$.

For instance: $\TSE{\varsigma}{\kwdNULL}{\alpha}{\TSEmap{}} = (\alpha
\CEeqS \kwdNULL)$ means that the set constraint associated to the
\kwdNULL{} type, whose set of terms is denoted by $\alpha$, is $\alpha
\CEeqS \kwdNULL$; that is to say the terms are exactly the \kwdNULL{} set
constant. The solution of this constraint is the substitution that
maps the variable $\alpha$ to the set $\{\kwdNULL{}\}$. Let us consider
the constraint collection from the \kwdCHOICE{} type:

\begin{center}
\begin{tabular}{rl}
$\TSE{\varsigma}{\kwdCHOICE \,\,
    \FieldEnv{\{l\}}}{\alpha}{\TSEmap{}}$ = 
 & \hspace*{-3mm}
   $\ocamlkwd{let} \,\, \beta \,\, \emph{be a fresh variable}$\\ 
 & \hspace*{-3mm}
   $\ocamlkwd{in} \,\,
   \TSE{\varsigma}{\FieldEnv{}(l)}{\beta}{\TSEmap{}} 
   \CEand 
   \alpha \CEeqS l \pmb{:} \beta$
\end{tabular}
\end{center}

This equation is for \kwdCHOICE{} types having only one component,
labeled $l$. The expression corresponding to the component
$\FieldEnv{}(l)$ is denoted by the variable $\beta$ which is used to
build the constraint for the \kwdCHOICE{} type.

\medskip

\noindent
\begin{tabular}{rcl}
     $\TSE{\varsigma}{\kwdCHOICE \,\, \FieldEnv{\{l,m\} \disjunion
      \Labels}}{\alpha}{\TSEmap{}}$
   & \hspace*{-4mm} = & \\
     \multicolumn{3}{r}{
       \begin{tabular}{ll}   
            $\ocamlkwd{let}$ 
         & \hspace*{-4mm} 
           $\beta \,\, \emph{and} \,\, \gamma \,\, \emph{be fresh
            variables}$\\  
           $\ocamlkwd{in}$ 
         & \hspace*{-4mm}
           $\TSE{\varsigma}{\kwdCHOICE \,\, \FieldEnv{\{m\} \disjunion
            \Labels}}{\gamma}{\TSEmap{}} \CEand
            \TSE{\varsigma}{\FieldEnv{}(l)}{\beta}{\TSEmap{}}$\\
         & \hspace*{-4mm}
           $\CEand \alpha \CEeqS (l \pmb{:} \beta) \, \SEcup \,
            \gamma$
       \end{tabular}
     }
\end{tabular}

\medskip

This equation is for \kwdCHOICE{} types having at least two
components, $l$ and $m$. First, we compute the set constraint for the
\kwdCHOICE{} with the same components except $l$ (the expression is
named $\gamma$). Then there is the constraint for the component $l$
(variable $\beta$), and finally the constraint on the whole
\kwdCHOICE{} type, whose expression is $\alpha$ and is the 
union of the expression of the component $l$ and $\gamma$.

\medskip

\noindent
\begin{tabular}{rcl}
    $\TSE{f}{\kwdSETOF \,\, \T_0}{\alpha}{\TSEmap{}}$
  & \hspace*{-4mm} = & \\
    \multicolumn{3}{r}{
      \begin{tabular}{ll}
          $\ocamlkwd{let}$
        & \hspace*{-4mm}
          $\beta \,\, \emph{and} \,\,
          \gamma \,\, \emph{be fresh variables}$\\
          $\ocamlkwd{in}$
        & \hspace*{-4mm}
          $\TSE{\I}{\T_0}{\beta}{\TSEmap{}} \!
           \CEand \gamma \CEeqS \Cons \, (\beta, \gamma) \,
           \SEcup \, \Nil
           \CEand \alpha \PSCEeq \gamma \SEdiam \mathbb{N}^{+}$
      \end{tabular}
    }
\end{tabular}
\medskip

The variable $\beta$ denotes the expression from type \T{}, defined
by the constraint $\TSE{\I}{\T}{\beta}{\TSEmap{}}$. The constraint
$\gamma \CEeqS \Cons \, (\beta, \gamma) \, \SEcup \, \Nil$
defines the powerset $\gamma$ over $\beta$ sets. These sets are always
finite because the types in \core are well-founded (see
section~\ref{well_founded_types}). Finally, the powerset constraint
$\alpha \PSCEeq \gamma \SEdiam \mathbb{N}^{+}$ states that the powerset
expression we are looking for is the pair of $\gamma$ (the
powerset itself) and $\mathbb{N}^{+}$ (the allowed cardinals of the
elements of $\gamma$). Note that, because we are in \core, the
\kwdSETOF{} applies to a type (\T) which is always a reference without
subtyping constraint (see section~\ref{mapping},
step~\ref{types_in_SETOF}).

\begin{center}
$\TSE{f}{\kwdSEQUENCEOF \,\, \T}{\alpha}{\TSEmap{}}
 = \TSE{f}{\kwdSETOF \,\, \T}{\alpha}{\TSEmap{}}$
\end{center}

This equation defines the set of terms of the \kwdSEQUENCEOF{} type as
being the same as the \kwdSETOF{} terms. Indeed, the difference
between these two types is only meaningful for the encoding rules
(hence, at the application level).

\begin{center}
\begin{tabular}{rcl} 
    $\TSE{f}{\kwdSEQUENCE \, \{\}}{\alpha}{\TSEmap{}}$
  & \hspace*{-4mm} = 
  & \hspace*{-4mm}
    $\alpha \CEeqS \Nil$
\end{tabular}
\end{center}

\noindent
This equation is for empty \kwdSEQUENCE{} types.

\medskip

\begin{tabular}{rcl}
    $\TSE{f}{\kwdSEQUENCE \, \CompEnv{\{l\} \disjunion
    \Labels}}{\alpha}{\TSEmap{}}$
  & \hspace*{-4mm} = \\
    \multicolumn{3}{l}{
    \begin{tabular}{ll} 
        $\ocamlkwd{let}$ 
      & \hspace*{-4mm}
        $\beta \,\, \emph{and} \,\, \gamma \,\,
         \emph{be fresh variables}$\\
        $\ocamlkwd{in}$
      & \hspace*{-4mm}
        $\TSE{f}{\kwdSEQUENCE \,
         \CompEnv{\Labels}}{\gamma}{\TSEmap{}}$\\
      & \hspace*{-7mm}
        \begin{tabular}{rl}
            $\CEand$
          & \hspace*{-5mm}
            $\ocamlkwd{match} \,\, \CompEnv{}(l) \,\,
             \ocamlkwd{with}$\\
          & \hspace*{-6mm}
            \begin{tabular}{rl}
                & \hspace*{-4mm}
                  $(\T, \None \mid \Some \, (\kwdDEFAULT \,
                   \wildcard)) \rightarrow$\\
                & 
                  $\TSE{\I}{\T}{\beta}{\TSEmap{}}
                   \CEand \alpha \CEeqS \Bind \, (l, \beta,
                   \gamma)$\\
              $\mid$
                & \hspace*{-4mm}
                  $(\T, \Some \, \kwdOPTIONAL) \rightarrow$\\
                & 
                  $\TSE{\I}{\T}{\beta}{\TSEmap{}}
                   \CEand 
                   \alpha \CEeqS \Bind \, (l, \beta, \gamma) \,
                   \SEcup \, \gamma$
            \end{tabular}
        \end{tabular}
    \end{tabular}
    }
\end{tabular}

\medskip

This equation assumes that the set has at least one component, labeled
$l$.  We need first to extract the constraint from the
\kwdSEQUENCE{} without this component: $\TSE{f}{\kwdSEQUENCE \,
\CompEnv{\Labels}}{\gamma}{\TSEmap{}}$. Next, a case analysis is done
on the component, $\CompEnv{}(l)$. In all cases, the variable $\beta$
denote the set of terms of the component type:
$\TSE{f}{\T}{\beta}{\TSEmap{}}$. Hence we add the constraint $\alpha
\CEeqS \Bind \, (l, \beta, \gamma)$, for it represents the set of
terms when the component $l$ is always present. That is why, when this
component can be omitted (see the \kwdOPTIONAL{} case) we complete the
set with $\gamma$. Note that, because we are in \core, the (possible)
default value is a reference (see section~\ref{mapping},
step~\ref{default_values}), hence we do not care here about it (in
particular, we do not constrain it to belong to the set of terms of
the component). Indeed, this value, like all initial top-level values,
have been introduced in a so-called \emph{single value} subtyping
constraint (see step~\ref{types_from_values}) upon their expected
type and which will be considered independently from the current
case. Another detail worth mentioning is the lack, in \core, of
\kwdCOMPONENTSOF{} clause (see section~\ref{mapping},
steps~\ref{COMPONENTS_OF_reference}
and~\ref{COMPONENTS_OF_unfolding}).

\begin{center}
$\TSE{f}{\kwdSET \, \CompEnv{}}{\alpha}{\TSEmap{}}$
= $\TSE{f}{\kwdSEQUENCE \, \CompEnv{}}{\alpha}{\TSEmap{}}$
\end{center}

This equation states that the constraint for the \kwdSET{}
type is the one for the \kwdSEQUENCE{} with the same components. This
is in fact an approximation. Indeed, the difference between \kwdSET{}
and \kwdSEQUENCE{} is that the values of the latter must be given in
the same order as the components are given. Introducing explicitly the
proper combinatorics for the \kwdSET{} values would result in
exponential size of the term set. So we approach the \kwdSET{}
values as if they were \kwdSEQUENCE{} values. In theory, after the
equation solving, we would have to consider these values modulo
permutation (except for the validation purpose, which only requires
the existence of one term in the set).

\begin{center}
\begin{tabular}{rcl}
$\TSE{f}{\kwdINTEGER}{\alpha}{\TSEmap{}}$
  & \hspace*{-4mm} = & \hspace*{-4mm}
        $\alpha \CEeqI \mathbb{N}$\\ 
$\TSE{f}{\kwdREAL}{\alpha}{\TSEmap{}}$
  & \hspace*{-4mm} = & \hspace*{-4mm}
        $\alpha \CEeqF \mathbb{R}$\\ 
$\TSE{f}{\kwdBOOLEAN}{\alpha}{\TSEmap{}}$
  & \hspace*{-4mm} = & \hspace*{-4mm}
        $\alpha \CEeqS \kwdTRUE \,\, \SEcup \,\, \kwdFALSE$\\
$\TSE{f}{\kwdNULL}{\alpha}{\TSEmap{}}$
  & \hspace*{-4mm} = & \hspace*{-4mm}
        $\alpha \CEeqS \kwdNULL$\\
$\TSE{f}{\String}{\alpha}{\TSEmap{}}$
  & \hspace*{-4mm} = & \hspace*{-4mm}
        $\alpha \CEeqR \Regexp \, \stringof{.*}$\\
$\TSE{f}{\kwdBITSTRING}{\alpha}{\TSEmap{}}$
  & \hspace*{-4mm} = & \hspace*{-4mm}
        $\alpha \CEeqR \Regexp \, \texttt{"}\textsf{[}\symbol{92}\textsf{s01]*}\texttt{"}$ \\
$\TSE{f}{\kwdOCTETSTRING}{\alpha}{\TSEmap{}}$
  & \hspace*{-4mm} = & \hspace*{-4mm}
        $\alpha \CEeqR \Regexp \, \texttt{"}\textsf{[}\symbol{92}\textsf{s}\symbol{92}\textsf{da-fA-F]*}\texttt{"}$
\end{tabular}
\end{center}

\begin{center}
\begin{tabular}{rcl}
    $\TSE{f}{\kwdENUMERATED \, [a]}{\alpha}{\TSEmap{}}$
  & \hspace*{-4mm} = 
  & \hspace*{-4mm}
    $\alpha \CEeqS \Enum \, (a)$\\
    $\TSE{f}{\kwdENUMERATED \, (a \CONS b \CONS
     \textnormal{I})}{\alpha}{\TSEmap{}}$
  & \hspace*{-4mm} = 
  & \\
    \multicolumn{3}{r}{
      \begin{tabular}{l}
         $\ocamlkwd{let} \,\, \beta \,\, \emph{be a fresh variable}$\\
         $\ocamlkwd{in} \,\, \TSE{\I}{\kwdENUMERATED \, (b \CONS
          \textnormal{I})}{\beta}{\TSEmap{}}
          \CEand
          \alpha \CEeqS \Enum \, (a) \, \SEcup \, \beta$
      \end{tabular}
    }
\end{tabular}
\end{center}

These equations deal with the constraints from the basic types of
\ASN. You may notice that the terms of the types $\kwdBITSTRING$,
$\kwdOCTETSTRING$ and $\String$ are encoded using the regular
expressions. Also, we do not care about the enumerated and bit
string constants. Indeed, in \core, these values, like all
initially declared values, have been introduced in single value
subtyping constraints (see section~\ref{mapping},
steps~\ref{constants} and~\ref{types_from_values}) on their expected
type, and which wil be considered independently from the current
case. Another detail worth citing is that there is no \kwdINTEGER{}
type defining constants in \core (see section~\ref{mapping},
steps~\ref{constants} and~\ref{constants_unfolding}).

\begin{center}
\begin{tabular}{rcl}
$\TSE{f}{\TRef \, (x)}{\alpha}{\TSEmap{\R}}$
  & \hspace*{-4mm} = & \\
\multicolumn{3}{r}{
  \begin{tabular}{l}
     $\ocamlkwd{if} \,\, x \in \R \,\, \ocamlkwd{then} \,\, \alpha
      \CEeq \TSEmap{}(x)$\\
     $\ocamlkwd{else} \,\, \ocamlkwd{match} \,\, \TypeEnv(x) \,\,
      \ocamlkwd{with}$\\
     \hspace*{5mm}
     \begin{tabular}{rl}
               & \hspace*{-5mm}
                  $(\T_0, \None) \rightarrow \,\,
                  \denotC{\I}{\T_0}{\alpha}{\TSEmap{}
                  \funupdate \{x \mapsto \alpha\}}$\\
        $\mid$ & \hspace*{-5mm}
                 $(\T_0, \Some \, \C_0) \! \rightarrow \!
                 \denotC{f}{\T_0, \C_0}{\alpha}{\TSEmap{} \funupdate \{x
                 \mapsto \alpha\}}$
     \end{tabular}
  \end{tabular}
}
\end{tabular}
\end{center}

This equation defines the constraint collected from a type
reference. Two situations can occur. First, if the reference name $x$
is already in the domain $\R$ of the mapping $\TSEmap{}$, it means
that we already analysed this type before. Then we just emit
an $\alpha \CEeq \TSEmap{}(x)$ constraint that expresses that the
terms of the reference $\alpha$ are exactly the terms of the
referenced type $\TSEmap{}(x)$. Otherwise, we analyse the referenced
type: it can be either a type (first pattern) or a subtype (second
pattern).

In the first case, we just analyse the referenced type, without
forgetting to add ($\funupdate$) a binding $x \mapsto \alpha$ to
$\TSEmap{}$, in order to avoid a loop at run-time in presence of
recursive types\footnote{Hence there is no fixpoint-based approach in
the collection, it lies in the solving procedure instead.}. A valid
situation can be simply illustrated by the declaration \texttt{\small
T ::= CHOICE} \verb+{+\texttt{a REAL,} \texttt{b T}\verb+}+.

In the second case, we use the constraint collection from subtypes,
presented in the next section. A valid situation is: 
\texttt{\small T ::= SET} \verb+{+\texttt{a} \texttt{\small U}\verb+}+
\ \texttt{\small U ::= REAL (0..5)}. This case is the only dependence
between constraints from types and constraints from subtypes.

Let us consider now an example of constraint collection from a
type.\label{example_CHOICE} Let \texttt{\small T ::= CHOICE}
\verb+{+\texttt{item} \texttt{\small INTEGER,} \texttt{and}
\texttt{\small SET OF T,} \texttt{not} \texttt{\small T}\verb+}+. We
want the constraint whose solution in $\alpha$ is the set of terms of
\texttt{T}, that is to say, $\denotC{\I}{\TRef \,
  (\textsf{"T"})}{\alpha}{\{\}} = \denotC{\I}{\kwdCHOICE \,
  \FieldEnv{\Labels}}{\alpha}{{\cal Q}}$, where $\Labels =
\{\textsf{"item"}, \textsf{"and"}, \textsf{"not"}\}$ and ${\cal Q} =
\{\textsf{"T"} \mapsto \alpha\}.$

Let
$\left\{
  \hspace*{-2mm}
  \begin{tabular}{l}
     $\FieldEnv{} \, (\textsf{"item"}) = \kwdINTEGER$\\
     $\FieldEnv{} \, (\textsf{"and"}) = \kwdSETOF \, (\TRef \,
      (\textsf{"T"}))$\\
     $\FieldEnv{} \, (\textsf{"not"}) = \TRef \, (\textsf{"T"})$
  \end{tabular}
 \right.$

\noindent
Then we get: 

\begin{tabular}{rcll}
    $\denotC{\I}{\kwdCHOICE \, \FieldEnv{\Labels}}{\alpha}{{\cal Q}}$
  & \hspace*{-4mm} = 
  & 
  & \hspace*{-6mm}
    $\denotC{\I}{\kwdCHOICE \, \FieldEnv{\Labels \backslash
    \{\textsf{"item"}\}}}{\gamma}{{\cal Q}}$\\
  & 
  & \hspace*{-6mm}
    $\CEand$
  & \hspace*{-6mm}
    $\denotC{\I}{\FieldEnv{} \,
    (\textsf{"item"})}{\beta}{{\cal Q}}$\\
  & 
  & \hspace*{-6mm}
    $\CEand$
  & \hspace*{-6mm}
    $\alpha \CEeqS (\textsf{"item"} \pmb{:} \beta) \,
    \SEcup \, \gamma$\\
%%
    $\denotC{\I}{\kwdCHOICE \, \FieldEnv{\Labels \backslash
     \{\textsf{"item"}\}}}{\gamma}{{\cal Q}}$
  & \hspace*{-4mm} = 
  & 
  & \hspace*{-6mm}
    $\denotC{\I}{\kwdCHOICE \,
    \FieldEnv{\{\textsf{"not"}\}}}{\delta}{{\cal Q}}$\\
  & 
  & \hspace*{-6mm}
    $\CEand$   
  & \hspace*{-6mm}
    $\denotC{\I}{\FieldEnv{} \,
     (\textsf{"and"})}{\varepsilon}{{\cal Q}}$\\
  &
  & \hspace*{-6mm}
    $\CEand$
  & \hspace*{-6mm}
    $\gamma \CEeqS (\textsf{"and"} \pmb{:} \varepsilon)
     \, \SEcup \, \delta$\\
%%
    $\denotC{\I}{\kwdCHOICE \,
     \FieldEnv{\{\textsf{"not"}\}}}{\delta}{{\cal Q}}$
  & \hspace*{-4mm} = 
  & 
  & \hspace*{-6mm}
    $\denotC{\I}{\FieldEnv{} \, (\textsf{"not"})}{\zeta}{{\cal Q}}$\\
  & 
  & \hspace*{-6mm}
    $\CEand$
  & \hspace*{-6mm}
    $\delta \CEeqS \textsf{"not"} \pmb{:} \zeta$\\
    $\denotC{\I}{\FieldEnv{} \, (\textsf{"item"})}{\beta}{{\cal Q}}$
  & \hspace*{-4mm} = 
  & \multicolumn{2}{l}{
      \hspace*{-4mm}
      $\denotC{\I}{\kwdINTEGER}{\beta}{{\cal Q}}$
    }\\
  & \hspace*{-4mm} = 
  & \multicolumn{2}{l}{
      \hspace*{-4mm}
      $\beta \CEeqI \mathbb{N}$
    }\\ 
%%
    $\denotC{\I}{\FieldEnv{} \, (\textsf{"and"})}{\varepsilon}{{\cal
     Q}}$
  & \hspace*{-4mm} = 
  & \multicolumn{2}{l}{
      \hspace*{-4mm}
      $\denotC{\I}{\kwdSETOF \, (\TRef \,
       (\textsf{"T"}))}{\varepsilon}{{\cal Q}}$
    }\\
  & \hspace*{-4mm} = 
  &
  & \hspace*{-4mm}
    $\denotC{}{\TRef \, (\textsf{"T"})}{\eta}{{\cal Q}}$\\
  & 
  & \hspace{-4mm}
    $\CEand$
  & \hspace*{-4mm}
    $\varepsilon \PSCEeq \theta \SEdiam \mathbb{N}^{+}$\\
  &  
  & \hspace*{-4mm}
    $\CEand$
  & \hspace*{-4mm}
    $\theta \CEeq \Cons \, (\eta, \theta) \, \SEcup \, \Nil$\\
%%
    $\denotC{\I}{\FieldEnv{} \, (\textsf{"not"})}{\zeta}{{\cal Q}}$
  & \hspace*{-4mm} = 
  & \multicolumn{2}{l}{
      \hspace*{-4mm}
      $\denotC{\I}{\TRef \, (\textsf{"T"})}{\zeta}{{\cal Q}}$
    }\\
  & \hspace*{-4mm} = 
  & \multicolumn{2}{l}{
      \hspace*{-4mm}
      $\zeta \CEeq \alpha$
    }\\
%%
    $\denotC{\I}{\TRef \, (\textsf{"T"})}{\eta}{{\cal Q}}$
  & \hspace*{-4mm} = 
  & \multicolumn{2}{l}{
      \hspace*{-4mm}
      $\eta \CEeq \alpha$
    }
\end{tabular}

\smallskip

\noindent
After substitution, we get:

\begin{tabular}{rcl}
    $\denotC{f}{\TRef \, (\textsf{"T"})}{\alpha}{\{\}}$
  & \hspace*{-4mm} = &\\
    \multicolumn{3}{r}{
      \hspace*{2mm}
      \begin{tabular}{ll}
        & \hspace*{-4mm}
          $\zeta \CEeq \alpha \,\,
           \CEand \,\, \delta \CEeqS \textsf{"not"} \pmb{:} \zeta \,\, 
           \CEand \,\, \eta \CEeq \alpha$\\
          $\CEand$
        & \hspace*{-4mm}
          $\varepsilon \PSCEeq \theta \SEdiam \mathbb{N}^{+}
           \,\, \CEand \,\, \theta \CEeqS \Cons \, (\eta, \theta) \,
           \SEcup \, \Nil$\\
          $\CEand$
        & \hspace*{-4mm}
          $\gamma \CEeqS (\textsf{"and"} \pmb{:} \varepsilon) \,
           \SEcup \, \delta \,\,
           \CEand \,\,\beta \CEeqI \mathbb{N}$\\
          $\CEand$
        & \hspace*{-4mm}
          $\alpha \CEeqS (\textsf{"item"} \pmb{:} \beta)
           \, \SEcup \, \gamma$
      \end{tabular}
    }
\end{tabular}

\noindent
This constraint implies the system (we do not define formally this
notion in this paper, because we rely on the solving algorithm of
Aiken and Wimmers~\cite{AikenWimmers:1992}):

\centerline{
$\left\{
  \begin{array}{l}
    \alpha \CEeqS (\textsf{"item"} \pmb{:} \beta) \,
    \SEcup \, (\textsf{"and"} \pmb{:} \varepsilon) \, \SEcup \,
    (\textsf{"not"} \pmb{:} \alpha)\\
    \beta \CEeqI \mathbb{N}\\
    \varepsilon \PSCEeq \theta \SEdiam \mathbb{N}^{+}\\
    \theta \CEeqS \Cons \, (\alpha, \theta) \, \SEcup \, \Nil
  \end{array}
\right.$}

The following examples is an illegal type
definition.\label{illegal_type} In section~\ref{validation}
(validation) we mentionned: \texttt{\small T ::= REAL (ALL EXCEPT
  T)}. We want the constraint whose solution in $\alpha$ is the set of
terms of \texttt{\small T}:

\smallskip

$\denotC{}{\TRef \, (\textsf{"T"})}{\alpha}{\{\}} =$\\
\hspace*{1mm} 
$\denotC{}{\kwdREAL, \kwdALLEXCEPT \, (\kwdINCLUDES \,
\textsf{"T"})}{\alpha}{\{\textsf{"T"} \mapsto \alpha\}}$

\smallskip

Let ${\cal Q} = \{\textsf{"T"} \mapsto \alpha\}$. Then we get:

\smallskip

\begin{tabular}{rl}
  $\bullet$ & \hspace*{-4mm}
    $\denotC{}{\kwdREAL}{\beta}{{\cal Q}} = \beta \CEeqF \mathbb{R}$\\
%%
  $\bullet$ & \hspace*{-4mm}
    $\denotC{}{\kwdREAL, \kwdINCLUDES \, (\textsf{"T"})}{\gamma}{{\cal
    Q}}$\\
            & \hspace*{-4mm}
    \begin{tabular}{rl}
      = & \hspace*{-4mm}
          $\denotC{}{\kwdREAL}{\delta}{{\cal Q}} \!
           \CEand \! \denotC{}{\TRef \,
           (\textsf{"T"})}{\varepsilon}{{\cal Q}} 
           \CEand \! \gamma \CEeqI \delta \, \SEcap \, \varepsilon$\\
      = & \hspace*{-4mm}
          $\delta \CEeqF \mathbb{R} 
           \CEand \varepsilon \CEeq \alpha 
           \CEand \gamma \CEeqI \delta \, \SEcap \, \varepsilon$ 
    \end{tabular}\\
%%
  $\bullet$ & \hspace*{-4mm}
    $\denotC{}{\TRef \, (\textsf{"T"})}{\alpha}{\{\}}$\\
            & \hspace*{-4mm}
    \begin{tabular}{rll}
        = 
      & 
      & \hspace*{-5mm}
        $\denotC{}{\kwdREAL}{\beta}{{\cal Q}}
         \CEand \alpha \CEeqI \beta \, \SEdiff \, \gamma$\\
      & \hspace*{-5mm}
        $\CEand$
      & \hspace*{-5mm}
        $\denotC{}{\kwdREAL, \kwdINCLUDES \,
         (\textsf{"T"})}{\gamma}{{\cal Q}}$\\
        = 
      &
      & \hspace*{-5mm}
        $\beta \CEeqF \mathbb{R} \CEand \delta \CEeqF \mathbb{R}
         \CEand \alpha \CEeqI \beta \, \SEdiff \, \gamma$\\
      & \hspace*{-5mm}
        $\CEand$ 
      & \hspace*{-5mm}
        $\varepsilon \CEeq \alpha
         \CEand \gamma \CEeqI \delta \, \SEcap \, \varepsilon$
    \end{tabular}

\end{tabular}

This constraint implies (we do not define formally this notion in this
paper): $\alpha \CEeqI \SEneg\alpha$, which has no solutions. Thus the
declaration of subtype \texttt{\small T} must be rejected.




