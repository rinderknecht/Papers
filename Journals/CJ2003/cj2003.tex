%%-*-latex-*-

\documentclass[11pt,a4paper]{article}

% Pre-defined packages
%
\usepackage[british]{babel}
\usepackage[T1]{fontenc}       % Required for hyphenation and \DJ
\usepackage[utf8]{inputenc}    % UTF-8 encoding
\usepackage{amsmath,amsthm,amssymb,stmaryrd}
\usepackage{latexsym}
\usepackage{url}
\usepackage{xspace}
\usepackage{float}
\usepackage{supertabular}
\usepackage{cite}

% Typesetting of Abstract Syntax
%
\input{asn1}

% Typesetting of Inference Rules
%
\usepackage{mathpartir}
\input{inference-commands}

% Hyperlinks in PDF (user-defined)
%
%\usepackage{hrefpdf}

%% New definitions, theorems, remarks etc.
%%
\newtheorem{Def}{Definition}[section]
\newtheorem{Rem}{Remark}[section]
\newtheorem{Algo}{Algorithm}[section]
\newtheorem{Theorem}{Theorem}[section]
\newtheorem{Proof}{Proof}[section]

%% Main document
%%
\title{An Algorithm for Validating \ASN~(\mbox{X.680}) Specifications
       using Set Constraints}
\author{Christian Rinderknecht\\
Network Architecture Laboratory \\
Information and Communications University \\
58-4 Hwaam-dong, Yuseong-gu, Daejeon, \\
305-732, Republic of Korea\\
{\small\url{rinderkn@nalab.icu.kr}}}
\date{July 2003}

%% \shortauthors{Christian Rinderknecht}
%% \volumeyear{2003}
%% \volumenumber{46}
%% \issuenumber{3}

\begin{document}

\maketitle

%%-*-latex-*-

\begin{abstract}
Recursion is a powerful programming technique which is notoriously
difficult to master, especially in functional languages because they
prominently feature structural recursion as the main control\hyp{}flow
mechanism. We propose several hypotheses to understand the issue and
put some to the test by designing an open\hyp{}source interactive
interface based on a tangible block\hyp{}world with augmented reality
and software feedback. Stacks of blocks are used as an analogy for the
list data structure, which enables the simplest form of structural
recursion. After using this application, students are expected to
transfer their training to directly write recursive programs in
sequential \erlang, a purely functional language.
\end{abstract}


\noindent\textbf{Keywords:} \mbox{ASN.1}, abstract syntax notation,
validation, compilation, set constraints.

\section*{Introduction}
\addcontentsline{toc}{section}{Introduction}
\label{introduction}
%%-*-latex-*-

\section{Introduction}

The wide variety of software and hardware architectures in distributed
systems and telecommunications makes it valuable to use a common
high-level data notation in protocol specifications. To fulfill this
need, the ISO organization and the International Telecommunication
Union (ITU) defined the Abstract Syntax Notation One (\ASN) series of
standards. \ASN~\cite{Dubuisson:2000,X.680:2002,X.681:2002,X.682:2002,X.683:2002}
is a language for data types allowing the protocol designer to capture
numerous networking concepts, such as protocol data units, without
worrying about the possible environment and implementation
heterogeneity of the peers. The peers share a set of \ASN modules and
agree upon a set of \emph{encoding rules}, such
as~\cite{X.690:2002,X.691:2002}, which is a method for encoding values
produced at run-time by the communicating applications, into series of
bits. \ASN has been adopted for a wide range of applications, such as
network management, secure e-mail, mobile telephony, air traffic
control etc.
%, video conferencing over the internet, electronic commerce,
%digital certificates, radio paging, financial service systems etc.

\medskip

In the last few years, the press has reported several alleged
vulnerabilities of \ASN and the Basic Encoding Rules (BER) related to
network protocols like SNMP and, more recently, OpenSSL. Each time, an
accurate description of the problem has been finally published,
showing that the weakness lay in \emph{implementations} poorly written
and insufficiently tested. The real vulnerabilities were almost all
related to improper decoding of ill-formed BER encodings (or
\emph{codes}) causing buffer overflows, unspecified
(non-deterministic) behaviours, stack corruptions and, in the end, a
possible denial of service.

\medskip

From now on, it is important to understand and remember that \ASN and
the BER, intrinsically, have nothing to do with security or
cryptographic protocols. Both are used for modeling and handling the
data part of protocols, not the control. As a consequence, the
soundness property we aim at in this article must not be considered as
a security property about \emph{control} but as mere correctness of
composition of encoding and decoding with the BER of \emph{values}
specified by means of \ASN. For instance, there are no attackers, no
nonces etc. here. Nevertheless, the difficulty is not lesser.

\medskip

More precisely, in this work we want to prove that the design of the
BER themselves is flawless, whatever the network protocol is and
whatever the values to be transmitted are. To achieve this goal we
need the support of formal methods. We start by a formal modeling of
the BER which abstracts away low-level details but captures the design
principles. Then we define a soundness property representing the
security warranty we require and finally we prove that this property
holds for all values that can be specified with \ASN.



\section{Presentation of \ASN}
\label{presentation}
 %%-*-latex-*-

This section provides a very short overview of \ASN. For a more
detailed introduction, please refer to Dubuisson's
book~\cite{Dubuisson:2000}. The subtyping features will be presented
in section~\ref{constraints_from_subtypes} together with the
constraint collection from subtypes. \ASN provides basic types as
follows.

\begin{itemize}
 
  \item The \texttt{\small BOOLEAN} type has two predefined values
        \texttt{\small TRUE} and \texttt{\small FALSE},
        \emph{e.g.,} \texttt{ok} \texttt{\small BOOLEAN ::= TRUE} defines a
        value \texttt{\small TRUE} whose name is \texttt{ok} 
        and whose type is \texttt{\small BOOLEAN}.

  \item The \texttt{\small NULL} type only has one value, also noted
        \texttt{\small NULL}. This type is often used as a placeholder
        in many real complete specifications to indicate that no
        additional information is needed, or it is used to test
        incomplete specifications.

  \item The \texttt{\small INTEGER} type matches the mathematical set
    $\mathbb{Z}$, \emph{e.g.,} \texttt{zero} \texttt{\small INTEGER
    ::= 0}. The syntax also allows some constants to be distinguished:
    \texttt{\small DayIn\-The\-Year ::= INTEGER \{first (1), last
      (365)\}} defines the type \texttt{Day\-In\-The\-Year} as being
    \texttt{\small INTEGER}, and distinguishes two integers named
    \texttt{first} and \texttt{last}, whose respective values are
    \texttt{\small 1} and \texttt{\small 365}. Then
    \texttt{newYearsEve} \texttt{DayInTheYear} \texttt{\small ::=}
    \texttt{last} defines a value \texttt{newYearsEve}. The definition
    is valid because \texttt{last} is in the scope of
    \texttt{Day\-In\-The\-Year}; the name \texttt{newYearsEve} is
    bound to the value \texttt{\small 365}.

  \item The \texttt{\small ENUMERATED} type defines a collection
    of (constant) names, like \texttt{\small
      Synchro\-Indicator ::= ENUMERATED \{serial, parallel\}} allows
    the following value definition: \texttt{synchro
      Synchro\-Indi\-ca\-tor} \texttt{\small ::=} \texttt{serial}. It
    is possible, though not recommended, to specify the
    \emph{encoding} of an enumerated value, like
    \texttt{PositiveLogics} \texttt{\small ::= ENUMERATED}
    \verb+{+\texttt{false} \texttt{\small (0),} \texttt{true}
    \texttt{\small(1)}\verb+}+, but this has no impact on the values
    themselves.

  \item The \texttt{\small REAL} type corresponds to the mathematical
        decimal numbers, defined either with a dotted notation,
        \emph{e.g.,} \texttt{\small 5.7}, or a sequence,
        \emph{e.g.,} \verb+{+\texttt{mantissa} \texttt{\small 1,}
        \texttt{base} \texttt{10}\texttt{\small,} \texttt{exponent}
        \texttt{\small -3}\verb+}+.

  \item The \texttt{\small BIT STRING} type corresponds to strings of
    bits, \emph{e.g.,} \texttt{\small '1101'B} (binary) or
    \texttt{\small '0D'H} (hexadecimal). The syntax also allows some
    bits to be distinguished. Given \texttt{\small T ::= BIT STRING}
    \verb+{+\texttt{msb} \texttt{\small (7),} \texttt{lsb}
    \texttt{\small (0)}\verb+}+, the definition \texttt{v}
    \texttt{\small T ::=} \verb+{+\texttt{msb}\texttt{\small ,}
    \texttt{lsb}\verb+}+ stands for \texttt{v} \texttt{\small T ::=
      '10000001'B}.  It is also possible to restrict the size of the
    string using a subtyping constraint: \texttt{StringOf32Bits ::=
      BIT STRING (SIZE (32))}.

  \item The \texttt{\small OCTET STRING} type is similar to the
        \texttt{\small BIT STRING}, except that the encoded strings
        must contain a number of bits that is a multiple of eight (and
        no bit can be distinguished by a name).

  \item The \texttt{\small OBJECT IDENTIFIER} and \texttt{\small
        RELATIVE-OID} types are used to reference other \ASN
        modules at an international level, by means of a path in a
        standard tree. They can also identify a physical object, such
        as a printer on a network, or a postal package, or an ASN.1
        type which is carried in some larger message. They are not
        considered here.

  \item For historical reasons there are plenty of string types in
        \ASN, like \texttt{Nu\-mer\-ic\-String}, \texttt{IA5String},
        \texttt{UTF8String}, \texttt{GeneralString} etc. They mainly
        differ in the alphabet they are built upon\footnote{There are
        other factors besides just the alphabets. Some string types
        such as \texttt{Ge\-ne\-ral\-String} allow escape characters
        to kick into alternate character sets (such as those for
        different languages) while others such as \texttt{UTF8String}
        can represent characters of all languages directly.}. Here, we
        will not make any difference between these strings,
        and assume there is only one kind, called \texttt{String}.

\end{itemize}

\noindent
These basic types can be used to construct other types:

\begin{itemize}

  \item The \texttt{\small SET} type corresponds to the record-like
        structures in programming languages, \emph{e.g.,} \texttt{PersonInfo}
        \texttt{\small ::= SET} \verb+{+\texttt{age} \texttt{\small
        INTEGER,} \texttt{married} \texttt{\small BOOLEAN}\verb+}+ of
        which one value may be: \texttt{i PersonInfo} \texttt{\small
        ::=} \verb+{+\texttt{married} \texttt{\small TRUE,} \texttt{age}
        \texttt{\small 32}\verb+}+. Some components may be marked as
        optional or having a default value, \emph{e.g.,} \texttt{\small Point
        ::= SET} \verb+{+\texttt{x} \texttt{\small REAL DEFAULT 0.0,}
        \texttt{y} \texttt{\small REAL DEFAULT 0.0}\verb+}+ allows
        defining the value \texttt{origin} \texttt{\small Point ::=}
        \verb+{}+, which is the same as \texttt{origin} \texttt{\small
        Point ::=} \verb+{+\texttt{x} \texttt{\small 0.0,} \texttt{y}
        \texttt{\small 0.0}\verb+}+. Here is an example from a real
        protocol:\\
{\small
\verb+DataAcknowledgementTPDU ::= SET {+\\
\verb+  destRef        Reference,+\\
\verb+  yr-tu-nr       TPDUnumber,+\\
\verb+  checkSum       CheckSum OPTIONAL,+\\
\verb+  subSeqNr       SubSequenceNumber DEFAULT 0,+\\
\verb+  flowControlCnf FlowCntlConf OPTIONAL}+}

  \item The \texttt{\small SEQUENCE} type is the same as the
        \texttt{\small SET} type, except that the component values
        must be given in the same order as they are declared, \emph{e.g.,},
        given \texttt{\small Point ::= SEQUENCE} \verb+{+\texttt{x}
        \texttt{\small REAL,} \texttt{y} \texttt{\small REAL}\verb+}+,
        the value \texttt{origin} \texttt{\small Point ::=}
        \verb+{+\texttt{y} \texttt{\small 0.0,} \texttt{x}
        \texttt{\small 0.0}\verb+}+ is rejected.

  \item The \texttt{\small SET OF} type corresponds to the
    mathematical notion of sets with repetition: all elements are of
    the same type, but their number is not known beforehand (unless
    the set's size is constrained to a given value), and they can be
    repeated, \emph{e.g.,} \texttt{\small T ::= SET OF INTEGER} allows
    the value definitions \texttt{empty} \texttt{\small T ::=}
    \verb+{}+ and \texttt{small} \texttt{\small T ::=}
    \verb+{+\texttt{\small 7, 9, 1, 1, 3}\verb+}+.
 
  \item The \texttt{\small SEQUENCE OF} type corresponds to the dynamic
        arrays or lists of some programming languages. It is similar
        to the \texttt{\small SET OF} type, except that the elements
        will be encoded in the specified order. Since the encoding
        rules are out of the scope of this paper, this difference is
        not relevant.

  \item The \texttt{\small CHOICE} type corresponds to a
        \textsf{union} in C, a \textsf{case} in Pascal, or a sum type
        in ML. For instance \texttt{\small T ::= CHOICE}
        \verb+{+\texttt{x} \texttt{\small REAL,} \texttt{y}
        \texttt{\small BOOLEAN}\verb+}+ allows the following
        declarations: \texttt{u} \texttt{\small T ::=} \texttt{x :}
        \texttt{\small 0.5}, where the component \texttt{x} is chosen
        to build the value, and \texttt{v} \texttt{\small T ::=}
        \texttt{y :} \texttt{\small FALSE} where the component
        \texttt{y} is used. The protocol data units are \texttt{\small
        CHOICE} types, because they model all the possible queries and
        responses between two peers. As we show later, a
        \texttt{\small CHOICE} type may be recursive, like the other
        constructed types. An example from a network management
        protocol is\\ {\small \verb+CMISFilter ::= CHOICE {+\\
        \verb+ item FilterItem,+\\ \verb+ and SET OF CMISFilter,+\\
        \verb+ or SET OF CMISFilter,+\\ \verb+ not CMISFilter}+}

\end{itemize}


\section{Core \ASN}
\label{core}
%%-*-latex-*-

It is difficult to separate the different concepts throughout the
syntax. The \emph{types}, \emph{values} and \emph{subtyping
  constraints} may depend on each other: a type may contain
constraints (on components) and values (\emph{e.g.,} default values),
a value has a declared type, and constraints rely upon types
(\emph{e.g.,} inclusion constraint) and values (\emph{e.g.,} value
constraint). Another related difficulty is the large number of
syntactic constructs.

In order to allow a clearer presentation of the constraint collection
(sections~\ref{constraints_from_types},
\ref{constraints_from_subtypes}
and~\ref{full_collection_and_solving}), we define a strict subset of
\mbox{X.680}, which we call from now on \core (versus \emph{full}
\ASN), that will be used in the rest of this paper. In \core,
  \begin{itemize}

    \item there are no \texttt{\small COMPONENTS OF} or selection
      types;

    \item the \texttt{\small INTEGER} type does not allow defining
          constants;

    \item component types are references;

    \item \texttt{\small SET OF} and \texttt{\small SEQUENCE OF} apply
          to references;

    \item default values are references;

    \item enumerated and bit string constants are references;

    \item types of declared values are references;

    \item default, enumerated, integer and bit string values appear in
          a constraint upon their expected type;
 
    \item types in inclusion constraints are references;

    \item there is no type reference just after the symbol
      `\texttt{\small ::=}' and constraints appear only at top-level,
      \emph{i.e.,} the extended Backus-Naur Form for type declarations
      is: {\small $<$\textsf{type reference}$>$ \textsf{::=}
        $<$\textsf{non-reference type without inner constraints}$>$
        \textsf{[}"("$<$\textsf{subtyping constraint}$>$")"\textsf{]}
      }
    \item there are no infinite values, \emph{i.e.,} recursive values.

  \end{itemize}
Since we have not yet introduced the collection, it is awkward to
explain here the rationale behind \core. As a consequence, this
information will be given later and the reader may skip the
next section when reading this for the first time.


\subsection{Mapping full \ASN into \core}
\label{mapping}

Full \ASN is mapped into \core by applying a series of rewritings.  It
is important to note that each step strictly preserves the
expressiveness of full \ASN. In other words: \core can express all
that can be expressed in full \ASN and nothing more.

Another useful property is that each simplification output can be
given in (the syntax of) full \ASN, making presentation easier. As
software tools use a specific internal data representation, the
practical bonus is that pretty-printing is then possible at each stage
with the \emph{same} initial pretty-printer (\emph{i.e.,} for full
\ASN).

It is assumed that the following transformations and checkings apply
to an \ASN module whose syntax complies with
\mbox{X.680}~\cite{X.680:2002}. (The attentive reader will note that
not all the rewritings commute, \emph{i.e.,} the following enumeration
cannot be arbitrarily shuffled.)
\begin{enumerate}

  \item \label{default_values} We extract the default constant values
    from the \texttt{\small SEQUENCE} and \texttt{\small SET} types,
    following the example

        {\small
         \verb+T ::= SET {a REAL DEFAULT 0.0}+\\
         $\longrightarrow
         \left\{
                \begin{tabular}{l}
                  \verb+T ::= SET {a A DEFAULT v}+\\
                  \verb+A ::= REAL+\\
                  \verb+v A ::= 0.0+
                \end{tabular}
              \right.$
        }

        where \texttt{A} is a fresh type reference and \texttt{v} is
        a fresh value reference.

  \item \label{constants}
        We lift the enumeration constants (enumerated, integer and bit
        string constants) to the top-level, as shown by
        (\texttt{v} is a fresh value reference):

        {\small
           \verb+T ::= ENUMERATED {a(x), b}+\\
            $\longrightarrow$
            $\left\{
                \begin{tabular}{l}
                  \verb+T ::= ENUMERATED {a(v), b}+\\
                  \verb+v INTEGER ::= x+
                \end{tabular}
              \right.$

           \verb+T ::= INTEGER {a(x)}+\\
            $\longrightarrow$
            $\left\{
                \begin{tabular}{l}
                   \verb+T ::= INTEGER {a(v)}+\\
                   \verb+v INTEGER ::= x+
                \end{tabular}
              \right.$

          \verb+T ::= BIT STRING {a(x)}+\\
           $\longrightarrow$
           $\left\{
                \begin{tabular}{l} 
                   \verb+T ::= BIT STRING {a(v)}+\\
                   \verb+v INTEGER (0..MAX) ::= x+
                \end{tabular}
              \right.$
        }

  \item \label{types_from_values}
        For each value declaration, we extract the given type and
        create a corresponding type declaration for it. We also create
        another type declaration where the previous type is required 
        to contain the originally declared value. For instance,
        consider\\
        {\small
         \begin{tabular}{l}
           \verb+y REAL(0..9) ::= 1+
           $\,\, \longrightarrow \,\,$
           \hspace*{-3.9pt}
           $\left\{
                \begin{tabular}{l} 
                   \verb+y A ::= 1+\\
                   \verb+A ::= REAL(0..9)+\\
                   \verb+B ::= A (y)+
                \end{tabular}
              \right.$
         \end{tabular}
        }\\
        where \verb+A+ and \verb+B+ are fresh type references. This
        way the declared type in a value definition is bound in a
        type definition (\texttt{\small A}). Also, the
        typechecking of a value (\texttt{y}) can be done with our
        algorithm (through \texttt{\small B}), since it deals with
        subtyping constraints.

  \item \label{COMPONENTS_OF_reference}
        The types which appear in
        \texttt{\small COM\-PO\-NENTS OF} constructions are
        replaced by fresh type references (in the following,
        \texttt{A} is a fresh type reference)
        {\small
         \verb+ T ::= SET {COMPONENTS OF SET {a REAL}}+\\
         $\longrightarrow$
         $\left\{
            \begin{tabular}{l}
               \verb+T ::= SET {COMPONENTS OF A}+\\
               \verb+A ::= SET {a REAL}+
            \end{tabular}
          \right.$}

  \item \label{contained_reference}
        We want to relax the dependence between
        subtyping constraints and types. Hence, for each inclusion
        constraint, we replace the included type by 
        a fresh reference and add a corresponding new type
        declaration, like (\texttt{\small A} is a fresh type
        reference):\\
        {\small
         \begin{tabular}{l}
           \verb+T ::= U(SET{a REAL})+
           $\rightarrow
           \left\{\hspace*{-2pt}
                \begin{tabular}{l}
                   \verb+T ::= U (A)+ \\
                   \verb+A ::= SET {a REAL}+
                \end{tabular}
              \right.$
         \end{tabular}}

  \item \label{component_types}
        We replace each component type by a reference:\\
        {\small
         \verb+T ::= SET {a REAL, b SET {d INTEGER}, c U (V)}+\\
         $\rightarrow
          \left\{\hspace*{-5pt}
            \begin{tabular}{l}
               \verb+T ::= SET {a A, b B, c C}+\\
               \verb+A ::= REAL  B ::= SET{d D}  C ::= U(V)+\\
               \verb+D ::= INTEGER+
            \end{tabular}
          \right.$}\\
        where \texttt{\small A}, \texttt{\small B}, \texttt{\small C}
        and \texttt{\small D} are fresh type references.

  \item \label{types_in_SETOF}
        At this step, we replace each type to which a \texttt{\small
        SET OF} or a \texttt{\small SEQUENCE OF} applies, by a
        reference:\\
        {\small
            \verb+T ::= SET OF REAL (C)+
            $\longrightarrow$
            $\left\{\hspace*{-5pt}
               \begin{tabular}{l}    
                  \verb+T ::= SET OF A+\\
                  \verb+A ::= REAL (C)+
               \end{tabular}
             \right.$}\\
        where \texttt{\small A} is a fresh type reference.

  \item We remove the selection types (at top-level)\\
        {\small
        \begin{tabular}{l} 
          $\left\{\hspace*{-5pt}
             \begin{tabular}{l}
                \verb+A ::= i < B+\\
                \verb+B ::= C+\\
                \verb+C ::= CHOICE{i D}+\\
                \verb+D ::= INTEGER+
             \end{tabular}
            \right.
          \longrightarrow
          \left\{\hspace*{-5pt}
               \begin{tabular}{l}
                  \verb+A ::= INTEGER+\\
                  \verb+B ::= C+\\
                  \verb+C ::= CHOICE{i D}+\\
                  \verb+D ::= INTEGER+
               \end{tabular}
             \right.$
        \end{tabular}}\\
       You must also be aware of the possibly misleading case:\\
       {\small
       \begin{tabular}{@{}r@{\;}c@{\;}l@{}}
         $\left\{
            \begin{tabular}{@{}l@{}}
              \verb+A ::= SET OF S+\\
              \verb+B ::= A (SIZE (7))+
            \end{tabular}
          \right.$
         & $\longrightarrow$
         & $\left\{
            \begin{tabular}{@{}l@{}}
              \verb+A ::= SET OF S+\\ 
              \verb+B ::= SET (SIZE (7)) OF S+
            \end{tabular}
            \right.$
       \end{tabular}
       }
   
       The result \verb+B ::= SET OF S (SIZE (7))+ would be wrong!

       This step is difficult because it removes all recursive types
       declarations that do not lead to a uniquely defined type, like
       \texttt{T ::= T} or \texttt{T ::= CHOICE \{a a < T\}}
       etc. 

        Note that the selection types that do not define a unique type
        lead to recursive type definitions whose pattern is
        \texttt{\small X ::= X}, as \verb+T ::= CHOICE{a a < T}+
        $\longrightarrow$\\
        {\small
         $\left\{\hspace*{-4pt}
            \begin{tabular}{l}
               \verb+T ::= CHOICE {a A}+\\
               \verb+A ::= a < T+
            \end{tabular}
          \right.
         \longrightarrow
         \left\{\hspace*{-4pt}
             \begin{tabular}{l}
                \verb+T ::= CHOICE {a A}+\\
                \verb+A ::= A+
             \end{tabular}
          \right.$
        }

        From now on we know exactly what a referenced
        type is, and thus what is the type of a value.

  \item \label{type_references} The top-level type references are
    unfolded, \emph{i.e.,} the type references at the declaration
    level are replaced by the type they reference\\
        {\small
         $\left\{
             \begin{tabular}{l}
                \verb+T ::= U (C)+\\
                \verb+U ::= REAL (D)+
             \end{tabular}
          \right.$
          $\longrightarrow$
          $\left\{
             \begin{tabular}{l}
                \verb+T ::= REAL (D ^ C)+\\
                \verb+U ::= REAL (D)+
             \end{tabular}
          \right.$}\\
        During this step, ill-formed recursive definitions, like
        \texttt{\small X ::= X}, are rejected.

  \item The default values are expanded, like\\
        {\small
         $\left\{
            \begin{tabular}{ll}
                 \verb+v T ::= {}+ 
               & \verb+T ::= SET {a U DEFAULT w}+
            \end{tabular}
          \right.$\\
          $\longrightarrow$
            \verb+ v T ::= {a w}   T ::= SET {a U DEFAULT w}+
        }

  \item \label{COMPONENTS_OF_unfolding}
        The type references in the \texttt{\small COMPONENTS OF}
        clauses are replaced by their corresponding components\\
        {\small
         $\left\{
            \begin{tabular}{l}
               \verb+T ::= SET {COMPONENTS OF A}+\\
               \verb+A ::= SET {a REAL}+
            \end{tabular}
          \right.$\\
         $\longrightarrow$
         $\left\{
            \begin{tabular}{l}
              \verb+T ::= SET {a REAL}+\\
              \verb+A ::= SET {a REAL}+
            \end{tabular}
          \right.$}

  \item \label{constants_unfolding}
        Integer and bit string constants are unfolded\\
        {\small
           $\left\{
              \begin{tabular}{l}
                 \verb+T ::= INTEGER {c(x)}+\\
                 \verb+v T ::= c+
              \end{tabular}
            \right.
           \longrightarrow
           \left\{
              \begin{tabular}{l}
                \verb+T ::= INTEGER+\\
                \verb+v T ::= x+
              \end{tabular}
           \right.$
        }\\
        In the case of bit string values which are specified by
        means of a series of bit names, we unfold their associated
        references and replace the value by an equivalent one
        without those names\\
        {\small
           $\left\{
              \begin{tabular}{l}
                \verb+T ::= BIT STRING {msb(x),lsb(y)}+\\
                \verb+v T ::= {msb,lsb}+
              \end{tabular}
            \right.$\\
            $\longrightarrow
            \left\{
              \begin{tabular}{l}
                \verb+T ::= BIT STRING+\\
                \verb+v T ::= '10000001'B+
              \end{tabular}
            \right.$
        }\\
        where 
        {\small
           $\left\{
              \begin{tabular}{l}
                \verb+x INTEGER (0..MAX) ::= 7+\\
                \verb+y INTEGER (0..MAX) ::= 0+
              \end{tabular}
           \right.$
        }
        (see step~\ref{constants}).

        This step may reveal some recursive values\\
        {\small
           $\left\{
              \begin{tabular}{l}
                 \verb+T ::= INTEGER {c(v)}+\\
                 \verb+v T ::= c+
              \end{tabular}
            \right.
           \longrightarrow
           \left\{
              \begin{tabular}{l}
                \verb+T ::= INTEGER+\\
                \verb+v T ::= v+
              \end{tabular}
            \right.$
        }

  \item We disallow the recursive values, like {\small
  \verb+v T ::= {v}+} or\\
        {\small
         $\left\{
            \begin{tabular}{ll}
                \verb+v T ::= {v}+
              & \verb+T ::= SET OF T+\\
                \textnormal{or}
              & \\
                \verb+v T ::= {}+
              & \verb+T ::= SET {a T DEFAULT v}+
            \end{tabular}
          \right.$
        }

\end{enumerate}



\subsection{Validation issues}\label{validation}

In \core, it is possible that

\begin{enumerate}
  
  \item \label{finiteness} types have only infinite values:
        \texttt{\small T ::= SET} \verb+{+\texttt{a} \texttt{\small
        T}\verb+}+

  \item \label{type_conformance} values are ill-typed:
        \texttt{v} \texttt{\small T ::= ""} \hspace*{6mm}
        \texttt{\small T ::= REAL}

  \item \label{type_compatibility} in particular, value references
        may be ill-typed:\\
        {\small 
        $\left\{
           \begin{tabular}{l}
              \verb+a A ::= b     A ::= INTEGER+\\
              \verb+b B ::= 1.5   B ::= REAL+
           \end{tabular}
        \right.$
        }

  \item \label{constraint_consistence} constraints are
        inconsistent: \texttt{\small T ::= REAL (SIZE(7))}

  \item \label{subtype_non_emptiness} subtypes are empty:\\
     \texttt{\small T ::= SET ((SIZE(1)) INTERSECTION (SIZE(2)))
        OF REAL};
        
  \item \label{solvability} subtypes have no value set:
        \texttt{\small T ::= A(ALL EXCEPT T)}

\end{enumerate}
These cases can be classified into the different problems: the
\emph{finiteness} problem (case~\ref{finiteness}), the
\emph{typechecking} problem (case~\ref{type_conformance}), the
\emph{type compatibility} problem (case~\ref{type_compatibility}), the
\emph{constraint consistence} problem
(case~\ref{constraint_consistence}), the \emph{non-emptiness} problem
(case~\ref{subtype_non_emptiness}) and the \emph{solvability} problem
(case~\ref{solvability}). The type compatibility problem is a sub-case
of the typechecking problem, and constraint consistence together with
non-emptiness are sub-cases of the solvability problem, because we
will explicitly construct the values of each (sub)type when the system
is solved. Moreover, since we added a new type declaration for each
value declaration at rewriting step~\ref{types_from_values}, the
solvability of the subtyping constraints will cope with the
typechecking problem. So, finiteness and solvability are enough to get
a full validation of \mbox{X.680} specifications and we need, as a
starting point, to express formally those concepts.

\subsection{Algorithmic meta-language}

We shall use as \emph{meta language} for the description of our
algorithm a version of the functional language ML:
OCaml\footnote{{\small\url{http://www.ocaml.org/}}}~\cite{CousineauMauny:1998},
which is a full-fledged programming language, as well as,
historically, a logic meta-language. Therefore our algorithm is close
to an actual implementation and is also a formal (operational)
model. Readers familiar with ML may skip this section, which gives a
crash overview of its syntax and semantics. (This presentation is
based on Pidgin ML~\cite{Huet:2002}.)

The core language has types and values. Thus, $1$ is a value of
predefined type \ocamltypename{int}, whereas \textsf{"CL"} is a
\ocamltypename{string}. Pairs of values inhabit the corresponding
product type. Therefore, $(1,\textsf{"CL"})$ has type
$(\ocamltypename{int} \times \ocamltypename{string})$. Recursive type
declarations create new types, whose values are inductively built from
the associated constructors. Thus a type modeling a binary tree of
integers could be declared as a sum by: $\ocamlkwd{type} \,\,
\ocamltypename{ib\_tree} = \ocamlconstr{Node} \, \ocamlkwd{of} \,
\ocamltypename{ib\_tree} \times \ocamltypename{int} \times
\ocamltypename{ib\_tree} \, \mid \ocamlconstr{Leaf}$. Parametric types
give rise to polymorphism: if \ocamlvaluename{x} is of type
\ocamltypename{t} and \ocamlvaluename{l} is of type (\ocamltypename{t}
\ocamlconstr{list}), we construct the list adding \ocamlvaluename{x}
to \ocamlvaluename{l} as $\ocamlvaluename{x} \CONS
\ocamlvaluename{l}$. The empty list is \emptyL, of (polymorphic) type
$('\ocamltypename{a} \,\, \ocamlconstr{list})$. Although the language
is strongly typed, explicit type specifications are rarely needed from
the programmer, since principal types may be inferred mechanically.

The language is functional in the sense that functions are first class 
objects. Therefore the integer doubling function may be written as 
$\ocamlkwd{fun} \,\, \ocamlvaluename{x} \rightarrow
\ocamlvaluename{x}+\ocamlvaluename{x}$, and it has type
$\ocamltypename{int} \rightarrow \ocamltypename{int}$. It may be
associated to the name \ocamlvaluename{double} by declaring:
$\ocamlkwd{let} \,\, \ocamlvaluename{double} = \ocamlkwd{fun} \,
\ocamlvaluename{x} \rightarrow
\ocamlvaluename{x}+\ocamlvaluename{x}$. Equivalently we could write:
$\ocamlkwd{let} \,\, \ocamlvaluename{double} \, \ocamlvaluename{x} =
\ocamlvaluename{x}+\ocamlvaluename{x}$. Its application to value
\ocamlvaluename{n} is written as $(\ocamlvaluename{double} \,
\ocamlvaluename{n})$ or even $\ocamlvaluename{double} \,
\ocamlvaluename{n}$ when there is no ambiguity. Application associates
to the left, and thus $\ocamlvaluename{f} \, \ocamlvaluename{x} \,
\ocamlvaluename{y}$ stands for $((\ocamlvaluename{f} \,
\ocamlvaluename{x}) \, \ocamlvaluename{y})$.
Recursive functional values are declared with the keyword
\ocamlkwd{rec}. Thus we may define the factorial function as:
$\ocamlkwd{let} \,\, \ocamlkwd{rec} \,\, \ocamlvaluename{fact} \,
\ocamlvaluename{n} = \ocamlkwd{if} \, \ocamlvaluename{n} \leqslant 0 \,
\ocamlkwd{then} \, 1 \, \ocamlkwd{else} \, \ocamlvaluename{n} \times
(\ocamlvaluename{fact} (\ocamlvaluename{n}-1))$.
Functions may be defined by pattern matching. Thus the first
projection of pairs could be defined by:
$\ocamlkwd{let} \,\, \ocamlvaluename{fst} = \ocamlkwd{fun} \,
(\ocamlvaluename{x},\ocamlvaluename{y}) \rightarrow
\ocamlvaluename{x}$ or equivalently (since there is only one pattern
in this case) by: $\ocamlkwd{let} \,\, \ocamlvaluename{fst} \,
(\ocamlvaluename{x}, \ocamlvaluename{y}) = \ocamlvaluename{x}$.
Pattern-matching is also usable in \ocamlkwd{match} expressions which
generalise case analysis, such as:
$\ocamlkwd{match} \,\, \ocamlvaluename{l} \,\, \ocamlkwd{with} \,\,
\emptyL \rightarrow \ocamlkwd{true} \mid \wildcard \rightarrow
\ocamlkwd{false}$, which tests if list \ocamlvaluename{l} is
empty, using underscore as catch-all pattern.

Evaluation is strict, which means that \ocamlvaluename{x} is evaluated
before invoking \ocamlvaluename{f} in the evaluation of
$(\ocamlvaluename{f} \, \ocamlvaluename{x})$. The \ocamlkwd{let}
expressions allow the sequentialization of computations, and the
sharing of sub-computations. Thus $\ocamlkwd{let} \,\,
\ocamlvaluename{x} = \ocamlvaluename{fact} \, 10 \,\, \ocamlkwd{in}
\,\, \ocamlvaluename{x}+\ocamlvaluename{x}$ will compute
$\ocamlvaluename{fact} \, 10$ first, and only once.

Exceptions are declared with the type of their parameters, like in:
$\ocamlkwd{exception} \,\, \ocamlconstr{Failure} \, \ocamlkwd{of} \,
\ocamltypename{string}$. An exceptional value may be raised, like in: 
$\ocamlvaluename{raise}$ $(\ocamlconstr{Failure}$ $\textsf{"div
0"})$ and handled by a \ocamlkwd{try} switching on exception patterns,
like: $\ocamlkwd{try} \, \ocamlvaluename{expression} \,
\ocamlkwd{with} \, \ocamlconstr{Failure} \, \ocamlvaluename{s}
\rightarrow \ldots$ Other imperative constructs may be used, such as 
references, mutable arrays, while loops and I/O commands, 
but we shall seldom need them. Sequences of instructions are 
evaluated in left to right regime in bloc expressions such as:
$\ocamlkwd{begin} \,\, {\ocamlvaluename{e}_1; ...;
\ocamlvaluename{e}_n} \,\, \ocamlkwd{end}$. 

ML is a \emph{modular} language, in the sense that sequences of type,
value and exception declarations may be packed in a structural unit
called a \emph{module}, amenable to separate treatment. 
Modules have types themselves, called \emph{signatures}. Parametric 
modules are called \emph{functors}. The algorithms presented in this
paper will only use this modularity structure to access some library
functions --- the syntax ought to be self-evident.

Despite the focus in this paper is algorithmic, the readers
uninterested in computational details may think of ML
definitions as recursive equations over inductively defined
algebras.


\subsection{Abstract grammar}\label{abstract_grammar}

Let us use OCaml's algebraic type declarations to define the
\emph{abstract grammar} of \core. This grammar captures the
syntactically correct constructs of \core, except those which involve
\emph{tags}~\cite[\S{3.6.69}, \S{8}]{X.680:2002} (since they are
related to the encoding rules) and the \kwdOBJECTIDENTIFIER{} and
\kwdRELATIVEOID{} types and values (for the sake of brevity). The
parser's output is a pair of a type environment and a value
environment. The former is a mapping from type names to subtypes,
corresponding to the type declarations in the \ASN specification, and
the latter is a mapping from value names to values, corresponding to
the value declarations. The subtypes and values are \emph{abstract
syntax trees}, complying with the abstract grammar. We do not follow
the syntactic conventions of OCaml exactly, as detailed below.
\begin{itemize}

  \item in mutual recursive polymorphic variant definitions, we shall
    allow type names instead of a variant, like $\ocamlkwd{type} \,
    \ocamltypename{t} = \textsf{[}\ocamlpvar{K}\textsf{]} \,
    \ocamlkwd{and} \, \ocamltypename{u} = \textsf{[}\ocamlpvar{L} \mid
      \ocamltypename{t}\textsf{]}$ (this limitation can be
    circumvented by an implementation trick out of scope here);

   \item We allow \ASN symbols or keywords as data constructors, like
         `$\pmb{<}$', `$\pmb{..}$', \kwdMINUSINFINITY{} or
         \kwdOCTETSTRING; we then use underscores to 
         denote the location of their arguments, as in 
         `$\wildcard \pmb{<} \wildcard$';

   \item We sometimes write $\kwdCOMPONENTSOF \,\, \T \, \sigma$
     instead of the correct (non-currified) $\kwdCOMPONENTSOF \, (\T,
     \sigma)$, and similarly for other data constructors.

\end{itemize}
The abstract grammar for \core values is defined as follows.  Firstly,
we assume that the parser removes the ambiguity between
enumeration constants~\cite[\S{19}]{X.680:2002} and value
references~\cite[\S{11.4}]{X.680:2002}. For instance, in `\texttt{a}
\texttt{\small T} ::= \texttt{b}', the token \texttt{b} can denote
either an enumeration constant or a value reference, depending on the
definition of the type \texttt{\small T}. The ambiguity can always be
removed just by looking at the type definition (this is easy in
\core). We start by defining the OCaml type \ocamltypename{v\_ref} for
value references, the type \ocamltypename{item}, which is used
later in the definition of enumerated constants, and the type
\ocamltypename{label}, which denotes component names

\smallskip

\noindent
\ocamlkwd{type} \ocamltypename{v\_ref} = \textsf{[}\VRef{}
\ocamlkwd{of} \ocamltypename{string}\textsf{]}

\noindent 
\ocamlkwd{type} \ocamltypename{item} = \ocamltypename{string} 
\ocamlkwd{and} \ocamltypename{label} = \ocamltypename{string}

\smallskip

\noindent
The values of \VRef's argument are noted $y$. Values of type
\ocamltypename{item} are noted $a$, $b$, $c$, and lists of such
values are noted I. Values of type \ocamltypename{label} are noted
$l$, and sets of labels are noted $\Labels$. Let us now define the
numeric types straightforwardly:

\smallskip

\noindent 
\ocamlkwd{type} \ocamltypename{integer} = 
       \textsf{[}\PosInt{} \ocamlkwd{of} \ocamltypename{int}
$\mid$ \NegInt{} \ocamlkwd{of} \ocamltypename{int}\textsf{]}

\noindent \ocamlkwd{type} \ocamltypename{real} = 
       \textsf{[}\PosReal{} \ocamlkwd{of} \ocamltypename{float} 
$\mid$ \NegReal{} \ocamlkwd{of} \ocamltypename{float}\textsf{]}

\smallskip

\noindent
where \PosInt{} means `Positive or null integer'. Next, the OCaml
type for \ASN values is named $\V$:

\smallskip

\noindent 
\ocamlkwd{type} $\V$ = 
\textsf{[}\List{} \ocamlkwd{of} $\V$ \ocamlconstr{list}
$\mid$ \Map{} \ocamlkwd{of} \ocamltypename{label} $\rightarrow$
       $\V$\\
\hspace*{-2mm}
\begin{tabular}{rl}
  $\mid$ & \hspace*{-4mm} 
           \Nil{}
  $\mid$   \kwdTRUE{} 
  $\mid$   \kwdFALSE{}
  $\mid$   \pvString{} \ocamlkwd{of} \ocamltypename{string}
  $\mid$   \ocamltypename{integer}\\
  $\mid$ & \hspace*{-4mm} 
           \wildcard $\pmb{:}$ \wildcardof \ocamltypename{label}
           $\times$ $\V$
  $\mid$   \Enum{} \ocamlkwd{of} \ocamltypename{item}\\
  $\mid$ & \hspace*{-4mm} 
           \HexStr{} \ocamlkwd{of} \ocamltypename{string}
  $\mid$   \BinStr{} \ocamlkwd{of} \ocamltypename{string}
  $\mid$   \kwdNULL{}
  $\mid$   \ocamltypename{real}\\
  $\mid$ & \hspace*{-4mm} \kwdPLUSINFINITY{}
  $\mid$   \hspace*{-5pt}
           \kwdMINUSINFINITY{}
  $\mid$   \ocamltypename{v\_ref}\textsf{]}\\
\end{tabular}

\smallskip

\noindent
where \List{} corresponds to values of \kwdSETOF{} and
\kwdSEQUENCEOF{} types~\cite[\S{25}, \S{27}]{X.680:2002}; \Map{}
models values of the \kwdSET{} and \kwdSEQUENCE{} types~\cite[\S{24},
\S{26}]{X.680:2002} (the argument is a mapping from labels to values);
\Nil{} captures the ambiguous value \verb+{}+, which can be of type
\kwdSETOF, \kwdSEQUENCEOF, \kwdSET{}, \kwdSEQUENCE{} or \kwdBITSTRING;
\kwdTRUE{} and \kwdFALSE{} are the values of the \kwdBOOLEAN{} type;
\pvString{} stands for all kinds of character strings;
$\wildcard~\pmb{:}~\wildcard$ corresponds to \kwdCHOICE{}
values~\cite[\S{28}]{X.680:2002} (thus its argument is a pair of a
label and a value); \Enum{} models enumerated constants; \HexStr{}
corresponds to \kwdOCTETSTRING{} values~\cite[\S{22}]{X.680:2002};
\BinStr{} represents \kwdBITSTRING{}
constants~\cite[\S{21}]{X.680:2002}; \kwdNULL{} captures the special
\kwdNULL{} value~\cite[\S{23}]{X.680:2002}; \kwdPLUSINFINITY{} and
\kwdMINUSINFINITY{} are special values of type \kwdREAL{}.

OCaml values of type $\V$ will be noted $v$, and lists of values
L. The value environments are noted $\ValueEnv$, and have type
$\ocamltypename{string} \rightarrow \V$. The argument of the \Map{}
data constructor is noted $\IdEnv{}$ or $\IdEnv{\Labels}$ when the set
of labels (the domain of the mapping) is $\Labels$ (the empty mapping
is simply noted $\{\}$). The OCaml recursive type ${\cal T}$
represents the \core types:

\smallskip

\noindent 
\ocamlkwd{type} ${\cal T}$ = 
           \textsf{[}\kwdCHOICE{} \ocamlkwd{of} \ocamltypename{label}
           $\rightarrow$ \ocamltypename{t\_ref}
  $\mid$   \kwdOCTETSTRING{}\\
\hspace*{-2mm}
\begin{tabular}{rl}
  $\mid$ & \hspace*{-4mm}
           \kwdSET{} \ocamlkwd{of} \ocamltypename{components}
  $\mid$   \kwdINTEGER{}
  $\mid$   \pvString{}
  $\mid$   \kwdNULL{}\\
  $\mid$ & \hspace*{-4mm}
           \kwdSEQUENCEOF{} \ocamlkwd{of} \ocamltypename{t\_ref}
  $\mid$   \kwdREAL{}
  $\mid$   \kwdBITSTRING{}\\
  $\mid$ & \hspace*{-4mm}
           \kwdSETOF{} \ocamlkwd{of} \ocamltypename{t\_ref}
  $\mid$   \kwdSEQUENCE{} \ocamlkwd{of}
           \ocamltypename{components}\\
  $\mid$ & \hspace*{-4mm}
           \kwdENUMERATED{} \ocamlkwd{of}
           \ocamltypename{item} \ocamlconstr{list}
  $\mid$   \kwdBOOLEAN{}
  $\mid$   \ocamltypename{t\_ref}\textsf{]}
\end{tabular}

\noindent
\ocamlkwd{and} \ocamltypename{t\_ref} = \textsf{[}\TRef{}
\ocamlkwd{of} ${\cal R}$\textsf{]} 
\ocamlkwd{and} ${\cal R}$ = \ocamltypename{string}

\noindent 
\ocamlkwd{and} \ocamltypename{components} =\\
\hspace*{2mm} 
\ocamltypename{label} $\!\rightarrow\!$ \ocamltypename{t\_ref}
$\!\times\!$ 
\textsf{[}\kwdOPTIONAL{} $\!\mid\!$ \kwdDEFAULT{} \ocamlkwd{of}
\ocamltypename{v\_ref}\textsf{]} \ocamlconstr{option}

\smallskip

\noindent
The type \ocamltypename{t\_ref} denotes type references. The type
${\cal R}$ is the countable set of type reference names.  The type
\ocamltypename{components} defines the components of \kwdSET{} and
\kwdSEQUENCE{} \core types: it is a function from labels to pairs
whose first projection is the type reference (in \core, component
types are references) and whose second projection models the
component's optional annotation, which tells whether the component is
mandatory, optional or has a default value.

OCaml values of type ${\cal T}$ are noted \T. Values of type ${\cal
R}$ are noted $x$. The mapping of type $\ocamltypename{label}
\rightarrow \ocamltypename{t\_ref}$, which is the argument of
\kwdCHOICE, is noted $\FieldEnv{}$, or $\FieldEnv{\Labels}$ when the
labels range over $\Labels$ (in general, the domain of a mapping can
be put in subscript), \emph{e.g.,} $\kwdCHOICE \, \FieldEnv{\Labels}$. Values
of type \ocamltypename{components} are mappings noted $\Phi$,
\emph{e.g.,} $\kwdSET \, \Phi$.

\ASN subtyping constraints~\cite[\S{45}]{X.680:2002} are modelled
by the following type ${\cal C}$, whose values are noted C:

\smallskip

\noindent \ocamlkwd{type} ${\cal C}$ = 
\textsf{[}\wildcard \kwdUNION{} \wildcardof 
   ${\cal C}$ $\times$ ${\cal C}$
$\mid$   \ocamltypename{interval}\\
\begin{tabular}{rl}
  $\mid$ & \hspace*{-4mm}
           \wildcard \kwdINTERSECTION{} \wildcardof 
           ${\cal C}$ $\times$ ${\cal C}$
  $\mid$   $\V$\\
  $\mid$ & \hspace*{-4mm}
           \wildcard \kwdEXCEPT{} \wildcardof
           ${\cal C}$ $\times$ ${\cal C}$
  $\mid$   \kwdALLEXCEPT{} \ocamlkwd{of} ${\cal C}$\\
  $\mid$ & \hspace*{-4mm}
           \kwdFROM{} \ocamlkwd{of} ${\cal C}$ 
  $\mid$   \kwdSIZE{} \ocamlkwd{of} ${\cal C}$
  $\mid$   \kwdINCLUDES{} \ocamlkwd{of} \ocamltypename{t\_ref}\\
  $\mid$ & \hspace*{-4mm}
           \kwdWITHCOMPONENTS{} \ocamlkwd{of}\\
         & \ocamltypename{kind}
           $\times$ (\ocamltypename{label} $\rightarrow$ ${\cal C}$
           \ocamlconstr{option}
           $\times$ \ocamltypename{status} \ocamlconstr{option})\\
  $\mid$ & \hspace*{-4mm}
           \kwdWITHCOMPONENT{} \ocamlkwd{of} ${\cal C}$ 
  $\mid$   \kwdPATTERN{} \ocamlkwd{of}
           \ocamltypename{string}\textsf{]} 
\end{tabular}

\noindent \ocamlkwd{and} \ocamltypename{kind} = 
\Partial{} $\mid$ \Full

\noindent \ocamlkwd{and} \ocamltypename{status} = 
       \textsf{[}\kwdPRESENT{}
$\mid$ \kwdABSENT{}
$\mid$ \kwdOPTIONAL\textsf{]}

\noindent \ocamlkwd{and} \ocamltypename{interval} =
\wildcard \wildcard $\pmb{..}$ \wildcard \wildcardof\\
\hspace*{2mm}
\ocamltypename{bound} $\times$
\ocamltypename{in\_out} $\times$ \ocamltypename{in\_out}
$\times$  \ocamltypename{bound}

\noindent \ocamlkwd{and} \ocamltypename{bound} = 
         \textsf{[}\kwdMIN{}
$\mid$   \kwdMAX{}
$\mid$   \pvString{} \ocamlkwd{of} \ocamltypename{string}
$\mid$   \ocamltypename{real}\\
\begin{tabular}{rl}
$\mid$ & \hspace*{-4mm}
         \HexStr{} \ocamlkwd{of} \ocamltypename{string}
$\mid$   \BinStr{} \ocamlkwd{of} \ocamltypename{string}
$\mid$   \ocamltypename{integer}\textsf{]}
\end{tabular}

\noindent \ocamlkwd{and} \ocamltypename{in\_out} = 
$\pmb{<}$ $\mid$ $\pmb{\leqslant}$

\smallskip

The OCaml type \ocamltypename{subtype} models \ASN subtypes. An \ASN
subtype is a pair of a type and an optional subtyping constraint.
OCaml values of type ${\cal C}$ \ocamlconstr{option} are noted
$\sigma$. The type environments are noted $\TypeEnv$, and have type
$\ocamltypename{string} \rightarrow \ocamltypename{subtype}$.

\smallskip

\noindent \ocamlkwd{type} \ocamltypename{subtype} = 
  ${\cal T}$ $\times$ (${\cal C}$ \ocamlconstr{option})


\section{Well-founded Types}
\label{well_founded_types}
%%-*-latex-*-

In this section we tackle the finiteness problem of \ASN types. The
solution shall be used later for solving the same problem over
subtypes. At this point, let us notice again that \emph{some types
have only infinite values}. For instance, \texttt{\small T ::= SET}
\verb+{+\texttt{a} \texttt{\small T}\verb+}+ is forbidden by the
standard since the current encoding rules cannot handle values of such
a type (the encoding engine generated by the \ASN compiler would loop
forever). Let us call \emph{well-founded} a \core type which has at
least one finite value. Let $\TypeEnv$ be a type environment. Let us
write $\TypeEnv \Vdash \T$ if and only if $\T$ is well-founded in
$\TypeEnv$.\footnote{It is possible to give a formal and direct
definition of this concept. Here is a sketch. First, we define a
function $h_{\TypeEnv} : \V \rightarrow \mathbb{N} \cup \{+\infty\}$,
which computes the maximum height of the abstract syntax tree
corresponding to a value in the environment $\TypeEnv$, modulo
references: an empty tree, a leaf and a non-cyclic reference add no
height (the reference is unfolded then); a node adds a height of 1,
and a cyclic reference returns $+\infty$. Second, we define a
predicate $\TypeEnv \vdash v : \T$, read: `$v$ is of type \T{} in the
environment $\TypeEnv$'. Then \T{} is well-founded if, and only if, $\exists v \in
\V$ such as $h_{\TypeEnv}(v) \in \mathbb{N}$ and $\TypeEnv \vdash v :
\T$. We should then prove that our forthcoming axiomatisation of
$\TypeEnv \Vdash \T$ is equivalent to $\exists v \in \V$ such as
$h_{\TypeEnv}(v) \in \mathbb{N}$ and $\TypeEnv \vdash v : \T$.} For
technical reasons, we need another definition: $\TypeEnv, \Path \Vdash
\T$ where $\Path$ (read `history') is a set of type reference
names. These names correspond to the previously encountered type
references: they allow some recursions to be detected and rejected. By
definition: $\TypeEnv \Vdash \T$ is equivalent to $\TypeEnv,
\varnothing \Vdash \T$. This latter relationship is the smallest one
induced by the closure of the following inference rules. Please note
that we use `$x \lhd y$' as a short-hand for `$\ocamlkwd{match} \,\, x
\,\, \ocamlkwd{with} \,\, y \rightarrow \ocamlkwd{true}$' in OCaml
(projection), and \ocamlkwd{as} has the same meaning as in OCaml
(pattern binder):
\begin{mathpar}
\inferrule*[right=Axioms]
  {\neg(\T \lhd \TRef \, \wildcard 
                \mid \kwdCHOICE \, \wildcard)\\
   \neg(\T \lhd \kwdSET \, \wildcard 
                \mid \kwdSEQUENCE \, \wildcard)}
  {\TypeEnv, \Path \Vdash \T}
\end{mathpar}
The rule \textsf{Axioms} states that the types that differ from \TRef,
\kwdCHOICE, \kwdSET{} and \kwdSEQUENCE{} are always well-founded,
e.g. \texttt{\small T ::= SET OF T} is well-founded.
\begin{mathpar}
\inferrule*[right=Ref]
  {x \not\in \Path\\
   \TypeEnv(x) \lhd (\T,\sigma)\\
   \TypeEnv, \{x\} \cup \Path \Vdash \T}
  {\TypeEnv, \Path \Vdash \TRef \, (x)}
\end{mathpar}
The rule \RefTirName{Ref} handles the case of the type references. The
first premise is a look-up in the history to check for a previous
occurrence of the reference name: if present, the type is rejected,
like \texttt{\small T ::= CHOICE} \verb+{+\texttt{a} \texttt{\small
T}\verb+}+. The second premise is a look-up in the specification for
the definition of the referenced type. The last premise is the
checking of the referenced type.
\begin{mathpar}
\inferrule*[right=Choice]
  {\TypeEnv, \Path \Vdash \FieldEnv{}(l)}
  {\TypeEnv, \Path \Vdash \kwdCHOICE \, \FieldEnv{\{l\} \disjunion \Labels}}
\end{mathpar}
In the rule \RefTirName{Choice}, we use the the symbol $\disjunion$ for
the disjoint set union. This rule handles the case of \kwdCHOICE{}
types. The first premise is the projection of a component, which is
checked in the following premise: a \kwdCHOICE{} is well-founded if
and only if one of its component is well-founded, e.g. \texttt{\small
T ::= CHOICE} \verb+{+\texttt{a} \texttt{\small T,} \texttt{b}
\texttt{\small INTEGER}\verb+}+.
\begin{mathpar}
\inferrule*[right=Seq]
  {\TypeEnv, \Path \Vdash \kwdSET \,\, \CompEnv{}} 
  {\TypeEnv, \Path \Vdash \kwdSEQUENCE \,\, \CompEnv{}}
\end{mathpar}
The rule \RefTirName{Seq} simply states that the proof of well-foundedness
of a \kwdSEQUENCE{} type is the same as the one for the \kwdSET{} with
the same components. 
\begin{mathpar}
\inferrule*[right=$\varnothing$-Set]
  {}
  {\TypeEnv, \Path \Vdash \kwdSET \,\, \{\}}
\end{mathpar}
The axiom \textsf{$\varnothing$-Set} says that an
empty \kwdSET{} type is well-founded. 
\begin{mathpar}
\inferrule*[right=SetOpt]
  {\CompEnv{}(l) \lhd (\T', \Some \, \kwdOPTIONAL)\\
   \TypeEnv, \Path \Vdash \kwdSET \,\, \CompEnv{\Labels}}
  {\TypeEnv, \Path \Vdash \kwdSET \,\, \CompEnv{\{l\} \disjunion \Labels}}
\end{mathpar}
The rule \RefTirName{SetOpt} applies when a component is marked as
\kwdOPTIONAL. In this case, it is ignored, and the remaining
components are checked. Indeed, an optional component can be absent in
the value definition, thus any recursion throughout it is valid.
\begin{mathpar}
\inferrule*[right=SetDef]
  {\CompEnv{}(l) \lhd (\T', \None \mid \Some \, (\kwdDEFAULT \wildcard))\\
   \TypeEnv, \Path \Vdash \T'
   \and
   \TypeEnv, \Path \Vdash \kwdSET \,\, \CompEnv{\Labels}
  }
  {\TypeEnv, \Path \Vdash \kwdSET \,\, \CompEnv{\{l\} \disjunion \Labels}}
\end{mathpar}
The rule \RefTirName{SetDef} applies when a component is not marked as
\kwdOPTIONAL{} (first premise): then the type of this
component is checked. This allows to reject for instance
\texttt{\small T ::= SET} \verb+{+\texttt{a} \texttt{\small
T}\verb+}+. The last premise corresponds to the checking of the
remaining components. 

It is not too difficult to see that if a type satisfies our formal
criterion, then it has at least one finite value (the proof tree is
isomorphic to value construction steps, i.e. each judgement $\TypeEnv
\Vdash \T$ can be associated to a value of \T). Also, our inference
system can be considered as an algorithm: just consider the rules and
the premises ordered as they are written. Also, the implicit
existential quantifiers (on $l$ in rules \RefTirName{Choice},
\RefTirName{SetOpt} and \RefTirName{SetDef}) are easy to make
constructive: simply try the components in the given order.

It is worth remarking that the well-foundedness of a type does not
imply that its subtypes have at least a finite value. For example,
\texttt{\small T ::= CHOICE} \verb+{+\texttt{a} \texttt{\small T,}
\texttt{b} \texttt{\small REAL}\verb+}+ is well-founded, but
\texttt{\small U ::= T (WITH COMPONENTS} \verb+{+\texttt{\small ...,}
\texttt{b} \texttt{\small ABSENT}\verb+}+\texttt{)} has no finite
value. That is why the relationship $\Vdash$ will be reused in the
forthcoming constraint-collecting algorithm, which assumes that all
the types in \core are well-founded.


\section{Constraints}
\label{constraints}
%%-*-latex-*-

Till now, we have considered all the problems except solvability of
subtyping constraints (see section~\ref{validation}). Let us start by
quoting Olivier Dubuisson~\cite[\S{13.11}]{Dubuisson:2000}:

\begin{quotation}
`[...] the set operators potentially apply to any subtype constraints;
the problem lies in the interpretation of the constraints in terms of
sets of values to actually determine the possible values of the
resulting type. The designers should be aware, however, that the \ASN
compilers are not likely to check for [...] eccentric constraint
combinations.'
\end{quotation}

The aim of this section is to provide a formal definition of
the constraints, which will allow us (in
section~\ref{full_collection_and_solving}) to determine the values of
each subtype in a specification. The integration of this procedure into
the analysis phase of an \ASN compiler frees the protocol designer
from being `eccentric' or not. The idea is to collect constraints
from the specification, and then solve them. Most of our constraints
are or rely upon \emph{set constraints}. In the 1990s, the field of
the set constraints has been extensively explored, and both major
theoretical results (such as complexity for several classes of
constraints, or links to other fields such as the tree automata
theory) and practical achievements (such as constraint programming
languages or static programme analysis) have been brought to the
light.

On one hand, the \ASN values have a tree structure, thus fit perfectly
the usual domain of set constraints. On the other hand, the \ASN
subtyping constraints are too general to be modeled only with set
constraints: there are also constraints expressed in terms of
intervals, regular expressions and powersets. (Our specific
contribution is the use and resolution of these special constraints.)
So, in order to unify the representation of these concepts, we need to
abstract the values and make them simple constraints. We could
directly reuse the abstract grammar (type $\V$ in
section~\ref{abstract_grammar}), nevertheless it is worth abstracting
further the values at this stage; for instance, from a semantic
point of view, it is more suitable to consider that bit strings, octet
strings, and general strings are all described by regular expressions.
Also, it is more uniform to consider that an integer is in fact an
interval reduced to one element. In order to simplify the introduction
of the collection algorithm, we gather in the same definition the
abstracted values, intervals, regular expressions, sets and powersets
(they will be separated before the solving procedure, because they
require specific algorithms):

\begin{Def}[Expressions]\label{expressions} 
An \emph{expression} $e$ is an element of the set $\E$ defined using
the following grammar and OCaml type definitions:

\medskip

\noindent
\begin{tabular}{ll}
    $\E$ \ASSIGN 
  & \hspace*{-4mm}
    $\SEtop$
    $\mid$ $\SEbot$
    $\mid$ $\alpha$
    $\mid$ $\SEneg\E$
    $\mid$ $\E_0$ $\SEcup$ $\E_1$ 
    $\mid$ $\E_0$ $\SEcap$ $\E_1$
    $\mid$ $\E_0$ $\SEdiff$ $\E_1$\\
  & \hspace*{-4mm} 
    $\mid$ \ocamltypename{series}
    $\mid$ \ocamltypename{map}
    $\mid$ \wildcard $\pmb{:}$ \wildcardof
           \ocamltypename{label} $\times$ $\E$
    $\mid$ \kwdNULL{}\\
  & \hspace*{-4mm} 
    $\mid$ \ocamltypename{real\_interval}
    $\mid$ \ocamltypename{closed\_int\_interval}
    $\mid$ \kwdTRUE{}\\
%\end{tabular}
%\hspace*{7.2mm}
%\begin{tabular}{ll}
  & \hspace*{-4mm}
    $\mid$ \kwdFALSE{}
    $\mid$ \Enum{} \ocamlkwd{of} \ocamltypename{item}
    $\mid$ \Regexp{} \ocamlkwd{of} \ocamltypename{string}\\
  & \hspace*{-4mm}
%%    $\mid$ \ocamltypename{integer}
%%    $\mid$ \ocamltypename{real}
    $\mid$ $\E \SEdiam \ocamltypename{closed\_int\_interval}$
\end{tabular}

\smallskip

\noindent
\ocamlkwd{and} \ocamltypename{series} = 
\ocamlconstr{[}\Cons{} \ocamlkwd{of} $\E$ $\times$
\ocamltypename{series} 
$\mid$ \Nil \ocamlconstr{]}

\noindent
\ocamlkwd{and} \ocamltypename{map} =
\ocamlconstr{[}\Bind{} \ocamlkwd{of} \ocamltypename{identifier}
$\times$ $\E$ $\times$ \ocamltypename{map} $\mid$ \Nil
\ocamlconstr{]} 

\smallskip

\noindent
\ocamlkwd{and} \ocamltypename{real\_interval} =
\ocamlconstr{[}\wildcard \wildcard \asnkwdconstr{..} \wildcard
\wildcardof\\
\hspace*{2mm}
\ocamltypename{real\_bound} $\times$ \ocamltypename{in\_out} 
$\times$ \ocamltypename{in\_out} $\times$
\ocamltypename{real\_bound}\ocamlconstr{]}

\noindent
\ocamlkwd{and} \ocamltypename{real\_bound} =
\ocamlconstr{[}\ocamltypename{real} 
$\mid$ \MinInfReal{} 
$\mid$ \PlusInfReal \ocamlconstr{]}

\noindent
\ocamlkwd{and} \ocamltypename{closed\_int\_interval} =\\
\hspace*{2mm}
\ocamlconstr{[}\Interval{} \ocamlkwd{of} \ocamltypename{int\_bound}
$\times$ \ocamltypename{int\_bound}\ocamlconstr{]}

\noindent
\ocamlkwd{and} \ocamltypename{int\_bound} =
\ocamlconstr{[}\ocamltypename{integer}{} 
$\mid$ \MinInfInt{}
$\mid$ \PlusInfInt \ocamlconstr{]}

\smallskip

\noindent
As a special case, we define some useful constants:

\noindent
\begin{tabular}{lll}
    $\ocamlkwd{let}$ 
  & \hspace*{-4mm}
    $\mathbb{N}$ 
  & \hspace*{-5mm}
    $= \Interval \, (\MinInfInt, \PlusInfInt)$\\
    $\ocamlkwd{and}$
  & \hspace*{-4mm}
    $\mathbb{N}^{+}$
  & \hspace*{-5mm}
    $= \Interval \, (\PosInt \, (0), \PlusInfInt)$\\
    $\ocamlkwd{and}$
  & \hspace*{-4mm}
    $\mathbb{R}$
  & \hspace*{-5mm}
    $= \MinInfReal \, \pmb{<} \asnkwdconstr{..} \pmb{<} \,
     \PlusInfReal$ 
\end{tabular}

The constant $\SEtop$ denotes the set of all terms (see
definition~\ref{terms}); $\SEbot$ the empty set; $\alpha$ is a
variable denoting an expression (see
definition~\ref{variables_in_expressions}); $\SEneg$ is the complement
operator; $\SEcup$, $\SEcap$ and $\SEdiff$ are straightforward. These
operators are the same as Aiken and Wimmers' ones (except their
notation is not dotted). Next is the \ocamltypename{series} type,
which captures a list of expressions; it will be used for encoding
sets of \ASN values of type \kwdSETOF{} and \kwdSEQUENCEOF. Next comes
the \ocamltypename{map} type, which represents mappings from
identifiers to expressions; it will be used to denote sets of \ASN
values of type \kwdSET{} and \kwdSEQUENCE. Next, there is the
$\wildcard\pmb{:}\!\wildcard$ constructor that associates a
\ocamltypename{label} with an expression; it will be used for encoding
values of the \ASN{} type \kwdCHOICE. The \kwdNULL{} constructor is
for the \kwdNULL{} value. Next, we find the
\ocamltypename{real\_interval} type that defines an interval on the
\kwdREAL{} values. Next, we get the
\ocamltypename{closed\_int\_interval} type which represents the closed
intervals on the \kwdINTEGER{} values (in theory, it is not necessary
to have closed intervals, but this design decision makes the algorithm
simpler). The constructors \kwdTRUE{} and \kwdFALSE{} are
obvious. Following, we have the \Enum{} constructor which denotes the
enumerated constants; then \Regexp{} which denotes the \ASN regular
expressions (we call \emph{regexp} for short); then the
\ocamltypename{integer} type; then the \ocamltypename{real} type. The
constructor $\SEdiam$ denotes a powerset (see below). The constructors
\MinInfReal, \PlusInfReal{} stand respectively for $-\infty$ and
$+\infty$ on the \ASN real numbers; \MinInfInt, \PlusInfInt{} are
$-\infty$ and $+\infty$ on the integers.
\end{Def}

The mapping from abstracted values to expressions is quite
straightforward. Just notice that integer and real numbers are mapped
into intervals, and that the value reference names (of type
\ocamltypename{string}) are cast into the set $V$ of variables:

\begin{Def}[From values to expressions]\label{mu}
\noindent
$\ocamlkwd{let} \,\, \ocamlkwd{rec} \,\, \mu : \V \rightarrow \E
= \ocamlkwd{function}$\\
\noindent
\begin{tabular}{rl}
           & \hspace*{-4mm}
             $\List \, [\,] \mid \Map \, \{\} \rightarrow \Nil$\\
         $\mid$ 
           & \hspace*{-4mm}
             $\List \, (\ocamlvaluename{v} \CONS \textnormal{L})
              \rightarrow
              \Cons \, (\mu \, (\ocamlvaluename{v}), \mu \, (\List
              \, (\textnormal{L})))$\\
         $\mid$
           & \hspace*{-4mm}
             $\Map \, (\IdEnv{\{l\} \disjunion \Labels}) \rightarrow
              \Bind \, (l, \mu \, (\IdEnv{}(l)),
              \mu \, (\Map \, (\IdEnv{\Labels})))$\\
         $\mid$
           & \hspace*{-4mm}
             $\BinStr \, (s) \mid \HexStr \, (s) \mid \pvString \, (s)
              \rightarrow \Regexp \, (s)$\\
         $\mid$ 
           & \hspace*{-4mm} 
             $(\PosInt \, \wildcard \mid \NegInt \, \wildcard) \AS
             \ocamlvaluename{v} \rightarrow 
             \Interval \, (\ocamlvaluename{v}, \ocamlvaluename{v})$\\
         $\mid$
           & \hspace*{-4mm}
             $(\PosReal \, \wildcard \mid \NegReal \, \wildcard) \AS
             \ocamlvaluename{v} \rightarrow \ocamlvaluename{v}
             \, \pmb{\leqslant} \asnkwdconstr{..} \pmb{\leqslant} \,
             \ocamlvaluename{v}$\\
         $\mid$
           & \hspace*{-4mm}
             $\kwdMINUSINFINITY \rightarrow \MinInfReal \,
             \pmb{\leqslant} \asnkwdconstr{..} \pmb{\leqslant}
             \MinInfReal$\\
         $\mid$
           & \hspace*{-4mm}
             $\kwdPLUSINFINITY \rightarrow \PlusInfReal \,
             \pmb{\leqslant} \asnkwdconstr{..} \pmb{\leqslant}
             \PlusInfReal$\\
         $\mid$ 
           & \hspace*{-4mm}
             $\ocamlvaluename{l}\/ \pmb{:} \ocamlvaluename{v}
             \rightarrow \ocamlvaluename{l}\/ \pmb{:} \mu 
             (\ocamlvaluename{v})
             \mid
             \VRef \, (y) \rightarrow (y : V)$\\
         $\mid$
           & \hspace*{-4mm}
             $\ocamlvaluename{v} \rightarrow \ocamlvaluename{v}$
\end{tabular}
\end{Def}

The next step is to build the constraints on top of the expressions:

\begin{Def}[Constraints]\label{constraints_def}
A \emph{constraint} $\kappa$ is a conjunction of inclusions over
expressions. The set of constraints is:
${\cal K}$ \ASSIGN ${\cal K}_0 \CEand {\cal K}_1$
            $\mid$ $\E_0 \CEsubseteq \E_1$
            $\mid$ $\E_0 \CEeq \E_1$.
\end{Def}

Note that the notations for the operators defining the constraints are
double-dotted, e.g. $\CEand\!$. The operator $\CEeq$ stands for the
double inclusion.

%% In the next sections, it will be used with several annotations:
%% equality between two integer expressions ($e_0 \CEeqI e_1$),
%% equality between two real expressions ($e_0 \CEeqF e_1$), equality
%% between two regexp expressions ($e_0 \CEeqR e_1$), equality between
%% two basic expressions ($e_0 \CEeqS e_1$), equality over whatever
%% expressions ($e_0 \CEeq e_1$) and equality between two powerset
%% expressions ($\pi_0 \PSCEeq \pi_1$).

Let us consider the sets of value sets associated with \kwdSETOF{} and
\kwdSEQUENCEOF{} types, i.e. powersets of values. For instance,
\texttt{\small A ::= SET (SIZE (4..7)) OF INTEGER} denotes the set of
integer sets whose cardinals belong to the interval $[4;7]$. For each
value of the cardinal it is possible to give the corresponding integer
set expression, and then make the union to get the expression
associated with type \texttt{\small A}. In general, this encoding is
bulky, it makes the constraint solving inefficient and it is unable
to cope with the infinite bound $\mathbb{N}^{+}\!$. A better idea is
to keep the interval of cardinals \emph{together with} an
over-approximation of the powerset itself (this latter contains sets
of any size). The powerset expression $e \SEdiam \varsigma$ is a pair
of an expression $e$ which denotes a powerset (coming from \kwdSETOF{}
and \kwdSEQUENCEOF{} types), and an interval $\varsigma$ of the
elements' cardinals (coming from \kwdSIZE{} subtyping constraints).

%% \begin{Def}[Powerset Expressions]\label{powerset_expressions} A
%% \emph{powerset expression} $\pi$ is an element of the set $\Pi$
%% defined by: $\Pi$ \ASSIGN $\alpha$ $\mid$ $\E \SEdiam
%% \ocamltypename{closed\_int\_interval}$ $\mid$ $\Pi_0 \PSEsum \Pi_1$
%% $\mid$ $\Pi_0 \PSEprod \Pi_1$ $\mid$ $\Pi_0 \PSEdiff \Pi_1$.  It
%% can be a variable $\alpha$ denoting another powerset expression.

%% The expression $\pi_0 \PSEsum \pi_1$ is the union of the powerset
%% expressions $\pi_0$ and $\pi_1$, the expression $\pi_0 \PSEprod
%% \pi_1$ is the intersection of $\pi_0$ and $\pi_1$, and $\pi_0
%% \PSEdiff \pi_1$ is the difference between $\pi_0$ and $\pi_1$. If
%% $\varsigma_1$ and $\varsigma_2$ are disjoint, we can write $\alpha
%% \SEdiam \varsigma_1 \stackrel{\bot}{\PSEsum} \beta \SEdiam
%% \varsigma_2$ instead of $\alpha \SEdiam \varsigma_1 \PSEsum \alpha
%% \SEdiam \varsigma_2$.  \end{Def}

%% \begin{Def}[Powerset Constraints]\label{powerset_constraints} A
%% \emph{powerset constraint} is defined similarly to the (basic)
%% constraints, but using powerset expressions instead of (basic)
%% expressions: ${\cal P}$ \ASSIGN ${\cal P}_0 \! \CEand \! {\cal
%% P}_1$ $\mid$ $\Pi_0 \CEsubseteq \Pi_1$ $\mid$ $\Pi_0 \CEeq
%% \Pi_1$.  \end{Def}

%% We shall use the annotation P (for \emph{powerset}) in $\pi_0
%% \PSCEeq \pi_1$. For the sake of commodity, we shall mix powerset
%% constraints (${\cal P}$) and basic constraints (${\cal K}$) in the
%% same constraint, which will be split after the collection into
%% different systems (i.e. sets of constraints), because the solving
%% procedure is different.

Our idea is to analyse the subtypes in \core and to produce a mixed
constraint for each one. Since component types are all type references
(see section~\ref{mapping}, step~\ref{component_types}) and type
declarations are of the form {\small $<$\textsf{type reference}$>$
\textsf{::=} $<$\textsf{non-reference type without inner
constraints}$>$ \textsf{(}$<$\textsf{subtyping
constraint}$>$\textsf{)}} or simply {\small $<$\textsf{type
reference}$>$ \textsf{::=} $<$\textsf{non-reference type without inner
constraints}$>$} (see step~\ref{type_references}), we can parse each
subtyping constraint without destructuring the type it applies to
(indeed, the constraints on component types are only found at the
top-level). Thanks to this specific shape of \core, the collection
process has two weakly interdependent aspects: the collection on types
and the collection on (proper) subtypes. The link between the two
collections is due to the component types, i.e. type references,
because a type reference can denote either a type or a proper subtype;
hence this link will appear in the treatment of the \TRef{}
constructor.





\section{Constraints from Types}
\label{constraints_from_types}
%%-*-latex-*-

In this section we define the collection of constraints from types
(i.e. declarations of pattern {\small $<$\textsf{type reference}$>$
\textsf{::=} $<$\textsf{non-reference type without inner
constraints}$>$}). It is a mapping ${\cal T} \rightarrow V \rightarrow
(\R \rightarrow V) \rightarrow {\cal K}$, where ${\cal T}$ is the set
of types, $V$ is the set of variables (ranged over by $\alpha$,
$\beta$, $\gamma$ etc.), $\R$ is the countable set of type reference
names (ranged over by $x$) and ${\cal K}$ is the set of
constraints. The notation for the mapping is
$\TSE{\varsigma}{\T}{\alpha}{\TSEmap{}}$, where \T{} is the analysed
type, $\alpha$ is the variable denoting the terms corresponding to the
values of \T{} and $\TSEmap{}$ is a mapping from type reference names
into the variables, which allows the proper handling of the recursive
types. For the sake of clarity, we write sometimes $\TSEmap{\R}$ to
show that the domain of $\TSEmap{}$ is $\R$.

For instance: $\TSE{\varsigma}{\kwdNULL}{\alpha}{\TSEmap{}} = (\alpha
\CEeqS \kwdNULL)$ means that the set constraint associated to the
\kwdNULL{} type, whose set of terms is denoted by $\alpha$, is $\alpha
\CEeqS \kwdNULL$; that is to say the terms are exactly the \kwdNULL{} set
constant. The solution of this constraint is the substitution that
maps the variable $\alpha$ to the set $\{\kwdNULL{}\}$. Let us consider
the constraint collection from the \kwdCHOICE{} type:

\begin{center}
\begin{tabular}{rl}
$\TSE{\varsigma}{\kwdCHOICE \,\,
    \FieldEnv{\{l\}}}{\alpha}{\TSEmap{}}$ = 
 & \hspace*{-3mm}
   $\ocamlkwd{let} \,\, \beta \,\, \emph{be a fresh variable}$\\ 
 & \hspace*{-3mm}
   $\ocamlkwd{in} \,\,
   \TSE{\varsigma}{\FieldEnv{}(l)}{\beta}{\TSEmap{}} 
   \CEand 
   \alpha \CEeqS l \pmb{:} \beta$
\end{tabular}
\end{center}

This equation is for \kwdCHOICE{} types having only one component,
labeled $l$. The expression corresponding to the component
$\FieldEnv{}(l)$ is denoted by the variable $\beta$ which is used to
build the constraint for the \kwdCHOICE{} type.

\medskip

\noindent
\begin{tabular}{rcl}
     $\TSE{\varsigma}{\kwdCHOICE \,\, \FieldEnv{\{l,m\} \disjunion
      \Labels}}{\alpha}{\TSEmap{}}$
   & \hspace*{-4mm} = & \\
     \multicolumn{3}{r}{
       \begin{tabular}{ll}   
            $\ocamlkwd{let}$ 
         & \hspace*{-4mm} 
           $\beta \,\, \emph{and} \,\, \gamma \,\, \emph{be fresh
            variables}$\\  
           $\ocamlkwd{in}$ 
         & \hspace*{-4mm}
           $\TSE{\varsigma}{\kwdCHOICE \,\, \FieldEnv{\{m\} \disjunion
            \Labels}}{\gamma}{\TSEmap{}} \CEand
            \TSE{\varsigma}{\FieldEnv{}(l)}{\beta}{\TSEmap{}}$\\
         & \hspace*{-4mm}
           $\CEand \alpha \CEeqS (l \pmb{:} \beta) \, \SEcup \,
            \gamma$
       \end{tabular}
     }
\end{tabular}

\medskip

This equation is for \kwdCHOICE{} types having at least two
components, $l$ and $m$. First, we compute the set constraint for the
\kwdCHOICE{} with the same components except $l$ (the expression is
named $\gamma$). Then there is the constraint for the component $l$
(variable $\beta$), and finally the constraint on the whole
\kwdCHOICE{} type, whose expression is $\alpha$ and is the 
union of the expression of the component $l$ and $\gamma$.

\medskip

\noindent
\begin{tabular}{rcl}
    $\TSE{f}{\kwdSETOF \,\, \T_0}{\alpha}{\TSEmap{}}$
  & \hspace*{-4mm} = & \\
    \multicolumn{3}{r}{
      \begin{tabular}{ll}
          $\ocamlkwd{let}$
        & \hspace*{-4mm}
          $\beta \,\, \emph{and} \,\,
          \gamma \,\, \emph{be fresh variables}$\\
          $\ocamlkwd{in}$
        & \hspace*{-4mm}
          $\TSE{\I}{\T_0}{\beta}{\TSEmap{}} \!
           \CEand \gamma \CEeqS \Cons \, (\beta, \gamma) \,
           \SEcup \, \Nil
           \CEand \alpha \PSCEeq \gamma \SEdiam \mathbb{N}^{+}$
      \end{tabular}
    }
\end{tabular}
\medskip

The variable $\beta$ denotes the expression from type \T{}, defined
by the constraint $\TSE{\I}{\T}{\beta}{\TSEmap{}}$. The constraint
$\gamma \CEeqS \Cons \, (\beta, \gamma) \, \SEcup \, \Nil$
defines the powerset $\gamma$ over $\beta$ sets. These sets are always
finite because the types in \core are well-founded (see
section~\ref{well_founded_types}). Finally, the powerset constraint
$\alpha \PSCEeq \gamma \SEdiam \mathbb{N}^{+}$ states that the powerset
expression we are looking for is the pair of $\gamma$ (the
powerset itself) and $\mathbb{N}^{+}$ (the allowed cardinals of the
elements of $\gamma$). Note that, because we are in \core, the
\kwdSETOF{} applies to a type (\T) which is always a reference without
subtyping constraint (see section~\ref{mapping},
step~\ref{types_in_SETOF}).

\begin{center}
$\TSE{f}{\kwdSEQUENCEOF \,\, \T}{\alpha}{\TSEmap{}}
 = \TSE{f}{\kwdSETOF \,\, \T}{\alpha}{\TSEmap{}}$
\end{center}

This equation defines the set of terms of the \kwdSEQUENCEOF{} type as
being the same as the \kwdSETOF{} terms. Indeed, the difference
between these two types is only meaningful for the encoding rules
(hence, at the application level).

\begin{center}
\begin{tabular}{rcl} 
    $\TSE{f}{\kwdSEQUENCE \, \{\}}{\alpha}{\TSEmap{}}$
  & \hspace*{-4mm} = 
  & \hspace*{-4mm}
    $\alpha \CEeqS \Nil$
\end{tabular}
\end{center}

\noindent
This equation is for empty \kwdSEQUENCE{} types.

\medskip

\begin{tabular}{rcl}
    $\TSE{f}{\kwdSEQUENCE \, \CompEnv{\{l\} \disjunion
    \Labels}}{\alpha}{\TSEmap{}}$
  & \hspace*{-4mm} = \\
    \multicolumn{3}{l}{
    \begin{tabular}{ll} 
        $\ocamlkwd{let}$ 
      & \hspace*{-4mm}
        $\beta \,\, \emph{and} \,\, \gamma \,\,
         \emph{be fresh variables}$\\
        $\ocamlkwd{in}$
      & \hspace*{-4mm}
        $\TSE{f}{\kwdSEQUENCE \,
         \CompEnv{\Labels}}{\gamma}{\TSEmap{}}$\\
      & \hspace*{-7mm}
        \begin{tabular}{rl}
            $\CEand$
          & \hspace*{-5mm}
            $\ocamlkwd{match} \,\, \CompEnv{}(l) \,\,
             \ocamlkwd{with}$\\
          & \hspace*{-6mm}
            \begin{tabular}{rl}
                & \hspace*{-4mm}
                  $(\T, \None \mid \Some \, (\kwdDEFAULT \,
                   \wildcard)) \rightarrow$\\
                & 
                  $\TSE{\I}{\T}{\beta}{\TSEmap{}}
                   \CEand \alpha \CEeqS \Bind \, (l, \beta,
                   \gamma)$\\
              $\mid$
                & \hspace*{-4mm}
                  $(\T, \Some \, \kwdOPTIONAL) \rightarrow$\\
                & 
                  $\TSE{\I}{\T}{\beta}{\TSEmap{}}
                   \CEand 
                   \alpha \CEeqS \Bind \, (l, \beta, \gamma) \,
                   \SEcup \, \gamma$
            \end{tabular}
        \end{tabular}
    \end{tabular}
    }
\end{tabular}

\medskip

This equation assumes that the set has at least one component, labeled
$l$.  We need first to extract the constraint from the
\kwdSEQUENCE{} without this component: $\TSE{f}{\kwdSEQUENCE \,
\CompEnv{\Labels}}{\gamma}{\TSEmap{}}$. Next, a case analysis is done
on the component, $\CompEnv{}(l)$. In all cases, the variable $\beta$
denote the set of terms of the component type:
$\TSE{f}{\T}{\beta}{\TSEmap{}}$. Hence we add the constraint $\alpha
\CEeqS \Bind \, (l, \beta, \gamma)$, for it represents the set of
terms when the component $l$ is always present. That is why, when this
component can be omitted (see the \kwdOPTIONAL{} case) we complete the
set with $\gamma$. Note that, because we are in \core, the (possible)
default value is a reference (see section~\ref{mapping},
step~\ref{default_values}), hence we do not care here about it (in
particular, we do not constrain it to belong to the set of terms of
the component). Indeed, this value, like all initial top-level values,
have been introduced in a so-called \emph{single value} subtyping
constraint (see step~\ref{types_from_values}) upon their expected
type and which will be considered independently from the current
case. Another detail worth mentioning is the lack, in \core, of
\kwdCOMPONENTSOF{} clause (see section~\ref{mapping},
steps~\ref{COMPONENTS_OF_reference}
and~\ref{COMPONENTS_OF_unfolding}).

\begin{center}
$\TSE{f}{\kwdSET \, \CompEnv{}}{\alpha}{\TSEmap{}}$
= $\TSE{f}{\kwdSEQUENCE \, \CompEnv{}}{\alpha}{\TSEmap{}}$
\end{center}

This equation states that the constraint for the \kwdSET{}
type is the one for the \kwdSEQUENCE{} with the same components. This
is in fact an approximation. Indeed, the difference between \kwdSET{}
and \kwdSEQUENCE{} is that the values of the latter must be given in
the same order as the components are given. Introducing explicitly the
proper combinatorics for the \kwdSET{} values would result in
exponential size of the term set. So we approach the \kwdSET{}
values as if they were \kwdSEQUENCE{} values. In theory, after the
equation solving, we would have to consider these values modulo
permutation (except for the validation purpose, which only requires
the existence of one term in the set).

\begin{center}
\begin{tabular}{rcl}
$\TSE{f}{\kwdINTEGER}{\alpha}{\TSEmap{}}$
  & \hspace*{-4mm} = & \hspace*{-4mm}
        $\alpha \CEeqI \mathbb{N}$\\ 
$\TSE{f}{\kwdREAL}{\alpha}{\TSEmap{}}$
  & \hspace*{-4mm} = & \hspace*{-4mm}
        $\alpha \CEeqF \mathbb{R}$\\ 
$\TSE{f}{\kwdBOOLEAN}{\alpha}{\TSEmap{}}$
  & \hspace*{-4mm} = & \hspace*{-4mm}
        $\alpha \CEeqS \kwdTRUE \,\, \SEcup \,\, \kwdFALSE$\\
$\TSE{f}{\kwdNULL}{\alpha}{\TSEmap{}}$
  & \hspace*{-4mm} = & \hspace*{-4mm}
        $\alpha \CEeqS \kwdNULL$\\
$\TSE{f}{\String}{\alpha}{\TSEmap{}}$
  & \hspace*{-4mm} = & \hspace*{-4mm}
        $\alpha \CEeqR \Regexp \, \stringof{.*}$\\
$\TSE{f}{\kwdBITSTRING}{\alpha}{\TSEmap{}}$
  & \hspace*{-4mm} = & \hspace*{-4mm}
        $\alpha \CEeqR \Regexp \, \texttt{"}\textsf{[}\symbol{92}\textsf{s01]*}\texttt{"}$ \\
$\TSE{f}{\kwdOCTETSTRING}{\alpha}{\TSEmap{}}$
  & \hspace*{-4mm} = & \hspace*{-4mm}
        $\alpha \CEeqR \Regexp \, \texttt{"}\textsf{[}\symbol{92}\textsf{s}\symbol{92}\textsf{da-fA-F]*}\texttt{"}$
\end{tabular}
\end{center}

\begin{center}
\begin{tabular}{rcl}
    $\TSE{f}{\kwdENUMERATED \, [a]}{\alpha}{\TSEmap{}}$
  & \hspace*{-4mm} = 
  & \hspace*{-4mm}
    $\alpha \CEeqS \Enum \, (a)$\\
    $\TSE{f}{\kwdENUMERATED \, (a \CONS b \CONS
     \textnormal{I})}{\alpha}{\TSEmap{}}$
  & \hspace*{-4mm} = 
  & \\
    \multicolumn{3}{r}{
      \begin{tabular}{l}
         $\ocamlkwd{let} \,\, \beta \,\, \emph{be a fresh variable}$\\
         $\ocamlkwd{in} \,\, \TSE{\I}{\kwdENUMERATED \, (b \CONS
          \textnormal{I})}{\beta}{\TSEmap{}}
          \CEand
          \alpha \CEeqS \Enum \, (a) \, \SEcup \, \beta$
      \end{tabular}
    }
\end{tabular}
\end{center}

These equations deal with the constraints from the basic types of
\ASN. You may notice that the terms of the types $\kwdBITSTRING$,
$\kwdOCTETSTRING$ and $\String$ are encoded using the regular
expressions. Also, we do not care about the enumerated and bit
string constants. Indeed, in \core, these values, like all
initially declared values, have been introduced in single value
subtyping constraints (see section~\ref{mapping},
steps~\ref{constants} and~\ref{types_from_values}) on their expected
type, and which wil be considered independently from the current
case. Another detail worth citing is that there is no \kwdINTEGER{}
type defining constants in \core (see section~\ref{mapping},
steps~\ref{constants} and~\ref{constants_unfolding}).

\begin{center}
\begin{tabular}{rcl}
$\TSE{f}{\TRef \, (x)}{\alpha}{\TSEmap{\R}}$
  & \hspace*{-4mm} = & \\
\multicolumn{3}{r}{
  \begin{tabular}{l}
     $\ocamlkwd{if} \,\, x \in \R \,\, \ocamlkwd{then} \,\, \alpha
      \CEeq \TSEmap{}(x)$\\
     $\ocamlkwd{else} \,\, \ocamlkwd{match} \,\, \TypeEnv(x) \,\,
      \ocamlkwd{with}$\\
     \hspace*{5mm}
     \begin{tabular}{rl}
               & \hspace*{-5mm}
                  $(\T_0, \None) \rightarrow \,\,
                  \denotC{\I}{\T_0}{\alpha}{\TSEmap{}
                  \funupdate \{x \mapsto \alpha\}}$\\
        $\mid$ & \hspace*{-5mm}
                 $(\T_0, \Some \, \C_0) \! \rightarrow \!
                 \denotC{f}{\T_0, \C_0}{\alpha}{\TSEmap{} \funupdate \{x
                 \mapsto \alpha\}}$
     \end{tabular}
  \end{tabular}
}
\end{tabular}
\end{center}

This equation defines the constraint collected from a type
reference. Two situations can occur. First, if the reference name $x$
is already in the domain $\R$ of the mapping $\TSEmap{}$, it means
that we already analysed this type before. Then we just emit
an $\alpha \CEeq \TSEmap{}(x)$ constraint that expresses that the
terms of the reference $\alpha$ are exactly the terms of the
referenced type $\TSEmap{}(x)$. Otherwise, we analyse the referenced
type: it can be either a type (first pattern) or a subtype (second
pattern).

In the first case, we just analyse the referenced type, without
forgetting to add ($\funupdate$) a binding $x \mapsto \alpha$ to
$\TSEmap{}$, in order to avoid a loop at run-time in presence of
recursive types\footnote{Hence there is no fixpoint-based approach in
the collection, it lies in the solving procedure instead.}. A valid
situation can be simply illustrated by the declaration \texttt{\small
T ::= CHOICE} \verb+{+\texttt{a REAL,} \texttt{b T}\verb+}+.

In the second case, we use the constraint collection from subtypes,
presented in the next section. A valid situation is: 
\texttt{\small T ::= SET} \verb+{+\texttt{a} \texttt{\small U}\verb+}+
\ \texttt{\small U ::= REAL (0..5)}. This case is the only dependence
between constraints from types and constraints from subtypes.

Let us consider now an example of constraint collection from a
type.\label{example_CHOICE} Let \texttt{\small T ::= CHOICE}
\verb+{+\texttt{item} \texttt{\small INTEGER,} \texttt{and}
\texttt{\small SET OF T,} \texttt{not} \texttt{\small T}\verb+}+. We
want the constraint whose solution in $\alpha$ is the set of terms of
\texttt{T}, that is to say, $\denotC{\I}{\TRef \,
  (\textsf{"T"})}{\alpha}{\{\}} = \denotC{\I}{\kwdCHOICE \,
  \FieldEnv{\Labels}}{\alpha}{{\cal Q}}$, where $\Labels =
\{\textsf{"item"}, \textsf{"and"}, \textsf{"not"}\}$ and ${\cal Q} =
\{\textsf{"T"} \mapsto \alpha\}.$

Let
$\left\{
  \hspace*{-2mm}
  \begin{tabular}{l}
     $\FieldEnv{} \, (\textsf{"item"}) = \kwdINTEGER$\\
     $\FieldEnv{} \, (\textsf{"and"}) = \kwdSETOF \, (\TRef \,
      (\textsf{"T"}))$\\
     $\FieldEnv{} \, (\textsf{"not"}) = \TRef \, (\textsf{"T"})$
  \end{tabular}
 \right.$

\noindent
Then we get: 

\begin{tabular}{rcll}
    $\denotC{\I}{\kwdCHOICE \, \FieldEnv{\Labels}}{\alpha}{{\cal Q}}$
  & \hspace*{-4mm} = 
  & 
  & \hspace*{-6mm}
    $\denotC{\I}{\kwdCHOICE \, \FieldEnv{\Labels \backslash
    \{\textsf{"item"}\}}}{\gamma}{{\cal Q}}$\\
  & 
  & \hspace*{-6mm}
    $\CEand$
  & \hspace*{-6mm}
    $\denotC{\I}{\FieldEnv{} \,
    (\textsf{"item"})}{\beta}{{\cal Q}}$\\
  & 
  & \hspace*{-6mm}
    $\CEand$
  & \hspace*{-6mm}
    $\alpha \CEeqS (\textsf{"item"} \pmb{:} \beta) \,
    \SEcup \, \gamma$\\
%%
    $\denotC{\I}{\kwdCHOICE \, \FieldEnv{\Labels \backslash
     \{\textsf{"item"}\}}}{\gamma}{{\cal Q}}$
  & \hspace*{-4mm} = 
  & 
  & \hspace*{-6mm}
    $\denotC{\I}{\kwdCHOICE \,
    \FieldEnv{\{\textsf{"not"}\}}}{\delta}{{\cal Q}}$\\
  & 
  & \hspace*{-6mm}
    $\CEand$   
  & \hspace*{-6mm}
    $\denotC{\I}{\FieldEnv{} \,
     (\textsf{"and"})}{\varepsilon}{{\cal Q}}$\\
  &
  & \hspace*{-6mm}
    $\CEand$
  & \hspace*{-6mm}
    $\gamma \CEeqS (\textsf{"and"} \pmb{:} \varepsilon)
     \, \SEcup \, \delta$\\
%%
    $\denotC{\I}{\kwdCHOICE \,
     \FieldEnv{\{\textsf{"not"}\}}}{\delta}{{\cal Q}}$
  & \hspace*{-4mm} = 
  & 
  & \hspace*{-6mm}
    $\denotC{\I}{\FieldEnv{} \, (\textsf{"not"})}{\zeta}{{\cal Q}}$\\
  & 
  & \hspace*{-6mm}
    $\CEand$
  & \hspace*{-6mm}
    $\delta \CEeqS \textsf{"not"} \pmb{:} \zeta$\\
    $\denotC{\I}{\FieldEnv{} \, (\textsf{"item"})}{\beta}{{\cal Q}}$
  & \hspace*{-4mm} = 
  & \multicolumn{2}{l}{
      \hspace*{-4mm}
      $\denotC{\I}{\kwdINTEGER}{\beta}{{\cal Q}}$
    }\\
  & \hspace*{-4mm} = 
  & \multicolumn{2}{l}{
      \hspace*{-4mm}
      $\beta \CEeqI \mathbb{N}$
    }\\ 
%%
    $\denotC{\I}{\FieldEnv{} \, (\textsf{"and"})}{\varepsilon}{{\cal
     Q}}$
  & \hspace*{-4mm} = 
  & \multicolumn{2}{l}{
      \hspace*{-4mm}
      $\denotC{\I}{\kwdSETOF \, (\TRef \,
       (\textsf{"T"}))}{\varepsilon}{{\cal Q}}$
    }\\
  & \hspace*{-4mm} = 
  &
  & \hspace*{-4mm}
    $\denotC{}{\TRef \, (\textsf{"T"})}{\eta}{{\cal Q}}$\\
  & 
  & \hspace{-4mm}
    $\CEand$
  & \hspace*{-4mm}
    $\varepsilon \PSCEeq \theta \SEdiam \mathbb{N}^{+}$\\
  &  
  & \hspace*{-4mm}
    $\CEand$
  & \hspace*{-4mm}
    $\theta \CEeq \Cons \, (\eta, \theta) \, \SEcup \, \Nil$\\
%%
    $\denotC{\I}{\FieldEnv{} \, (\textsf{"not"})}{\zeta}{{\cal Q}}$
  & \hspace*{-4mm} = 
  & \multicolumn{2}{l}{
      \hspace*{-4mm}
      $\denotC{\I}{\TRef \, (\textsf{"T"})}{\zeta}{{\cal Q}}$
    }\\
  & \hspace*{-4mm} = 
  & \multicolumn{2}{l}{
      \hspace*{-4mm}
      $\zeta \CEeq \alpha$
    }\\
%%
    $\denotC{\I}{\TRef \, (\textsf{"T"})}{\eta}{{\cal Q}}$
  & \hspace*{-4mm} = 
  & \multicolumn{2}{l}{
      \hspace*{-4mm}
      $\eta \CEeq \alpha$
    }
\end{tabular}

\smallskip

\noindent
After substitution, we get:

\begin{tabular}{rcl}
    $\denotC{f}{\TRef \, (\textsf{"T"})}{\alpha}{\{\}}$
  & \hspace*{-4mm} = &\\
    \multicolumn{3}{r}{
      \hspace*{2mm}
      \begin{tabular}{ll}
        & \hspace*{-4mm}
          $\zeta \CEeq \alpha \,\,
           \CEand \,\, \delta \CEeqS \textsf{"not"} \pmb{:} \zeta \,\, 
           \CEand \,\, \eta \CEeq \alpha$\\
          $\CEand$
        & \hspace*{-4mm}
          $\varepsilon \PSCEeq \theta \SEdiam \mathbb{N}^{+}
           \,\, \CEand \,\, \theta \CEeqS \Cons \, (\eta, \theta) \,
           \SEcup \, \Nil$\\
          $\CEand$
        & \hspace*{-4mm}
          $\gamma \CEeqS (\textsf{"and"} \pmb{:} \varepsilon) \,
           \SEcup \, \delta \,\,
           \CEand \,\,\beta \CEeqI \mathbb{N}$\\
          $\CEand$
        & \hspace*{-4mm}
          $\alpha \CEeqS (\textsf{"item"} \pmb{:} \beta)
           \, \SEcup \, \gamma$
      \end{tabular}
    }
\end{tabular}

\noindent
This constraint implies the system (we do not define formally this
notion in this paper, because we rely on the solving algorithm of
Aiken and Wimmers~\cite{AikenWimmers:1992}):

\centerline{
$\left\{
  \begin{array}{l}
    \alpha \CEeqS (\textsf{"item"} \pmb{:} \beta) \,
    \SEcup \, (\textsf{"and"} \pmb{:} \varepsilon) \, \SEcup \,
    (\textsf{"not"} \pmb{:} \alpha)\\
    \beta \CEeqI \mathbb{N}\\
    \varepsilon \PSCEeq \theta \SEdiam \mathbb{N}^{+}\\
    \theta \CEeqS \Cons \, (\alpha, \theta) \, \SEcup \, \Nil
  \end{array}
\right.$}

The following examples is an illegal type
definition.\label{illegal_type} In section~\ref{validation}
(validation) we mentionned: \texttt{\small T ::= REAL (ALL EXCEPT
  T)}. We want the constraint whose solution in $\alpha$ is the set of
terms of \texttt{\small T}:

\smallskip

$\denotC{}{\TRef \, (\textsf{"T"})}{\alpha}{\{\}} =$\\
\hspace*{1mm} 
$\denotC{}{\kwdREAL, \kwdALLEXCEPT \, (\kwdINCLUDES \,
\textsf{"T"})}{\alpha}{\{\textsf{"T"} \mapsto \alpha\}}$

\smallskip

Let ${\cal Q} = \{\textsf{"T"} \mapsto \alpha\}$. Then we get:

\smallskip

\begin{tabular}{rl}
  $\bullet$ & \hspace*{-4mm}
    $\denotC{}{\kwdREAL}{\beta}{{\cal Q}} = \beta \CEeqF \mathbb{R}$\\
%%
  $\bullet$ & \hspace*{-4mm}
    $\denotC{}{\kwdREAL, \kwdINCLUDES \, (\textsf{"T"})}{\gamma}{{\cal
    Q}}$\\
            & \hspace*{-4mm}
    \begin{tabular}{rl}
      = & \hspace*{-4mm}
          $\denotC{}{\kwdREAL}{\delta}{{\cal Q}} \!
           \CEand \! \denotC{}{\TRef \,
           (\textsf{"T"})}{\varepsilon}{{\cal Q}} 
           \CEand \! \gamma \CEeqI \delta \, \SEcap \, \varepsilon$\\
      = & \hspace*{-4mm}
          $\delta \CEeqF \mathbb{R} 
           \CEand \varepsilon \CEeq \alpha 
           \CEand \gamma \CEeqI \delta \, \SEcap \, \varepsilon$ 
    \end{tabular}\\
%%
  $\bullet$ & \hspace*{-4mm}
    $\denotC{}{\TRef \, (\textsf{"T"})}{\alpha}{\{\}}$\\
            & \hspace*{-4mm}
    \begin{tabular}{rll}
        = 
      & 
      & \hspace*{-5mm}
        $\denotC{}{\kwdREAL}{\beta}{{\cal Q}}
         \CEand \alpha \CEeqI \beta \, \SEdiff \, \gamma$\\
      & \hspace*{-5mm}
        $\CEand$
      & \hspace*{-5mm}
        $\denotC{}{\kwdREAL, \kwdINCLUDES \,
         (\textsf{"T"})}{\gamma}{{\cal Q}}$\\
        = 
      &
      & \hspace*{-5mm}
        $\beta \CEeqF \mathbb{R} \CEand \delta \CEeqF \mathbb{R}
         \CEand \alpha \CEeqI \beta \, \SEdiff \, \gamma$\\
      & \hspace*{-5mm}
        $\CEand$ 
      & \hspace*{-5mm}
        $\varepsilon \CEeq \alpha
         \CEand \gamma \CEeqI \delta \, \SEcap \, \varepsilon$
    \end{tabular}

\end{tabular}

This constraint implies (we do not define formally this notion in this
paper): $\alpha \CEeqI \SEneg\alpha$, which has no solutions. Thus the
declaration of subtype \texttt{\small T} must be rejected.






\section{Constraints from Subtypes}
\label{constraints_from_subtypes}
%%-*-latex-*-

In this section we define the collection of constraints from
subtypes, \emph{i.e.,} on the declarations of pattern {\small $<$\textsf{type
reference}$>$  \textsf{::=} $<$\textsf{non-reference type without
inner constraints}$>$ \textsf{(}$<$\textsf{subtyping
constraint}$>$\textsf{)}}. It is a function of type ${\cal T} \times
{\cal C} \rightarrow V \rightarrow (\R \rightarrow V) \rightarrow
{\cal K}$, where ${\cal C}$ is the set of subtyping constraints. We
use a similar notation for this mapping as in
section~\ref{constraints_from_types},
\emph{e.g.,} $\denotC{f}{\T,\C}{\alpha}{\TSEmap{\R}}$.


\subsection{Regular expression constraint}

The strings can be constrained to belong to a regular language
defined by means of a regular expression (similarly to the Perl
scripting language, or the XML, or the \textsf{grep} Unix command)
introduced by the \kwdPATTERN{} keyword. The following specifies date
and time in format `{\sc dd/mm/yyyy-hh:mm}': \texttt{\small 
DateAndTime ::= VisibleString(PATTERN "}{\small \verb+\+}\texttt{\small d\#2/}{\small \verb+\+}\texttt{\small d\#2/}{\small \verb+\+}\texttt{\small d\#4-}{\small \verb+\+}\texttt{\small d\#2:}{\small \verb+\+}\texttt{\small d\#2")}. Formally, the semantics is expressed as follows:

\begin{center}
\begin{tabular}{rcl}
    $\denotC{f}{\String, \kwdPATTERN \,\, (s)}{\alpha}{\TSEmap{}}$
  & \hspace*{-4mm} = 
  & \hspace*{-4mm}
    $\alpha \CEeqR \Regexp \, (s)$
\end{tabular}
\end{center}

\noindent
This equation defines the constraint collected from a regular
expression constraint (introduced by the \kwdPATTERN{} keyword). This
case is straightforward since we have a built-in notion of
regular expression ($\Regexp$).

\subsection{Union constraint}\label{union_constraint}

Given two constraints, it is possible to create a new constraint that
is the union of both, using the keyword \texttt{\small UNION} or the
symbol `\texttt{|}'. The semantics is then that the new subtype
contains the values of the first subtype and of the second
subtype. For instance, let us define \texttt{\small Day ::=
ENUMERATED} \verb+{+\texttt{mon\-day, tues\-day, wed\-nes\-day,}
\texttt{thurs\-day,} \texttt{fri\-day,} \texttt{sa\-tur\-day,}
\texttt{sun\-day}\verb+}+. Then: \texttt{\small WeekEnd
::= Day (}\texttt{sa\-tur\-day} \texttt{\small UNION}
\texttt{sun\-day)}.

\medskip

\noindent
\begin{tabular}{rcl}
    $\denotC{f}{\T, \C_0 \,\, \kwdUNION \,\, \C_1}{\alpha}{\TSEmap{}}$
  & \hspace*{-4mm} = & \hspace*{-4mm}\\
    \multicolumn{3}{r}{
      \begin{tabular}{ll}
          $\ocamlkwd{let}$ 
        & \hspace*{-4mm}
          $\beta \,\, \emph{and} \,\, \gamma \,\,
           \emph{be fresh variables}$\\
          $\ocamlkwd{in}$ 
        & \hspace*{-4mm}
          $\denotC{f}{\T,\C_0}{\beta}{\TSEmap{}}
          \CEand 
          \denotC{f}{\T,\C_1}{\gamma}{\TSEmap{}}
          \CEand
          \alpha \CEeq \beta \, \SEcup \, \gamma$
      \end{tabular}
    }
\end{tabular}

\medskip

This equation defines the collection of constraints from a union of
subtyping constraints, $\C_0$ and $\C_1$, which apply to a type
\T. First, we collect the constraints from the subtypes $(\T, \C_0)$
and $(\T, \C_1)$. The associated sets are respectively $\beta$ and
$\gamma$, and $\alpha$ is $\beta \, \SEcup \, \gamma$, as expected.


\subsection{Intersection constraint}\label{intersection_constraint}

Given two subtyping constraints, we can create a new one which is the
intersection of both, using the \texttt{\small INTERSECTION} keyword
or the symbol `\texttt{\symbol{94}}'. The semantics is that the new
subtype contains only the values that belong to the two subtypes. For
instance, we can define a type for the French telephone numbers:
\texttt{\small PhoneNumber ::= NumericString ((FROM("0".."9"))
INTERSECTION (SIZE(10)))}, using an alphabet constraint
(section~\ref{alphabet_constraint}) and a size constraint
(section~\ref{size_constraint}). The formal semantics is as follows:

\medskip

\noindent
\begin{tabular}{rcl}
    $\denotC{f}{\T,\C_0 \,\, \kwdINTERSECTION \,\,
     \C_1}{\alpha}{\TSEmap{}}$
  & \hspace*{-4mm} = & \\
    \multicolumn{3}{r}{
      \begin{tabular}{ll} 
          $\ocamlkwd{let}$
        & \hspace*{-4mm} 
          $\beta \,\, \emph{and} \,\, \gamma \,\,
           \emph{be fresh variables}$\\
          $\ocamlkwd{in}$
        & \hspace*{-4mm}
          $\denotC{f}{\T,\C_0}{\beta}{\TSEmap{}} \CEand
           \denotC{f}{\T,\C_1}{\gamma}{\TSEmap{}}
           \CEand
           \alpha \CEeq \beta \, \SEcap \, \gamma$
      \end{tabular}
    }
\end{tabular}

\medskip

\noindent
It is the dual case of section~\ref{union_constraint}.


\subsection{Inclusion constraint}\label{contained_type_constraint}

It is possible to restrict a subtype to only have the values of a
given subtype, using the \texttt{\small INCLUDES} keyword. For instance,
following the example given in section~\ref{union_constraint}, we can
define: \texttt{\small LongWeekEnd ::= Day (INCLUDES WeekEnd |}
\texttt{monday}\texttt{\small )}, or, as a short-hand
\texttt{\small LongWeekEnd ::= Day (WeekEnd |}
\texttt{monday}\texttt{\small )}. Formally, the semantics is

\medskip

\noindent
\begin{tabular}{rcl}
    $\denotC{f}{\T,\kwdINCLUDES \, \T'}{\alpha}{\TSEmap{}}$
  & \hspace*{-4mm} = & \hspace*{-4mm}\\
    \multicolumn{3}{r}{
      \begin{tabular}{ll}
          $\ocamlkwd{let}$ 
        & \hspace*{-4mm}
          $\beta \,\, \emph{and} \,\, \gamma \,\,
           \emph{be fresh variables}$\\
          $\ocamlkwd{in}$ 
        & \hspace*{-4mm}
          $\TSE{\I}{\T}{\beta}{\TSEmap{}} \CEand
           \denotC{f}{\T'}{\gamma}{\TSEmap{}}
           \CEand
           \alpha \CEeq \beta \, \SEcap \, \gamma$
      \end{tabular}
    }
\end{tabular}

\medskip

\noindent
The values of the type \T{} are restricted to be in the set of values
of the type \T'{}. Note that, since we work in \core, the type
\T'{} is a reference (section~\ref{mapping},
step~\ref{contained_reference}). This case is very similar to the
intersection constraint presented in
section~\ref{intersection_constraint}, and naturally has a very
similar semantics.


\subsection{Exclusion constraint}\label{complement_constraint}

The protocol designer can restrict the values of a type to \emph{not}
belong to another subtype, by means of a constraint preceded by
\texttt{\small ALL EXCEPT}, or two constraints separated by
\texttt{\small EXCEPT}. Consider for instance \texttt{\small Lipogram
  ::= IA5String (FROM (ALL EXCEPT ("e" | "E")))}, defining the
set of strings which do not contain the characters \texttt{\small "e"}
and \texttt{\small "E"}. Formally, we have

\medskip

\noindent
\begin{tabular}{rcl}
     $\denotC{f}{\T,\C_0 \,\, \kwdEXCEPT \,\, \C_1}{\alpha}{\TSEmap{}}$
  & \hspace*{-4mm} = &\\ 
    \multicolumn{3}{l}{
      \begin{tabular}{r}
        $\denotC{f}{\T,\C_0 \,\, \kwdINTERSECTION \,\, (\kwdALLEXCEPT
         \,\, \C_1)}{\alpha}{\TSEmap{}}$
      \end{tabular}
    }
\end{tabular}

\medskip

\noindent
This first equation defines the semantics of \texttt{\small EXCEPT}
using the \texttt{\small ALL EXCEPT} constraint. The underlying
ratio\-nale is the following equality on sets: $\beta \backslash
\gamma = \beta \, \cap \, (\alpha \backslash \gamma)$, for all $\beta
\subseteq \alpha$ and $\gamma \subseteq \alpha$. The remaining
situation is

\medskip

\noindent
\begin{tabular}{rcl}
    $\denotC{f}{\T,\kwdALLEXCEPT \,\, \C_0}{\alpha}{\TSEmap{}}$
  & \hspace*{-4mm} = & \\
    \multicolumn{3}{l}{
      \begin{tabular}{ll}
          $\ocamlkwd{let}$
        & \hspace*{-4mm}
          $\beta \,\, \emph{and} \,\,
           \gamma \,\, \emph{be fresh variables}$\\
          $\ocamlkwd{in}$
        & \hspace*{-6mm}
           $\TSE{\I}{\T}{\beta}{\TSEmap{}}
            \CEand \denotC{}{\T,\C_0}{\gamma}{\TSEmap{}}
            \CEand 
            \alpha \CEeq \beta \, \SEdiff \, \gamma$
      \end{tabular}
    }
\end{tabular}

%\vspace*{-5mm}

\subsection{Alphabet constraint}\label{alphabet_constraint}

The strings can be restricted to be built upon a given alphabet, using
the keyword \texttt{\small FROM}. For instance, we can define a
subtype whose values are strings made of capital and small letters:
\texttt{\small CapitalAndSmall ::= IA5String (FROM ("A".."Z" |
  "a".."z"))}. Contrast with the following: \texttt{\small
  CapitalOrSmall ::= IA5String (FROM("A".."Z") | FROM
  ("a".."z"))}. These examples combine an alphabet constraint and an
interval constraint (section~\ref{interval_constraint}).

\medskip

\noindent
\begin{tabular}{rcl}
    $\denotC{f}{\String,\kwdFROM \,\, \C_0}{\alpha}{\TSEmap{}}$
  & \hspace*{-4mm} = & \\
    \multicolumn{3}{r}{
      \begin{tabular}{ll}
          $\ocamlkwd{let}$
        & \hspace*{-4mm}
          $\beta \,\, \emph{be a fresh variable} \,\,
           \ocamlkwd{in}$\\
        & \hspace*{-4mm}
          $\ocamlkwd{let} \,\, \C' = (\kwdSIZE \, (\PosInt \,\, 1))
           \, \kwdINTERSECTION \, \C_0 \, \ocamlkwd{in}$\\
        & \hspace*{-4mm}
          $\ocamlkwd{let} \,\, \kappa =
          \denotC{\I}{\String, \C'}{\beta}{\TSEmap{}}$\\
          $\ocamlkwd{in}$
        & \hspace*{-5mm} 
          $\ocamlkwd{match} \,\, \ocamlvaluename{solve\_regexp} \, 
           (\kappa) \, (\beta) \,\, \ocamlkwd{with}$\\
        & \hspace*{-1mm}
          $\Some \, (\ocamlvaluename{s}) \rightarrow 
           \alpha \CEeq \Regexp \, (\ocamlvaluename{s})$
      \end{tabular}
    }
\end{tabular}

\medskip

This equation defines the constraint collection from the alphabet
subtyping constraint. The constraint $\C_0$ models a set of strings
allowed by \T. We construct a subtyping constraint $\C'$ that defines
the \emph{characters} of $\C_0$ (characters are strings of size one)
in order to build the corresponding semantic constraint $\kappa$. We
must resolve it at this stage because we want to use a regexp
constraint to model the alphabet subtyping constraint. In other words,
we need to compute a string following the \ASN regular expression
syntax and representing $\kappa$. This is done by the function
$\ocamlvaluename{solve\_regexp} : {\cal K} \rightarrow V \rightarrow
\ocamltypename{string} \, \ocamltypename{option}$ which takes as
arguments a constraint denoting an alphabet, and a variable occurring
in this constraint.

First, $\ocamlvaluename{solve\_regexp}\, (\kappa) \, (\beta)$
constructs the set of constraints which are the conjuncts of
$\kappa$. By analysing all the cases of our algorithm
(sections~\ref{constraints_from_types}
and~\ref{constraints_from_subtypes}), it turns out that their patterns
must be of five kinds only: $\alpha \CEeqR \beta \, \SEcup \, \gamma$,
$\alpha \CEeqR \beta \, \SEcap \, \gamma$, $\alpha \CEeqR \beta \,
\SEdiff \, \gamma$, $\alpha \CEeqR \Regexp \, (r)$ or $\alpha \CEeqR
\SEbigcup_{1 \leqslant i \leqslant n}{\Regexp \, (r_i)}$, otherwise we
reject the constraint as inconsistent. The last kind is transformed
into the equivalent: $\alpha \CEeqR \Regexp \,
\textsf{"(}r_1\textsf{)}\mid\textsf{(}r_2\textsf{)}\mid\ldots\mid\textsf{(}r_n\textsf{)"}$.
These constraints form a non-recursive system, which can be solved in
$\beta$ by simple substitution. The last step is to compute strings
denoting intersections, unions and complements of regular expressions
(the result is always a regular expression because regular expressions
are closed under such operations). Consider now the following
example. Let
\begin{align*}
\kappa &= (\alpha \CEeqR \beta \, \SEcup \, \SEneg\gamma)
\, \CEand \, (\beta \CEeqR \Regexp \, \textsf{"[0-9A-F]*"})\\
  & \hphantom{=} \; \CEand \, (\gamma \CEeqR \Regexp \, \textsf{"[01]*"}),
\end{align*}
then $\ocamlvaluename{solve\_regexp} \, (\kappa) \, (\alpha) = \Some
\, \textsf{"([0-9A-F]*)}\mid\textsf{(\symbol{94}[01]*)"}$. If the
return value is \None{}, then it is an inconsistency.


\subsection{Size constraint}\label{size_constraint}

The values of string types can be constrained to have given sizes,
by introducing a subtyping constraint by the keyword \texttt{\small
SIZE}. For instance:

\noindent
\texttt{\small Exactly31BitsString ::= BIT STRING} 
\texttt{\small (SIZE (31))}

\noindent
\texttt{\small StringOf31BitsAtTheMost::= BIT STRING(SIZE(0..31))}

\noindent
\texttt{\small NonEmptyString ::= OCTET STRING (SIZE (1..MAX))}

The size constraint can also apply to \kwdSETOF{} and \kwdSEQUENCEOF{}
types. In that case, the semantics is very different: the values of
the types are sets whose \emph{cardinals} are specified by the size
constraint. Moreover, the constraint must appear between the keywords
\texttt{\small SET} or \texttt{\small SEQUENCE}, and \texttt{\small
  OF}, as in \texttt{\small List\-Of5\-Strings ::= SEQUENCE (SIZE (5))
  OF Print\-able\-String}. Contrast this with \texttt{\small
  List\-Of\-Strings\-Of5\-Char::= SEQUENCE OF Print\-able\-String
  (SIZE (5))}, where the strings themselves are constrained, not the
cardinal of the \kwdSEQUENCEOF. See Figure~\ref{fig:size_constraint}.
\begin{figure}
\centering
\begin{tabular}{rcl}
    $\denotC{f}{\T, \kwdSIZE \,\, \C_0}{\alpha}{\TSEmap{}}$
  & \hspace*{-4mm} = & \\
    \multicolumn{3}{r}{
      \begin{tabular}{lll}
          $\ocamlkwd{let}$
        & \multicolumn{2}{l}{
            \hspace*{-5mm}
            $\beta, \gamma \,\, \emph{and} \,\, \delta \,\,
             \emph{be fresh variables}$
          }\\
          $\ocamlkwd{and}$
        & \multicolumn{2}{l}{
            \hspace*{-5mm}
            $\ocamlvaluename{apply} \,
             (\ocamlvaluename{kind} \!:\! \ocamltypename{string}) \,
             (\Interval \, (\PosInt(\ocamlvaluename{n}),
             \ocamlvaluename{ub})) =$
          }\\
        & \multicolumn{2}{l}{
            \hspace*{-4mm}
            \begin{tabular}{lll}
                $\ocamlkwd{let}$
              & \hspace*{-4.5mm}
                $\ocamlvaluename{up} =$
              & \hspace*{-4.5mm}
                $\ocamlkwd{match} \,\, \ocamlvaluename{ub} \,\,
                 \ocamlkwd{with} \,\, \PlusInfInt \rightarrow \,
                 \textsf{""}$\\
              & 
              & \hspace*{-8mm}
                \begin{tabular}{l}
                  \begin{tabular}{ll}
                      $\mid$
                    & \hspace*{-4mm}
                      $\PosInt \, (m) \rightarrow
                       \ocamlvaluename{string\_of\_int} \, (m)$
                  \end{tabular}
                \end{tabular}\\
                $\ocamlkwd{in}$
              & \multicolumn{2}{l}{
                  \hspace*{-4mm}
                  $\textsf{"(" \symbol{94}
                   \ocamlvaluename{kind} \symbol{94} "{\small \#}("
                   \symbol{94} \ocamlvaluename{string\_of\_int}
                   (\ocamlvaluename{n})}$
                }\\ 
               & \multicolumn{2}{l}{  
                  \hspace*{-4mm}
                  $\textsf{\symbol{94} "," \symbol{94}
                   \ocamlvaluename{up} \symbol{94} "))"}$
                }
            \end{tabular}
          }\\
          $\ocamlkwd{and}$
        & \hspace*{-4mm}
          $\C =$
        & \hspace*{-4mm}
            $(\PosInt \, (0) \, \pmb{\leqslant} \asnkwdconstr{..}
             \pmb{<} \, \kwdMAX)$\\
        &
        & \hspace*{-4mm}
          $\kwdINTERSECTION \,\, \C_0 \,\, \ocamlkwd{in}$\\
          $\ocamlkwd{let}$
        & \multicolumn{2}{l}{
            \hspace*{-6mm}
            $\kappa = \denotC{\I}{\kwdINTEGER,
             \C}{\delta}{\TSEmap{}}$
          }\\
          $\ocamlkwd{in}$
        & \multicolumn{2}{l}{
            \hspace*{-7mm}
            $\ocamlkwd{match} \,\, \ocamlvaluename{solve\_integers} \,
             (\kappa) \, (\delta)$
          }\\
        & \multicolumn{2}{l}{
            \hspace*{-7mm} 
            $\ocamlkwd{with} \,\, \ocamlvaluename{dnf} \,\,
             \ocamlkwd{when} \,\, \ocamlvaluename{dnf} \not=
             \varnothing \,\, \textsf{\small \&\&} \,\,
            \ocamlvaluename{dnf} \not= \{\SEbot\} \rightarrow$
          }\\
        & \multicolumn{2}{l}{
            \hspace*{-8mm}
            \begin{tabular}{l}
              $(\ocamlkwd{match} \,\, \T \,\, \ocamlkwd{with} \,\,
               \kwdBITSTRING \rightarrow$\\
              \begin{tabular}{ll}
                   & \hspace*{-3mm}
                     $\alpha \CEeqR \SEbigcup_{\varsigma
                     \in \ocamlvaluename{dnf}}{\Regexp \,
                     (\!\ocamlvaluename{apply} \, 
                     \textsf{"[01]"} \, (\varsigma))}$\\ 
                $\mid$ 
                   & \hspace*{-5mm}
                     $\kwdOCTETSTRING \rightarrow$\\
                   & \hspace*{-3mm}
                     $\alpha \CEeqR \SEbigcup_{\varsigma
                     \in \ocamlvaluename{dnf}}{\Regexp \,
                     (\!\ocamlvaluename{apply} \, 
                     \textsf{"[0-9A-F]"} \, (\varsigma))}$\\
                $\mid$ 
                   & \hspace*{-5mm}
                     $\String \rightarrow
                     \alpha \CEeqR \SEbigcup_{\varsigma
                     \in \ocamlvaluename{dnf}}{\Regexp \,
                     (\!\ocamlvaluename{apply} \, 
                     \textsf{"."} \, (\varsigma))}$\\
                $\mid$ 
                   & \hspace*{-5mm}
                     $\kwdSETOF \, \T_0 \mid \kwdSEQUENCEOF \, \T_0
                      \rightarrow \TSE{f}{\T_0}{\beta}{\TSEmap{}}
                      \CEand$\\
                   & \hspace{-3mm}
                     $\gamma \CEeqS \Cons \, (\beta, \gamma) \,
                      \SEcup \, \Nil \CEand \alpha \PSCEeq
                      \SEbigcup_{\varsigma \in
                      \ocamlvaluename{dnf}}{\gamma \SEdiam \varsigma})$
               \end{tabular}
            \end{tabular}
          }
      \end{tabular}
    }
\end{tabular}
\caption{Size constraint\label{fig:size_constraint}}
\end{figure}

This equation defines the constraint collected from a size subtyping
constraint $\C_0$. The function \ocamlvaluename{apply} takes as its
first argument a string \ocamlvaluename{kind}, and a positive (closed)
interval as its second argument. The kind is a string that determines
the alphabet of the string \T: \textsf{"[01]"} for bit strings,
\textsf{"[0-9A-F]"} for octet strings, and \textsf{"."} for general
strings. The interval is used to build an \ASN regular expression
determining the repetition of the characters in the kind. Both are
finally combined by \ocamlvaluename{apply}. For example:
$\ocamlvaluename{apply} \, (\textsf{"[01]"}) \, (\PosInt \, (3),
\PlusInfInt) = \textsf{"([01]{\small \#}(3,))"}$, for bit strings
whose size is at least three.

We can construct a subtyping constraint \C{} which is a restriction of
$\C_0$ to the positive integers (indeed, a size must be a positive
integer). Then we can compute the constraint corresponding to the
sizes: `$\ocamlkwd{let} \,\, \kappa = \denotC{\I}{\kwdINTEGER,
  \C}{\delta}{\TSEmap{}}$'. Our aim is to use the regexp constraint
expressions to model the \kwdSIZE{} subtyping constraint, hence we
need to extract from $\kappa$ a suitable set of constraints: a
\emph{disjunctive normal form} which is the smallest union of closed
intervals. This is the purpose of $\ocamlvaluename{solve\_integers} :
{\cal K} \rightarrow V \rightarrow \textsf{[$>$} \,
  \ocamltypename{clo\-sed\_int\_in\-ter\-val} \, \textsf{$\mid$} \,
  \SEbot\textsf{]} \, \ocamltypename{set}$ (as a short-hand, we assume
that we have an abstract polymorphic type \textsf{'a}
\ocamltypename{set}). The returned value of
$\ocamlvaluename{solve\_integers} \, (\kappa) \, (\delta)$ is called
\ocamlvaluename{dnf}, where $\delta$ is the unknown in $\kappa$ we are
interested in. If it is $\varnothing$, it means an inconsistency error
(note that the property $\SEbot \in \ocamlvaluename{dnf} \Rightarrow
\ocamlvaluename{dnf} = \{\SEbot\}$ always holds). The analysis of all
the cases of our algorithm shows that each conjunct in $\kappa$ must
have the pattern $\alpha \CEeqI \beta \, \SEcup \, \gamma$, $\alpha
\CEeqI \beta \, \SEcap \, \gamma$, $\alpha \CEeqI \beta \, \SEdiff \,
\gamma$ or $\alpha \CEeqI \Interval \, (\ocamlvaluename{lb},
\ocamlvaluename{ub})$. They form a non-recursive system of equations
on closed intervals, whose left-hand side variables are unique, hence
resolution by substitution is straightforward.

The type \T{} can be a string. In this case, the constraint is built
as the union ($\SEcup$) of the constraints ($\Regexp \ldots$)
associated (according to the actual kind of \T) to each interval
($\varsigma$) of sizes ($\ocamlvaluename{dnf}$).

Otherwise, \T{} is actually either a \kwdSETOF{} or a \kwdSEQUENCEOF{}
type, and the semantics is very different. There is no easy way to
encode sets whose sizes (cardinals) range over an interval with our
constraints, \emph{e.g.,} a \kwdSET{} value whose size is 3.000, encoded as
the embedding of 3.000 \Cons{} constructors. So we decide to issue a
powerset constraint made of the expression collected from \T{} and the
actual cardinals of its elements. More precisely, we return a
constraint similar to the constraint $\denotC{}{\T}{\alpha}{{\cal Q}}$
(see section~\ref{constraints_from_types}), but whose powerset
constraint is $\SEbigcup_{\varsigma \in \ocamlvaluename{dnf}}{\gamma
\SEdiam \varsigma}$ instead of $\gamma \SEdiam \mathbb{N}^{+}\!$. As
an example, let us consider now the declaration: \texttt{\small A ::=
SET (SIZE (3..8|7..10|12)) OF REAL}. Then the constraint collected
from the type \texttt{\small A} is:
\begin{align*}
\denotC{\I}{\TRef \, \textsf{"A"}}{\alpha}{\{\}} 
&= \denotC{\I}{\kwdSETOF \,\,
\kwdREAL}{\beta}{\textsf{"A"} \mapsto \alpha}\\
&\begin{tabular}{rcll}
    $\CEand \, \alpha$
  & \hspace*{-4mm} 
    $\PSCEeq$
  &
  & \hspace*{-4mm}
    $\beta \SEdiam \Interval \, (\PosInt \, (3), \PosInt \, (10))$\\
  & 
  & \hspace*{-4mm} 
    $\SEcup$
  & \hspace*{-4mm}
    $\beta \SEdiam \Interval \, (\PosInt \,(12), \PosInt \,(12)).$
\end{tabular}
\end{align*}


\subsection{Interval constraint}\label{interval_constraint}

The \kwdINTEGER, \kwdREAL{} and (almost all) string types have totally
ordered values, hence allowing interval definitions for their
values. Consider for instance

\vspace*{-1mm}

{\small
\begin{verbatim}
PositiveOrZeroInteger ::= INTEGER (0..MAX)
PositiveInteger ::= INTEGER (0<..MAX)
NegativeOrZeroInteger ::= INTEGER (MIN..0)
NegativeInteger ::= INTEGER (MIN..<0)
PositiveReal ::= REAL (0<..PLUS-INFINITY)
NegativeReal ::= REAL (MINUS-INFINITY..<0)
RealInterval ::= REAL (4e-5..1e-4)
\end{verbatim}
}

\vspace*{-1mm}

\noindent
\texttt{\small PositiveOrZeroInteger ::= INTEGER (0..MAX)}\\
\noindent
\texttt{\small PositiveInteger ::= INTEGER (0<..MAX)}\\
\noindent
\texttt{\small NegativeOrZeroInteger ::= INTEGER (MIN..0)}\\
\noindent
\texttt{\small NegativeInteger ::= INTEGER (MIN..<0)}\\
\noindent
\texttt{\small PositiveReal ::= REAL (0<..PLUS-INFINITY)}\\
\noindent
\texttt{\small NegativeReal ::= REAL (MINUS-INFINITY..<0)}\\
\noindent
\texttt{\small RealInterval ::= REAL (4e-5..1e-4)}

\noindent
The formal semantics of this kind of constraint is:

\medskip

\noindent
\begin{tabular}{rcl}
    $\denotC{f}{\T, v_0 \, b_0 \, \asnkwdconstr{..} \, b_1 \,
     v_1}{\alpha}{\TSEmap{}}$
  & \hspace*{-4mm} = & \\
    \multicolumn{3}{r}{
      \begin{tabular}{ll}
          $\ocamlkwd{let}$
        & \hspace*{-5mm}
          $\ocamlvaluename{norm\_int\_interval}:$\\
        & \hspace*{-5mm}
          $\ocamltypename{interval} \rightarrow
           \ocamltypename{closed\_int\_interval} = \ldots$\\
%      \end{tabular}
%    }
%\end{tabular}
%\begin{tabular}{rcl}
%    \multicolumn{3}{r}{
%      \begin{tabular}{ll}
          $\ocamlkwd{and}$
        & \hspace*{-5mm}
          $\ocamlvaluename{norm\_real\_interval}:$\\
        & \hspace*{-5mm}
          $\ocamltypename{interval} \rightarrow
           \ocamltypename{real\_interval} = \ldots$\\
          $\ocamlkwd{and}$
        & \hspace*{-5mm}
          $\ocamlvaluename{regexp\_of\_str\_interval}:$\\
        & \hspace*{-5mm}
          $\ocamltypename{interval}\rightarrow \textsf{[}\textsf{$>$}
           \Regexp \, \ocamlkwd{of} \,\ocamltypename{string}\textsf{]}
           = \ldots$\\
          $\ocamlkwd{in}$
        & \hspace*{-7mm}
          $\ocamlkwd{match} \,\, \T \,\, \ocamlkwd{with}$\\ 
          \multicolumn{2}{l}{
            \begin{tabular}{ll} 
              & \hspace*{-4.5mm}
                $\kwdINTEGER \! \rightarrow \! \alpha \! \CEeqI \!\!
                 \ocamlvaluename{norm\_int\_interval} (v_0 \,
                 b_0 \, \asnkwdconstr{..} \, b_1 \, v_1)$\\
                $\mid$ 
              & \hspace*{-4.5mm}
                $\kwdREAL \! \rightarrow \! \alpha \! \CEeqF \!\!
                 \ocamlvaluename{norm\_real\_interval} (v_0 \,
                 b_0 \, \asnkwdconstr{..} \, b_1 \, v_1)$\\
                 $\mid$ 
              & \hspace*{-4.5mm}
                $\String \! \rightarrow \! \alpha \! \CEeqR \!\!
                 \ocamlvaluename{regexp\_of\_str\_interval}
                 (v_0 \, b_0 \, \asnkwdconstr{..} \, b_1 \, v_1)$
            \end{tabular}  
          }     
      \end{tabular}
    }
\end{tabular}

\medskip

This equation defines the constraint collection from value range
subtyping constraints, in other words, interval constraints. These
constraints can be applied to the \kwdINTEGER, \kwdREAL{} and
\String{} types, because the values of these types can be totally
ordered. So, according to the actual type, the specified value range
constraint will be transformed respectively into an OCaml value of
type \ocamltypename{closed\_int\_interval} (\emph{i.e.,} a closed integer
interval), \ocamltypename{real\_interval}, or \textsf{[$>$} \Regexp \,
\ocamlkwd{of} \, \ocamltypename{string}\textsf{]} (that is to say, it
is a regular expression). Due to the lack of room, we cannot give the
pieces of codes of the three functions implementing these
transformations, but there is nothing really difficult here.


\subsection{Value constraint}\label{value_constraint}

It is possible to restrict the set of values of a type to be a
singleton, by simply specifying this unique value between
parenthesis. Consider again the example of
section~\ref{union_constraint}: \texttt{\small Wednesday ::= Day
(}\texttt{\small wednesday}\texttt{\small )}, and the type
\texttt{\small LongWeekEnd} in
section~\ref{contained_type_constraint}. As we showed in
section~\ref{mapping} at step~\ref{types_from_values}, all the
declared values in \core appear in value constraints (of their
expected type). Therefore, this section provides the solution to the
type compatibility problem (section~\ref{validation},
item~\ref{type_compatibility}). Typically, we want to compute the
constraint $\denotC{}{\T, \VRef (y_0)}{\alpha}{\TSEmap{}}$, with the
pseudo-specification excerpt
\begin{center}
\tt y$_0$ X$_1$ ::= y$_1$ \ y$_1$ X$_2$ ::= y$_2$ ... y$_{n-1}$ X$_n$
::= $v$,
\end{center}
where $v$ is not a value reference. We have to unfold the reference
\texttt{y$_0$} and get: $\denotC{}{\T, \VRef (y_0)}{\alpha}{\TSEmap{}}
= \denotC{}{\T, v}{\alpha}{\TSEmap{}}$. Moreover, the semantic model
of \ASN implies that the following condition must hold:
\begin{align*}
\denotC{}{\T}{\beta}{\{\}} &= \denotC{}{\TRef \,
  (\textsf{"}\texttt{X}_1\textsf{"})}{\beta}{\{\}} = \denotC{}{\TRef
  \, (\textsf{"}\texttt{X}_2\textsf{"})}{\beta}{\{\}} = \ldots\\
 &=
\denotC{}{\TRef \, (\textsf{"}\texttt{X}_n\textsf{"})}{\beta}{\{\}},
\end{align*}
\emph{i.e.,} all the types must have the same value sets. For
instance, the \core specification:

\medskip

{\small
 $\left\{
    \begin{tabular}{ll}
        \texttt{a} \texttt{\small A} \texttt{\small ::=} \texttt{b}
      &
        \texttt{\small A} \texttt{\small ::=} \texttt{\small SET}
        \verb+{+\texttt{x} \texttt{\small REAL OPTIONAL}\verb+}+\\
        \texttt{b} \texttt{\small B} \texttt{\small ::= 0.0}
      & \texttt{\small B} \texttt{\small ::=} \texttt{\small SET}
        \verb+{+\texttt{x} \texttt{\small REAL}\verb+}+
    \end{tabular}
  \right.$
}

\medskip

\noindent
must be rejected because the value \verb+{}+ does not belong to the
type \texttt{\small B}, despite \texttt{\small 0.0} belonging to both
\texttt{\small A} and \texttt{\small B}. The following equation
formally defines this semantics:

\medskip

\noindent
\begin{tabular}{rcl}
    $\denotC{}{\T, \VRef \, (y_0)}{\alpha}{\TSEmap{}}$
  & \hspace*{-4mm} = & \\
    \multicolumn{3}{l}{
      \begin{tabular}{ll}
          $\ocamlkwd{let}$
        & \hspace*{-4mm}
          $\beta \,\, \emph{be a fresh variable} \,\, \ocamlkwd{in}$\\
          \multicolumn{2}{l}{
            \begin{tabular}{ll}
                $\ocamlkwd{let}$
              & \hspace*{-4mm}
                $\ocamlkwd{rec} \,\, \ocamlvaluename{unfold} \, (f :
                 \V \rightarrow {\cal K}) = \,\,
                 \ocamlkwd{function}$\\
                \multicolumn{2}{l}{
                  \begin{tabular}{ll}
                    & \hspace*{-4mm}
                      $((x_1, \VRef \, (y_1)) : \R \times
                       \ocamltypename{v\_ref}) \rightarrow$\\ 
                    & \hspace*{-3mm}
                      \begin{tabular}{ll}
                          $\ocamlkwd{let}$
                        & \hspace*{-4mm} 
                          $f' \, (v) = f (v) \CEand
                           \denotC{}{\TRef \, (x_1)}{\beta}{\{\}}$\\
                          $\ocamlkwd{in}$
                        & \hspace*{-5mm}
                          $\ocamlvaluename{unfold} \,\, f' \,
                           (\ValueEnv (y_1))$
                      \end{tabular}\\
                      $\mid$
                    & \hspace*{-4mm}
                      $(\wildcard, v) \rightarrow f \, (v)$
                  \end{tabular}
                }
            \end{tabular}
          }\\
          $\ocamlkwd{in}$
        & \hspace*{-6mm}
          $\ocamlvaluename{unfold} \,\, (\ocamlkwd{fun} \,\, v
           \rightarrow \denotC{}{\T, v}{\alpha}{\{\}} \!\CEand\!
           \denotC{}{\T}{\beta}{\{\}}) \,\, (\ValueEnv (y_0))$
      \end{tabular} 
    }
\end{tabular}

\medskip

\noindent
If the value $v$ is not a reference, the following equations apply,
in accordance with the type \T:

\medskip

\noindent
\begin{tabular}{rcl}
    $\denotC{}{(\wildSETOF \mid \wildSEQUENCEOF),
     v}{\alpha}{\TSEmap{}}$
  & \hspace*{-4mm} = &\\
    \multicolumn{3}{l}{
      \begin{tabular}{l}
        \begin{tabular}{lll}
            $\ocamlkwd{let}$
          & \hspace*{-4mm}
            $\ocamlvaluename{length} =$
          & \hspace*{-4mm}
            $\ocamlkwd{match} \,\, v \,\, \ocamlkwd{with}$\\
          &
          & \hspace*{-5mm}
            \begin{tabular}{ll}
              & \hspace*{-4mm}
                $\List \, (\textnormal{L}) \rightarrow 
                 \ocamlmodulepath{List}.\ocamlvaluename{length} \,
                 (\textnormal{L})$\\
                $\mid$
              & \hspace*{-4mm}
                   $\Nil \rightarrow 0$
            \end{tabular}\\
            $\ocamlkwd{in}$
          & 
            \multicolumn{2}{l}{
              \hspace*{-4mm}
              $\alpha \PSCEeq \mu(v) \SEdiam \mu (\PosInt
               (\ocamlvaluename{length}))$
            }
        \end{tabular}
      \end{tabular}
    }
\end{tabular}

\medskip

\noindent
This equation applies for \kwdSETOF{} and \kwdSEQUENCEOF{} types. The
value $v$ must be either \List{} or \Nil. We compute the length
\ocamlvaluename{length} of $v$. Hence $\mu \, (\PosInt \,
(\ocamlvaluename{length}))$ is the expression corresponding to a
closed integer interval containing only the size of $v$. Finally, we
form the powerset expression $\mu (v) \SEdiam \mu (\ldots)$.

\begin{itemize}

  \item $\denotC{}{\kwdREAL, v}{\alpha}{\TSEmap{}} = \,\, \alpha
         \CEeqF \mu \, (\ocamlvaluename{normalise\_real} \,
         (\ocamlvaluename{v}))$

  \item $\denotC{}{\kwdINTEGER, v}{\alpha}{\TSEmap{}} = \,\, \alpha
         \CEeqI \mu \, (\ocamlvaluename{v})$

  \item $\denotC{}{(\kwdBITSTRING \! \mid \kwdOCTETSTRING \! \mid
         \String), v}{\alpha}{\TSEmap{}} =$\\ 
         \hspace*{3mm} $\alpha \CEeqR \mu \, (\ocamlvaluename{v})$

  \item $\denotC{}{\T, v}{\alpha}{\TSEmap{}} = \,\, \alpha
        \CEeqS \mu \, (\ocamlvaluename{v})$ otherwise.

\end{itemize}

\medskip

These equations handle the remaining cases. If \T{} is \kwdREAL, we
need to normalise the value $v$ by means of the
\ocamlvaluename{normalise\_real} function (whose code is not shown
here). It consists of rewriting the values of the type associated to
\kwdREAL{} using the decimal (dotted) notation:
\verb+{+\texttt{mantissa} \texttt{\small 1,} \texttt{base}
\texttt{\small 10,} \texttt{exponent} \texttt{\small -3}\verb+}+
$\longrightarrow$ \texttt{\small 1E-3}, and each integer literal of
the \kwdREAL{} type is rewritten with the decimal notation:
\texttt{\small 5} $\longrightarrow$ \texttt{\small 5.0}. Then $\mu$ is
applied: $\mu \, (\ocamlvaluename{normalise\_real} \,
(\ocamlvaluename{v}))$. The remaining equations are obvious.


\subsection{Inner type constraints}
 
When some \kwdSETOF{} or \kwdSEQUENCEOF{} type is imported from
another \ASN module, we cannot syntactically insert a constraint on
its elements between the keywords \texttt{\small SET} or
\texttt{\small SEQUENCE} and \texttt{\small OF} (see
section~\ref{size_constraint}). That is why there is another way of
achieving the same goal: using the \texttt{\small WITH COMPONENT}
constraint. For instance, quoting
Dubuisson~\cite[\S{13.8.11}]{Dubuisson:2000}, given \texttt{\small
  TextBlock ::= SEQUENCE OF VisibleString}, we can further refine this
type for lines of no more than 32 characters: \texttt{\small
  Address\-Block ::= Text\-Block (WITH COM\-PO\-NENT (SIZE (1..32)))},
or for digits and spaces: \texttt{\small Digit\-Block ::= Text\-Block
  (WITH COM\-PO\-NENT (Numeric\-String))}. These two subtypes have the
same values as: \texttt{\small Address\-Block ::= SE\-QUEN\-CE OF
  Visible\-String(SIZE(1..32))} and \texttt{\small Digit\-Block
  ::=SE\-QUEN\-CE OF Vi\-si\-ble\-String(Numeric\-String)}. Formally:

\medskip

\noindent
\begin{tabular}{lll}
    $\llbracket (\kwdSETOF \,\, \T_0 \mid \kwdSEQUENCEOF \,\, \T_0)$,
  & & \\
    \multicolumn{3}{l}{
      \hspace*{1mm}
      ${\kwdWITHCOMPONENT \,\, \C_0}\rrbracket_{\alpha}({\TSEmap{}})
       =$
    } \\
    \multicolumn{3}{r}{
      \begin{tabular}{ll}
          $\ocamlkwd{let}$
        & \hspace*{-4mm}
          $\beta \,\, \emph{and} \,\, \gamma \,\, \emph{be fresh
           variables}$\\ 
          $\ocamlkwd{in}$
        & \hspace*{-4mm}
          $\denotC{\I}{\T_0,\C_0}{\beta}{\TSEmap{}}
           \,\, \CEand \,\, \gamma \CEeqS \Cons \, (\beta, \gamma) \,
           \SEcup \, \Nil$\\
        & \hspace*{-4mm}
          $\CEand \,\, \alpha \PSCEeq \gamma\SEdiam\mathbb{N}^{+}$
      \end{tabular}
    }
\end{tabular}

\medskip

\noindent
This equation defines the constraint collection from the
\kwdWITHCOMPONENT{} subtyping constraint. It applies to \kwdSETOF{}
and \kwdSEQUENCEOF{} types, corresponding to powersets of values, by
constraining their elements (but not their cardinals). For instance
\texttt{\small A ::= SET (WITH COMPONENT (0..7)) OF INTEGER} is the
type of the set of integers ranging from 0 to 7. First, we extract the
expression corresponding to the type of the elements \emph{as
constrained by the clause \kwdWITHCOMPONENT}, and bind it to the fresh
variable $\beta$: $\denotC{\I}{\T_0,\C_0}{\beta}{\TSEmap{}}$. Then we
form the constraint $\gamma \CEeqS \Cons \, (\beta, \gamma) \, \SEcup
\, \Nil$ which denotes the powerset over sets of $\beta$
elements. Finally, we get the powerset constraint $\alpha \PSCEeq
\gamma\SEdiam\mathbb{N}^{+}$ which assigns the set $\mathbb{N}^{+}\!$
as the cardinals for the elements of $\gamma$.

\medskip

Now let us introduce the \emph{partial constraints} of \kwdSETOF,
\kwdSEQUENCE{} and \kwdCHOICE{} types. Given 
{\small
\begin{verbatim}
Quadruple ::= SEQUENCE {
  alpha ENUMERATED {in, out} OPTIONAL,
  beta  IA5String OPTIONAL,
  gamma SEQUENCE OF INTEGER,
  delta BOOLEAN DEFAULT TRUE}
\end{verbatim}
}

\noindent
we can derive a subtype whose component \verb+alpha+ is always present
and equals \verb+in+, and the component \verb+gamma+ always has
five elements:
{\small
\begin{verbatim}
Quadruple1 ::= Quadruple (
    WITH COMPONENTS {..., alpha (in) PRESENT, 
                          gamma (SIZE (5))})
\end{verbatim}
}
\noindent
This subtype has the same values as
{\small 
\begin{verbatim}
Quadruple1 ::= SEQUENCE {
  alpha ENUMERATED {in, out} (in),
  beta  IA5String OPTIONAL,
  gamma SEQUENCE SIZE (5) OF INTEGER,
  delta BOOLEAN DEFAULT TRUE}
\end{verbatim}
}
\noindent
The symbol `\verb+...+' means that we constrain only the listed
components: this is a \emph{partial constraint}. If the symbol
`\verb+...+' is missing, there is an implicit constraint
\texttt{\small PRESENT} on the listed components, and \texttt{\small
ABSENT} on the others:
{\small
\begin{verbatim}
Quadruple1 ::= Quadruple (WITH COMPONENTS 
    {alpha (in), gamma (SIZE (5)), delta})
\end{verbatim}
}
\noindent
This subtype has the same values as
{\small 
\begin{verbatim}
Quadruple1 ::= SEQUENCE {
  alpha ENUMERATED {in, out} (in),
  gamma SEQUENCE SIZE (5) OF INTEGER,
  delta BOOLEAN DEFAULT TRUE
}
\end{verbatim}
}

\noindent
This is called a \emph{complete} or \emph{fully specified}
constraint~\cite[\S{47.8.6}]{X.680:2002}. The following equation
defines the constraint collection from this kind of constraint. They
can be expressed in terms of the so-called partial constraints. Let us
first examine the case of the \kwdCHOICE{} type:

\medskip

\noindent
\begin{tabular}{lll}
    $\denotC{f}{\T, \kwdWITHCOMPONENTS \,\, (\Full,
     \FieldConst{\Labels'})}{\alpha}{\TSEmap{}}$
  & \hspace*{-4mm} = &\\
    \multicolumn{3}{l}{
      $\,\, \ocamlkwd{match} \,\, \T \,\, \ocamlkwd{with}$
    }\\
    \multicolumn{3}{l}{
      \begin{tabular}{ll}
        & \hspace*{-4mm}
          $\kwdCHOICE \,\, \FieldEnv{\Labels} \,\, \ocamlkwd{when}
           \,\, \Labels' \subseteq \Labels \rightarrow$\\
        & \hspace*{-4mm}
          \begin{tabular}{ll}
              $\ocamlkwd{let}$
            & \hspace*{-4mm}
              $\textnormal{A} = \{l \mapsto (\None, \Some \,
               \kwdABSENT)\}_{l \in  \Labels \backslash \Labels'} \,\,
               \ocamlkwd{in}$\\
            & \hspace*{-6mm}
              $\ocamlkwd{let} \,\, \NormFieldConst{\Labels'} =
              \FieldConst{\Labels'} \funupdate \textnormal{A}$\\
              $\ocamlkwd{in}$
            & \hspace*{-5mm}
              $\denotC{f}{\T, \kwdWITHCOMPONENTS \,\, (\Partial,
               \NormFieldConst{\Labels'})}{\alpha}{\TSEmap{}}$
          \end{tabular}\\
          $\mid$
        & \hspace*{-4mm}
          $\ldots$
      \end{tabular}%    
    }%
\end{tabular}%

\medskip

The component names (\emph{labels}) which carry constraints,
$\Labels'$, must occur in the type definition: $\Labels' \subseteq
\Labels$ (this is a consistence checking). The fact that the
constraint is fully specified implies that the components not
explicitly constrained must be absent (A) in the value notation,
and, since moreover the type is a \kwdCHOICE, no further constraint is
added to the explicitly constrained components. For instance:
\texttt{\small A ::= CHOICE} \verb+{+\texttt{a} \texttt{\small
INTEGER,} \texttt{b} \texttt{\small T}\verb+}+ \texttt{\small (WITH
COMPONENTS} \verb+{+\texttt{a} \texttt{(3..10)}\verb+}+\texttt{\small
)} is equivalent to \texttt{\small A ::= CHOICE} \verb+{+\texttt{a}
\texttt{\small INTEGER,} \texttt{b} \texttt{\small T}\verb+}+
\texttt{\small (WITH COMPONENTS} \verb+{+\texttt{a} \texttt{\small
(3..10),} \texttt{b} \texttt{\small ABSENT,} \texttt{\small
...}\verb+}+\texttt{\small )}.

This semantics may lead to an assumption we made being broken. Indeed,
we assumed that the types in \core (section~\ref{core}) are
well-founded (section~\ref{well_founded_types}), \emph{i.e.,} the
types have at least a finite value. This property may not hold here
anymore because of this `component-cancellation' semantics. Consider the following type declaration:
\begin{center}
\tt\small A ::= CHOICE \{a A, b INTEGER\} (WITH COMPONENTS \{a\}).
\end{center}
It is equivalent to \texttt{\small A ::= CHOICE} \verb+{+\texttt{a}
\texttt{\small A}\verb+}+, which is obvioulsy not a well-founded
type. That is why we must check again this property on the
\kwdCHOICE{} with only its explicitly constrained components:
$\TypeEnv \Vdash \kwdCHOICE \,\, \FieldEnv{\Labels'}$.

The inner subtyping constraint is represented by
$\FieldConst{\Labels'}$, which is a mapping from labels (belonging to
the set $\Labels'$) to pairs of (possibly optional) subtyping
constraint and (possibly optional) presence constraint. For example
\texttt{\small WITH COMPONENTS} \verb+{+\texttt{a} \texttt{\small
(0..9) PRESENT,} \texttt{b} \texttt{\small ABSENT,} \texttt{c}
\texttt{\small (7)}\verb+}+ corresponds to:

\noindent
$\left\{
\begin{tabular}{lcl}
    $\FieldConst{}(\textsf{"a"})$ 
  & \hspace*{-4mm} = 
  & \hspace*{-4mm} 
    $(\Some \, ( \PosInt (0) \LEQ \pmb{..}
     \LEQ \PosInt (9)),$\\
  &
  & \hspace*{-3mm} 
    $\Some \, \kwdPRESENT)$\\
    $\FieldConst{}(\textsf{"b"})$
  & \hspace*{-4mm} = 
  & \hspace*{-4mm}
    $(\None, \Some \, \kwdABSENT)$\\
    $\FieldConst{}(\textsf{"c"})$
  & \hspace*{-4mm} =
  & \hspace*{-4mm}
    $(\Some \, (\PosInt (7)), \None)$
\end{tabular}
\right.$

When there is no explicit presence constraint, like for the component
\texttt{c} in our last example, there is implicitly a \kwdPRESENT{}
constraint~\cite[\S{48.8.9.2}]{X.680:2002}. This rule is formally
specified in the lines `$\ocamlkwd{let} \,\, \ocamlvaluename{p} =
\ldots \,\, \ocamlkwd{in} \,\, \ocamlkwd{let} \,\,
\NormFieldConst{\Labels'} = \ldots$' (let us recall that the notation
`$v \lhd p$' is a short-hand for `$\ocamlkwd{match} \,\, v \,\,
\ocamlkwd{with} \,\, p \rightarrow \ocamlkwd{true}$').

We can finally create a subtyping constraint that is \emph{partial},
instead of being complete:
\begin{equation*}
\denotC{f}{(\kwdCHOICE \,\, \FieldEnv{\Labels'}), \kwdWITHCOMPONENTS
    \, (\Partial, \ldots)}{\alpha}{\TSEmap{}}.
\end{equation*}
This way we are able to factorize as much as possible the computations
and checks: we reduce the complete constraints to partial ones.

Let us consider the second pattern. It filters the \kwdSET{} and
\kwdSEQUENCE{} types:

\medskip

\noindent
\begin{tabular}{lll}
    \multicolumn{3}{l}{%
      \begin{tabular}{ll}%
          $\mid$
        & \hspace*{-4mm}
          $\ldots$\\
          $\mid$ 
        & \hspace*{-4mm}
          $(\kwdSET \, \CompEnv{\Labels} \mid \kwdSEQUENCE \,
           \CompEnv{\Labels}) \,\, \ocamlkwd{when} \,\, \Labels'
           \subseteq \Labels \rightarrow$\\
        & \hspace*{-4mm}
          \begin{tabular}{lll}
              $\ocamlkwd{let}$
            & \multicolumn{2}{l}{
                \hspace*{-4mm}
                $\textnormal{A} = \{l \mapsto (\None, \Some \,
                 \kwdABSENT)\}_{l \in \Labels \backslash \Labels'}
                 \,\, \ocamlkwd{in}$
              }\\
            & \hspace*{-6mm} 
              $\ocamlkwd{let} \,\, \textnormal{P} =$
            & \hspace*{-4mm}
              $\{l' \mapsto (\sigma, \Some \, \kwdPRESENT) \mid$\\
            &
            & \hspace*{-2mm}
              $\FieldConst{}(l') \lhd (\sigma, \None)\}_{l'\in
               \Labels'} \,\, \ocamlkwd{in}$\\
            & \multicolumn{2}{l}{
                \hspace*{-6mm}
                $\ocamlkwd{let} \,\,
                 \NormFieldConst{\Labels'} = \FieldConst{\Labels'} \,
                 \funupdate \, \textnormal{A} \, \funupdate \,
                 \textnormal{P}$
              }\\
              $\ocamlkwd{in}$
            & \multicolumn{2}{l}{
                \hspace*{-5mm}
                $\denotC{f}{\T,
                 \kwdWITHCOMPONENTS \,\, (\Partial,
                 \NormFieldConst{\Labels'})}{\alpha}{\TSEmap{}}$
              }
          \end{tabular}
      \end{tabular} 
    }
\end{tabular}

\medskip

First we perform the same consistence checking as for \kwdCHOICE{}
types, $\Labels' \subseteq \Labels$. As for \kwdCHOICE{} types, since
the constraint is fully specified, it implies that the components not
explicitly constrained must be absent in the value notation
(A). But, contrary to \kwdCHOICE{} types, the components which are
not explicitly constrained are further constrained to be present
(P)~\cite[\S{47.8.9.3}]{X.680:2002}. For instance: \texttt{\small A
::= SET} \verb+{+\texttt{a} \texttt{\small REAL OPTIONAL}\verb+}+
\texttt{\small (WITH COMPONENTS} \verb+{+\texttt{a} \texttt{\small
(0.0)}\verb+}+\texttt{\small )} is equivalent to \texttt{\small A ::=
SET} \verb+{+\texttt{a} \texttt{\small REAL OPTIONAL}\verb+}+
\texttt{\small (WITH COMPONENTS} \verb+{+\texttt{a} \texttt{\small
(0.0)} \texttt{\small PRESENT,} \texttt{\small
...}\verb+}+\texttt{\small )}.

\medskip

\noindent
\begin{tabular}{rcl}
    $\denotC{f}{\T, \kwdWITHCOMPONENTS \,\, (\Partial,
     \FieldConst{\Labels'})}{\alpha}{\TSEmap{}}$
  & \hspace*{-4mm} = &\\
    \multicolumn{3}{l}{
      $\,\, \ocamlkwd{match} \,\, \T \,\, \ocamlkwd{with}$ 
    }\\
    \multicolumn{3}{l}{
      \begin{tabular}{ll}
        & \hspace*{-4mm}
          $\kwdCHOICE \,\, \FieldEnv{\Labels} \,\, \ocamlkwd{when}
          \,\, \Labels' \subseteq \Labels \rightarrow \ldots$%\\
      \end{tabular}%
    }%
\end{tabular}%

\medskip

This equation defines the constraint collection from the partially
specified subtyping constraints on \kwdCHOICE{} types.  First, we
check the consistence $\Labels' \subseteq \Labels$. Then we use a
non-standard construct $\ocamlkwd{cases} \,\, b_0 \rightarrow e_0 \mid
\ldots \mid b_n \rightarrow e_n \,\, \ocamlkwd{end}$, which
means:`$\ocamlkwd{if}$ $b_0$ $\ocamlkwd{then}$ $e_0$ $\ocamlkwd{else}$
$\ldots$ $\ocamlkwd{if}$ $b_n$ $\ocamlkwd{then}$ $e_n$
$\ocamlkwd{else}$ $\ocamlkwd{fail}$':

\medskip

\noindent
\begin{tabular}{rcl}%
    \multicolumn{3}{l}{%
      \begin{tabular}{ll}%
        & $\ldots$\\
        & $\ocamlkwd{cases}$\\
        & \begin{tabular}{rll}
            & \multicolumn{2}{l}{
                \hspace*{-4mm}
                 $\Labels' = \varnothing \rightarrow \,
                 \TSE{f}{\T}{\alpha}{\TSEmap{}}$
              }\\
              $\mid$ 
            & \hspace*{-4mm}
              $\exists! l' \in \Labels'.$
            & \hspace*{-5mm}
              $(\FieldConst{}(l')
                \lhd (\sigma, \Some \, \kwdPRESENT)$\\
            &
            & \hspace*{-4mm}
              $\BOOLand \TypeEnv \Vdash \kwdCHOICE \,\,
               \FieldEnv{\{l'\}}) \rightarrow$\\
            & \multicolumn{2}{l}{
                \hspace*{-4mm}
                \begin{tabular}{ll}
                    $\ocamlkwd{let}$
                  & \hspace*{-4mm}
                    $\NormFieldConst{} = \{ l' \mapsto (\sigma, \None)
                     \}$\\
                    $\ocamlkwd{and}$
                  & \hspace*{-4mm}
                    $\T' = \kwdCHOICE \,\, \FieldEnv{\{l'\}}$\\
                    \multicolumn{2}{l}{
                      $\ocamlkwd{in} \,
                       \denotC{f}{\T',\kwdWITHCOMPONENTS \,
                       (\Partial,
                        \NormFieldConst{}\,)}{\alpha}{\TSEmap{}}$
                    }
                \end{tabular}
              }\\
              $\mid$
            & \hspace*{-4mm}
              $\ldots$ & %
          \end{tabular}%%  
      \end{tabular}%
    }%
\end{tabular}%

\medskip

The first case corresponds to the lack of an actual inner subtyping
constraint, \emph{i.e.,} $\Labels' = \varnothing$. In this case, the
constraint is simply collected from the type \T{} (this means, for
example, that \texttt{\small A ::= CHOICE} \verb+{+\texttt{a}
\texttt{\small B}\verb+}+ \texttt{\small (WITH COMPONENTS}
\verb+{+\texttt{\small ...}\verb+}+\texttt{\small )} is equivalent to
\texttt{\small A ::= CHOICE} \verb+{+\texttt{a} \texttt{\small
B}\verb+}+).

The second case applies when there is only one \kwdPRESENT{}
constraint applying to a component $l'$ and when the \kwdCHOICE{}
type restricted to $l'$ is not recursive: $\TypeEnv
\Vdash \kwdCHOICE \,\, \FieldEnv{\{l'\}}$, \emph{e.g.,} \texttt{\small A ::=
CHOICE} \verb+{+\texttt{a} \texttt{\small A,} \texttt{b} \texttt{\small
REAL}\verb+}+ \texttt{\small (WITH COMPONENTS} \verb+{+\texttt{a}
\texttt{\small PRESENT, ...}\verb+}+\texttt{\small )} is rejected
because \texttt{\small A ::= CHOICE} \verb+{+\texttt{a} \texttt{\small
A}\verb+}+ is recursive. In this case, the collected constraint is the
same as for the restricted type with no presence constraint,
\emph{e.g.,} \texttt{\small A ::= CHOICE} \verb+{+\texttt{a} \texttt{\small
REAL,}  \texttt{b} \texttt{\small INTEGER}\verb+}+\texttt{\small (WITH
COMPONENTS} \verb+{+\texttt{a} \texttt{\small (0.0)
PRESENT,...}\verb+}+\texttt{\small )} is equivalent to \texttt{\small
A ::= CHOI\-CE} \verb+{+\texttt{a} \texttt{\small REAL}\verb+}+
\texttt{\small (WITH COMPO\-NENTS} \verb+{+\texttt{a} \texttt{\small
(0.0),...}\verb+}+\texttt{\small )}. 

\medskip

\noindent
\begin{tabular}{rcl}%
    \multicolumn{3}{l}{%
      \begin{tabular}{ll}%
        & \begin{tabular}{rll}
              $\mid$
            & \hspace*{-4mm}
              $\ldots$ & \\
              $\mid$
            &
              \multicolumn{2}{l}{
                \hspace*{-4mm}
                $\neg\exists l' \in \Labels'.\FieldConst{}(l')
                 \lhd (\wildcard, \Some \, \kwdPRESENT)$
              }\\
            & \multicolumn{2}{l}{
                \hspace*{-4mm}
                $\!\!\BOOLand \exists l' \in
                 \Labels'.\FieldConst{}(l') \lhd (\sigma,
                 \wildcard) \rightarrow$ 
              }\\
            & \multicolumn{2}{l}{
                \hspace*{-4mm} 
                \begin{tabular}{ll}
                    $\ocamlkwd{let}$ 
                  & \hspace*{-4mm}
                    $\beta \,\, \emph{and} \,\, \gamma \,\,
                     \emph{be fresh variables}$\\
                    $\ocamlkwd{and}$
                  & \hspace*{-4mm}
                    $\NormFieldConst{} = \FieldConst{\Labels'
                     \backslash \{l'\}} \,\,
                     \ocamlkwd{and} \,\, 
                     \T' = \kwdCHOICE \,\, 
                     \FieldEnv{\Labels' \backslash \{l'\}}$\\
                    \multicolumn{2}{l}{
                      $\ocamlkwd{in} \,
                       \denotC{f}{\T',\kwdWITHCOMPONENTS \,
                       (\Partial,
                        \NormFieldConst{}\,)}{\gamma}{\TSEmap{}}$
                    }\\
                  & \hspace*{-7mm}
                    $\!\!\CEand \alpha \CEeqS (l' \pmb{:} \beta)
                     \, \SEcup \, \gamma$\\
                  & \hspace*{-7mm}
                    $\!\! \CEand \ocamlkwd{match}
                     \,\, \sigma \,\, \ocamlkwd{with} \,\, \None
                     \rightarrow \,
                     \TSE{\I}{\FieldEnv{}(l')}{\beta}{\TSEmap{}}$\\
                  & \hspace*{-4mm}
                    \begin{tabular}{ll}
                        $\mid$
                      & \hspace*{-4mm}
                        $\Some \, (\C) \rightarrow \,
                         \denotC{\I}{\FieldEnv{}(l'),
                         \C}{\beta}{\TSEmap{}}$
                    \end{tabular}
                \end{tabular}
              }
          \end{tabular} \\                
        & $\ocamlkwd{end} \,\,\, \textsf{(*} \,\, \ocamlkwd{cases} \,\,
           \textsf{*)}$\\
          $\mid$
        & \hspace*{-4mm}
          $\ldots$
      \end{tabular}
    }
\end{tabular}

\medskip


The last case applies when there is no \kwdPRESENT{} constraint. A
component labelled $l'$ is chosen and the constraint corresponding to
the \kwdCHOICE{} type without $l'$ is collected:
\begin{equation*}
\denotC{f}{\T',\kwdWITHCOMPONENTS \, (\Partial,
  \NormFieldConst{}\,)}{\gamma}{\TSEmap{}}.
\end{equation*}
If there is actually a subtyping constraint $\C$ for the component
$l'$, then the constraint for it is collected:
$\denotC{\I}{\FieldEnv{}(l'), \C}{\beta}{\TSEmap{}}$. Otherwise the
constraint from the component alone is collected:
$\TSE{\I}{\FieldEnv{}(l')}{\beta}{\TSEmap{}}$. The next cases
correspond to the \kwdSET{} and \kwdSEQUENCE{} types:

\medskip

\noindent
\begin{tabular}{ll}
    $\mid$
  & \hspace*{-4mm}
    $\ldots$\\
    $\mid$
  & \hspace*{-4mm}
    $(\kwdSET \, \CompEnv{\Labels} \mid \kwdSEQUENCE \,
     \CompEnv{\Labels}) \,\, \ocamlkwd{when} \,\, \Labels'
     \subseteq \Labels \rightarrow$\\
  & \hspace*{-4mm}
    $\ocamlkwd{cases} \,\, \Labels' = \varnothing \rightarrow \,
     \TSE{f}{\T}{\alpha}{\TSEmap{}}$\\
  & \hspace*{-4mm}
    \begin{tabular}{rll}
        $\mid$ 
      & \hspace*{-4mm}
        $\exists l' \in \Labels'.$
      & \hspace*{-5mm}
        $(\FieldConst{}(l') \lhd (\sigma, \Some \, \kwdOPTIONAL)$\\
      &
      & \hspace*{-4mm}
        $\BOOLand \CompEnv{}(l') \lhd (\T_0, \Some \,
         \kwdOPTIONAL)) \rightarrow$\\
      & \multicolumn{2}{l}{
          \hspace*{-4mm}
          \begin{tabular}{ll}
              $\ocamlkwd{let}$
            & \hspace*{-4mm}
              $\NormFieldConst{\Labels'} = \FieldConst{\Labels'}
               \funupdate \{ l' \mapsto (\sigma, \None) \}$\\
              $\ocamlkwd{in}$
            & \hspace*{-5mm}
              $\denotC{f}{\T, \kwdWITHCOMPONENTS \,\, (\Partial,
               \NormFieldConst{\Labels'})}{\alpha}{\TSEmap{}}$
          \end{tabular}
        }\\
        $\mid$
      & \hspace*{-4mm}
        $\ldots$
    \end{tabular}%
\end{tabular}%

\medskip

The first pattern corresponds to the situation where there is an
\kwdOPTIONAL{} constraint applying to a component marked as
\kwdOPTIONAL{} in the type definition. Hence the presence constraint
is removed, \emph{e.g.,} \texttt{\small A ::= SET} \verb+{+\texttt{a}
\texttt{\small REAL OPTION\-AL}\verb+}+ \texttt{\small (WITH
COM\-PO\-NENTS} \verb+{+\texttt{a} \texttt{\small (0.0) OPTION\-AL,
...}\verb+}+\texttt{\small )} is equivalent to \texttt{\small A ::=
SET} \verb+{+\texttt{a} \texttt{\small REAL OPTION\-AL}\verb+}+
\texttt{\small (WITH COM\-PO\-NENTS} \verb+{+\texttt{a} \texttt{\small
(0.0)} \texttt{\small ...}\verb+}+\texttt{\small )}.

\medskip

\noindent%
\begin{tabular}{ll}%
  & \hspace*{-4mm}%
    \begin{tabular}{rll}%
        $\mid$%
      & \hspace*{-4mm}
        $\dots$\\%
        $\mid$ 
      & \hspace*{-4mm}
        $\exists l' \in \Labels'.$
      & \hspace*{-5mm}
        $(\FieldConst{}(l') \lhd (\None, \Some \, \kwdABSENT)$\\
      &
      & \hspace*{-4mm}
        $\BOOLand \CompEnv{}(l') \lhd (\T_0, \Some \,
         \kwdOPTIONAL)) \rightarrow$\\
      & \multicolumn{2}{l}{
          \hspace*{-4mm}
          \begin{tabular}{ll}
              $\ocamlkwd{let}$
            & \hspace*{-4mm}
              $\C' = \kwdWITHCOMPONENTS \,\, (\Partial,
               \FieldConst{\Labels' \backslash \{l'\}})$\\
              $\ocamlkwd{in}$
            & \hspace*{-4mm}
              $\denotC{f}{\kwdSET \,\, \CompEnv{\Labels \backslash
               \{l'\}}, \C'}{\alpha}{\TSEmap{}}$
          \end{tabular}
        }\\
        $\mid$
      & \hspace*{-4mm}
        $\ldots$
    \end{tabular}
\end{tabular}
       
\medskip

The second pattern rules if an \kwdABSENT{} constraint \emph{alone}
applies to a component marked as \kwdOPTIONAL. Then the component and
the constraints are simply discarded, \emph{e.g.,} \texttt{\small A ::= SET}
\verb+{+\texttt{a} \texttt{\small REAL OPTION\-AL}\verb+}+
\texttt{\small (WITH COM\-PO\-NENTS} \verb+{+\texttt{a} \texttt{\small
ABSENT,...}\verb+}+\texttt{\small )} is equivalent to \texttt{\small A
::= SET} \verb+{}+. If the \kwdABSENT{} constraint was associated to
another kind of constraint, like a value constraint, then it is an
error (in the implementation, this behaviour can be turned into a
warning). For instance, the semantics of \texttt{\small A ::= SET}
\verb+{+\texttt{a} \texttt{\small REAL OPTION\-AL}\verb+}+
\texttt{\small (WITH COM\-PO\-NENTS} \verb+{+\texttt{a} \texttt{\small
(0.0) ABSENT,...}\verb+}+\texttt{\small )} is undefined.

\medskip

\noindent
\begin{tabular}{ll}
  & \hspace*{-4mm}
    \begin{tabular}{rll}    
        $\mid$
      & \hspace*{-4mm}
        $\ldots$\\
        $\mid$ 
      & \hspace*{-4mm}
        $\exists l' \in \Labels'.$
      & \hspace*{-5mm}
        $\FieldConst{}(l') \lhd (\sigma, \Some \, \kwdPRESENT)
        \rightarrow$ \\
      & \multicolumn{2}{l}{
          \hspace*{-4mm}
          \begin{tabular}{ll}
              $\ocamlkwd{let}$
            & \hspace*{-4mm}
              $\NormFieldConst{\Labels'} =
               \FieldConst{\Labels'} \funupdate \{ l' \mapsto (\sigma,
               \None) \} \,\, \ocamlkwd{in}$\\
              \multicolumn{2}{l}{
                \begin{tabular}{l}
                  $\ocamlkwd{let} \,\, \C' = \kwdWITHCOMPONENTS \,\,
                   (\Partial, \NormFieldConst{\Labels'})$
                \end{tabular}
              }\\
              $\ocamlkwd{in}$
            & \hspace*{-6mm}
              $(\ocamlkwd{match} \,\, \CompEnv{}(l') \,\,
               \ocamlkwd{with}$\\
            & \hspace*{-6mm}
              \begin{tabular}{rl}
                & \hspace*{-4mm}
                  $(\T_0, \Some \, \kwdOPTIONAL) \rightarrow$\\
                & \hspace*{-4mm}
                  \begin{tabular}{ll}
                      $\ocamlkwd{let}$
                    & \hspace*{-4mm} 
                      $\NormCompEnv{\Labels} =
                       \CompEnv{\Labels} \funupdate \{ l' \mapsto (\T_0,
                       \None) \}$\\
                      $\ocamlkwd{in}$
                    & \hspace*{-6mm}
                      $(\ocamlkwd{match} \,\,
                       \TypeEnv \Vdash \kwdSET \,\, 
                       \NormCompEnv{\{ l' \}} \,\,
                       \ocamlkwd{with}$\\
                    & \hspace*{-3mm}
                      $\ocamlkwd{true} \rightarrow \,
                       \denotC{f}{\kwdSET \,\,
                       \NormCompEnv{\Labels},
                       \C'}{\alpha}{\TSEmap{}})$
                    \end{tabular}\\
                  $\mid$ 
                & \hspace*{-4mm} 
                  $\wildcard \,\, \rightarrow \,\, \denotC{f}{\kwdSET
                   \,\, \CompEnv{\Labels}, \C'}{\alpha}{\TSEmap{}})$
              \end{tabular}\\
          \end{tabular}
        }\\
        $\mid$
      & \hspace*{-4mm}
        $\ldots$
  \end{tabular}
\end{tabular}

\medskip

The third pattern is appropriate when there is some \kwdPRESENT{}
constraint. If the component in question is not optional, then the
presence constraint is discarded, otherwise its mark \kwdOPTIONAL{} is
discarded and so is the constraint (we need to check also that the
resulting type is still well-founded), \emph{e.g.,} \texttt{\small A
  ::= SET} \verb+{+\texttt{a} \texttt{\small A OPTION\-AL}\verb+}+
\texttt{\small (WITH COM\-PO\-NENTS} \verb+{+\texttt{a} \texttt{\small
  PRE\-SENT, ...}\verb+}+\texttt{\small )} is equivalent to
\texttt{\small A ::= SET} \verb+{+\texttt{a} \texttt{\small
  A}\verb+}+, which has only infinite values, hence must be
rejected. The last case is:

\medskip

\noindent
\begin{tabular}{ll}
  & \hspace*{-4mm}
    \begin{tabular}{rll}
        $\mid$
      & \hspace*{-4mm}
        $\ldots$\\
        $\mid$
      & \hspace*{-4mm} 
        $\exists l' \in \Labels'.$
      & \hspace*{-5mm} 
        $\FieldConst{}(l') \lhd (\sigma, \None) \rightarrow$\\
      & \multicolumn{2}{l}{
          \hspace*{-4mm}
          \begin{tabular}{ll}
              $\ocamlkwd{let}$
            & \hspace*{-4mm} 
              $\beta \,\, \emph{and} \,\, \gamma \,\, \emph{be fresh
               variables}$\\
              $\ocamlkwd{and}$
            & \hspace*{-4mm}
              $\C' = \kwdWITHCOMPONENTS \,\,
               (\Partial, \FieldConst{\Labels' \backslash \{l'\}})$\\
              $\ocamlkwd{in}$
            & \hspace*{-7mm} 
              $\denotC{f}{\kwdSET \,\, \CompEnv{\Labels \backslash
               \{l'\}}, \C'}{\gamma}{\TSEmap{}} \CEand$\\
            & \hspace*{-7mm}
              $\ocamlkwd{match} \,\, \CompEnv{}(l') \,\,
               \ocamlkwd{with}$\\
            & \hspace*{-8mm}
              \begin{tabular}{rl}
                & \hspace*{-4mm}
                  $(\T_0, \None \mid \Some \, (\kwdDEFAULT
                   \wildcard)) \rightarrow$\\ 
                & $\alpha \CEeqS \Bind \, (l', \beta, \gamma)
                  \CEand$\\
                & $(\ocamlkwd{match}
                   \,\, \sigma \,\, \ocamlkwd{with} \,\, \None \,
                  \rightarrow \TSE{\I}{\T_0}{\beta}{\TSEmap{}}$\\
                & \hspace*{-1mm} 
                  \begin{tabular}{ll}
                      $\mid$
                    & \hspace*{-4mm}
                      $\Some \, (\C_0) \rightarrow
                       \TSE{\I}{\T_0,\C_0}{\beta}{\TSEmap{}})$\\
                  \end{tabular}\\
                  $\mid$
                & \hspace*{-4mm}
                  $(\T_0, \Some \, \kwdOPTIONAL) \rightarrow$\\
                & 
                  $\alpha \CEeqS \Bind \, (l', \beta, \gamma) \,
                   \SEcup \, \gamma \,\, \CEand$\\
                & $\ocamlkwd{match}
                   \,\, \sigma \,\, \ocamlkwd{with} \,\, \None \,
                   \rightarrow \TSE{\I}{\T_0}{\beta}{\TSEmap{}}$\\
                & \hspace*{-1mm} 
                  \begin{tabular}{ll}
                      $\mid$
                    & \hspace*{-4mm}
                      $\Some \, (\C_0) \rightarrow
                       \TSE{\I}{\T_0,\C_0}{\beta}{\TSEmap{}}$
                  \end{tabular}
              \end{tabular}
          \end{tabular}
        }
    \end{tabular}\\
  & \hspace*{-4mm}
    $\ocamlkwd{end} \,\,\, \textsf{(*} \,\, \ocamlkwd{cases} \,\,
    \textsf{*)}$
\end{tabular}

\medskip

This last pattern applies when there is no presence constraint on the
component (see \ocamlconstr{None} in $\FieldConst{}(l') \lhd (\sigma,
\None)$). A component labelled $l'$ is chosen, and the constraint
corresponding to \T{} without $l'$ is collected: $\denotC{f}{\kwdSET
\,\, \CompEnv{\Labels \backslash \{l'\}}, \C'}{\gamma}{\TSEmap{}}$ (if
\T{} is actually a \kwdSEQUENCE{} type, we nevertheless transform it
in \kwdSET{} because their semantics is the same in our work).

Next, if there is actually a subtyping constraint $\C$ for the
component $l'$, then the constraint for this subtyped component is
collected: $ \TSE{\I}{\T_0,\C_0}{\beta}{\TSEmap{}}$. Otherwise the
constraint from the component alone is collected:
$\TSE{\I}{\T_0}{\beta}{\TSEmap{}}$. Finally, if the component $l'$ is
not \kwdOPTIONAL{}, then the constraint for \T{} is collected: $\alpha
\CEeqS \Bind \, (l', \beta, \gamma)$. Otherwise, we have to take into
account that the component may be missing in the value notation:
$\alpha \CEeqS \Bind \, (l', \beta, \gamma) \, \SEcup \, \gamma$.


\subsection{Type reference}\label{collection_from_type_references}

\noindent
\begin{tabular}{rcl}
     $\denotC{f}{\TRef \, (x), \C}{\alpha}{\TSEmap{\R}}$
   & \hspace*{-4mm} = & \\
   \multicolumn{3}{r}{
      \begin{tabular}{l}
        $\ocamlkwd{if} \,\, x \in \R
         \,\, \ocamlkwd{then} \,\, \alpha \CEeq \TSEmap{}(x)$\\
        $\ocamlkwd{else} \,\, \ocamlkwd{match} \,\, \TypeEnv(x) \,\,
         \ocamlkwd{with}$\\
        \hspace*{5mm}
        \begin{tabular}{rl}
             & \hspace*{-4mm}
               $(\T_0, \None) \rightarrow \,\,
               \denotC{f}{\T_0,\C}{\alpha}{\TSEmap{}
               \funupdate \{x \mapsto \alpha\}}$\\
          $\mid$
             & \hspace*{-4mm}
               $(\T_0, \Some \, \C_0) \rightarrow$\\
             & 
               $\denotC{f}{\T_0, \C \,\, \kwdINTERSECTION
                \,\, \C_0}{\alpha}{\TSEmap{} \funupdate \{x \mapsto
                \alpha\}}$
        \end{tabular}
      \end{tabular}
   }
\end{tabular}

\medskip

This equation defines the constraint collection from type references
\emph{with a subtyping constraint} --- here $\C$. This situation can
only happen when a component, which is always a reference in \core
(see section~\ref{component_types}), is constrained by $\C$. It is
similar to the equation $\TSE{f}{\TRef \, (x)}{\alpha}{\TSEmap{\R}}$
we gave in the section~\ref{constraints_from_types}. As before, two
cases can occur. First, if the reference $x$ has an image through the
mapping $\TSEmap{}$, it means we previously analysed the referenced
type, and we collect $\alpha \CEeq \TSEmap{}(x)$.

Otherwise, we analyse the referenced subtype $\TypeEnv (x)$. The first
pattern corresponds to the case when it is not a proper subtype,
\emph{i.e.,} it is actually a type $\T_0$: we collect the constraints from it
\emph{with the subtyping constraint $\C$}:
$\denotC{f}{\T_0,\C}{\alpha}{\TSEmap{} \funupdate \{x \mapsto
\alpha\}}$. This case means that the values of \texttt{\small A} in
\texttt{\small A ::= SET} \verb+{+\texttt{a} \texttt{\small B}\verb+}+
\texttt{\small (WITH COM\-PO\-NENTS} \verb+{+\texttt{a} \texttt{\small
(0.0)}\verb+}+\texttt{\small )}, where \texttt{\small B ::= REAL}, are
the same as \texttt{\small A} in \texttt{\small A ::= SET}
\verb+{+\texttt{a} \texttt{\small B}\verb+}+, where \texttt{\small B
::= REAL (0.0)}.

The second pattern applies when the referenced type $\T_0$ is itself
constrained by $\C_0$. Then we form the constraint $\C \,\,
\kwdINTERSECTION \,\, \C_0$ and to apply it to $\T_0$. This case means
that the values of \texttt{\small A} in \texttt{\small A ::= SET}
\verb+{+\texttt{a} \texttt{\small B}\verb+}+ \texttt{\small (WITH
  COM\-PO\-NENTS} \verb+{+\texttt{a} \texttt{\small
  (0.0)}\verb+}+\texttt{\small )}, where \texttt{\small B ::= REAL
  (-2..15)}, are the same as \texttt{\small A} in \texttt{\small A ::=
  SET} \verb+{+\texttt{a} \texttt{\small B}\verb+}+, where
\texttt{\small B ::= REAL ((0.0) INTERSECTION (-2..15))}.

In the following, we show the constraint collection from a complex
subtype and the solved form.\label{subtype_example} Let us consider
the following specification excerpt:\\

\begin{verbatim}
T ::= SET (ALL EXCEPT 
             ((SIZE (6..9))
              INTERSECTION 
              (WITH COMPONENT (16..19))))
      OF INTEGER
\end{verbatim}

We want the constraint whose solution in $\alpha$ is the set of terms
of \texttt{T}, shown in Figure~\ref{fig:TRef}.
\begin{figure}[!t]
\centering
\begin{tabular}{rcll}
    \multicolumn{4}{l}{
      $\denotC{}{\TRef \, (\textsf{"T"})}{\alpha}{\{\}}$
    }\\
  & \hspace*{-4mm} =
  &
  & \hspace*{-5mm}
    $\denotC{}{\kwdSETOF \, \kwdINTEGER}{\beta}{\{\}}$\\
  & 
  & \hspace*{-5mm}
    $\CEand$
  & \hspace*{-5mm}
    $\llbracket \kwdSETOF \, \kwdINTEGER,$\\
  &
  &
  & \hspace*{-3mm}
    $\kwdSIZE \, (\Interval \, (\PosInt \, (6), \PosInt \, (9)))$\\
  & 
  &  
  & \hspace*{-3mm}
    $\kwdINTERSECTION \, (\kwdWITHCOMPONENT$\\
  &
  &
  & \hspace*{-2mm}
    $(\Interval \, (\PosInt \, (16), \PosInt \, (19))))
     \rrbracket_{\varepsilon}(\{\})$\\
  &
  & \hspace*{-5mm}
    $\CEand$
  & \hspace*{-5mm}
    $\alpha \CEeq \beta \, \SEdiff \, \varepsilon$\\
  & \hspace*{-4mm} =
  & 
  & \hspace*{-5mm}
    $(\denotC{}{\kwdINTEGER}{\gamma}{\{\}}
      \CEand
      \delta \CEeq \Cons \, (\gamma, \delta) \, \SEcup \, \Nil$\\
  &
  &
  & \hspace*{-5mm}
    $\CEand \beta \CEeq \delta \SEdiam \mathbb{N}^{+}\!)$\\
  &
  & \hspace*{-5mm}
    $\CEand$
  & \hspace*{-5mm}
    $(\llbracket \kwdSETOF \, \kwdINTEGER, \kwdSIZE$\\
  &
  &
  & $(\Interval \, (\PosInt \, (6), \PosInt \,
     (9))) \rrbracket_{\zeta}(\{\})$\\
  &
  &
  & \hspace*{-5mm}
    $\CEand \llbracket \kwdSETOF \, \kwdINTEGER, \kwdWITHCOMPONENT$\\
  &
  &
  & $(\Interval \, (\PosInt \, (16), \PosInt \,
     (19)))) \rrbracket_{\lambda}(\{\})$\\
  &
  & 
  & \hspace*{-5mm}
    $\CEand \varepsilon \CEeq \zeta \, \SEcap \, \lambda)$\\
  &
  & \hspace*{-5mm}
    $\CEand$
  & \hspace*{-5mm}
    $\alpha \CEeq \beta \, \SEdiff \, \varepsilon$\\
  & \hspace*{-4mm} =
  & 
  & \hspace*{-5mm}
    $\gamma \CEeq \mathbb{N}
     \CEand \delta \CEeq \Cons \, (\gamma, \delta) \, \SEcup \, \Nil
     \CEand \beta \CEeq \delta \SEdiam \mathbb{N}^{+}$\\
  &
  & \hspace*{-5mm}
    $\CEand$
  & \hspace*{-5mm}
    $(\denotC{}{\kwdINTEGER}{\theta}{\{\}}
     \CEand 
     \eta \CEeq \Cons \, (\theta, \eta) \, \SEcup \, \Nil$\\
  &
  & 
  & \hspace*{-5mm}
    $\CEand
     \zeta \CEeq \eta \SEdiam \Interval \, (\PosInt \, (6), \PosInt
     \, (9)))$\\
  &
  & \hspace*{-5mm}
    $\CEand$
  & \hspace*{-5mm}
    $\!(\denotC{}{\kwdINTEGER, \PosInt (16) \! \pmb{\leqslant} \!
     \pmb{..} \! \pmb{\leqslant} \! \PosInt (19)}{\iota}{\{\}}$\\
  &
  &
  & \hspace*{-5mm}
    $\CEand \xi \CEeq \Cons \, (\iota, \xi) \, \SEcup \, \Nil
     \CEand \lambda \CEeq \xi \SEdiam \mathbb{N}^{+}\!)$\\
  &
  & \hspace*{-5mm} 
    $\CEand$
  & \hspace*{-5mm}
    $\varepsilon \CEeq \zeta \, \SEcap \, \lambda
     \CEand
     \alpha \CEeq \beta \, \SEdiff \, \varepsilon$\\
  & \hspace*{-4mm} =
  & 
  & \hspace*{-5mm}
    $\gamma \CEeqI \mathbb{N} 
     \CEand
     \delta \CEeqS \Cons \, (\gamma, \delta) \, \SEcup \, \Nil
     \CEand 
     \beta \CEeq \delta \SEdiam \mathbb{N}^{+}$\\
  &
  & \hspace*{-5mm}
    $\CEand$
  & \hspace*{-5mm}
    $\theta \CEeqI \mathbb{N}
     \CEand
     \eta \CEeqS \Cons \, (\theta, \eta) \, \SEcup \, \Nil$\\
  &
  & \hspace*{-5mm}
    $\CEand$
  & \hspace*{-5mm}
    $\zeta \CEeq \eta \SEdiam \Interval \, (\PosInt \, (6), \PosInt \,
     (9))$\\
  &
  & \hspace*{-5mm}
    $\CEand$
  & \hspace*{-5mm}
    $\iota \CEeqI \Interval \, (\PosInt \, (16), \PosInt \, (19))$\\
  &
  & \hspace*{-5mm}
    $\CEand$
  & \hspace*{-5mm}
    $\xi \CEeqS \Cons \, (\iota, \xi) \, \SEcup \, \Nil
     \CEand
     \lambda \CEeq \xi \SEdiam \mathbb{N}^{+}$\\
  &
  & \hspace*{-5mm}
    $\CEand$
  & \hspace*{-5mm}
    $\varepsilon \CEeq \zeta \, \SEcap \, \lambda
     \CEand
     \alpha \CEeq \beta \, \SEdiff \, \varepsilon$
\end{tabular}
\caption{Constraint collection from a complex subtype\label{fig:TRef}}
\end{figure}

The simplified result of the solving procedure (not presented here),
is:

\medskip

\noindent
$\left\{
\begin{tabular}{rcllll}
    $\alpha$
  & \hspace*{-4mm}
    $\CEeq$
  & 
  & \multicolumn{3}{l}{
      \hspace*{-4mm}
      $\omega \SEdiam \Interval \, (\PosInt \, (6), \PosInt \, (9))$ 
    }\\
  & 
  & \hspace*{-4mm}
    $\SEcup$
  & \hspace*{-4mm}
    $\delta \SEdiam ($
  &
  & \hspace*{-4mm}
    $\Interval \, (\PosInt \, (0), \PosInt \, (5))$\\
  & 
  & 
  &
  & \hspace*{-4mm}
    $\SEcup$
  & \hspace*{-4mm}
    $\Interval \, (\PosInt \, (10), \PlusInfInt))$\\
    $\omega$
  & \hspace*{-4mm}
    $\CEeqS$
  & \multicolumn{4}{l}{
       \hspace*{-4mm}
       $\Cons \, (\phi, \nu)$
    }\\
    $\phi$
  & \hspace*{-4mm}
    $\CEeqI$
  & 
  & \multicolumn{3}{l}{
      \hspace*{-4mm}
      $\Interval \, (\MinInfInt, \PosInt \, (15))$
    }\\
  & 
  & \hspace*{-4mm}
    $\SEcup$
  & \multicolumn{3}{l}{
      \hspace*{-4mm}
      $\Interval \, (\PosInt \, (20), \PlusInfInt)$
    }\\
    $\nu$
  & \hspace*{-4mm}
    $\CEeqS$
  & \multicolumn{4}{l}{
      \hspace*{-4mm}
      $\Cons \, (\phi, \nu) \, \SEcup \, \Nil$
    }\\
    $\delta$
  & \hspace*{-4mm}
    $\CEeqS$
  & \multicolumn{4}{l}{
      \hspace*{-4mm}
      $\Cons \, (\mathbb{N}, \delta) \, \SEcup \, \Nil$
    }
\end{tabular}
\right.$

\medskip

So, the values of type \texttt{T} are either of size $[6;9]$ and made
of integers in $]-\infty;15] \, \cup \, [20;+\infty[$, or of size
$[0;5] \, \cup \, [10;+\infty[$ and made of integers in
$\mathbb{N}$. As a corollary, this type contains at least a finite
value, hence is correct.




\section{Full Collection and Solving}
\label{full_collection_and_solving}
%%-*-latex-*-

This section completes the two previous sections which present the
constraint collection from types and subtypes and it describes the
solving procedure for the collected constraints. As a result, either
the constraints have no solutions (and the corresponding \ASN
specification must be rejected) or the value sets can be finitely
represented. It is straightforward to determine whether these value
sets are empty; if they are empty then the specification is rejected.

The constraint modeling an \ASN specification is 
$$\overline{\kappa} = \CEbigand{x \in {\cal X}}{\denotC{\I}{\TRef \,
(x)}{\TSEmap{}(x)}{\TSEmap{{\cal X} \backslash \{x\}}}}$$

\noindent
where ${\cal X} \subset {\cal R}$ is the finite set of the top-level
type names of the specification and $\TSEmap{} : {\cal X} \rightarrow
V$ maps these type names to fresh variables. The construction of a
constraint modeling an entire specification is convenient during the
collection (less parenthesis involved), but it is more comfortable for
the solving procedure to use a system of constraints.

\begin{Def}[Systems of constraints]\label{systems_of_constraints}
A \emph{system of constraints} is simply a set of constraints. The
system corresponding to the \ASN specification is $\Xi \,
(\overline{\kappa})$, where

\begin{center}
\begin{tabular}{rcl}
    $\Xi \, (\kappa_0 \CEand \kappa_1)$
  & = &
    $\Xi \, (\kappa_0) \, \cup \, \Xi \, (\kappa_1)$\\
    $\Xi (\kappa)$
  & = &
    $\{\kappa\}$ otherwise.
\end{tabular}
\end{center}

\end{Def}

We introduce the solving procedure as the semantics of a system of
constraints. In this aim, we first need to define a strict subset of
the expressions \E{} containing no variables: the \emph{terms}. They
can be characterised by the variables occurring in the expressions:

\begin{Def}[Variables in expressions]\label{variables_in_expressions} 
The countable set of variables $\alpha$, $\beta$, $\gamma$ etc. is
noted $V$. The variables of an expression $e$ are given by:

\begin{center}
\begin{tabular}{rcl}
$\VAR{\alpha}$
              & = & $\{\alpha\}$\\
$\VAR{\SEneg{e}}$
              & = & $\VAR{e}$\\
$\VAR{e_0 \, \SEcup \, e_1}$
              & = & $\VAR{e_0} \, \cup \, \VAR{e_1}$\\
$\VAR{e_0 \, \SEcap \, e_1}$
              & = & $\VAR{e_0} \, \cup \, \VAR{e_1}$\\
$\VAR{e_0 \, \SEdiff \, e_1}$
              & = & $\VAR{e_0} \, \cup \, \VAR{e_1}$\\
$\VAR{\Cons \, (e_0,e_1)}$
              & = & $\VAR{e_0} \, \cup \, \VAR{e_1}$\\
$\VAR{\Bind \, (l,e_0,e_1)}$
              & = & $\VAR{e_0} \, \cup \, \VAR{e_1}$\\
$\VAR{l \pmb{:} e}$
              & = & $\VAR{e}$\\
$\VAR{e \SEdiam \varsigma}$
              & = & $\VAR{e}$\\
$\VAR{e}$      
              & = & $\varnothing$ otherwise.
\end{tabular}
\end{center}
\end{Def}

Now we can define the ground expressions, or terms.

\begin{Def}[Terms]\label{terms} 
A \emph{term} $t$ is an element of the set $\Herbrand \triangleq \{e
\in \E \mid \VAR{e} = \varnothing\}$. The set of all subsets of
$\Herbrand$ is noted $\wp \,(\Herbrand)$.
\end{Def}

Note that the terms do not only correspond to the \ASN well-typed
values, for example the term $\Cons \, (\PosInt \, (0), \Cons \,
(\String \, \texttt{""}, \Nil))$ is not a typable term, i.e. there is
no \ASN type for  the \ASN value \verb+{+\texttt{\small 0,} \verb+""}+
denoted by this term.

Another necessary step is to characterise different kind of
expressions, because the constraints in which they appear are solved
by means of a specific algorithm: one for the integer intervals, one
for the real intervals, one for the regular expressions, one for the
powersets and one for the sets.

\begin{Def}[Characterization of integer intervals]
An expression $e \in \E$ denotes an integer interval if
$\chi_{\textnormal{I}} \, (e) = \ocamlkwd{true}$,
where

\medskip

\noindent
\begin{tabular}{l}
  $\ocamlkwd{let} \,\, \ocamlkwd{rec} \,\, \chi_{\textnormal{I}}
   = \ocamlkwd{function} \,\, \SEneg{e} \rightarrow
   \chi_{\textnormal{I}} (e)$\\
  \begin{tabular}{ll}
      $\mid$
    & \hspace*{-4mm}
      $e_0 \SEcup e_1 \mid e_0 \SEcap e_1 \mid e_0 
      \SEdiff e_1 \rightarrow \chi_{\textnormal{I}} (e_0)
      \,\, \textsf{\small \&\&} \,\, \chi_{\textnormal{I}} (e_1)$\\
      $\mid$
    & \hspace*{-4mm}
      $\alpha \,\, \ocamlkwd{when} \,\, \exists e \in \E.(\alpha \CEeq
       e) \in \Xi \, (\overline{\kappa}) \rightarrow
       \chi_{\textnormal{I}} (e)$\\
      $\mid$
    & \hspace*{-4mm}
      $\SEbot \mid \Interval \, \wildcard
       \rightarrow \ocamlkwd{true} \, \mid \,
       \wildcard \rightarrow \ocamlkwd{false}$
  \end{tabular}
\end{tabular}

\end{Def}

\begin{Def}[Characterization of real intervals]
An expression $e \in \E$ denotes a real interval if
$\chi_{\textnormal{F}} \, (e) = \ocamlkwd{true}$, where

\medskip

\noindent
\begin{tabular}{l}
  $\ocamlkwd{let} \,\, \ocamlkwd{rec} \,\, \chi_{\textnormal{F}}
   = \ocamlkwd{function} \,\, \SEneg{e} \rightarrow
   \chi_{\textnormal{F}} (e)$\\
  \begin{tabular}{ll}
      $\mid$
    & \hspace*{-4mm}
      $e_0 \SEcup e_1 \mid e_0 \SEcap e_1 \mid e_0 
      \SEdiff e_1 \rightarrow \chi_{\textnormal{F}} (e_0)
      \,\, \textsf{\small \&\&} \,\, \chi_{\textnormal{F}} (e_1)$\\
      $\mid$
    & \hspace*{-4mm}
      $\alpha \,\, \ocamlkwd{when} \,\, \exists e \in \E.(\alpha \CEeq
       e) \in \Xi \, (\overline{\kappa}) \rightarrow
       \chi_{\textnormal{F}} (e)$\\
      $\mid$
    & \hspace*{-4mm}
      $\SEbot \mid \wildcard \, \wildcard \,\, 
      \asnkwdconstr{..} \,\, \wildcard \, \wildcard 
      \rightarrow \ocamlkwd{true} \, \mid \,
      \wildcard \rightarrow \ocamlkwd{false}$
  \end{tabular}
\end{tabular}

\end{Def}

\begin{Def}[Characterization of regular expressions]
An expression $e \in \E$ denotes a regular expression if
$\chi_{\textnormal{R}} \, (e) = \ocamlkwd{true}$, where

\medskip

\noindent
\begin{tabular}{l}
  $\ocamlkwd{let} \,\, \ocamlkwd{rec} \,\, \chi_{\textnormal{R}}
   = \ocamlkwd{function} \,\, \SEneg{e} \rightarrow
   \chi_{\textnormal{R}} (e)$\\
  \begin{tabular}{ll}
      $\mid$
    & \hspace*{-4mm}
      $e_0 \SEcup e_1 \mid e_0 \SEcap e_1 \mid e_0 
      \SEdiff e_1 \rightarrow \chi_{\textnormal{R}} (e_0)
      \,\, \textsf{\small \&\&} \,\, \chi_{\textnormal{R}} (e_1)$\\
      $\mid$
    & \hspace*{-4mm}
      $\alpha \,\, \ocamlkwd{when} \,\, \exists e \in \E.(\alpha \CEeq
       e) \in \Xi \, (\overline{\kappa}) \rightarrow
       \chi_{\textnormal{R}} (e)$\\
      $\mid$
    & \hspace*{-4mm}
      $\Regexp \, \wildcard \rightarrow \ocamlkwd{true} \, 
      \mid \, \wildcard \rightarrow \ocamlkwd{false}$
  \end{tabular}
\end{tabular}

\end{Def}


\begin{Def}[Characterization of powerset expressions]
An expression $e \in \E$ denotes a powerset if $\chi_{\textnormal{P}}
\, (e) = \ocamlkwd{true}$, where

\medskip

\noindent
\begin{tabular}{l}
  $\ocamlkwd{let} \,\, \ocamlkwd{rec} \,\, \chi_{\textnormal{P}}
   = \ocamlkwd{function} \,\, \SEneg{e} \rightarrow
   \chi_{\textnormal{P}} (e)$\\
  \begin{tabular}{ll}
      $\mid$
    & \hspace*{-4mm}
      $e_0 \SEcup e_1 \mid e_0 \SEcap e_1 \mid e_0 
      \SEdiff e_1 \rightarrow \chi_{\textnormal{P}} (e_0)
      \,\, \textsf{\small \&\&} \,\, \chi_{\textnormal{P}} (e_1)$\\
      $\mid$
    & \hspace*{-4mm}
      $\alpha \,\, \ocamlkwd{when} \,\, \exists e \in \E.(\alpha \CEeq
       e) \in \Xi \, (\overline{\kappa}) \rightarrow
       \chi_{\textnormal{P}} (e)$\\
      $\mid$
    & \hspace*{-4mm}
      $e \SEdiam \varsigma \rightarrow \ocamlkwd{true} \,
       \mid \wildcard \rightarrow \ocamlkwd{false}$
  \end{tabular}
\end{tabular}

\end{Def}


\begin{Def}[Characterization of set expressions]
An expression $e \in \E$ denotes a set if $\chi_{\textnormal{S}} \,
(\varnothing) \, (e) = \ocamlkwd{true}$, where

\medskip

\noindent
\begin{tabular}{l}
  $\ocamlkwd{let} \,\, \ocamlkwd{rec} \,\, \chi_{\textnormal{S}} \,
   (\Path) = \ocamlkwd{function} \,\, \SEneg{e} \rightarrow
   \chi_{\textnormal{S}} \, (\Path) \, (e)$\\
  \begin{tabular}{ll}
      $\mid$
    & \hspace*{-4mm}
      $e_0 \SEcup e_1 \mid e_0 \SEcap e_1 \mid e_0 
      \SEdiff e_1 \rightarrow \chi_{\textnormal{S}} \, (\Path) \, (e_0)
      \,\, \textsf{\small \&\&} \,\, \chi_{\textnormal{S}} \, (\Path)
      \, (e_1)$\\
      $\mid$
    & \hspace*{-4mm}
      $\alpha \,\, \ocamlkwd{when} \,\, \alpha \in \Path \rightarrow
       \ocamlkwd{true}$\\
      $\mid$
    & \hspace*{-4mm}
      $\alpha \,\, \ocamlkwd{when} \,\, 
       \exists e \in \E.(\alpha \CEeq e) \in \Xi \,
       (\overline{\kappa}) \rightarrow 
       \chi_{\textnormal{S}} \, (\{\alpha\} \cup \Path) \, (e)$\\
      $\mid$
    & \hspace*{-4mm}
      $\Interval \wildcard 
       \mid \wildcard \, \wildcard \,\, \asnkwdconstr{..} \,\,
            \wildcard \, \wildcard  
       \mid \Regexp \wildcard 
       \mid e \SEdiam \varsigma \rightarrow \ocamlkwd{false}$\\
      $\mid$
    & \hspace*{-4mm}
      $\wildcard \rightarrow \ocamlkwd{true}$
  \end{tabular}
\end{tabular}

\end{Def}

\noindent
Now we can define the semantics of expressions.

\begin{Def}[Semantics of expressions]\label{semantics_of_SE}
A \emph{substitution} $\sigma$ is a function $\sigma : V \rightarrow
\wp \,(\Herbrand)$ from vari\-ables to sets of terms. The set of
substitutions is noted $\Sigma$. The standard semantics $\SEmu$ of
expressions is a mapping $\SEmu : \E \times \Sigma \rightarrow \wp \,
(\Herbrand)$ from the Cartesian product of expressions and
substitutions into sets of terms.
\end{Def}
\begin{itemize}

  \item 
    if $\chi_{\textnormal{I}} \, (e) = \ocamlkwd{true}$ then $\SEmu \,
    (e, \sigma) = \|e\|$, where\\ $\| \wildcard \,\| \colon \E
    \rightarrow \textsf{[$>$}
      \ocamltypename{clo\-sed\_int\_in\-ter\-val} \,\, \textsf{$\mid$}
      \,\, \SEbot \textsf{]}$ is a function defined as follows:

    \begin{center}
    \begin{tabular}{lll}
        $\| e \,\| =$ 
      & \hspace*{-4mm}
        $\ocamlkwd{let}$
      & \hspace*{-4mm}
        $\alpha \,\, \emph{be a fresh variable}$\\
      & \hspace*{-4mm}
        $\ocamlkwd{in}$
      & \hspace{-4mm}
        $\ocamlvaluename{solve\_integers} \, (\alpha \CEeqI e) \,
         (\alpha)$
    \end{tabular}
    \end{center}

  \item 
    if $\chi_{\textnormal{F}} \, (e) = \ocamlkwd{true}$ then $\SEmu \,
    (e, \sigma)$ is computed by a function similar to
    $\|\wildcard\,\|$, which applies to real intervals (not presented
    for the sake of brevity);
 
    \bigskip

  \item 
    if $\chi_{\textnormal{R}} \, (e) = \ocamlkwd{true}$ then

    \begin{center}
    \begin{tabular}{rcl}
        $\SEmu \, (e, \sigma)$
      & \hspace*{-4mm} = &\\
        \multicolumn{3}{l}{
          \begin{tabular}{ll}
              $\ocamlkwd{let}$
            & \hspace*{-4mm}
              $\alpha \,\, \emph{be a fresh variable}$\\
              $\ocamlkwd{in}$
            & \hspace*{-5mm}
              $\ocamlkwd{match} \,\, \ocamlvaluename{solve\_regexp} \,
               (\alpha \CEeqR e) \, (\alpha) \,\, \ocamlkwd{with}$\\
            & 
              $s \,\, \ocamlkwd{when} \,\, s \not= \textsf{""}
               \rightarrow \{\Regexp \, (s)\}$
          \end{tabular}
        }
    \end{tabular}
    \end{center}

  \item\label{semantics_of_PSE}
   if $\chi_{\textnormal{P}} \, (e) = \ocamlkwd{true}$ then $\SEmu$ is
   defined by the equations

   \begin{center}
   \begin{supertabular}{rcl} %XXX
       $\SEmu (\alpha, \sigma)$
     & = &
       $\sigma \, (\alpha)$\\
       $\SEmu (e \SEdiam \varsigma, \sigma)$ 
     & = & 
       $\{t \SEdiam \varsigma \mid t \in \SEmu (e, \sigma)\}$\\
       $\SEmu (\pi_0 \SEcap \pi_1, \sigma)$
     & = &
       $\ocamlkwd{let} \,\, \overline{\pi} = \SEmu (\pi_0, \sigma),
        \SEmu (\pi_1, \sigma)$\\
     & &
       $\ocamlkwd{in} \,\, \varphi \, (\overline{\pi}, \SEcap,
        \SEcap)$\\  
       $\SEmu (\pi_0 \SEdiff \pi_1, \sigma)$ 
     & = &
       $\ocamlkwd{let} \,\, \overline{\pi} = \SEmu (\pi_0, \sigma),
        \SEmu (\pi_1, \sigma)$\\ 
     & &
       $\ocamlkwd{in} \,\, \varphi \, (\overline{\pi}, \SEcap,
        \SEdiff) \, \disjunion \, \varphi \, (\overline{\pi},
        \SEdiff, \ocamlvaluename{fst})$\\
       $\SEmu (\pi_0 \SEcup \pi_1, \sigma)$
     & = &\\
       \multicolumn{3}{r}{
         \begin{tabular}{ll}
             $\ocamlkwd{let}$
           & \hspace*{-4mm}
             $(\overline{\pi}_0, \overline{\pi}_1) \AS \overline{\pi}
              = \SEmu (\pi_0, \sigma), \SEmu (\pi_1, \sigma)$\\
             $\ocamlkwd{in}$
           & \hspace*{-5mm}
             $\varphi \, (\overline{\pi}, \SEcap, \SEcup) \,
              \disjunion
              \, \varphi \, (\overline{\pi}, \SEdiff,
              \ocamlvaluename{fst}) \, \disjunion \, \varphi \,
              ((\overline{\pi}_1, \overline{\pi}_0), \SEdiff,
              \ocamlvaluename{fst})$ 
         \end{tabular}
       }\\
   \end{supertabular}
   \end{center}

The second equation relies on the semantics of set expressions
presented in definition~\ref{semantics_of_SE} ($e$ ranges over the set
expressions). The second equation gives the semantics of the
intersection of two powersets. $\overline{\pi}$ is the pair of the
semantics of $\pi_0$ and $\pi_1$. The function $\varphi$ is defined as
follows:

\vspace*{-4mm}

\begin{center}
\begin{tabular}{rcl}
    $\varphi \, ((\overline{\pi}_0, \overline{\pi}_1), f, g)$
  & \hspace*{-4mm} $\triangleq$
  & \hspace*{-4mm}
    $\{e \SEdiam i \mid e_0 \SEdiam i_0 \in \overline{\pi}_0, 
       e_1 \SEdiam i_1 \in \overline{\pi}_1,$\\
    \multicolumn{3}{r}{
      $e \in \|g (e_0, e_1)\| \backslash \{\SEbot\}, i \in \|f (i_0,
       i_1)\| \backslash \{\SEbot\}\}$ 
    }
\end{tabular}
\end{center}

\noindent
The first argument of $\varphi$ is the pair $(\overline{\pi}_0,
\overline{\pi}_1)$ of semantics. The second, $f$, is the set operator
to be applied to the intervals of the elements of $\overline{\pi}_0$
and $\overline{\pi}_1$. The last, $g$, is the set operator to be
applied to the set expressions of $\overline{\pi}_0$ and
$\overline{\pi}_1$. For instance, $\varphi \, (\overline{\pi}, \SEcap,
\SEcap) = \{e \SEdiam i \mid e_0 \SEdiam i_0 \in \overline{\pi}_0, e_1
\SEdiam i_1 \in \overline{\pi}_1, e \in \|e_0 \, \SEcap \, e_1\|
\backslash \{\SEbot\}, i \in \|i_0 \, \SEcap \, i_1\| \backslash
\{\SEbot\}\}$. The intuition is that the intersection of two powersets
is the powerset whose elements are the intersection of the initial
elements with the \emph{same} cardinal (hence we compute the
intersection of the intervals).

The fourth equation defines the semantics of the difference of two
powersets: $\varphi \, (\overline{\pi}, \SEcap, \SEdiff) \, \disjunion
\, \varphi \, (\overline{\pi}, \SEdiff, \ocamlvaluename{fst})$. The
symbol $\disjunion$ is the disjunctive set union $\cup$. The elements
of the powerset $\varphi \, (\overline{\pi}, \SEcap, \SEdiff)$ are the
difference between the initial elements of same cardinal. The elements
of the powerset $\varphi \, (\overline{\pi}, \SEdiff,
\ocamlvaluename{fst})$ are the elements of $\overline{\pi}_0$ whose
cardinals are different from the cardinals of the elements of
$\overline{\pi}_1$. The rationale of the equation comes from the
trivial formula $A = (A \, \cap \, B) \, \disjunion \, (A \,
\backslash \, B)$.

The last equation defines the semantics of the union of two powersets
as $\varphi \, (\overline{\pi}, \SEcap, \SEcup) \, \disjunion \,
\varphi \, (\overline{\pi}, \SEdiff, \ocamlvaluename{fst}) \,
\disjunion \, \varphi \, ((\overline{\pi}_1, \overline{\pi}_0),
\SEdiff, \ocamlvaluename{fst})$. First, the elements of the powerset
$\varphi \, (\overline{\pi}, \SEcap, \SEcup)$ are the union of the
initial elements of same cardinal. Second, the elements of the
powerset $\varphi \, (\overline{\pi}, \SEdiff, \ocamlvaluename{fst})$
are the elements of $\overline{\pi}_0$ whose cardinals are different
from the cardinals of the elements of $\overline{\pi}_1$. Last, the
elements of the powerset $\varphi \, ((\overline{\pi}_1,
\overline{\pi}_0), \SEdiff, \ocamlvaluename{fst})$ are the elements of
$\overline{\pi}_1$ whose cardinals are different from the cardinals of
the elements of $\overline{\pi}_0$. The rationale of this equation is
the formula: $A \, \cup \, B = (A \, \cap \, B) \, \disjunion
\, (A \, \backslash \, B) \, \disjunion \, (B \, \backslash \, A)$.

  \item 
    Otherwise, if $\chi_{\textnormal{S}} \, (\varnothing) \, (e) =
    \ocamlkwd{true}$ then $\SEmu$ is defined by the following
    equations:

\begin{center}
\begin{supertabular}{rcl} %XXX
$\SEmu(\SEbot,\sigma)$ & = & $\varnothing$ \\
$\SEmu(\SEtop,\sigma)$ & = & $\Herbrand$ \\
$\SEmu(\alpha,\sigma)$    & = & $\sigma \, (\alpha)$ \\
$\SEmu(e_0 \, \SEcup \, e_1, \sigma)$
              & = & $\SEmu(e_0,\sigma) \, \cup \, \SEmu(e_1,\sigma)$ \\
$\SEmu(e_0 \, \SEcap \, e_1,\sigma)$
              & = & $\SEmu(e_0,\sigma) \, \cap \, \SEmu(e_1,\sigma)$ \\
$\SEmu(\SEneg{e},\sigma)$ 
              & = & $\Herbrand \, \backslash \, \SEmu(e,\sigma)$ \\ 
$\SEmu(e_0 \SEdiff e_1, \sigma)$
              & = & $\SEmu(e_0 \, \SEcap \, \SEneg{e_1}, \sigma)$\\
$\SEmu(\ocamlvaluename{l}\/ \pmb{:} e,\sigma)$
              & = & $\{ \ocamlvaluename{l}\/ \pmb{:} t \mid t \in
		    \SEmu(e,\sigma) \}$ \\
$\SEmu(\Cons \, (e_0, e_1),\sigma)$
              & = &\\
\multicolumn{3}{r}{$\{\Cons \, (t_0, t_1) \mid 
  t_0 \in \SEmu(e_0,\sigma), t_1 \in \SEmu(e_1,\sigma)\}$}\\
$\SEmu(\Bind \, (\ocamlvaluename{y}, e_0, e_1),\sigma)$
              & = &\\
\multicolumn{3}{r}{$\{\Bind \, (\ocamlvaluename{y}, t_0, t_1) \mid
  t_0 \in \SEmu(e_0,\sigma), t_1 \in \SEmu(e_1,\sigma)\}$}\\
$\SEmu(\ocamlvaluename{e},\sigma)$
              & = & $\{ \ocamlvaluename{e} \}$ otherwise.\\
\end{supertabular}
\end{center}

\end{itemize}

Now we can define the semantics of constraints on top of the semantics
of expressions.

\begin{Def}[Semantics of constraints]
We define the semantics $\CEmu : {\cal K} \times \Sigma \rightarrow
\ocamltypename{bool}$ of constraints as a predicate on the Cartesian
product of constraints and substitutions:

\begin{center}
\begin{tabular}{rcl}
  $\CEmu(\kappa_0 \CEand \kappa_1,\sigma)$
    & = & $\CEmu(\kappa_0,\sigma) \BOOLand
          \CEmu(\kappa_1,\sigma)$ \\ 
  $\CEmu(e_0 \, \CEsubseteq \, e_1,\sigma)$
    & = & $\SEmu(e_0,\sigma) \subseteq \SEmu(e_1,\sigma)$\\
  $\CEmu(e_0 \, \CEeq \, e_1,\sigma)$
    & = & 
  $\CEmu((e_0 \, \CEsubseteq \, e_1)
         \CEand 
         (e_1 \, \CEsubseteq \, e_0),
         \sigma)$
\end{tabular}
\end{center}

\noindent
where $\!\!\BOOLand\!\!$ is the logical boolean operator and
$\subseteq$ is the inclusion over mathematical sets (of terms).
\end{Def}

\begin{Def}[Solutions]
The \emph{set of solutions} ${\cal S}(\kappa)$ of a constraint
$\kappa$ is the set of all substitutions that satisfy the semantics of
$\kappa$: ${\cal S}(\kappa) = \{ \sigma \in \Sigma \, \mid \,
\CEmu(\kappa,\sigma) \}$.
\end{Def}

The algorithm constructing the solutions of a system of set
constraints may be the algorithm which was published by Aiken and
Wimmers in 1992~\cite{AikenWimmers:1992}. We constructed our
constraint expressions in order to almost exactly fit the input of
this algorithm (see definition~\ref{expressions}). The only thing to
do is to replace $\E_0 \, \SEdiff \, \E_1$ by $\E_0 \, \SEcap \,
\SEneg\E_1$, since these two set expressions have the same semantics
(see definition~\ref{semantics_of_SE}).

\subsection{Worst-case complexity analysis}

Let us say some words about the complexity of the solving
procedure. The collecting algorithm we have presented is obviously not
optimised, in any way. For instance, a type which is constrained by a
combination of subtyping constraints will be analysed each time a
basic constraint is analysed. The reason for this is that we want to
provide an algorithm that can also be considered as a reference formal
model, thus it must be as readable as possible. Therefore, any
optimisation is postponed until the implementation phase (as using
hash-consing or caches to solve the mentioned problem). Anyway, the
reader can convince himself that the worst case complexity of the
collecting algorithm is proportional to the size of the subtypes plus
the complexity of the computations on integer intervals and regular
expressions.

The worst case complexity of the solving procedure is the same as the
complexity of Aiken and Wimmers' algorithm. In their paper, they prove
that their algorithm is in class NEXPTIME (they also published,
together with Kozen and Vardi, a general study about the complexity of
solving set constraints~\cite{AikenKozenVardiWimmers:1993}). This
result may be very disappointing, but it is inherent to the high
expressiveness of \ASN subtyping, in particular the use of a
complement operator (hence not monotonic) with indiscriminate unions
and intersections. As a future work, it would be interesting to
implement our algorithm and to make some benchmarks. Also, there
exists constraint subclasses with polynomial complexity in the
worst case and it would be worth studying whether it is possible, in
practice, to model \ASN with them.


\section*{Conclusion}
\addcontentsline{toc}{section}{Conclusion}
%%-*-latex-*-

\section{Conclusion\label{conclusion}}

In this paper we have presented a global test architecture for  
distributed services including the generation of test sequences for
service components. Our approach was validated by a case study: a
France Telecom \audio service.

The full service was described using the SDL language, and the running
of the Hit-or-Jump algorithm showed no deadlocks and produced the test
sequences for all the components of the studied service.

For sake of simplicity, we have selected a component of the \audio service
that is at the very heart of the service, the conference bridge, and
which coordinates the other components and illustrates clearly what
one imagines a \audio service is. The other components were generic
components that can be present in other kind of telecommunication
services, and for which we also generated the corresponding tests. We
have produced the tests for the bridge component in its context, and
we translated them in the TTCN and MSC formats.

We have also defined an architecture for the tester, which combines an
active part (based on a stimulation of the implementation) and a
passive one (based on the observation of the exchanges between the
CORBA objects). 

The results we got show that the use of formal methods considerably
eases the task of the service designers and developers, and that they
are usable for real services. Since the design phase to the
implementation and test phases, we used formal description techniques
(SDL, TTCN, MSC) and a formal test methodology. Moreover, we showed it
is possible to test the service components in the context of the
others (and not artificially in isolation). We think this is a notable
step towards the validation and the design of reusable
service-components.



\bibliographystyle{unsrt}
\bibliography{cj2003}

\end{document}
