%%-*-latex-*-

Till now, we have considered all the problems except solvability of
subtyping constraints (see section~\ref{validation}). Let us start by
quoting Olivier Dubuisson~\cite[\S{13.11}]{Dubuisson:2000}:

\begin{quotation}
`[...] the set operators potentially apply to any subtype constraints;
the problem lies in the interpretation of the constraints in terms of
sets of values to actually determine the possible values of the
resulting type. The designers should be aware, however, that the \ASN
compilers are not likely to check for [...] eccentric constraint
combinations.'
\end{quotation}

The aim of this section is to provide a formal definition of
the constraints, which will allow us (in
section~\ref{full_collection_and_solving}) to determine the values of
each subtype in a specification. The integration of this procedure into
the analysis phase of an \ASN compiler frees the protocol designer
from being `eccentric' or not. The idea is to collect constraints
from the specification, and then solve them. Most of our constraints
are or rely upon \emph{set constraints}. In the 1990s, the field of
the set constraints has been extensively explored, and both major
theoretical results (such as complexity for several classes of
constraints, or links to other fields such as the tree automata
theory) and practical achievements (such as constraint programming
languages or static programme analysis) have been brought to the
light.

On one hand, the \ASN values have a tree structure, thus fit perfectly
the usual domain of set constraints. On the other hand, the \ASN
subtyping constraints are too general to be modeled only with set
constraints: there are also constraints expressed in terms of
intervals, regular expressions and powersets. (Our specific
contribution is the use and resolution of these special constraints.)
So, in order to unify the representation of these concepts, we need to
abstract the values and make them simple constraints. We could
directly reuse the abstract grammar (type $\V$ in
section~\ref{abstract_grammar}), nevertheless it is worth abstracting
further the values at this stage; for instance, from a semantic
point of view, it is more suitable to consider that bit strings, octet
strings, and general strings are all described by regular expressions.
Also, it is more uniform to consider that an integer is in fact an
interval reduced to one element. In order to simplify the introduction
of the collection algorithm, we gather in the same definition the
abstracted values, intervals, regular expressions, sets and powersets
(they will be separated before the solving procedure, because they
require specific algorithms):

\begin{Def}[Expressions]\label{expressions} 
An \emph{expression} $e$ is an element of the set $\E$ defined using
the following grammar and OCaml type definitions:

\medskip

\noindent
\begin{tabular}{ll}
    $\E$ \ASSIGN 
  & \hspace*{-4mm}
    $\SEtop$
    $\mid$ $\SEbot$
    $\mid$ $\alpha$
    $\mid$ $\SEneg\E$
    $\mid$ $\E_0$ $\SEcup$ $\E_1$ 
    $\mid$ $\E_0$ $\SEcap$ $\E_1$
    $\mid$ $\E_0$ $\SEdiff$ $\E_1$\\
  & \hspace*{-4mm} 
    $\mid$ \ocamltypename{series}
    $\mid$ \ocamltypename{map}
    $\mid$ \wildcard $\pmb{:}$ \wildcardof
           \ocamltypename{label} $\times$ $\E$
    $\mid$ \kwdNULL{}\\
  & \hspace*{-4mm} 
    $\mid$ \ocamltypename{real\_interval}
    $\mid$ \ocamltypename{closed\_int\_interval}
    $\mid$ \kwdTRUE{}\\
%\end{tabular}
%\hspace*{7.2mm}
%\begin{tabular}{ll}
  & \hspace*{-4mm}
    $\mid$ \kwdFALSE{}
    $\mid$ \Enum{} \ocamlkwd{of} \ocamltypename{item}
    $\mid$ \Regexp{} \ocamlkwd{of} \ocamltypename{string}\\
  & \hspace*{-4mm}
%%    $\mid$ \ocamltypename{integer}
%%    $\mid$ \ocamltypename{real}
    $\mid$ $\E \SEdiam \ocamltypename{closed\_int\_interval}$
\end{tabular}

\smallskip

\noindent
\ocamlkwd{and} \ocamltypename{series} = 
\ocamlconstr{[}\Cons{} \ocamlkwd{of} $\E$ $\times$
\ocamltypename{series} 
$\mid$ \Nil \ocamlconstr{]}

\noindent
\ocamlkwd{and} \ocamltypename{map} =
\ocamlconstr{[}\Bind{} \ocamlkwd{of} \ocamltypename{identifier}
$\times$ $\E$ $\times$ \ocamltypename{map} $\mid$ \Nil
\ocamlconstr{]} 

\smallskip

\noindent
\ocamlkwd{and} \ocamltypename{real\_interval} =
\ocamlconstr{[}\wildcard \wildcard \asnkwdconstr{..} \wildcard
\wildcardof\\
\hspace*{2mm}
\ocamltypename{real\_bound} $\times$ \ocamltypename{in\_out} 
$\times$ \ocamltypename{in\_out} $\times$
\ocamltypename{real\_bound}\ocamlconstr{]}

\noindent
\ocamlkwd{and} \ocamltypename{real\_bound} =
\ocamlconstr{[}\ocamltypename{real} 
$\mid$ \MinInfReal{} 
$\mid$ \PlusInfReal \ocamlconstr{]}

\noindent
\ocamlkwd{and} \ocamltypename{closed\_int\_interval} =\\
\hspace*{2mm}
\ocamlconstr{[}\Interval{} \ocamlkwd{of} \ocamltypename{int\_bound}
$\times$ \ocamltypename{int\_bound}\ocamlconstr{]}

\noindent
\ocamlkwd{and} \ocamltypename{int\_bound} =
\ocamlconstr{[}\ocamltypename{integer}{} 
$\mid$ \MinInfInt{}
$\mid$ \PlusInfInt \ocamlconstr{]}

\smallskip

\noindent
As a special case, we define some useful constants:

\noindent
\begin{tabular}{lll}
    $\ocamlkwd{let}$ 
  & \hspace*{-4mm}
    $\mathbb{N}$ 
  & \hspace*{-5mm}
    $= \Interval \, (\MinInfInt, \PlusInfInt)$\\
    $\ocamlkwd{and}$
  & \hspace*{-4mm}
    $\mathbb{N}^{+}$
  & \hspace*{-5mm}
    $= \Interval \, (\PosInt \, (0), \PlusInfInt)$\\
    $\ocamlkwd{and}$
  & \hspace*{-4mm}
    $\mathbb{R}$
  & \hspace*{-5mm}
    $= \MinInfReal \, \pmb{<} \asnkwdconstr{..} \pmb{<} \,
     \PlusInfReal$ 
\end{tabular}

The constant $\SEtop$ denotes the set of all terms (see
definition~\ref{terms}); $\SEbot$ the empty set; $\alpha$ is a
variable denoting an expression (see
definition~\ref{variables_in_expressions}); $\SEneg$ is the complement
operator; $\SEcup$, $\SEcap$ and $\SEdiff$ are straightforward. These
operators are the same as Aiken and Wimmers' ones (except their
notation is not dotted). Next is the \ocamltypename{series} type,
which captures a list of expressions; it will be used for encoding
sets of \ASN values of type \kwdSETOF{} and \kwdSEQUENCEOF. Next comes
the \ocamltypename{map} type, which represents mappings from
identifiers to expressions; it will be used to denote sets of \ASN
values of type \kwdSET{} and \kwdSEQUENCE. Next, there is the
$\wildcard\pmb{:}\!\wildcard$ constructor that associates a
\ocamltypename{label} with an expression; it will be used for encoding
values of the \ASN{} type \kwdCHOICE. The \kwdNULL{} constructor is
for the \kwdNULL{} value. Next, we find the
\ocamltypename{real\_interval} type that defines an interval on the
\kwdREAL{} values. Next, we get the
\ocamltypename{closed\_int\_interval} type which represents the closed
intervals on the \kwdINTEGER{} values (in theory, it is not necessary
to have closed intervals, but this design decision makes the algorithm
simpler). The constructors \kwdTRUE{} and \kwdFALSE{} are
obvious. Following, we have the \Enum{} constructor which denotes the
enumerated constants; then \Regexp{} which denotes the \ASN regular
expressions (we call \emph{regexp} for short); then the
\ocamltypename{integer} type; then the \ocamltypename{real} type. The
constructor $\SEdiam$ denotes a powerset (see below). The constructors
\MinInfReal, \PlusInfReal{} stand respectively for $-\infty$ and
$+\infty$ on the \ASN real numbers; \MinInfInt, \PlusInfInt{} are
$-\infty$ and $+\infty$ on the integers.
\end{Def}

The mapping from abstracted values to expressions is quite
straightforward. Just notice that integer and real numbers are mapped
into intervals, and that the value reference names (of type
\ocamltypename{string}) are cast into the set $V$ of variables:

\begin{Def}[From values to expressions]\label{mu}
\noindent
$\ocamlkwd{let} \,\, \ocamlkwd{rec} \,\, \mu : \V \rightarrow \E
= \ocamlkwd{function}$\\
\noindent
\begin{tabular}{rl}
           & \hspace*{-4mm}
             $\List \, [\,] \mid \Map \, \{\} \rightarrow \Nil$\\
         $\mid$ 
           & \hspace*{-4mm}
             $\List \, (\ocamlvaluename{v} \CONS \textnormal{L})
              \rightarrow
              \Cons \, (\mu \, (\ocamlvaluename{v}), \mu \, (\List
              \, (\textnormal{L})))$\\
         $\mid$
           & \hspace*{-4mm}
             $\Map \, (\IdEnv{\{l\} \disjunion \Labels}) \rightarrow
              \Bind \, (l, \mu \, (\IdEnv{}(l)),
              \mu \, (\Map \, (\IdEnv{\Labels})))$\\
         $\mid$
           & \hspace*{-4mm}
             $\BinStr \, (s) \mid \HexStr \, (s) \mid \pvString \, (s)
              \rightarrow \Regexp \, (s)$\\
         $\mid$ 
           & \hspace*{-4mm} 
             $(\PosInt \, \wildcard \mid \NegInt \, \wildcard) \AS
             \ocamlvaluename{v} \rightarrow 
             \Interval \, (\ocamlvaluename{v}, \ocamlvaluename{v})$\\
         $\mid$
           & \hspace*{-4mm}
             $(\PosReal \, \wildcard \mid \NegReal \, \wildcard) \AS
             \ocamlvaluename{v} \rightarrow \ocamlvaluename{v}
             \, \pmb{\leqslant} \asnkwdconstr{..} \pmb{\leqslant} \,
             \ocamlvaluename{v}$\\
         $\mid$
           & \hspace*{-4mm}
             $\kwdMINUSINFINITY \rightarrow \MinInfReal \,
             \pmb{\leqslant} \asnkwdconstr{..} \pmb{\leqslant}
             \MinInfReal$\\
         $\mid$
           & \hspace*{-4mm}
             $\kwdPLUSINFINITY \rightarrow \PlusInfReal \,
             \pmb{\leqslant} \asnkwdconstr{..} \pmb{\leqslant}
             \PlusInfReal$\\
         $\mid$ 
           & \hspace*{-4mm}
             $\ocamlvaluename{l}\/ \pmb{:} \ocamlvaluename{v}
             \rightarrow \ocamlvaluename{l}\/ \pmb{:} \mu 
             (\ocamlvaluename{v})
             \mid
             \VRef \, (y) \rightarrow (y : V)$\\
         $\mid$
           & \hspace*{-4mm}
             $\ocamlvaluename{v} \rightarrow \ocamlvaluename{v}$
\end{tabular}
\end{Def}

The next step is to build the constraints on top of the expressions:

\begin{Def}[Constraints]\label{constraints_def}
A \emph{constraint} $\kappa$ is a conjunction of inclusions over
expressions. The set of constraints is:
${\cal K}$ \ASSIGN ${\cal K}_0 \CEand {\cal K}_1$
            $\mid$ $\E_0 \CEsubseteq \E_1$
            $\mid$ $\E_0 \CEeq \E_1$.
\end{Def}

Note that the notations for the operators defining the constraints are
double-dotted, e.g. $\CEand\!$. The operator $\CEeq$ stands for the
double inclusion.

%% In the next sections, it will be used with several annotations:
%% equality between two integer expressions ($e_0 \CEeqI e_1$),
%% equality between two real expressions ($e_0 \CEeqF e_1$), equality
%% between two regexp expressions ($e_0 \CEeqR e_1$), equality between
%% two basic expressions ($e_0 \CEeqS e_1$), equality over whatever
%% expressions ($e_0 \CEeq e_1$) and equality between two powerset
%% expressions ($\pi_0 \PSCEeq \pi_1$).

Let us consider the sets of value sets associated with \kwdSETOF{} and
\kwdSEQUENCEOF{} types, i.e. powersets of values. For instance,
\texttt{\small A ::= SET (SIZE (4..7)) OF INTEGER} denotes the set of
integer sets whose cardinals belong to the interval $[4;7]$. For each
value of the cardinal it is possible to give the corresponding integer
set expression, and then make the union to get the expression
associated with type \texttt{\small A}. In general, this encoding is
bulky, it makes the constraint solving inefficient and it is unable
to cope with the infinite bound $\mathbb{N}^{+}\!$. A better idea is
to keep the interval of cardinals \emph{together with} an
over-approximation of the powerset itself (this latter contains sets
of any size). The powerset expression $e \SEdiam \varsigma$ is a pair
of an expression $e$ which denotes a powerset (coming from \kwdSETOF{}
and \kwdSEQUENCEOF{} types), and an interval $\varsigma$ of the
elements' cardinals (coming from \kwdSIZE{} subtyping constraints).

%% \begin{Def}[Powerset Expressions]\label{powerset_expressions} A
%% \emph{powerset expression} $\pi$ is an element of the set $\Pi$
%% defined by: $\Pi$ \ASSIGN $\alpha$ $\mid$ $\E \SEdiam
%% \ocamltypename{closed\_int\_interval}$ $\mid$ $\Pi_0 \PSEsum \Pi_1$
%% $\mid$ $\Pi_0 \PSEprod \Pi_1$ $\mid$ $\Pi_0 \PSEdiff \Pi_1$.  It
%% can be a variable $\alpha$ denoting another powerset expression.

%% The expression $\pi_0 \PSEsum \pi_1$ is the union of the powerset
%% expressions $\pi_0$ and $\pi_1$, the expression $\pi_0 \PSEprod
%% \pi_1$ is the intersection of $\pi_0$ and $\pi_1$, and $\pi_0
%% \PSEdiff \pi_1$ is the difference between $\pi_0$ and $\pi_1$. If
%% $\varsigma_1$ and $\varsigma_2$ are disjoint, we can write $\alpha
%% \SEdiam \varsigma_1 \stackrel{\bot}{\PSEsum} \beta \SEdiam
%% \varsigma_2$ instead of $\alpha \SEdiam \varsigma_1 \PSEsum \alpha
%% \SEdiam \varsigma_2$.  \end{Def}

%% \begin{Def}[Powerset Constraints]\label{powerset_constraints} A
%% \emph{powerset constraint} is defined similarly to the (basic)
%% constraints, but using powerset expressions instead of (basic)
%% expressions: ${\cal P}$ \ASSIGN ${\cal P}_0 \! \CEand \! {\cal
%% P}_1$ $\mid$ $\Pi_0 \CEsubseteq \Pi_1$ $\mid$ $\Pi_0 \CEeq
%% \Pi_1$.  \end{Def}

%% We shall use the annotation P (for \emph{powerset}) in $\pi_0
%% \PSCEeq \pi_1$. For the sake of commodity, we shall mix powerset
%% constraints (${\cal P}$) and basic constraints (${\cal K}$) in the
%% same constraint, which will be split after the collection into
%% different systems (i.e. sets of constraints), because the solving
%% procedure is different.

Our idea is to analyse the subtypes in \core and to produce a mixed
constraint for each one. Since component types are all type references
(see section~\ref{mapping}, step~\ref{component_types}) and type
declarations are of the form {\small $<$\textsf{type reference}$>$
\textsf{::=} $<$\textsf{non-reference type without inner
constraints}$>$ \textsf{(}$<$\textsf{subtyping
constraint}$>$\textsf{)}} or simply {\small $<$\textsf{type
reference}$>$ \textsf{::=} $<$\textsf{non-reference type without inner
constraints}$>$} (see step~\ref{type_references}), we can parse each
subtyping constraint without destructuring the type it applies to
(indeed, the constraints on component types are only found at the
top-level). Thanks to this specific shape of \core, the collection
process has two weakly interdependent aspects: the collection on types
and the collection on (proper) subtypes. The link between the two
collections is due to the component types, i.e. type references,
because a type reference can denote either a type or a proper subtype;
hence this link will appear in the treatment of the \TRef{}
constructor.



