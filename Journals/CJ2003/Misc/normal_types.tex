%%-*-latex-*-

We give the list of rewritings to be performed on the original \ASN
specification in order to get its normal form. 

First, the following rewritings are aimed at placing at top-level some
types that are not subtrees of type terms, and hence would be ignored
by the checks on the types only.

\begin{enumerate}

  \item uncoupling of default values:

        \begin{tabular}{rcl} 
           \verb+T ::= SET {a REAL DEFAULT 0}+\\
             $\longrightarrow$
           & $\left\{
                \begin{tabular}{l}
                  \verb+T ::= SET {a A DEFAULT v}+\\
                  \verb+A ::= REAL+\\
                  \verb+v A ::= 0+
                \end{tabular}
              \right.$\\
        \end{tabular} \\
        where \texttt{A} is a fresh type reference and \texttt{v} is
        a fresh value reference;

  \item uncoupling of enumerated constants, integer constants and bit
        string constants:

        \begin{tabular}{rcl}
           \verb+T ::= ENUMERATED {a(0)}+\\
             $\longrightarrow$
           & $\left\{
                \begin{tabular}{l}
                  \verb+T ::= ENUMERATED {a(v)}+\\
                  \verb+v INTEGER ::= 0+
                \end{tabular}
              \right.$\\
            \verb+T ::= INTEGER {off(0)}+\\
              $\longrightarrow$
           & $\left\{
                \begin{tabular}{l}
                   \verb+T ::= INTEGER {off(v)}+\\
                   \verb+v INTEGER ::= 0+
                \end{tabular}
              \right.$\\
            \verb+T ::= BIT STRING {lsb(0)}+\\
              $\longrightarrow$
           & $\left\{
                \begin{tabular}{l} 
                   \verb+T ::= BIT STRING {lsb(v)}+\\
                   \verb+v INTEGER ::= 0+\\
                \end{tabular}
              \right.$
        \end{tabular} \\
        where \texttt{v} is a fresh value reference;

  \item for each value declaration:

        \begin{tabular}{rcl}
           \verb+y INTEGER (0..9) ::= 1+
           & $\longrightarrow$ 
           & $\left\{
                \begin{tabular}{l} 
                   \verb+y A ::= 1+ \\
                   \verb+A ::= INTEGER (0..9)+ \\
                   \verb+B ::= A (y)+
                \end{tabular}
              \right.$
        \end{tabular} \\
        where \verb+A+ and \verb+B+ are fresh type references;

  \item for each \kwdINCLUDES constraint, replace the included type by
        a fresh reference and add a corresponding new type
        declaration:

        \begin{tabular}{rcl}
           \verb+FrenchWE ::= Day (WeekEnd)+
           & $\longrightarrow$
           & $\left\{
                \begin{tabular}{l}
                   \verb+FrenchWE ::= Day (A)+ \\
                   \verb+A ::= WeekEnd+
                \end{tabular}
              \right.$
        \end{tabular} \\
        where \verb+A+ is a fresh type reference;

  \item unfolding of selection types and top-level reference types:

        \begin{tabular}{rcl} 
          $\left\{
             \begin{tabular}{l}
                \verb+A ::= i < B+\\
                \verb+B ::= C+\\
                \verb+C ::= CHOICE {i D}+\\
                \verb+D ::= INTEGER+
             \end{tabular}
            \right.$
          & $\longrightarrow$
          & $\left\{
               \begin{tabular}{l}
                  \verb+A ::= INTEGER+\\
                  \verb+B ::= CHOICE {i D}+\\
                  \verb+C ::= CHOICE {i D}+\\
                  \verb+D ::= INTEGER+
               \end{tabular}
             \right.$
        \end{tabular} \\
        beware:

        \begin{tabular}{rcl}
          $\left\{
             \begin{tabular}{l}
               \verb+A ::= SET OF S+\\
               \verb+B ::= A (SIZE (7))+
             \end{tabular}
           \right.$
          & $\longrightarrow$
          & $\left\{
             \begin{tabular}{l}
               \verb+A ::= SET OF S+\\ 
               \verb+B ::= SET (SIZE (7)) OF S+
             \end{tabular}
             \right.$
        \end{tabular}

  \item for each \kwdCOMPONENTSOF clause, replace the given type by
        its components;

  \item each value of the type associated to \kwdREAL are rewritten
        using the decimal (dotted) notation:
 
        \verb+v REAL ::= {mantissa 1, base 10, exponent -3}+\\
        \begin{tabular}{rcl}
           & $\longrightarrow$
           & \verb+v REAL ::= 1E-3+
        \end{tabular}

  \item each integer token of type \kwdREAL are rewritten with the
        decimal notation: \verb+v REAL ::= 5+ $\, \longrightarrow \,$
        \verb+v REAL ::= 5.0+

  \item resolution of the integer constants and the bit string
        constants:

        \begin{tabular}{lcl}
           $\left\{
              \begin{tabular}{l}
                 \verb+T ::= INTEGER {c(x)}+\\
                 \verb+v T ::= c+
              \end{tabular}
            \right.$
         & $\longrightarrow$
         & $\left\{
              \begin{tabular}{l}
                \verb+T ::= INTEGER+\\
                \verb+v T ::= x+
              \end{tabular}
            \right.$ \\
           $\left\{
              \begin{tabular}{l}
                 \verb+T ::= BIT STRING {msb(7)}+\\
                 \verb+v T ::= {msb}+
              \end{tabular}
            \right.$
         & $\longrightarrow$
         & $\left\{
              \begin{tabular}{l}
                 \verb+T ::= BIT STRING+\\
                 \verb+v T = '10000000'B+
              \end{tabular} 
            \right.$
        \end{tabular} \\
        beware:

        \begin{tabular}{rcl}
           $\left\{
              \begin{tabular}{l}
                 \verb+T ::= INTEGER {c(v)}+\\
                 \verb+v T ::= c+
              \end{tabular}
            \right.$
         & $\longrightarrow$
         & $\left\{
              \begin{tabular}{l}
                \verb+T ::= INTEGER+\\
                \verb+v T ::= v+
              \end{tabular}
            \right.$
        \end{tabular}

\end{enumerate}




We omit \kwdOBJECTIDENTIFIER and \kwdRELATIVEOID types, for sake of simplicity.

\noindent 
\ocamlkwd{type} $\overline{\cal T}$ = 
\kwdCHOICE \ocamlkwd{of} \ocamltypename{label}
           $\rightarrow$ $\overline{\cal T}$
$\mid$ \kwdSETOF \ocamlkwd{of} $\overline{\cal T}$
  $\mid$   \kwdSEQUENCEOF \ocamlkwd{of} $\overline{\cal T}$ \\
\begin{tabular}{rl}
    $\mid$ & \kwdSET \ocamlkwd{of} \ocamltypename{components} 
  $\mid$   \kwdSEQUENCE \ocamlkwd{of}
           \ocamltypename{components} \\
  $\mid$ & \TRef \ocamlkwd{of} \ocamltypename{string}
  $\mid$   \kwdBITSTRING
  $\mid$   \kwdENUMERATED \ocamlkwd{of}
           \ocamltypename{name} \ocamlconstr{list} \\
  $\mid$ & \kwdINTEGER 
  $\mid$   \kwdREAL
  $\mid$   \kwdBOOLEAN
  $\mid$   \kwdNULL
  $\mid$   \kwdOCTETSTRING
  $\mid$   \String 
%%  $\mid$ & \kwdOBJECTIDENTIFIER
%%  $\mid$   \kwdRELATIVEOID
  
\end{tabular}

\medskip

\noindent \ocamlkwd{and} \ocamltypename{components} = 
\ocamltypename{label} $\rightarrow$ $\overline{\cal T}$ $\times$  [\kwdOPTIONAL
$\mid$ \kwdDEFAULT \ocamlkwd{of} $\V$] \ocamlconstr{option}




