%%-*-latex-*-

It is a common belief among the practitioners in the networking area
that theoretical study cannot help the industrial audience. We
nevertheless think that many tools the engineers use can concretely
benefit from such an initiative, especially those based on protocol
languages and compilers. The work on \ASN we have presented in this
article follows this track, and can be considered in several
complementary ways.

From a theoretical point of view, we bring to the fore the first
complete and formal semantics of \mbox{X.680}, based on the set
constraints framework. Each syntactic construct is mapped to a
mathematical object in a consistent manner, thus bringing new insights
to obscure areas of the standard, like type compatibility in
assignments, type emptiness, type recursion in the presence of the
complement subtyping constraint, etc. The set constraints prove to be
a suitable model that allows one to encode all \mbox{X.680}. One of
the reasons is that the paradigm of \ASN types comes from the
telecommunications industry, where what matters are the values
exchanged between applications or network equipments --- hence types
are naturally considered as \emph{sets} of values. The tree-like
structure of the \ASN values also contributed to this successful
fitting. Nevertheless, some special values not found in the set
constraint litterature, such as regular expressions, numeric intervals
and powersets, motivated a specific contribution. Also, the usage of a
functional meta-language to express our algorithm brought elegance and
compactness. This work can be considered as an improvement of our
PhD~\cite{myPhD}, which introduced an \emph{ad-hoc} `operational
semantics' of X.680, based on the 1988/1990 standard --- as opposed to
the present `denotational semantics' based on the 1997/2001
standard. Indeed, this early framework did not take advantage of the
expressiveness of the set constraints, and their natural fitting to
the `types as sets' paradigm of \ASN. Also, the operational aspect
introduced too many aspects that are split here into collection and
solving. From a general standpoint, a formal semantics, if properly
documented, is a good complement to that provided by the
standards. They indeed provide definitions in English by following the
\ASN grammar in Backus-Naur Form. Unfortunately, this is ambiguous in
the sense that a specification can be derived using different
syntactical rules and, therefore, such a syntax-driven approach of
semantics may be misleading.

From a practical point of view, the existence of a compiler for a
plainer version of this meta-language, namely the OCaml language,
helps in bridging the gap between the semantics and its implementation
in an \ASN compiler. We identified the main issues in the validation
of \mbox{X.680} specifications, and stressed the importance of the `at
least one finite value' property for types in the Protocol Data
Units. We showed how our semantics \emph{is} naturally an algorithm.

Despite the present paper focusing on the telecommunication usages of
\ASN, we do not forget the database researchers. They are, in general,
more concerned about formal foundations, hence we hope this work will
convince them of the suitability of \ASN for their purposes. For
instance, ongoing joint work of the ISO and the ITU-T explores the
links between XML and \ASN, in particular the use of \ASN as a schema
notation for XML, and a mapping from XML schemas to \ASN modules
(\url{http://asn1.elibel.tm.fr/xml/}).

Immediate further work ranges from optimization of the collecting
algorithm, to extension to full \ASN (\mbox{X.681}, \mbox{X.682} and
\mbox{X.683}). A mid-term project would be to implement our algorithm
in OCaml. This will lead to the completion of our parser and
typechecker, \textsf{Asno} (available at
{\small\url{http://cristal.inria.fr/~rinderkn/}}), which lacks subtype
analysis. In the long term, a more challenging endeavour consists in
taking the benefit of our new semantics in order to formally model the
Packed Encoding Rules~\cite{X.691:2002}.

