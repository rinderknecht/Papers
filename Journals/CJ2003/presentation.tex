 %%-*-latex-*-

This section provides a very short overview of \ASN. For a more
detailed introduction, please refer to Dubuisson's
book~\cite{Dubuisson:2000}. The subtyping features will be presented
in section~\ref{constraints_from_subtypes} together with the
constraint collection from subtypes. \ASN provides basic types as
follows.

\begin{itemize}
 
  \item The \texttt{\small BOOLEAN} type has two predefined values
        \texttt{\small TRUE} and \texttt{\small FALSE},
        \emph{e.g.,} \texttt{ok} \texttt{\small BOOLEAN ::= TRUE} defines a
        value \texttt{\small TRUE} whose name is \texttt{ok} 
        and whose type is \texttt{\small BOOLEAN}.

  \item The \texttt{\small NULL} type only has one value, also noted
        \texttt{\small NULL}. This type is often used as a placeholder
        in many real complete specifications to indicate that no
        additional information is needed, or it is used to test
        incomplete specifications.

  \item The \texttt{\small INTEGER} type matches the mathematical set
    $\mathbb{Z}$, \emph{e.g.,} \texttt{zero} \texttt{\small INTEGER
    ::= 0}. The syntax also allows some constants to be distinguished:
    \texttt{\small DayIn\-The\-Year ::= INTEGER \{first (1), last
      (365)\}} defines the type \texttt{Day\-In\-The\-Year} as being
    \texttt{\small INTEGER}, and distinguishes two integers named
    \texttt{first} and \texttt{last}, whose respective values are
    \texttt{\small 1} and \texttt{\small 365}. Then
    \texttt{newYearsEve} \texttt{DayInTheYear} \texttt{\small ::=}
    \texttt{last} defines a value \texttt{newYearsEve}. The definition
    is valid because \texttt{last} is in the scope of
    \texttt{Day\-In\-The\-Year}; the name \texttt{newYearsEve} is
    bound to the value \texttt{\small 365}.

  \item The \texttt{\small ENUMERATED} type defines a collection
    of (constant) names, like \texttt{\small
      Synchro\-Indicator ::= ENUMERATED \{serial, parallel\}} allows
    the following value definition: \texttt{synchro
      Synchro\-Indi\-ca\-tor} \texttt{\small ::=} \texttt{serial}. It
    is possible, though not recommended, to specify the
    \emph{encoding} of an enumerated value, like
    \texttt{PositiveLogics} \texttt{\small ::= ENUMERATED}
    \verb+{+\texttt{false} \texttt{\small (0),} \texttt{true}
    \texttt{\small(1)}\verb+}+, but this has no impact on the values
    themselves.

  \item The \texttt{\small REAL} type corresponds to the mathematical
        decimal numbers, defined either with a dotted notation,
        \emph{e.g.,} \texttt{\small 5.7}, or a sequence,
        \emph{e.g.,} \verb+{+\texttt{mantissa} \texttt{\small 1,}
        \texttt{base} \texttt{10}\texttt{\small,} \texttt{exponent}
        \texttt{\small -3}\verb+}+.

  \item The \texttt{\small BIT STRING} type corresponds to strings of
    bits, \emph{e.g.,} \texttt{\small '1101'B} (binary) or
    \texttt{\small '0D'H} (hexadecimal). The syntax also allows some
    bits to be distinguished. Given \texttt{\small T ::= BIT STRING}
    \verb+{+\texttt{msb} \texttt{\small (7),} \texttt{lsb}
    \texttt{\small (0)}\verb+}+, the definition \texttt{v}
    \texttt{\small T ::=} \verb+{+\texttt{msb}\texttt{\small ,}
    \texttt{lsb}\verb+}+ stands for \texttt{v} \texttt{\small T ::=
      '10000001'B}.  It is also possible to restrict the size of the
    string using a subtyping constraint: \texttt{StringOf32Bits ::=
      BIT STRING (SIZE (32))}.

  \item The \texttt{\small OCTET STRING} type is similar to the
        \texttt{\small BIT STRING}, except that the encoded strings
        must contain a number of bits that is a multiple of eight (and
        no bit can be distinguished by a name).

  \item The \texttt{\small OBJECT IDENTIFIER} and \texttt{\small
        RELATIVE-OID} types are used to reference other \ASN
        modules at an international level, by means of a path in a
        standard tree. They can also identify a physical object, such
        as a printer on a network, or a postal package, or an ASN.1
        type which is carried in some larger message. They are not
        considered here.

  \item For historical reasons there are plenty of string types in
        \ASN, like \texttt{Nu\-mer\-ic\-String}, \texttt{IA5String},
        \texttt{UTF8String}, \texttt{GeneralString} etc. They mainly
        differ in the alphabet they are built upon\footnote{There are
        other factors besides just the alphabets. Some string types
        such as \texttt{Ge\-ne\-ral\-String} allow escape characters
        to kick into alternate character sets (such as those for
        different languages) while others such as \texttt{UTF8String}
        can represent characters of all languages directly.}. Here, we
        will not make any difference between these strings,
        and assume there is only one kind, called \texttt{String}.

\end{itemize}

\noindent
These basic types can be used to construct other types:

\begin{itemize}

  \item The \texttt{\small SET} type corresponds to the record-like
        structures in programming languages, \emph{e.g.,} \texttt{PersonInfo}
        \texttt{\small ::= SET} \verb+{+\texttt{age} \texttt{\small
        INTEGER,} \texttt{married} \texttt{\small BOOLEAN}\verb+}+ of
        which one value may be: \texttt{i PersonInfo} \texttt{\small
        ::=} \verb+{+\texttt{married} \texttt{\small TRUE,} \texttt{age}
        \texttt{\small 32}\verb+}+. Some components may be marked as
        optional or having a default value, \emph{e.g.,} \texttt{\small Point
        ::= SET} \verb+{+\texttt{x} \texttt{\small REAL DEFAULT 0.0,}
        \texttt{y} \texttt{\small REAL DEFAULT 0.0}\verb+}+ allows
        defining the value \texttt{origin} \texttt{\small Point ::=}
        \verb+{}+, which is the same as \texttt{origin} \texttt{\small
        Point ::=} \verb+{+\texttt{x} \texttt{\small 0.0,} \texttt{y}
        \texttt{\small 0.0}\verb+}+. Here is an example from a real
        protocol:\\
{\small
\verb+DataAcknowledgementTPDU ::= SET {+\\
\verb+  destRef        Reference,+\\
\verb+  yr-tu-nr       TPDUnumber,+\\
\verb+  checkSum       CheckSum OPTIONAL,+\\
\verb+  subSeqNr       SubSequenceNumber DEFAULT 0,+\\
\verb+  flowControlCnf FlowCntlConf OPTIONAL}+}

  \item The \texttt{\small SEQUENCE} type is the same as the
        \texttt{\small SET} type, except that the component values
        must be given in the same order as they are declared, \emph{e.g.,},
        given \texttt{\small Point ::= SEQUENCE} \verb+{+\texttt{x}
        \texttt{\small REAL,} \texttt{y} \texttt{\small REAL}\verb+}+,
        the value \texttt{origin} \texttt{\small Point ::=}
        \verb+{+\texttt{y} \texttt{\small 0.0,} \texttt{x}
        \texttt{\small 0.0}\verb+}+ is rejected.

  \item The \texttt{\small SET OF} type corresponds to the
    mathematical notion of sets with repetition: all elements are of
    the same type, but their number is not known beforehand (unless
    the set's size is constrained to a given value), and they can be
    repeated, \emph{e.g.,} \texttt{\small T ::= SET OF INTEGER} allows
    the value definitions \texttt{empty} \texttt{\small T ::=}
    \verb+{}+ and \texttt{small} \texttt{\small T ::=}
    \verb+{+\texttt{\small 7, 9, 1, 1, 3}\verb+}+.
 
  \item The \texttt{\small SEQUENCE OF} type corresponds to the dynamic
        arrays or lists of some programming languages. It is similar
        to the \texttt{\small SET OF} type, except that the elements
        will be encoded in the specified order. Since the encoding
        rules are out of the scope of this paper, this difference is
        not relevant.

  \item The \texttt{\small CHOICE} type corresponds to a
        \textsf{union} in C, a \textsf{case} in Pascal, or a sum type
        in ML. For instance \texttt{\small T ::= CHOICE}
        \verb+{+\texttt{x} \texttt{\small REAL,} \texttt{y}
        \texttt{\small BOOLEAN}\verb+}+ allows the following
        declarations: \texttt{u} \texttt{\small T ::=} \texttt{x :}
        \texttt{\small 0.5}, where the component \texttt{x} is chosen
        to build the value, and \texttt{v} \texttt{\small T ::=}
        \texttt{y :} \texttt{\small FALSE} where the component
        \texttt{y} is used. The protocol data units are \texttt{\small
        CHOICE} types, because they model all the possible queries and
        responses between two peers. As we show later, a
        \texttt{\small CHOICE} type may be recursive, like the other
        constructed types. An example from a network management
        protocol is\\ {\small \verb+CMISFilter ::= CHOICE {+\\
        \verb+ item FilterItem,+\\ \verb+ and SET OF CMISFilter,+\\
        \verb+ or SET OF CMISFilter,+\\ \verb+ not CMISFilter}+}

\end{itemize}
