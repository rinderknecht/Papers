\section{Functional Queues}

A functional queue is a linear data structure that is used in
\emph{functional languages}, whose semantics force the programmer to
model a \emph{queue} with two \emph{stacks}. Items can be added to a
stack, or \emph{pushed}, on only one of its ends, called the
\emph{top}. They can be removed, or \emph{popped}, only at the top:
\[
\begin{array}{@{}l@{}r@{\;}cc|c|c|c|c|c|cc@{\;}l@{}}
\cline{4-9}
& \text{Push, Pop (top)} & \leftrightsquigarrow & & a & b & c & d & e &&&\\
\cline{4-9}
\end{array}
\]
A queue is like a stack where items are added, or \emph{enqueued}, at
one end, called \emph{rear}, but taken out, or \emph{dequeued}, at the
other end, called \emph{front}:
\[
\begin{array}{@{}l@{}r@{\;}cc|c|c|c|c|c|cc@{\;}l@{}}
\cline{4-10}
& \text{Enqueue (rear end)} 
                & \rightsquigarrow & & a & b & c & d & e &
& \rightsquigarrow & \text{Dequeue (front end).}\\
\cline{4-10}
\end{array}
\]
Let us implement a queue with two stacks: one for enqueuing, called
the \emph{rear stack}, and one for dequeuing, called the \emph{front
  stack}. The previous queue is equivalent to
\[
\begin{array}{r@{\;}cc|c|c|c|c|c|c|cc@{\;}l}
  \cline{3-6}\cline{8-10}
  \text{Enqueue (rear stack)} & \rightsquigarrow & & a & b & c & & d & e & &
  \rightsquigarrow & \text{Dequeue (front stack).}\\
  \cline{3-6}\cline{8-10}
\end{array}
\]
Enqueuing is now pushing on the rear stack and dequeuing is popping on
the front stack. In the latter case, if the front stack is empty and
the rear stack is not, we swap the stacks and reverse the (new) front
stack. Graphically, dequeuing in the configuration 
\(\begin{array}{c|c|c|c|c|c}
  \cline{1-4}\cline{6-6}
  & a & b & c & &\\
  \cline{1-4}\cline{6-6}
\end{array}\)
requires first to make
\(\begin{array}{c|c|c|c|c|c}
  \cline{1-1}\cline{3-6}
  & & a & b & c &\\
  \cline{1-1}\cline{3-6}
\end{array}\)
and then dequeue \(c\).

As a side note, although our presentation is independent of any
programming language, programmers using
\textsf{Erlang}~\cite{Armstrong:2007,Armstrong:2010} may implement
enqueuings and dequeuings as in \fig~\ref{fig:erlang}. As a measure of
the input, we shall say that the queue has size~\(n\) if the total
number of items in both stacks is~\(n\). As a measure of time, we
shall count as one unit one item movement. Therefore, the cost of
enqueuing is \(\C{\texttt{enq}}{n} = 1\). The minimum cost for
dequeuing is \(\B{\texttt{deq}}{n} = 1\), when the front is not empty,
so exactly one item moves (out). The maximum cost is
\(\W{\texttt{deq}}{n} = n+1\), when the front stack is empty and the
rear contains \(n\) items: these move frontward and then the top moves
out.
\begin{figure}[!t]
\centering
\small
\begin{verbatim}
enqueue(Item,{Rear,Front}) -> {[Item|Rear],Front}.

dequeue({[Item|Rear],[]})    -> dequeue({[],rcat(Rear,[Item])});
dequeue({Rear,[Item|Front]}) -> {{Rear,Front},Item}.

rcat(         [],To) -> To;
rcat([Item|From],To) -> rcat(From,[Item|To]).
\end{verbatim}
\caption{Enqueuing and dequeuing in \textsf{Erlang}\label{fig:erlang}}
\end{figure}

Let \(\C{}{n}\) be the cost of a sequence of \(n\)~updates on a
functional queue originally empty. A first attempt at
assessing~\(\C{}{n}\) consists in ignoring any dependence on previous
operations and take the maximum cost individually. Since
\(\C{\texttt{enq}}{k} \leqslant \C{\texttt{deq}}{k}\), we consider a
series of \(n\)~dequeuings in their worst case, that is, with all the
items located in the rear stack. Besides, after \(k\)~updates, there
may be \(k\)~items in the queue, so we draw
\[
\C{}{n} \leqslant \sum_{k=1}^{n-1}{\W{\texttt{deq}}{k}} =
\frac{1}{2}{(n-1)(n+2)} \mathrel{\sim} \frac{1}{2}{n^2}.
\]
Actually, this is overly pessimistic and even unrealistic. First, one
cannot dequeue on an empty queue, therefore, at any time, the number
of enqueuings since the beginning is always greater or equal than the
number of dequeuings and the series must start with one
enqueuing. Second, when dequeuing with the front being empty, the rear
stack is reversed onto the front stack, so its items cannot be
reversed again during the next dequeuing, whose cost will
be~\(1\). Moreover, as remarked above, \(\C{\texttt{enq}}{k} \leqslant
\C{\texttt{deq}}{k}\), so the worst case for a series of
\(n\)~operations occurs when the number of dequeuings is maximum, that
is, when it is \(\lfloor{n/2}\rfloor\). If we denote by~\(e\) the
number of enqueuings and by~\(d\) the number of dequeuings, we have
the relationship \(n = e + d\) and the two requisites for a worst case
become \(e=d\) (\(n\)~even) or \(e=d+1\) (\(n\)~odd).
\begin{figure}
\centering
\subfloat[Enqueue\label{fig:enqueue}]{%
  \includegraphics[bb=71 662 130 721,scale=0.75]{enqueue}
}
\qquad
\subfloat[Dequeue\label{fig:dequeue}]{%
  \includegraphics[bb=71 662 130 721,scale=0.75]{dequeue}
}
\caption{Graphical representations of operations on queues\label{fig:enq_deq}}
\end{figure}

\medskip

\mypar{Dyck Path (\(\boldsymbol{e=d}\))} Let us represent graphically
the updates as in \fig~\ref{fig:enq_deq}.  Textually, we represent an
enqueuing as an opening parenthesis and a dequeuing as a closing
parenthesis. For example, \texttt{((()()(()))())} can be represented
in \fig~\ref{fig:dyck_path} as a \emph{Dyck path}, named in the honor
of the logician Walther (von) Dyck
(\oldstylenums{1856}--\oldstylenums{1934}).
For a broken line to qualify as a Dyck path of length~\(n\), it has to
start at the origin \((0,0)\) and end at coordinates \((n,0)\).  In
terms of a \emph{Dyck language}, an enqueuing is called a \emph{rise}
and a dequeuing is called a \emph{fall}. A rise followed by a fall,
that is, \texttt{()}, is called a \emph{peak}. For instance, in
\fig~\ref{fig:dyck_path}, there are four peaks. The number near each
rise or fall is the cost incurred by the corresponding operation. The
abscissa axis bears the ordinal of each operation.
\begin{figure}[h]
\centering
\includegraphics[bb=78 570 510 725,scale=0.75]{dyck_path}
\caption{Dyck path modeling queue operations (cost \(21\))
\label{fig:dyck_path}}
\end{figure}

When \(e=d\), the graphics is a Dyck path of length \(n=2e=2d\). In
order to deduce the total cost in this case, we must find a
\emph{decomposition} of the path, by which we mean to identify
patterns whose costs are easy to compute and which make up any path,
or to associate any path to another path whose cost is the same but
easy to find. \Fig~\ref{fig:dyck_eq1}
\begin{figure}[b]
\centering
\includegraphics[bb=78 570 510 725,scale=0.75]{dyck_eq1}
\caption{Dyck path equivalent to \fig~\ref{fig:dyck_path}
\label{fig:dyck_eq1}}
\end{figure}
shows how the previous path is mapped to an equivalent path only made
of a series of isosceles triangles whose bases belong to the abscissa
axis. Let us call them \emph{mountains} and their series a
\emph{range}. The mapping is simple: after the first fall, if we are
back to the abscissa axis, we have a mountain and we proceed with the
rest of the path. Otherwise, the next operation is a rise and we
exchange it with the first fall after it. This brings us down by~\(1\)
and the process resumes until the bottom line is reached. We call this
process \emph{rescheduling} because it amounts, in operational terms,
to reordering subsequences of operations a posteriori. For instance,
\fig~\ref{fig:rescheduling} displays the rescheduling of
\fig~\ref{fig:dyck_path}.
\begin{figure}
\centering
\subfloat[Initial\label{fig:initial}]{%
  \includegraphics[scale=0.75]{mountain0}
}
\qquad
\subfloat[Swapping \(4\nearrow 5\) and \(5\searrow 6\)]{%
  \includegraphics[scale=0.75]{mountain1}
}
\qquad
\subfloat[Swapping \(5\nearrow 6\) and \(8 \searrow
  9\)\label{fig:valley}]{%
  \includegraphics[scale=0.75]{mountain4}
}
\qquad
\subfloat[Last one]{%
  \includegraphics[bb=13 647 180 782,scale=0.75]{mountain5}
}
\caption{Rescheduling of \fig~\ref{fig:dyck_path} into
  \fig~\ref{fig:dyck_eq1}\label{fig:rescheduling}}
\end{figure}
Note that two different paths can be rescheduled into the same
path. What makes \fig~\ref{fig:valley} equivalent to
\fig~\ref{fig:initial} is the invariance of the cost because all
operations have cost~\(1\). This always holds because enqueuings
always have cost~\(1\) and the dequeuings involved in a rescheduling
have cost~\(1\), because they found the front stack non\hyp{}empty
after a peak. We proved that all paths are equivalent to a range with
the same cost. Therefore, the maximum cost can be found on ranges
alone. Let us note \(e_1, e_2, \dots, e_k\) the maximal subsequences
of rises; for example, in \fig~\ref{fig:dyck_eq1}, we have \(e_1=3\),
\(e_2 = 3\) and \(e_3 = 1\). Of course, \(e = e_1 + e_2 + \dots +
e_k\). The fall making up the \(i\)th peak incurs the cost
\(\W{\texttt{deq}}{e_i} = e_i + 1\), due to the front being empty
because we started the rises from the abscissa axis. The next
\(e_i-1\) falls have all cost~\(1\), because the front is not
empty. For the \(i\)th mountain, the cost is thus
\(e_i+(e_i+1)+(e_i-1) = 3e_i\). Then \(\C{}{n} = \sum_{i=1}^{k}{3e_i}
= 3e = \frac{3}{2}n\), since \(n=e+d=2e\).

\medskip

\mypar{Dyck Meander (\(\boldsymbol{e=d+1}\))} The other possibility
for a worst case is that \(e = d + 1\) and the graphics is then a
\emph{Dyck meander} whose extremity ends at ordinate \(e-d=1\). An
example is given in \fig~\ref{fig:dyck_meander1}, where the last
operation is a dequeuing.
\begin{figure}[t]
\centering
\includegraphics[bb=76 570 480 725,scale=0.75]{dyck_meander1}
\caption{Dyck meander modeling queue operations (total cost
  \(19\))\label{fig:dyck_meander1}}
\end{figure}
The dotted line delineates the result of applying the rescheduling we
used on Dyck paths. Here, the last operation becomes an
enqueuing. Another possibility is shown in
\fig~\ref{fig:dyck_meander2},
\begin{figure}[b]
\centering
\includegraphics[bb=76 570 480 725,scale=0.75]{dyck_meander2}
\caption{Dyck meander modeling queue operations (total cost
  \(20\))\label{fig:dyck_meander2}}
\end{figure}
where the last operation is left unchanged. The difference between the
two examples lies in the fact that the original last dequeuing has, in
the former case, a cost of~\(1\) (thus is changed) and, in the latter
case, a cost greater than~\(1\) (thus is invariant). The third kind of
Dyck meander is one ending with an enqueuing, but because this
enqueuing must start from the abscissa axis, this is the same
situation as the result of rescheduling a meander ending with a
dequeuing with cost~\(1\) (see dotted line in
\fig~\ref{fig:dyck_meander1} again). Therefore, we are left to compare
the results of rescheduling meanders ending with a dequeuing, that is,
we envisage two cases: either a range of \(n-1\)~operations followed
by an enqueuing or a range of \(n-3\)~operations followed by two rises
and one fall (of cost~\(5\)). In the former case, the total cost is
\(\C{}{n}=\C{}{n-1}+1=(3n-1)/2\), because \(n=e+d=2e-1\). In the
latter case, the cost is \(\C{}{n}=\C{}{n-3}+5=(3n+1)/2\), which is
slightly greater than all the previous costs, so it is the absolute
maximum cost.
