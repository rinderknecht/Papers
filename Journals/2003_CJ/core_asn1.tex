%%-*-latex-*-

It is difficult to separate the different concepts throughout the
syntax. The \emph{types}, \emph{values} and \emph{subtyping
  constraints} may depend on each other: a type may contain
constraints (on components) and values (\emph{e.g.,} default values),
a value has a declared type, and constraints rely upon types
(\emph{e.g.,} inclusion constraint) and values (\emph{e.g.,} value
constraint). Another related difficulty is the large number of
syntactic constructs.

In order to allow a clearer presentation of the constraint collection
(sections~\ref{constraints_from_types},
\ref{constraints_from_subtypes}
and~\ref{full_collection_and_solving}), we define a strict subset of
\mbox{X.680}, which we call from now on \core (versus \emph{full}
\ASN), that will be used in the rest of this paper. In \core,
  \begin{itemize}

    \item there are no \texttt{\small COMPONENTS OF} or selection
      types;

    \item the \texttt{\small INTEGER} type does not allow defining
          constants;

    \item component types are references;

    \item \texttt{\small SET OF} and \texttt{\small SEQUENCE OF} apply
          to references;

    \item default values are references;

    \item enumerated and bit string constants are references;

    \item types of declared values are references;

    \item default, enumerated, integer and bit string values appear in
          a constraint upon their expected type;
 
    \item types in inclusion constraints are references;

    \item there is no type reference just after the symbol
      `\texttt{\small ::=}' and constraints appear only at top-level,
      \emph{i.e.,} the extended Backus-Naur Form for type declarations
      is: {\small $<$\textsf{type reference}$>$ \textsf{::=}
        $<$\textsf{non-reference type without inner constraints}$>$
        \textsf{[}"("$<$\textsf{subtyping constraint}$>$")"\textsf{]}
      }
    \item there are no infinite values, \emph{i.e.,} recursive values.

  \end{itemize}
Since we have not yet introduced the collection, it is awkward to
explain here the rationale behind \core. As a consequence, this
information will be given later and the reader may skip the
next section when reading this for the first time.


\subsection{Mapping full \ASN into \core}
\label{mapping}

Full \ASN is mapped into \core by applying a series of rewritings.  It
is important to note that each step strictly preserves the
expressiveness of full \ASN. In other words: \core can express all
that can be expressed in full \ASN and nothing more.

Another useful property is that each simplification output can be
given in (the syntax of) full \ASN, making presentation easier. As
software tools use a specific internal data representation, the
practical bonus is that pretty-printing is then possible at each stage
with the \emph{same} initial pretty-printer (\emph{i.e.,} for full
\ASN).

It is assumed that the following transformations and checkings apply
to an \ASN module whose syntax complies with
\mbox{X.680}~\cite{X.680:2002}. (The attentive reader will note that
not all the rewritings commute, \emph{i.e.,} the following enumeration
cannot be arbitrarily shuffled.)
\begin{enumerate}

  \item \label{default_values} We extract the default constant values
    from the \texttt{\small SEQUENCE} and \texttt{\small SET} types,
    following the example

        {\small
         \verb+T ::= SET {a REAL DEFAULT 0.0}+\\
         $\longrightarrow
         \left\{
                \begin{tabular}{l}
                  \verb+T ::= SET {a A DEFAULT v}+\\
                  \verb+A ::= REAL+\\
                  \verb+v A ::= 0.0+
                \end{tabular}
              \right.$
        }

        where \texttt{A} is a fresh type reference and \texttt{v} is
        a fresh value reference.

  \item \label{constants}
        We lift the enumeration constants (enumerated, integer and bit
        string constants) to the top-level, as shown by
        (\texttt{v} is a fresh value reference):

        {\small
           \verb+T ::= ENUMERATED {a(x), b}+\\
            $\longrightarrow$
            $\left\{
                \begin{tabular}{l}
                  \verb+T ::= ENUMERATED {a(v), b}+\\
                  \verb+v INTEGER ::= x+
                \end{tabular}
              \right.$

           \verb+T ::= INTEGER {a(x)}+\\
            $\longrightarrow$
            $\left\{
                \begin{tabular}{l}
                   \verb+T ::= INTEGER {a(v)}+\\
                   \verb+v INTEGER ::= x+
                \end{tabular}
              \right.$

          \verb+T ::= BIT STRING {a(x)}+\\
           $\longrightarrow$
           $\left\{
                \begin{tabular}{l} 
                   \verb+T ::= BIT STRING {a(v)}+\\
                   \verb+v INTEGER (0..MAX) ::= x+
                \end{tabular}
              \right.$
        }

  \item \label{types_from_values}
        For each value declaration, we extract the given type and
        create a corresponding type declaration for it. We also create
        another type declaration where the previous type is required 
        to contain the originally declared value. For instance,
        consider\\
        {\small
         \begin{tabular}{l}
           \verb+y REAL(0..9) ::= 1+
           $\,\, \longrightarrow \,\,$
           \hspace*{-3.9pt}
           $\left\{
                \begin{tabular}{l} 
                   \verb+y A ::= 1+\\
                   \verb+A ::= REAL(0..9)+\\
                   \verb+B ::= A (y)+
                \end{tabular}
              \right.$
         \end{tabular}
        }\\
        where \verb+A+ and \verb+B+ are fresh type references. This
        way the declared type in a value definition is bound in a
        type definition (\texttt{\small A}). Also, the
        typechecking of a value (\texttt{y}) can be done with our
        algorithm (through \texttt{\small B}), since it deals with
        subtyping constraints.

  \item \label{COMPONENTS_OF_reference}
        The types which appear in
        \texttt{\small COM\-PO\-NENTS OF} constructions are
        replaced by fresh type references (in the following,
        \texttt{A} is a fresh type reference)
        {\small
         \verb+ T ::= SET {COMPONENTS OF SET {a REAL}}+\\
         $\longrightarrow$
         $\left\{
            \begin{tabular}{l}
               \verb+T ::= SET {COMPONENTS OF A}+\\
               \verb+A ::= SET {a REAL}+
            \end{tabular}
          \right.$}

  \item \label{contained_reference}
        We want to relax the dependence between
        subtyping constraints and types. Hence, for each inclusion
        constraint, we replace the included type by 
        a fresh reference and add a corresponding new type
        declaration, like (\texttt{\small A} is a fresh type
        reference):\\
        {\small
         \begin{tabular}{l}
           \verb+T ::= U(SET{a REAL})+
           $\rightarrow
           \left\{\hspace*{-2pt}
                \begin{tabular}{l}
                   \verb+T ::= U (A)+ \\
                   \verb+A ::= SET {a REAL}+
                \end{tabular}
              \right.$
         \end{tabular}}

  \item \label{component_types}
        We replace each component type by a reference:\\
        {\small
         \verb+T ::= SET {a REAL, b SET {d INTEGER}, c U (V)}+\\
         $\rightarrow
          \left\{\hspace*{-5pt}
            \begin{tabular}{l}
               \verb+T ::= SET {a A, b B, c C}+\\
               \verb+A ::= REAL  B ::= SET{d D}  C ::= U(V)+\\
               \verb+D ::= INTEGER+
            \end{tabular}
          \right.$}\\
        where \texttt{\small A}, \texttt{\small B}, \texttt{\small C}
        and \texttt{\small D} are fresh type references.

  \item \label{types_in_SETOF}
        At this step, we replace each type to which a \texttt{\small
        SET OF} or a \texttt{\small SEQUENCE OF} applies, by a
        reference:\\
        {\small
            \verb+T ::= SET OF REAL (C)+
            $\longrightarrow$
            $\left\{\hspace*{-5pt}
               \begin{tabular}{l}    
                  \verb+T ::= SET OF A+\\
                  \verb+A ::= REAL (C)+
               \end{tabular}
             \right.$}\\
        where \texttt{\small A} is a fresh type reference.

  \item We remove the selection types (at top-level)\\
        {\small
        \begin{tabular}{l} 
          $\left\{\hspace*{-5pt}
             \begin{tabular}{l}
                \verb+A ::= i < B+\\
                \verb+B ::= C+\\
                \verb+C ::= CHOICE{i D}+\\
                \verb+D ::= INTEGER+
             \end{tabular}
            \right.
          \longrightarrow
          \left\{\hspace*{-5pt}
               \begin{tabular}{l}
                  \verb+A ::= INTEGER+\\
                  \verb+B ::= C+\\
                  \verb+C ::= CHOICE{i D}+\\
                  \verb+D ::= INTEGER+
               \end{tabular}
             \right.$
        \end{tabular}}\\
       You must also be aware of the possibly misleading case:\\
       {\small
       \begin{tabular}{@{}r@{\;}c@{\;}l@{}}
         $\left\{
            \begin{tabular}{@{}l@{}}
              \verb+A ::= SET OF S+\\
              \verb+B ::= A (SIZE (7))+
            \end{tabular}
          \right.$
         & $\longrightarrow$
         & $\left\{
            \begin{tabular}{@{}l@{}}
              \verb+A ::= SET OF S+\\ 
              \verb+B ::= SET (SIZE (7)) OF S+
            \end{tabular}
            \right.$
       \end{tabular}
       }
   
       The result \verb+B ::= SET OF S (SIZE (7))+ would be wrong!

       This step is difficult because it removes all recursive types
       declarations that do not lead to a uniquely defined type, like
       \texttt{T ::= T} or \texttt{T ::= CHOICE \{a a < T\}}
       etc. 

        Note that the selection types that do not define a unique type
        lead to recursive type definitions whose pattern is
        \texttt{\small X ::= X}, as \verb+T ::= CHOICE{a a < T}+
        $\longrightarrow$\\
        {\small
         $\left\{\hspace*{-4pt}
            \begin{tabular}{l}
               \verb+T ::= CHOICE {a A}+\\
               \verb+A ::= a < T+
            \end{tabular}
          \right.
         \longrightarrow
         \left\{\hspace*{-4pt}
             \begin{tabular}{l}
                \verb+T ::= CHOICE {a A}+\\
                \verb+A ::= A+
             \end{tabular}
          \right.$
        }

        From now on we know exactly what a referenced
        type is, and thus what is the type of a value.

  \item \label{type_references} The top-level type references are
    unfolded, \emph{i.e.,} the type references at the declaration
    level are replaced by the type they reference\\
        {\small
         $\left\{
             \begin{tabular}{l}
                \verb+T ::= U (C)+\\
                \verb+U ::= REAL (D)+
             \end{tabular}
          \right.$
          $\longrightarrow$
          $\left\{
             \begin{tabular}{l}
                \verb+T ::= REAL (D ^ C)+\\
                \verb+U ::= REAL (D)+
             \end{tabular}
          \right.$}\\
        During this step, ill-formed recursive definitions, like
        \texttt{\small X ::= X}, are rejected.

  \item The default values are expanded, like\\
        {\small
         $\left\{
            \begin{tabular}{ll}
                 \verb+v T ::= {}+ 
               & \verb+T ::= SET {a U DEFAULT w}+
            \end{tabular}
          \right.$\\
          $\longrightarrow$
            \verb+ v T ::= {a w}   T ::= SET {a U DEFAULT w}+
        }

  \item \label{COMPONENTS_OF_unfolding}
        The type references in the \texttt{\small COMPONENTS OF}
        clauses are replaced by their corresponding components\\
        {\small
         $\left\{
            \begin{tabular}{l}
               \verb+T ::= SET {COMPONENTS OF A}+\\
               \verb+A ::= SET {a REAL}+
            \end{tabular}
          \right.$\\
         $\longrightarrow$
         $\left\{
            \begin{tabular}{l}
              \verb+T ::= SET {a REAL}+\\
              \verb+A ::= SET {a REAL}+
            \end{tabular}
          \right.$}

  \item \label{constants_unfolding}
        Integer and bit string constants are unfolded\\
        {\small
           $\left\{
              \begin{tabular}{l}
                 \verb+T ::= INTEGER {c(x)}+\\
                 \verb+v T ::= c+
              \end{tabular}
            \right.
           \longrightarrow
           \left\{
              \begin{tabular}{l}
                \verb+T ::= INTEGER+\\
                \verb+v T ::= x+
              \end{tabular}
           \right.$
        }\\
        In the case of bit string values which are specified by
        means of a series of bit names, we unfold their associated
        references and replace the value by an equivalent one
        without those names\\
        {\small
           $\left\{
              \begin{tabular}{l}
                \verb+T ::= BIT STRING {msb(x),lsb(y)}+\\
                \verb+v T ::= {msb,lsb}+
              \end{tabular}
            \right.$\\
            $\longrightarrow
            \left\{
              \begin{tabular}{l}
                \verb+T ::= BIT STRING+\\
                \verb+v T ::= '10000001'B+
              \end{tabular}
            \right.$
        }\\
        where 
        {\small
           $\left\{
              \begin{tabular}{l}
                \verb+x INTEGER (0..MAX) ::= 7+\\
                \verb+y INTEGER (0..MAX) ::= 0+
              \end{tabular}
           \right.$
        }
        (see step~\ref{constants}).

        This step may reveal some recursive values\\
        {\small
           $\left\{
              \begin{tabular}{l}
                 \verb+T ::= INTEGER {c(v)}+\\
                 \verb+v T ::= c+
              \end{tabular}
            \right.
           \longrightarrow
           \left\{
              \begin{tabular}{l}
                \verb+T ::= INTEGER+\\
                \verb+v T ::= v+
              \end{tabular}
            \right.$
        }

  \item We disallow the recursive values, like {\small
  \verb+v T ::= {v}+} or\\
        {\small
         $\left\{
            \begin{tabular}{ll}
                \verb+v T ::= {v}+
              & \verb+T ::= SET OF T+\\
                \textnormal{or}
              & \\
                \verb+v T ::= {}+
              & \verb+T ::= SET {a T DEFAULT v}+
            \end{tabular}
          \right.$
        }

\end{enumerate}



\subsection{Validation issues}\label{validation}

In \core, it is possible that

\begin{enumerate}
  
  \item \label{finiteness} types have only infinite values:
        \texttt{\small T ::= SET} \verb+{+\texttt{a} \texttt{\small
        T}\verb+}+

  \item \label{type_conformance} values are ill-typed:
        \texttt{v} \texttt{\small T ::= ""} \hspace*{6mm}
        \texttt{\small T ::= REAL}

  \item \label{type_compatibility} in particular, value references
        may be ill-typed:\\
        {\small 
        $\left\{
           \begin{tabular}{l}
              \verb+a A ::= b     A ::= INTEGER+\\
              \verb+b B ::= 1.5   B ::= REAL+
           \end{tabular}
        \right.$
        }

  \item \label{constraint_consistence} constraints are
        inconsistent: \texttt{\small T ::= REAL (SIZE(7))}

  \item \label{subtype_non_emptiness} subtypes are empty:\\
     \texttt{\small T ::= SET ((SIZE(1)) INTERSECTION (SIZE(2)))
        OF REAL};
        
  \item \label{solvability} subtypes have no value set:
        \texttt{\small T ::= A(ALL EXCEPT T)}

\end{enumerate}
These cases can be classified into the different problems: the
\emph{finiteness} problem (case~\ref{finiteness}), the
\emph{typechecking} problem (case~\ref{type_conformance}), the
\emph{type compatibility} problem (case~\ref{type_compatibility}), the
\emph{constraint consistence} problem
(case~\ref{constraint_consistence}), the \emph{non-emptiness} problem
(case~\ref{subtype_non_emptiness}) and the \emph{solvability} problem
(case~\ref{solvability}). The type compatibility problem is a sub-case
of the typechecking problem, and constraint consistence together with
non-emptiness are sub-cases of the solvability problem, because we
will explicitly construct the values of each (sub)type when the system
is solved. Moreover, since we added a new type declaration for each
value declaration at rewriting step~\ref{types_from_values}, the
solvability of the subtyping constraints will cope with the
typechecking problem. So, finiteness and solvability are enough to get
a full validation of \mbox{X.680} specifications and we need, as a
starting point, to express formally those concepts.

\subsection{Algorithmic meta-language}

We shall use as \emph{meta language} for the description of our
algorithm a version of the functional language ML:
OCaml\footnote{{\small\url{http://www.ocaml.org/}}}~\cite{CousineauMauny:1998},
which is a full-fledged programming language, as well as,
historically, a logic meta-language. Therefore our algorithm is close
to an actual implementation and is also a formal (operational)
model. Readers familiar with ML may skip this section, which gives a
crash overview of its syntax and semantics. (This presentation is
based on Pidgin ML~\cite{Huet:2002}.)

The core language has types and values. Thus, $1$ is a value of
predefined type \ocamltypename{int}, whereas \textsf{"CL"} is a
\ocamltypename{string}. Pairs of values inhabit the corresponding
product type. Therefore, $(1,\textsf{"CL"})$ has type
$(\ocamltypename{int} \times \ocamltypename{string})$. Recursive type
declarations create new types, whose values are inductively built from
the associated constructors. Thus a type modeling a binary tree of
integers could be declared as a sum by: $\ocamlkwd{type} \,\,
\ocamltypename{ib\_tree} = \ocamlconstr{Node} \, \ocamlkwd{of} \,
\ocamltypename{ib\_tree} \times \ocamltypename{int} \times
\ocamltypename{ib\_tree} \, \mid \ocamlconstr{Leaf}$. Parametric types
give rise to polymorphism: if \ocamlvaluename{x} is of type
\ocamltypename{t} and \ocamlvaluename{l} is of type (\ocamltypename{t}
\ocamlconstr{list}), we construct the list adding \ocamlvaluename{x}
to \ocamlvaluename{l} as $\ocamlvaluename{x} \CONS
\ocamlvaluename{l}$. The empty list is \emptyL, of (polymorphic) type
$('\ocamltypename{a} \,\, \ocamlconstr{list})$. Although the language
is strongly typed, explicit type specifications are rarely needed from
the programmer, since principal types may be inferred mechanically.

The language is functional in the sense that functions are first class 
objects. Therefore the integer doubling function may be written as 
$\ocamlkwd{fun} \,\, \ocamlvaluename{x} \rightarrow
\ocamlvaluename{x}+\ocamlvaluename{x}$, and it has type
$\ocamltypename{int} \rightarrow \ocamltypename{int}$. It may be
associated to the name \ocamlvaluename{double} by declaring:
$\ocamlkwd{let} \,\, \ocamlvaluename{double} = \ocamlkwd{fun} \,
\ocamlvaluename{x} \rightarrow
\ocamlvaluename{x}+\ocamlvaluename{x}$. Equivalently we could write:
$\ocamlkwd{let} \,\, \ocamlvaluename{double} \, \ocamlvaluename{x} =
\ocamlvaluename{x}+\ocamlvaluename{x}$. Its application to value
\ocamlvaluename{n} is written as $(\ocamlvaluename{double} \,
\ocamlvaluename{n})$ or even $\ocamlvaluename{double} \,
\ocamlvaluename{n}$ when there is no ambiguity. Application associates
to the left, and thus $\ocamlvaluename{f} \, \ocamlvaluename{x} \,
\ocamlvaluename{y}$ stands for $((\ocamlvaluename{f} \,
\ocamlvaluename{x}) \, \ocamlvaluename{y})$.
Recursive functional values are declared with the keyword
\ocamlkwd{rec}. Thus we may define the factorial function as:
$\ocamlkwd{let} \,\, \ocamlkwd{rec} \,\, \ocamlvaluename{fact} \,
\ocamlvaluename{n} = \ocamlkwd{if} \, \ocamlvaluename{n} \leqslant 0 \,
\ocamlkwd{then} \, 1 \, \ocamlkwd{else} \, \ocamlvaluename{n} \times
(\ocamlvaluename{fact} (\ocamlvaluename{n}-1))$.
Functions may be defined by pattern matching. Thus the first
projection of pairs could be defined by:
$\ocamlkwd{let} \,\, \ocamlvaluename{fst} = \ocamlkwd{fun} \,
(\ocamlvaluename{x},\ocamlvaluename{y}) \rightarrow
\ocamlvaluename{x}$ or equivalently (since there is only one pattern
in this case) by: $\ocamlkwd{let} \,\, \ocamlvaluename{fst} \,
(\ocamlvaluename{x}, \ocamlvaluename{y}) = \ocamlvaluename{x}$.
Pattern-matching is also usable in \ocamlkwd{match} expressions which
generalise case analysis, such as:
$\ocamlkwd{match} \,\, \ocamlvaluename{l} \,\, \ocamlkwd{with} \,\,
\emptyL \rightarrow \ocamlkwd{true} \mid \wildcard \rightarrow
\ocamlkwd{false}$, which tests if list \ocamlvaluename{l} is
empty, using underscore as catch-all pattern.

Evaluation is strict, which means that \ocamlvaluename{x} is evaluated
before invoking \ocamlvaluename{f} in the evaluation of
$(\ocamlvaluename{f} \, \ocamlvaluename{x})$. The \ocamlkwd{let}
expressions allow the sequentialization of computations, and the
sharing of sub-computations. Thus $\ocamlkwd{let} \,\,
\ocamlvaluename{x} = \ocamlvaluename{fact} \, 10 \,\, \ocamlkwd{in}
\,\, \ocamlvaluename{x}+\ocamlvaluename{x}$ will compute
$\ocamlvaluename{fact} \, 10$ first, and only once.

Exceptions are declared with the type of their parameters, like in:
$\ocamlkwd{exception} \,\, \ocamlconstr{Failure} \, \ocamlkwd{of} \,
\ocamltypename{string}$. An exceptional value may be raised, like in: 
$\ocamlvaluename{raise}$ $(\ocamlconstr{Failure}$ $\textsf{"div
0"})$ and handled by a \ocamlkwd{try} switching on exception patterns,
like: $\ocamlkwd{try} \, \ocamlvaluename{expression} \,
\ocamlkwd{with} \, \ocamlconstr{Failure} \, \ocamlvaluename{s}
\rightarrow \ldots$ Other imperative constructs may be used, such as 
references, mutable arrays, while loops and I/O commands, 
but we shall seldom need them. Sequences of instructions are 
evaluated in left to right regime in bloc expressions such as:
$\ocamlkwd{begin} \,\, {\ocamlvaluename{e}_1; ...;
\ocamlvaluename{e}_n} \,\, \ocamlkwd{end}$. 

ML is a \emph{modular} language, in the sense that sequences of type,
value and exception declarations may be packed in a structural unit
called a \emph{module}, amenable to separate treatment. 
Modules have types themselves, called \emph{signatures}. Parametric 
modules are called \emph{functors}. The algorithms presented in this
paper will only use this modularity structure to access some library
functions --- the syntax ought to be self-evident.

Despite the focus in this paper is algorithmic, the readers
uninterested in computational details may think of ML
definitions as recursive equations over inductively defined
algebras.


\subsection{Abstract grammar}\label{abstract_grammar}

Let us use OCaml's algebraic type declarations to define the
\emph{abstract grammar} of \core. This grammar captures the
syntactically correct constructs of \core, except those which involve
\emph{tags}~\cite[\S{3.6.69}, \S{8}]{X.680:2002} (since they are
related to the encoding rules) and the \kwdOBJECTIDENTIFIER{} and
\kwdRELATIVEOID{} types and values (for the sake of brevity). The
parser's output is a pair of a type environment and a value
environment. The former is a mapping from type names to subtypes,
corresponding to the type declarations in the \ASN specification, and
the latter is a mapping from value names to values, corresponding to
the value declarations. The subtypes and values are \emph{abstract
syntax trees}, complying with the abstract grammar. We do not follow
the syntactic conventions of OCaml exactly, as detailed below.
\begin{itemize}

  \item in mutual recursive polymorphic variant definitions, we shall
    allow type names instead of a variant, like $\ocamlkwd{type} \,
    \ocamltypename{t} = \textsf{[}\ocamlpvar{K}\textsf{]} \,
    \ocamlkwd{and} \, \ocamltypename{u} = \textsf{[}\ocamlpvar{L} \mid
      \ocamltypename{t}\textsf{]}$ (this limitation can be
    circumvented by an implementation trick out of scope here);

   \item We allow \ASN symbols or keywords as data constructors, like
         `$\pmb{<}$', `$\pmb{..}$', \kwdMINUSINFINITY{} or
         \kwdOCTETSTRING; we then use underscores to 
         denote the location of their arguments, as in 
         `$\wildcard \pmb{<} \wildcard$';

   \item We sometimes write $\kwdCOMPONENTSOF \,\, \T \, \sigma$
     instead of the correct (non-currified) $\kwdCOMPONENTSOF \, (\T,
     \sigma)$, and similarly for other data constructors.

\end{itemize}
The abstract grammar for \core values is defined as follows.  Firstly,
we assume that the parser removes the ambiguity between
enumeration constants~\cite[\S{19}]{X.680:2002} and value
references~\cite[\S{11.4}]{X.680:2002}. For instance, in `\texttt{a}
\texttt{\small T} ::= \texttt{b}', the token \texttt{b} can denote
either an enumeration constant or a value reference, depending on the
definition of the type \texttt{\small T}. The ambiguity can always be
removed just by looking at the type definition (this is easy in
\core). We start by defining the OCaml type \ocamltypename{v\_ref} for
value references, the type \ocamltypename{item}, which is used
later in the definition of enumerated constants, and the type
\ocamltypename{label}, which denotes component names

\smallskip

\noindent
\ocamlkwd{type} \ocamltypename{v\_ref} = \textsf{[}\VRef{}
\ocamlkwd{of} \ocamltypename{string}\textsf{]}

\noindent 
\ocamlkwd{type} \ocamltypename{item} = \ocamltypename{string} 
\ocamlkwd{and} \ocamltypename{label} = \ocamltypename{string}

\smallskip

\noindent
The values of \VRef's argument are noted $y$. Values of type
\ocamltypename{item} are noted $a$, $b$, $c$, and lists of such
values are noted I. Values of type \ocamltypename{label} are noted
$l$, and sets of labels are noted $\Labels$. Let us now define the
numeric types straightforwardly:

\smallskip

\noindent 
\ocamlkwd{type} \ocamltypename{integer} = 
       \textsf{[}\PosInt{} \ocamlkwd{of} \ocamltypename{int}
$\mid$ \NegInt{} \ocamlkwd{of} \ocamltypename{int}\textsf{]}

\noindent \ocamlkwd{type} \ocamltypename{real} = 
       \textsf{[}\PosReal{} \ocamlkwd{of} \ocamltypename{float} 
$\mid$ \NegReal{} \ocamlkwd{of} \ocamltypename{float}\textsf{]}

\smallskip

\noindent
where \PosInt{} means `Positive or null integer'. Next, the OCaml
type for \ASN values is named $\V$:

\smallskip

\noindent 
\ocamlkwd{type} $\V$ = 
\textsf{[}\List{} \ocamlkwd{of} $\V$ \ocamlconstr{list}
$\mid$ \Map{} \ocamlkwd{of} \ocamltypename{label} $\rightarrow$
       $\V$\\
\hspace*{-2mm}
\begin{tabular}{rl}
  $\mid$ & \hspace*{-4mm} 
           \Nil{}
  $\mid$   \kwdTRUE{} 
  $\mid$   \kwdFALSE{}
  $\mid$   \pvString{} \ocamlkwd{of} \ocamltypename{string}
  $\mid$   \ocamltypename{integer}\\
  $\mid$ & \hspace*{-4mm} 
           \wildcard $\pmb{:}$ \wildcardof \ocamltypename{label}
           $\times$ $\V$
  $\mid$   \Enum{} \ocamlkwd{of} \ocamltypename{item}\\
  $\mid$ & \hspace*{-4mm} 
           \HexStr{} \ocamlkwd{of} \ocamltypename{string}
  $\mid$   \BinStr{} \ocamlkwd{of} \ocamltypename{string}
  $\mid$   \kwdNULL{}
  $\mid$   \ocamltypename{real}\\
  $\mid$ & \hspace*{-4mm} \kwdPLUSINFINITY{}
  $\mid$   \hspace*{-5pt}
           \kwdMINUSINFINITY{}
  $\mid$   \ocamltypename{v\_ref}\textsf{]}\\
\end{tabular}

\smallskip

\noindent
where \List{} corresponds to values of \kwdSETOF{} and
\kwdSEQUENCEOF{} types~\cite[\S{25}, \S{27}]{X.680:2002}; \Map{}
models values of the \kwdSET{} and \kwdSEQUENCE{} types~\cite[\S{24},
\S{26}]{X.680:2002} (the argument is a mapping from labels to values);
\Nil{} captures the ambiguous value \verb+{}+, which can be of type
\kwdSETOF, \kwdSEQUENCEOF, \kwdSET{}, \kwdSEQUENCE{} or \kwdBITSTRING;
\kwdTRUE{} and \kwdFALSE{} are the values of the \kwdBOOLEAN{} type;
\pvString{} stands for all kinds of character strings;
$\wildcard~\pmb{:}~\wildcard$ corresponds to \kwdCHOICE{}
values~\cite[\S{28}]{X.680:2002} (thus its argument is a pair of a
label and a value); \Enum{} models enumerated constants; \HexStr{}
corresponds to \kwdOCTETSTRING{} values~\cite[\S{22}]{X.680:2002};
\BinStr{} represents \kwdBITSTRING{}
constants~\cite[\S{21}]{X.680:2002}; \kwdNULL{} captures the special
\kwdNULL{} value~\cite[\S{23}]{X.680:2002}; \kwdPLUSINFINITY{} and
\kwdMINUSINFINITY{} are special values of type \kwdREAL{}.

OCaml values of type $\V$ will be noted $v$, and lists of values
L. The value environments are noted $\ValueEnv$, and have type
$\ocamltypename{string} \rightarrow \V$. The argument of the \Map{}
data constructor is noted $\IdEnv{}$ or $\IdEnv{\Labels}$ when the set
of labels (the domain of the mapping) is $\Labels$ (the empty mapping
is simply noted $\{\}$). The OCaml recursive type ${\cal T}$
represents the \core types:

\smallskip

\noindent 
\ocamlkwd{type} ${\cal T}$ = 
           \textsf{[}\kwdCHOICE{} \ocamlkwd{of} \ocamltypename{label}
           $\rightarrow$ \ocamltypename{t\_ref}
  $\mid$   \kwdOCTETSTRING{}\\
\hspace*{-2mm}
\begin{tabular}{rl}
  $\mid$ & \hspace*{-4mm}
           \kwdSET{} \ocamlkwd{of} \ocamltypename{components}
  $\mid$   \kwdINTEGER{}
  $\mid$   \pvString{}
  $\mid$   \kwdNULL{}\\
  $\mid$ & \hspace*{-4mm}
           \kwdSEQUENCEOF{} \ocamlkwd{of} \ocamltypename{t\_ref}
  $\mid$   \kwdREAL{}
  $\mid$   \kwdBITSTRING{}\\
  $\mid$ & \hspace*{-4mm}
           \kwdSETOF{} \ocamlkwd{of} \ocamltypename{t\_ref}
  $\mid$   \kwdSEQUENCE{} \ocamlkwd{of}
           \ocamltypename{components}\\
  $\mid$ & \hspace*{-4mm}
           \kwdENUMERATED{} \ocamlkwd{of}
           \ocamltypename{item} \ocamlconstr{list}
  $\mid$   \kwdBOOLEAN{}
  $\mid$   \ocamltypename{t\_ref}\textsf{]}
\end{tabular}

\noindent
\ocamlkwd{and} \ocamltypename{t\_ref} = \textsf{[}\TRef{}
\ocamlkwd{of} ${\cal R}$\textsf{]} 
\ocamlkwd{and} ${\cal R}$ = \ocamltypename{string}

\noindent 
\ocamlkwd{and} \ocamltypename{components} =\\
\hspace*{2mm} 
\ocamltypename{label} $\!\rightarrow\!$ \ocamltypename{t\_ref}
$\!\times\!$ 
\textsf{[}\kwdOPTIONAL{} $\!\mid\!$ \kwdDEFAULT{} \ocamlkwd{of}
\ocamltypename{v\_ref}\textsf{]} \ocamlconstr{option}

\smallskip

\noindent
The type \ocamltypename{t\_ref} denotes type references. The type
${\cal R}$ is the countable set of type reference names.  The type
\ocamltypename{components} defines the components of \kwdSET{} and
\kwdSEQUENCE{} \core types: it is a function from labels to pairs
whose first projection is the type reference (in \core, component
types are references) and whose second projection models the
component's optional annotation, which tells whether the component is
mandatory, optional or has a default value.

OCaml values of type ${\cal T}$ are noted \T. Values of type ${\cal
R}$ are noted $x$. The mapping of type $\ocamltypename{label}
\rightarrow \ocamltypename{t\_ref}$, which is the argument of
\kwdCHOICE, is noted $\FieldEnv{}$, or $\FieldEnv{\Labels}$ when the
labels range over $\Labels$ (in general, the domain of a mapping can
be put in subscript), \emph{e.g.,} $\kwdCHOICE \, \FieldEnv{\Labels}$. Values
of type \ocamltypename{components} are mappings noted $\Phi$,
\emph{e.g.,} $\kwdSET \, \Phi$.

\ASN subtyping constraints~\cite[\S{45}]{X.680:2002} are modelled
by the following type ${\cal C}$, whose values are noted C:

\smallskip

\noindent \ocamlkwd{type} ${\cal C}$ = 
\textsf{[}\wildcard \kwdUNION{} \wildcardof 
   ${\cal C}$ $\times$ ${\cal C}$
$\mid$   \ocamltypename{interval}\\
\begin{tabular}{rl}
  $\mid$ & \hspace*{-4mm}
           \wildcard \kwdINTERSECTION{} \wildcardof 
           ${\cal C}$ $\times$ ${\cal C}$
  $\mid$   $\V$\\
  $\mid$ & \hspace*{-4mm}
           \wildcard \kwdEXCEPT{} \wildcardof
           ${\cal C}$ $\times$ ${\cal C}$
  $\mid$   \kwdALLEXCEPT{} \ocamlkwd{of} ${\cal C}$\\
  $\mid$ & \hspace*{-4mm}
           \kwdFROM{} \ocamlkwd{of} ${\cal C}$ 
  $\mid$   \kwdSIZE{} \ocamlkwd{of} ${\cal C}$
  $\mid$   \kwdINCLUDES{} \ocamlkwd{of} \ocamltypename{t\_ref}\\
  $\mid$ & \hspace*{-4mm}
           \kwdWITHCOMPONENTS{} \ocamlkwd{of}\\
         & \ocamltypename{kind}
           $\times$ (\ocamltypename{label} $\rightarrow$ ${\cal C}$
           \ocamlconstr{option}
           $\times$ \ocamltypename{status} \ocamlconstr{option})\\
  $\mid$ & \hspace*{-4mm}
           \kwdWITHCOMPONENT{} \ocamlkwd{of} ${\cal C}$ 
  $\mid$   \kwdPATTERN{} \ocamlkwd{of}
           \ocamltypename{string}\textsf{]} 
\end{tabular}

\noindent \ocamlkwd{and} \ocamltypename{kind} = 
\Partial{} $\mid$ \Full

\noindent \ocamlkwd{and} \ocamltypename{status} = 
       \textsf{[}\kwdPRESENT{}
$\mid$ \kwdABSENT{}
$\mid$ \kwdOPTIONAL\textsf{]}

\noindent \ocamlkwd{and} \ocamltypename{interval} =
\wildcard \wildcard $\pmb{..}$ \wildcard \wildcardof\\
\hspace*{2mm}
\ocamltypename{bound} $\times$
\ocamltypename{in\_out} $\times$ \ocamltypename{in\_out}
$\times$  \ocamltypename{bound}

\noindent \ocamlkwd{and} \ocamltypename{bound} = 
         \textsf{[}\kwdMIN{}
$\mid$   \kwdMAX{}
$\mid$   \pvString{} \ocamlkwd{of} \ocamltypename{string}
$\mid$   \ocamltypename{real}\\
\begin{tabular}{rl}
$\mid$ & \hspace*{-4mm}
         \HexStr{} \ocamlkwd{of} \ocamltypename{string}
$\mid$   \BinStr{} \ocamlkwd{of} \ocamltypename{string}
$\mid$   \ocamltypename{integer}\textsf{]}
\end{tabular}

\noindent \ocamlkwd{and} \ocamltypename{in\_out} = 
$\pmb{<}$ $\mid$ $\pmb{\leqslant}$

\smallskip

The OCaml type \ocamltypename{subtype} models \ASN subtypes. An \ASN
subtype is a pair of a type and an optional subtyping constraint.
OCaml values of type ${\cal C}$ \ocamlconstr{option} are noted
$\sigma$. The type environments are noted $\TypeEnv$, and have type
$\ocamltypename{string} \rightarrow \ocamltypename{subtype}$.

\smallskip

\noindent \ocamlkwd{type} \ocamltypename{subtype} = 
  ${\cal T}$ $\times$ (${\cal C}$ \ocamlconstr{option})
