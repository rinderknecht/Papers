%%-*-latex-*-

\abstract{\emph{Abstract Syntax Notation One} (\ASN) is a standard
  language for defining data types whose values may be exchanged
  across a network between two communicating applications,
  independently from the possible heterogeneity of the peers. \ASN has
  been adopted by a wide range of applications, such as network
  management, secure email, mobile telephony, voice over IP etc. It
  offers a very involved subtyping paradigm consisting of constraints
  upon recursive types, which restrict their sets of values in a
  set-theoretic manner or in a structural way. Because of this great
  expressiveness, most \ASN compilers are not likely to fully check
  arbitrary combinations of subtyping constraints. We propose to fully
  validate the \mbox{X.680} specifications, i.e., the main part of
  \ASN, by means of an algorithm which relies on the set constraints
  theory. Set constraints are inclusions between expressions
  interpreted over the domain of sets of trees which may be
  recursively defined. We define a system of constraints which can
  model all the specifications, we provide a complete collecting
  algorithm which extracts such constraints from a given
  specification, and, finally, we give a solving procedure which
  relies upon an algorithm of Aiken and Wimmers. As a result, either
  the constraints have no solutions (and the specification must be
  rejected), or the value sets can be finitely represented. It is
  straightforward to determine whether these value sets are empty; if
  they are empty then the specification is rejected. This article
  addresses both the network tool implementors and the theorist
  audience.}
