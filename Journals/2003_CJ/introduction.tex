%%-*-latex-*-

The wide variety of software and hardware architectures in distributed
systems and telecommunications makes it valuable to use a common
high-level data notation in protocol specifications. For this reason,
the ISO organization and the International Telecommunications Union
defined the \emph{Abstract Syntax Notation One}
(\ASN)~\cite{X.680:2002,X.681:2002,X.682:2002,X.683:2002} series of
standards. \ASN is a language of data types allowing the protocol
designer to capture numerous networking concepts, such as protocol
data units, without worrying about the possible environment and
implementation heterogeneity of the peers. The peers must share a set
of \ASN modules and agree upon a method for encoding values (which are
produced at run-time by the communicating applications) into series of
bits: the \emph{encoding rules}~\cite{X.690:2002, X.691:2002}. An \ASN
compiler accepts a set of \ASN modules and, according to a given set
of encoding rules and a peer-specific target programming language,
produces a set of data type definitions in that programming language,
together with a codec for the values to be exchanged. Then these
pieces of source code are compiled and linked separately against the
communicating application.

\ASN has been adopted for a wide range of applications, such as
network management, secure email, mobile telephony, air traffic
control, video conferencing over the Internet, electronic commerce,
digital certificates, radio paging, as well as emerging technologies
like interactive television and financial service systems. \ASN-based
software is used in Microsoft's Internet Explorer and Outlook. It is
also found in wireless applications from Nokia, Ericsson and
Motorola. \ASN is used to implement cryptographic protocols which
secure credit card purchases over the Internet. Biometrics, databases,
ATM transactions, plane take-offs and landings all rely on
\mbox{ASN.1}\footnote{See \url{http://asn1.elibel.tm.fr/} to learn
about more uses.}.

There are excellent books~\cite{Dubuisson:2000,Larmouth:2000} for the
audience of protocol designers and users, but it is still a challenge
to write an \ASN compiler. The main reason is that, in order to
fulfill its users' numerous needs, the language is extremely
expressive (without including functions). As a consequence, some
compilers may reject valid specifications or, worse, silently accept
invalid ones. Vendors argue that this is hardly a real problem because
such complex specifications are rarely found in
practice. Nevertheless, we claim that this pragmatic approach can
fruitfully be enhanced by a theoretical study which leads to an actual
implementation. Semantics, i.e. consistent mapping from the
syntactic constructs into some set of logical objects, leads to a
greater understanding of \ASN and the opportunity for a better
product. The aim of this work is to provide such a mapping for
\mbox{X.680}~\cite{X.680:2002}, the main part of
\mbox{ASN.1}\footnote{\mbox{X.680} does not contain information
objects, non-subtyping constraints or parameterization.}, by means of
an algorithm that can be implemented and integrated in the front-end
analyser of an \ASN compiler.

Since the paradigm of \ASN data types is `types as sets of
values'~\cite{Mortazavi:1996}, the main requirement that arises, at
least as far as telecommunication is concerned, is that \emph{types
must contain at least one finite value}. The finiteness condition
applies no matter what the encoding rules are, but it arises from the
fact that the current standard encoding rules cannot handle infinite
values, i.e. recursive values. The existence requirement is the main
aspect of the validation of \ASN specifications, because it is
directly related to the values that can be encoded, independently from
the encoding rules. It is also the most difficult, because it cannot
be dealt with by syntactic means only, i.e. it requires computations
or, more generally, inductions on mathematical objects.

In mainstream programming languages, typechecking (i.e. checking
whether a value complies with its declared type) is enough as far as
validation is concerned and the sets of values corresponding to types
are not considered explicitly. In \ASN, the great deal lies in an
involved notion of subtyping. It consists of constraints upon
recursive types, which restrict their sets of values in a
set-theoretic manner (e.g. by intersection) or in a structural way
(e.g. by requiring the omission of some fields in a record-like
construct).

In this article we introduce a set-theoretic interpretation which maps
types and subtypes into sets of values, by means of an algorithm, and
also allows a constructive decision procedure for the `at least one
finite value' property --- thus fully validates the \mbox{X.680}
specifications (except tagging). The algorithm deals with the entire
\mbox{X.680} standard. It brings new insights to obscure areas of the
standard (like type compatibility in assignments or recursion),
which, while not often used by the protocol designer, are unavoidable
for the tool implementor concerned with full conformance. The
algorithm is twofold: a collecting algorithm, which extracts some
constraints, and a solving procedure. Some of the constraints are set
constraints~\cite{Aiken:1994,PacholskiPodelski:1997}, so the solving
procedure relies upon Aiken and Wimmers'
algorithm~\cite{AikenWimmers:1992}. Set constraints are inclusions
between expressions interpreted over the domain of sets of trees.

First, in section~\ref{presentation}, we briefly introduce \ASN (the
subtyping constraints are presented step by step in
section~\ref{constraints_from_subtypes}). In section~\ref{core}, we
define a strict subset of \mbox{X.680} which has fewer ambiguities and
syntactic constructs; it also allows a much simpler presentation of
the collection algorithm of sections~\ref{constraints_from_types}
and~\ref{constraints_from_subtypes}. We give a procedure for rewriting
every \mbox{X.680} specification into \core. We provide, in
section~\ref{well_founded_types}, a formal predicate for the `at least
one finite value' property on (unconstrained) types in \core. This
property is a prerequisite for handling subtypes. Next, in
section~\ref{constraints}, we introduce our constraints. The
collecting algorithm is introduced in two steps: first, the collection
from types is given in section~\ref{constraints_from_types}; second,
the collection from proper subtypes appears in
section~\ref{constraints_from_subtypes}. We finally explain the
resolution process in section~\ref{full_collection_and_solving}.
