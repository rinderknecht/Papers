%%-*-latex-*-

\begin{thm}[Soundness]\hfill
\label{es1:soundness}
\begin{center}
If \(\smat{h}{p} \twoheadrightarrow \sigma\) 
then \(\subst{p}{\sigma} \sqsubseteq h\).
\end{center}
\end{thm}

\noindent\textsc{Proof~\ref{es1:soundness}} (Soundness). Let
\(\ind{U}(p, f, \sigma)\) be the proposition `\emph{If}
\(\smj{f}{p}{\sigma}\) \emph{then} \(\subst{p}{\sigma} \sqsubseteq
f\).'  So \(\ind{U}(p, [h], \sigma)\) is equivalent to the
soundness. Let
\begin{equation}
  \smj{f}{p}{\sigma} \label{es1:sound:head}
\end{equation}
(otherwise the theorem is trivially true). This means that there
exists a pattern matching derivation \(\Delta\) whose conclusion is
\(\smj{f}{p}{\sigma}\). This derivation is a tree; we hence can reason
by structural induction on it, i.e., we assume that \(\ind{U}\) holds
for the premises of the last rule in \(\Delta\) (this is the
\emph{induction hypothesis}) and then prove that \(\ind{U}\) holds for
\(\smj{f}{p}{\sigma}\). We proceed case by case on the kind of rule
that can end \(\Delta\).
\begin{enumerate}

  \item Case where \(\Delta\) ends with \textsf{END}.\\ We have
    \(p = \el\), \(f = \el\) and \(\sigma =
    \sigma_\varnothing\). Therefore \(\subst{p}{\sigma} =
    \subst{\el}{\sigma} \eqn{1} \el \sqsubseteq \el = f\).  Thus
    \(\ind{U}(\el, \el, \sigma_\varnothing)\) holds.

  \item Case where \textsf{ELIM} ends \(\Delta\).
    \begin{mathpar}
      \inferrule*[right=\text{\sf ELIM}]
         {\inferrule*
            {\inferrule*[vdots=1.5em]{}{ }}
            {\smj{f'}{p'}{\sigma}}}
         {\smj{\cons{l}{f'}}{\cons{l}{p'}}{\sigma}}
    \end{mathpar}
    where, since we assumed \eqref{es1:sound:head},
    \begin{enumerate}

      \item \label{es1:sound:24} \(p \triangleq
        \cons{l}{p'}\),

      \item \label{es1:sound:23} \(f \triangleq \cons{l}{f'}\).

    \end{enumerate}
    Let us assume that the induction hypothesis holds for the premise
    of \textsf{ELIM}, that is to say, \(\ind{U}(p', f',
    \sigma)\) holds. Thus
    \begin{gather}
      \subst{p'}{\sigma} \sqsubseteq f'. \label{es1:sound:02}
    \end{gather}
    Besides, we have
    \begin{align}
      \subst{p}{\sigma}
      &= \subst{\cons{l}{p'}}{\sigma}
       \eqn{3} \cons{l}{\subst{p'}{\sigma}}
       \sqsubseteq \cons{l}{f'} = f.
       &\text{by~\ref{es1:sound:24}, \eqref{es1:sound:02}, \textsf{EQ},
        \ref{es1:sound:23}}\label{es1:sound:09}      
    \end{align}
    As a conclusion, the induction hypothesis and
    \eqref{es1:sound:head} imply \eqref{es1:sound:09} in this case,
    i.e., \(\ind{U}(p, f, \sigma)\) holds.

  \item Case where \textsf{BIND}\(_1\) ends \(\Delta\).
    \begin{mathpar}
        \inferrule
          {\inferrule*
            {\inferrule*[vdots=1.5em]{}{ }}
            {\smj{f''}{p''}{\sigma'}}}
          {\smj{\cons{c(f_1), l}{\!f''}}%
               {\cons{\meta{x_{[c]}}, l}{p''}}%
               {\sigma' \oplus x \mapsto c(f_1)}}
      \end{mathpar}
      where, since we assumed \eqref{es1:sound:head},
      \begin{enumerate}

        \item \(t \triangleq c(f_1)\),

        \item \label{es1:sound:sigma'} \(\sigma' \subseteq \sigma'
          \oplus x \mapsto t\),

        \item \label{bind2_p} \(p \triangleq
          \cons{\meta{x_{[c]}}, l}{p''}\),

        \item \label{es1:sound:1} \(f \triangleq \cons{t, l}{f''}\),


         \item \label{es1:sound:sigma} \(\sigma \triangleq \sigma'
           \oplus x \mapsto t\).

      \end{enumerate}
      Let us assume that the induction hypothesis holds for the
      premise of \textsf{BIND}\(_1\), i.e., 
      \(\ind{U}(p'', f'', \sigma')\) holds:
      \begin{gather}
        \subst{p''}{\sigma'} \sqsubseteq f''. \label{es1:sound:M}
      \end{gather}
      From \ref{es1:sound:sigma'} and lemma \ref{minimality}, we draw
      \begin{align}
           \subst{p''}{\sigma'} 
        &= \subst{p''}{(\sigma' \oplus x \mapsto
             t)} = \subst{p''}{\sigma} \sqsubseteq f''.
        &\text{by} \,\; \text{\ref{es1:sound:sigma}} \,\; \text{and}
           \,\; \eqref{es1:sound:M}\label{es1:sound:00}
      \end{align}
      We have
      \begin{align}
           \sigma(x) 
        &= (\sigma' \oplus x \mapsto t)(x) =  t
        &\text{by} \,\; \text{\ref{es1:sound:sigma}} \,\; \text{and}
           \,\; \eqref{model:oplus}\label{es1:sound:N}. 
      \end{align}
      Besides, we have the following equalities:
      \begin{align}
          \subst{p}{\sigma} 
        &= \subst{\cons{\meta{x_{[c]}},l}{p''}}{\sigma}\notag
          \eqn{2} \cons{\sigma(x)}{\subst{\cons{l}{p''}}{\sigma}}
        &\text{by} \,\; \text{\ref{bind2_p}} \,\; \text{and} \,\;
        \text{Fig.~\ref{es1_subst_def}}\notag\\
        &\eqn{3} \cons{\sigma(x),l}{\subst{p''}{\sigma}}
        = \cons{t,l}{\subst{p''}{\sigma}}.
        &\text{Fig.~\ref{es1_subst_def}} \,\; \text{and} \,\;
        \eqref{es1:sound:N}\label{es1:sound:13}
      \end{align}
      Closed\hyp{}tree matching \eqref{es1:sound:00} and the derivation
      \begin{mathpar}
        \inferrule*[right=\text{\sf{EQ}}]
          {t \in {\cal H}
           \and
           \inferrule*[right=\text{\sf{EQ}}]
              {l \in {\cal H}
               \and
               \subst{p''}{\sigma} \sqsubseteq f''}
              {\cons{l}{\subst{p''}{\sigma}}
               \sqsubseteq \cons{l}{f''}}}
          {\cons{t, l}{\subst{p''}{\sigma}}
           \sqsubseteq \cons{t, l}{f''}}
      \end{mathpar}
      imply
      \begin{align}
           \subst{p}{\sigma} = \cons{t, l}{\subst{p''}{\sigma}} 
        &\sqsubseteq
           \cons{t, l}{f''} = f.           
        &\text{by} \,\; \eqref{es1:sound:13} \,\; 
        \text{and} \,\; \text{\ref{es1:sound:1}}\label{es1:sound:O}
      \end{align}
      As a conclusion, the induction hypothesis and
      \eqref{es1:sound:head} imply \eqref{es1:sound:O} in this case,
      i.e., \(\ind{U}(\cons{\meta{x_{[c]}}}{p'}, f, \sigma)\).

  \item Case where \textsf{BIND}\(_2\) ends \(\Delta\).
    \begin{mathpar}
      \inferrule
        {\inferrule*
           {\inferrule*[vdots=1.5em]{}{ }}
           {\smj{\cons{t_2}{f'}}{p'}{\sigma'}}}
        {\smj{\cons{c(f_1), t_2}{f'}}%
           {\cons{\meta{x_{[c]}}}{p'}}%
           {\sigma' \oplus x \mapsto c(f_1)}}
    \end{mathpar}
    where, since we assumed \eqref{es1:sound:head}, 
    \begin{enumerate}

      \item \(t_1 \triangleq c(f_1)\),

      \item \label{bind1_def_p} \(p \triangleq
        \cons{\meta{x_{[c]}}}{p'}\),

      \item \label{bind1_def_f} \(f \triangleq \cons{t_1, t_2}{f'}\),

      \item \label{es1:sound:BIND2:sigma} \(\sigma \triangleq \sigma'
        \oplus x \mapsto t_1\),

      \item \label{bind1_def_x} \(\sigma' \subseteq \sigma' \oplus x
        \mapsto t_1\).

    \end{enumerate}
    Let us assume that the induction hypothesis holds for the
    premise of \textsf{BIND}\(_2\), i.e.,
    \(\ind{U}(p', \cons{t_2}{f'}, \sigma')\):
    \begin{gather}
      \subst{p'}{\sigma'} \sqsubseteq \cons{t_2}{f'}. \label{es1:sound:05}
    \end{gather}
    From \ref{bind1_def_x} and lemma \ref{minimality}, we draw
    \begin{align}
         \subst{p'}{\sigma'}
      &= \subst{p'}{(\sigma' \oplus x \mapsto t_1)}
       = \subst{p'}{\sigma} \sqsubseteq \cons{t_2}{f'}.
      &\text{by} \,\;
         \text{\ref{es1:sound:BIND2:sigma}} \,\; \text{and} \,\;
         \eqref{es1:sound:05}\label{es1:sound:19}
    \end{align}
    We have
    \begin{align}
          \sigma(x) 
      &= (\sigma' \oplus x \mapsto t_1)(x) = t_1.
      &\text{by} \,\;
       \text{\ref{es1:sound:BIND2:sigma}} \,\; \text{and} \,\;
      \eqref{model:oplus}\label{es1:sound:P}
    \end{align}
    Furthermore,
    \begin{align}
         \subst{p}{\sigma} 
      &= \subst{\cons{\meta{x_{[c]}}}{p'}}{\sigma}\notag
      \eqn{2} \cons{\sigma(x)}{\subst{p'}{\sigma}}
      &\text{by} \,\; \text{\ref{bind1_def_p}} \,\;
      \text{and Fig.~\ref{es1_subst_def}}\notag\\
      &= \cons{t_1}{\subst{p'}{\sigma}}
      \sqsubseteq \cons{t_1, t_2}{f'} = f.
      &\text{by} \,\; \eqref{es1:sound:P},
      \eqref{es1:sound:19}, \textsf{EQ},
      \text{\ref{bind1_def_f}}\label{es1:sound:Q}
    \end{align}
    As a conclusion, the induction hypothesis and
    \eqref{es1:sound:head} imply \eqref{es1:sound:Q} in this case,
    i.e., \(\ind{U}(\cons{\meta{x_{[c]}}}{p'}, f, \sigma)\).

  \item Case where \textsf{BIND}\(_3\) ends \(\Delta\).
    \begin{mathpar}
      \inferrule*[right=\;\text{\sf BIND}\(_3\)]
         {}
         {\smj{[c(f)]}{[\meta{x_{[c]}}]}{\{x \mapsto c(f)\}}}
    \end{mathpar}
    where, since we assumed \eqref{es1:sound:head},
    \begin{enumerate}

      \item \(t \triangleq c(f)\),

      \item \label{es1:sound:15} \(p \triangleq [\meta{x_{[c]}}]\),

      \item \label{es1:sound:16} \(f \triangleq [t]\),

      \item \label{es1:sound:14} \(\sigma \triangleq \{x \mapsto t\}\).

    \end{enumerate}
    Because \textsf{BIND}\(_3\) is an axiom, we must prove
    \(\ind{U}([\meta{x_{[c]}}], [t], \{x \mapsto t\})\) without
    relying on the induction principle:
    \begin{align}
         \subst{p}{\sigma}
      &= \subst{[\meta{x_{[c]}}]}{\sigma}
      \eqn{2} \cons{\sigma(x)}{\subst{\el}{\sigma}}
      \eqn{1} \cons{\sigma(x)}{\el}
      &\text{cf. Fig.~\ref{es1_subst_def}} \,\; \text{and} \,\;
      \text{\ref{es1:sound:15}}\notag\\
      &\triangleq [\sigma(x)]
      = [t].
      & \text{by} \,\; \ref{es1:sound:14} \,\; \text{and} \,\;
        \eqref{model:oplus}\label{es1:sound:R}
    \end{align}
    Since we also have the derivation
    \begin{mathpar}
      \inferrule*[right=\text{\sf EQ}]
        {\inferrule*[right=\text{\sf EMP}]
          { }
          {\el \sqsubseteq \el}}
        {\cons{t}{\el} \sqsubseteq \cons{t}{\el}}
    \end{mathpar}
    we know that \([t] \sqsubseteq [t]\), which, in conjunction with
    \eqref{es1:sound:R}, implies
    \begin{align}
        \subst{p}{\sigma}
      &\sqsubseteq [t]  = f.
      & \text{by} \,\; \text{\ref{es1:sound:16}}\label{es1:sound:17}
    \end{align}
    As a conclusion, the induction hypothesis and
    \eqref{es1:sound:head} imply \eqref{es1:sound:17}, that is,
    \(\ind{U}([\meta{x_{[c]}}], [t], \{x \mapsto t\})\) holds.

  \item Case where \(\Delta\) ends with \textsf{UNPAR}\(_1\).
    \begin{mathpar}
      \inferrule*[right=\text{\sf UNPAR}\(_1\)]
        {\inferrule*
          {\inferrule*[vdots=1.5em]{}{(\Delta_1)}}
          {\smj{f_1}{p_1}{\sigma_1}}
         \and
         \inferrule*
          {\inferrule*[vdots=1.5em]{}{(\Delta_2)}}
          {\smj{f_2}{p_2}{\sigma_2}}}
        {\smj{\cons{c(f_1)\!}{\!f_2}}%
            {\cons{\pat{p_1}}{\!p_2}}%
            {\sigma_1 \oplus \sigma_2}
        }
    \end{mathpar}
    where, since we assumed \eqref{es1:sound:head},
    \begin{enumerate}
      
      \item \label{es1:sound:22} \(p \triangleq
        \cons{\pat{p_1}}{p_2}\),

      \item \label{es1:sound:21} \(f \triangleq \cons{c(f_1)}{f_2}\),

      \item \label{es1:sound:26} \(\sigma_1 \subseteq \sigma_1 \oplus
        \sigma_2\).
          
    \end{enumerate}
    The derivations \(\Delta_1\) and \(\Delta_2\) are
    sub\hyp{}derivations of \(\Delta\), therefore the induction
    hypothesis holds for their conclusions, i.e.,
    \(\ind{U}(p_1, f_1, \sigma_1)\) is true:
    \begin{gather}
      \subst{p_1}{\sigma_1} \sqsubseteq f_1 \label{es1:sound:01}
    \end{gather}
    and \(\ind{U}(p_2, f_2, \sigma_2)\) is true as well:
    \begin{gather}
      \subst{p_2}{\sigma_2} \sqsubseteq f_2. \label{es1:sound:F}
    \end{gather}
    Directly from the definition \eqref{model:oplus:ext}, it comes
    \begin{gather}
      \sigma_2 \subseteq \sigma_1 \oplus \sigma_2. \label{es1:sound:27}
    \end{gather}
    It follows
    \begin{align}
       \subst{p_1}{\sigma_1}
      &= \subst{p_1}{(\sigma_1 \oplus \sigma_2)}
      &\text{by} \,\; \text{\ref{es1:sound:26}} \,\; \text{and} \,\;
        \text{Lem.\ref{minimality}}\label{es1:sound:28}\\
        \subst{p_2}{\sigma_2}
      &= \subst{p_2}{(\sigma_1 \oplus \sigma_2)}
      &\text{by} \,\; \eqref{es1:sound:27} \,\; \text{and} \,\;
        \text{Lem.\ref{minimality}}.\label{es1:sound:29}
    \end{align}
    Let \(\sigma \triangleq \sigma_1 \oplus \sigma_2\). Besides, we
    have
    \begin{align}
         \subst{p}{\sigma}
      &= \subst{\cons{\pat{p_1}}{p_2}}{\sigma}
      \eqn{4}
       \cons{\pat{\subst{p_1}{\sigma}}}%
            {\subst{p_2}{\sigma}}
      &\text{by~\ref{es1:sound:22} and
              Fig.~\ref{es1_subst_def}}\notag\\
      &= \cons{\pat{\subst{p_1}{\sigma_1}}}%
              {\subst{p_2}{\sigma}}
      = \cons{\pat{\subst{p_1}{\sigma_1}}}%
              {\subst{p_2}{\sigma_2}}
      &\text{by} \,\; \eqref{es1:sound:28} 
       \,\; \text{and} \,\; \eqref{es1:sound:29}\notag\\
      &\sqsubseteq \cons{c(f_1)}{f_2} = f.
      &\text{by} \,\; \eqref{es1:sound:01}, \eqref{es1:sound:F},
       \textsf{PAT}, \text{\ref{es1:sound:21}}\label{es1:sound:G}
    \end{align}
    As a conclusion, the induction hypothesis and
    \eqref{es1:sound:head} imply \eqref{es1:sound:G},
    that is, \(\ind{U}(\cons{\pat{p_1}}{p_2}, f,
    \sigma)\) holds.

  \item \label{case_l_UNPAR2} Case where \textsf{UNPAR}\(_2\) ends
    \(\Delta\).
    \begin{mathpar}
      \inferrule*[right=\text{\sf UNPAR}\(_2\)]
         {\inferrule*
            {\inferrule*[vdots=1.5em]{}{ }}
            {\smj{f_1 \cdot f_2}{p}{\sigma}}}
         {\smj{\cons{c(f_2)\!}{\!f_2}}{p}{\sigma}}
    \end{mathpar}
    where, since we assumed \eqref{es1:sound:head},
    \begin{enumerate}

      \item \label{es1:sound:25} \(f \triangleq \cons{c(f_1)}{f_2}\).

    \end{enumerate}
    Let us assume that the induction hypothesis holds for the premise
    of \textsf{UNPAR}\(_2\), i.e., \(\ind{U}(p, f_1 \cdot
    f_2, \sigma)\) is true. Therefore
    \begin{align}
         \subst{p}{\sigma} 
       &\sqsubseteq f_1 \cdot f_2 \sqsubseteq \cons{c(f_1)}{f_2} = f.
       &\text{by} \,\; \textsf{SUB} \,
       (\text{Fig.~\ref{es1_tree_matching_def}}),
         \text{\ref{es1:sound:25}}\label{es1:sound:I}
    \end{align}
    As a conclusion, the induction hypothesis and
    \eqref{es1:sound:head} imply \eqref{es1:sound:I} in this case,
    i.e., \(\ind{U}(p, f, \sigma)\). (The structure of the pattern
    \(p\) is irrelevant here.)\hfill \(\Box\)

\end{enumerate}
