%%-*-latex-*-

\section{Implementation}
\label{implementation}

Avoiding the parsing of patterns dramatically simplifies the
implementation of pattern matching, as can be seen from the following
two implementations. Indeed, we saw that extending a programming
language grammar is difficult to implement in most existing tools. In
contrast, unparsing an AST is trivial to implement: it consists of
just printing the AST. In most cases, parser\hyp{}based tools already
include functions to pretty\hyp{}print the AST, for debugging
reasons. Moreover, AST pretty\hyp{}printers can be generated
automatically based on the grammar of a language. It is
straightforward to adapt an existing pretty\hyp{}printer to do
unparsing on demand (see below for an example).

Unparsed patterns were first implemented by the second author in the
context of a lightweight checking compiler called
\MyGCC,\footnote{\url{http://mygcc.free.fr}} which is an extensible
version of the \GCC compiler, able to perform user\hyp{}defined checks
on \Clang, \cpp, and \Ada code. Checks are expressed by defining
incorrect sequences of program operations, where each program
operation is described as an unparsed pattern or a disjunction of
unparsed patterns. The implementation of pattern matching within
\MyGCC accounts for only about 600 lines of new \Clang code, plus
about 250 lines of code adapting the existing tree pretty\hyp{}printer
of \GCC to perform unparsing on demand. The existing
pretty\hyp{}printer dumped the unparsed representation of a whole AST
in a debug file. We added a flag \verb|--lazy_mode| to switch between
the standard dumping behaviour and the new on\hyp{}demand
behaviour. When in on\hyp{}demand mode, the pretty\hyp{}printer
returns for a given AST the list of its direct children (either trees
or lexemes), instead of dumping the AST entirely to a file. This
modification was straightforward.

It is important to note that even though three different input
languages can be checked, every single line of the patch is
language\hyp{}independent. As a proof for that, the patched \GCC
compiler restricted to the \Clang front\hyp{}end was initially tested
only on \Clang code, as reported previously~\cite{ppdp}; subsequently,
by just recompiling \GCC with all the front\hyp{}ends enabled it
became possible to check \cpp{} and \Ada programs. Part of this
extreme language independence comes from the fact that all the three
front\hyp{}ends generate intermediate code in a language called
\Gimple, and the dumper for different languages shared a common
infrastructure based on \Gimple, which was modified just
once. However, this is not required for the pattern matching
framework. The only language\hyp{}dependent aspects used by the
matcher were already present in \GCC: a parser for each language, a
conversion from language\hyp{}specific ASTs to \Gimple ASTs, and a
dumping function of \Gimple ASTs for each language, sharing some
common infrastructure. Our patch of the dumper (briefly sketched
above) concerned only the common (or language\hyp{}independent)
infrastructure of the dumper. The \cpp{} and \Ada pattern matching
became possible by this combination of the language\hyp{}independent
matching algorithm with the language\hyp{}dependent unparsers already
present in \GCC.  For instance, a pattern such as \texttt{\%\_ =
operator[](\%x,\%y)} successfully matches any \cpp{} assignment of the
form \texttt{\%\_ = \%x[\%y]} in which the indexing operator has been
redefined, because the corresponding \Gimple ASTs are dumped using the
\texttt{operator[]} syntax captured by the pattern.  Had the common
infrastructure of the language\hyp{}specific dumpers not existed, each
of the dumpers should have been modified to make them unparse level by
level.

The \textit{ES}(1) pattern matching algorithm was also implemented by
the second author and is distributed as a freely available, standalone
prototype called
\Matchbox.\footnote{\url{http://mypatterns.free.fr/unparsed}} This
very simple prototype, consisting of 500 lines of \Clang code, takes a
parse tree represented in a \textsf{Lisp}\hyp{}like notation and an
unparsed pattern, prints a complete trace of all the rules applied,
and finally reports a successful match or a failure. The prototype may
already be used to reproduce all the examples in this paper (using the
\textit{ES}(1) algorithm). The aim of \Matchbox is to evolve into a
standalone library for unparsed\hyp{}pattern matching, which can be
linked from any parser\hyp{}based tool.

\subsection{Examples}

Here are some examples of pattern matching by \Matchbox. First, this
is the example shown in Figure~\ref{fig:match_ex}. Note how the parse
tree is written using parentheses and also how the concrete syntax
lexemes are quoted.
{\small
\begin{verbatim}
match "assign(var('a') '='
       sub(sub(var('a') '-' mul(var('b')'*'var('c'))) '-' var('d')))"
      "%x = %y - %z"
ok, sigma={x<-var('a') y<-sub(var('a')'-'mul(var('b')'*'var('c')))
           z<-var('d')}
\end{verbatim}
}
\noindent Now the same pattern but with a simplified parse tree in
which the constructors were omitted:
{\small
\begin{verbatim}
match "(('a')'='((('a')'-'(('b')'*'('c')))'-'('d')))"
      "%x = %y - %z"
ok, sigma={x<-('a') y<-(('a')'-'(('b')'*'('c'))) z<-('d')}
\end{verbatim}
}
Note the meta\-parentheses needed to avoid failure due to the
incompleteness of \textit{ES}(1):
{\small
\begin{verbatim}
match "(('a')'='((('a')'-'(('b')'*'('c')))'-'('d')))"
      "%x = %(%(%y - %z%) - %t%)"
ok, sigma = {t<-('d') x<-('a') y<-('a') z<-(('b')'*'('c'))}
\end{verbatim}
}
Consider the unparsed pattern ``\texttt{\%q \%t \%x;}'' against the
parse tree in Figure~\ref{fig:const_static}: the first meta\-variable,
which intends to capture the qualificators, actually binds the
qualified type as a whole, leading to a later failure:
{\small
\begin{verbatim}
match "decl(qtype(quals('const' 'static') type('int')) id('x') ';')"
      "%q %t %x;"
failed
\end{verbatim}
}
%% In order to avoid that, the unparsed pattern must be
%% meta\-parenthesized as \texttt{\%(\%(\%q \%t\%) \%x;\%)}.
%% {\small
%% \begin{verbatim}
%% match "decl(qtype(quals('const' 'static') type('int')) id('x') ';')"
%%       "%(%(%q %t%) %x;%)"
%% ok, sigma = {q<-quals('const' 'static') t<-type('int') x<-id('x')}
%% \end{verbatim}
%% }
In order to avoid that, the variable \(q\) must be typed as
\texttt{\%<quals>q}:
{\small
\begin{verbatim}
match "decl(qtype(quals('const' 'static') type('int')) id('x') ';')" 
      "%<quals>q %t %x;"
ok, sigma = {q<-quals('const' 'static') t<-type( 'int' ) x<-id( 'x' )}
\end{verbatim}
}
Mixing parentheses and meta\-parentheses with the parse
tree from ``\texttt{case v in 1) exit;; esac}'':
{\small
\begin{verbatim}
match "('case'('v')'in'(('1')')' ('exit')';;') 'esac')"
      "case %x in %y) %z;; esac"
ok, sigma = {x<-('v') y<-('1') z<-('exit')}
\end{verbatim}
}

\subsection{Assessment}

The technique of pattern matching based on unparsed patterns is
completely language independent. Patterns are simply strings, which
exist as a base type in any programming language. No pattern parser is
required, which means also that the implementation of the pattern
matcher is language independent. The only part tied to a specific
language is the unparser, but they can be automatically generated from
any grammar. 

Unparsed patterns are an unrestricted solution to match programs,
because they allow pattern variables to stand for any program
sub\-construct or sub\-expression---in fact, for any subtree in the
AST.  For instance, in the case of ASTs for the \Clang language,
pattern variables may stand for function names, structure field names,
or types, which are all subtrees. This makes it possible to naturally
express patterns such as \texttt{\%x = \%f (\%y)} catching function
names or expressions in pattern variable \texttt{f}, \texttt{static
  \%t a} catching the type of variable \texttt{a} in pattern variable
\texttt{t}, etc. As opposed to this unrestricted use, many
pattern\hyp{}based tools implement parsed patterns only for a few
common program constructs. By being an easy\hyp{}to\hyp{}implement and
a general solution, unparsed patterns have the potential to enable
widespread use of concrete syntax code pattern matching within many
parser\hyp{}based tools. There are two quite different ways to use
this enabling technology.

\subsection{Usages}

First, patterns can be used in the implementation of the tool itself,
to simplify various code analyses and transformations. For instance,
an analysis or optimisation pass could use pattern matching to look
for statements matching a given pattern, e.g., increment assignments.
This can simplify the implementation especially if the tool is written
in a language that does not provide any support for pattern matching,
like \Clang or \Java. However, as shown in the introduction, unparsed
patterns may be useful even when the tool is written in a language
that does provide a form of tree pattern matching (e.g., \ML): some
ASTs patterns ---especially verbose patterns--- are more easily
expressed in native syntax than in tree syntax. Thus, at least,
unparsed patterns offer an alternative for the programmer to freely
chose the more convenient form for each pattern.

The pattern matcher implemented in \MyGCC offers two programming
interfaces for internal usage.

The first interface implements unparsed patterns as used above, in
which pattern variables are represented by the \texttt{\%} escaping
character, followed by a variable, e.g, \texttt{\%x}, \texttt{\%y},
etc. Pattern variables are global variables. This interface is
available using the \Clang function:
\begin{verbatim}
bool tree_match(tree t, const char *format)
\end{verbatim}
which takes an AST \texttt{t} and a pattern format with named
pattern variables and returns true if the matching succeeded, and
false otherwise. 

The second interface implements unparsed patterns with unnamed pattern
variables, represented by the \texttt{\%} escaping character followed
by the type of the variable: \texttt{t} for a subtree, \texttt{c} for
a character, \texttt{d} for a number, e.g.,\texttt{\%t = \%t + \%t}
(currently only tree variables are implemented). This interface is
similar to the \Clang library functions performing I/O such as
\texttt{scanf}, and is available through the \Clang function:
\begin{verbatim}
bool tree_scanf(tree t,const char *pat,...)
\end{verbatim}
which takes an AST \texttt{t}, a pattern \texttt{pat} with unnamed
pattern variables, and a series of variable addresses corresponding to
the pattern variables occurring in format, and returns true if the
matching succeeded, and false otherwise. For example:
\begin{verbatim}
tree t, x, y;
succeeded=tree_scanf(t,"%t=%t+%t",&x,&x,&y);
\end{verbatim}
matches \texttt{t} if it represents an increment, that is, if the
variable on the left\hyp{}hand side is the same as the first term of
the addition. In case of successful match, it instantiates variable
\texttt{x} to the variable or expression being incremented, and
\texttt{y} to the increment.

Patterns can also be used to make the tool extensible by associating
user\hyp{}defined behaviour to some patterns. For example, in \MyGCC
users can define sequences of operations that constitute bugs. As a
particular case, \MyGCC allows searching all the code for a given
pattern, by using a new option \verb|--tree-check|. For instance, the
following command will issue a warning for all the read statements
whose result is unused, but otherwise compile the file as usual:
\begin{verbatim}
gcc --tree-check="read(%x, %y, %z);" foo.c   
\end{verbatim}
Indeed, because of the final semi\hyp{}colon, this pattern only
matches complete statements and not simple expressions consisting of a
call to the function \texttt{read}. Therefore, statements using the
value of the call will not be reported (which is the intended
behaviour).
