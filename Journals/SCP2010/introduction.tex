%%-*-latex-*-

\section{Introduction}

Pattern matching of source code is very useful for analysing and
transforming programs, as in compilers, interpreters, tools for legacy
program understanding, code inspectors, refactoring tools, model
checkers, code translators etc. Source code matching is especially
useful for building extensible versions of these tools with
user\hyp{}defined behaviour~\cite{mc,mj,mops,blast}. As the problem of
tree matching has been extensively studied, the problem of source code
matching has usually been reduced to tree matching, following two
different ways.

\noindent\textbf{Tree Patterns.} In the first approach, code patterns
are written as trees, using a domain\hyp{}specific notation to
describe an abstract syntax tree (AST). This approach based on tree
patterns has been used for a long time, either by using pattern
matching support available in the implementation language, for
instance, for tools written in \ML, or otherwise by explicitly
implementing a tree pattern matching mechanism, for instance in
inspection tools such as \textsf{tawk}~\cite{tawk} or
\Scruple~\cite{scruple} or in model checking tools such as
\textsf{MOPS}~\cite{mops}. More recently, some extensible code
inspectors such as
\textsf{PMD}~\footnote{\url{http://pmd.sourceforge.net/}} represent
ASTs in \XML (which can be considered, in general, as a standardised
formal notation for trees). This allows us to write tree patterns in
standardised languages such as \XPath (and the languages embedding it,
like \XQuery and \XSLT), and thus reuse the existing tree pattern
matchers. The main advantage of expressing patterns as trees is that
the implementation of pattern matching is simple, because any
appropriate tree matching algorithm can be directly used on this
representation. However, an important shortcoming of this approach is
that programmers writing patterns should be aware of the internal AST
representation of programs, and also of a specific notation for it. As
a motivating example, let us consider a very simple user\hyp{}defined
inspection rule over \Clang programs that searches for code fragments
resetting all the elements in an array to zero, such as
\verb|for(i=0; i<100; ++i) a[i]=0;|. When such initialisation code is
found, the code reviewer may suggest that the same operation would be
implemented more efficiently using the \texttt{bzero()} standard
library function. Alternatively, the same rule might be predefined in
a compiler in order to recognise such initialisations and
automatically implement them more efficiently using the
\texttt{bzero()} function. Depending on the AST representation of the
\Clang program in a particular tool, the tree pattern corresponding to
the example above would typically be expressed as follows: {\small
\begin{verbatim}
for_stmt(assign_expr(X,N), less_expr(X,M), preincr_expr(X),
         expr_stmt(assign_expr(array_expr(Y,X), int_literal(0))))
\end{verbatim}
}
\noindent The name of the array, its bound and its index have been
abstracted as variables of the pattern, called
\emph{meta\-variables}. As can be seen in the above tree pattern, the
programmer writing the inspection rule must be aware of both the AST
structure (for instance, that an assignment in \Clang is an expression
embedded in a statement) and of a specific notation for it (for
instance, that the assignment operator is called
\texttt{assign\_expr}, that the \texttt{for\_stmt} operator takes four
arguments in a given order, etc.). Writing the same pattern in a
language such as \XPath does not solve any of these issues, and the
pattern becomes even more verbose.

\noindent\textbf{Concrete Syntax Patterns.} In the second approach to
source code matching, code patterns are expressed using the native
syntax (i.e., the concrete syntax) of the subject programming
language, augmented with meta\-variables. Then patterns are parsed to
trees before being matched with the program AST. In this approach, the
pattern to be searched can be written much more naturally and
concisely as shown in Figure~\ref{intro:concrete_pattern} (where
meta\-variables are escaped by the special character \texttt{\%}).
\begin{figure}
\framebox[\columnwidth]{%
\texttt{for(\%x=0; \%x<\%n; ++\%x) \%y[\%x]=0;}
}%
\caption{A concrete syntax pattern\label{intro:concrete_pattern}}
\end{figure}
\noindent This second approach has been used, for instance, in several
extensible model checkers~\cite{cecil,mc,mj,blast}, as well as
extensible tools for legacy program understanding and
transformation~\cite{native,behavioral}. The main advantage of
concrete syntax patterns is that they are easy to write and read back
by any programmer, without knowledge of the AST
representation. However, as far as the implementation is concerned,
parsing the pattern requires a modified parser of the programming
language defining the application, extended to
\begin{itemize}

  \item allow meta\-variables, which, by definition, are variables
    that are never found in the source code;

  \item parse patterns that represent arbitrary program
    \emph{fragments} (statements, expressions, declarations), whereas
    the grammar specifies fixed syntactical subsets.
\end{itemize}
Extending the parser of a programming language in this way is a
challenging task, even if some frameworks automate the addition of the
extra grammar productions~\cite{refine}, because it entails usually
many supplementary \emph{ad hoc} choices in the parser strategy, often
called \emph{conflict resolutions}. These conflicts may correspond to
real ambiguities in the extended language (for example, in \Clang, the
pattern \texttt{f(\%x);} may represent either a function call or a
function declaration with an implicit return type of \texttt{int}), or
reveal a limited knowledge of the lexical context in the parsing
algorithm. Indeed, existing parsers commonly use limited lexical
lookahead, as those analysing LALR(1) or LL(1) grammars. In
LALR(1)\hyp{}based parsers, the conflicts present themselves as
dilemmas (reduce or reduce, shift or reduce) but choosing one option
rather than the other cannot always preserve the syntactic coverage
(i.e., some valid programs may be rejected), hence rewriting the
underlying grammar becomes a necessity. Systematically adding pattern
productions to an existing LALR(1) grammar usually creates a large
number of such conflicts, partly because patterns may represent any
program fragment instead of whole programs (like the \texttt{f(\%x);}
example above), partly because a pattern variable in a specific
context may stand for different sub\hyp{}constructs (for instance, in
the \Clang or \Java pattern \texttt{f(\%x)} variable \texttt{x} may
stand either for a unique argument or for a list of arguments). This
latter kind of conflicts arises for every production in the original
grammar allowing a non\hyp{}terminal to be derived into a single other
non\hyp{}terminal (in the example, an argument list non\hyp{}terminal
that may reduce to a single argument non\hyp{}terminal). In our
extensive experience with adding patterns to LALR(1) grammars for
re\hyp{}engineering projects, this single task takes at least several
man\hyp{}weeks to well\hyp{}trained developers when applied on
real\hyp{}size grammars, especially those of legacy languages designed
before grammar models have gained a wide adoption in the
industry. Furthermore, rewriting the grammar to cope with all these
pattern\hyp{}related conflicts leads to special pattern syntaxes (such
as \texttt{\#stmt f(\#list \%x);} in the previous example) that make
the patterns look less similar to native code. This is the reason why
most of the existing tools following this approach allow only
restricted forms of concrete syntax patterns, described by a limited
pattern grammar (for example, matching only assignments and function
calls~\cite{blast-ql}).

In theory, GLR (for Generalised LR) parsers~\cite{glr} can deal with
both kinds of conflicts, because they compute all possible parses:
conflicts due to a limited lookahead are solved later during the
parsing and conflicts due to the ambiguity of the grammar itself
result in several possible ASTs. However, mainstream parsers such as
\Bison\footnote{\url{http://www.gnu.org/software/bison/}} do not
implement the proper GLR algorithm, but rather implement under this
name an algorithm that may use exponential time and space when parsing
with an ambiguous grammar. Alternatively, the proper O($n^3$) time GLR
parsing algorithm~\cite{glr} was implemented in tools such as
\SDF~\cite{sdf}, used by \textsf{ASF+SDF}~\cite{asf+sdf} and
\textsf{Stratego/XT}~\cite{metaprog} to implement concrete syntax
rewriting rules, or very recently BRNGLR~\cite{brnglr}. Also, a hybrid
GLR/LALR parser called \textsf{Elkhound}~\cite{elkhound} has been
released and its running time is close to that of a standard LALR(1)
parser on all the portions of a grammar that are LALR(1). Other
parsing algorithms covering all context\hyp{}free grammars, such as
Earley's~\cite{earley}, may perform well on ambiguous grammars.

However, in the case of legacy parsers, that are specifically
addressed by our approach, there is a very important practical issue:
porting an existing parser to a different technology, like GLR or
Earley, may prove quite difficult and has profound impacts on the test
of the tool, which is unaffordable for most projects. An aspect that
may further complicate the problem is that in many such projects, it
is required not only to implement a new parser, but to reproduce
exactly the same behaviour as the legacy parser, including perhaps
some of its idiosyncrasies. Indeed, when a significant customer base
with potentially huge legacy assets relies on an existing compiler, it
may be economically unacceptable to require them migrating to a new,
perhaps better, but slightly incompatible parser.

\noindent\textbf{Our Proposal.} It is fair to say that there is no
simple solution today for adding concrete syntax pattern matching to
existing parsers without either profoundly re\-structuring the parser,
rewriting it in another framework, severe\-ly restricting the
patterns, or compromising performance. As a consequence, concrete
patterns are rarely used in existing parser\hyp{}based tools such as
mainstream compilers. The lack of concrete patterns is particularly
limiting the implementation of convenient user extensions in tools
such as extensible compilers, code inspectors, model checkers, etc. To
solve this problem in a pragmatic way, we propose a pattern matching
technique based on \emph{unparsed patterns}, which makes an efficient
use of unrestricted concrete syntax patterns, while requiring at the
same time no extension of the parser for the subject programming
language. This method is applicable to legacy parsers, based on any
parsing technology, without porting them to a different framework. In
a previous paper~\cite{ppdp}, one of us showed some concrete
applications of unparsed patterns within a checking compiler called
\MyGCC. However, no details were published about their implementation,
nor about their theoretical foundations; it introduced the idea of
unparsed patterns and briefly mentioned that they work by unparsing
the AST, rather than by parsing the pattern. The present article aims
at a complete exposition of unparsed patterns.

The rest of this paper is organised as follows. The next section
presents the formal model which we use later to define our algorithms
and related proofs. The reader familiar with functional languages or
logic programming can skip this section upon first reading.  Then,
section~\ref{backtracking} describes a matching algorithm for unparsed
patterns by means of backtracking. It allows the reader to become
familiar with the issues at hand, without delving too much on the
details. The following section~\ref{es1} presents our main
contribution, that is, a deterministic, linear\hyp{}time matching
algorithm, called \textit{ES}(1), together with some formal
proofs. After, section~\ref{implementation} presents our
implementation of our matching algorithm and discusses its strengths
and limitations. The penultimate section~\ref{related_work} presents
different related works and compares them to ours.
