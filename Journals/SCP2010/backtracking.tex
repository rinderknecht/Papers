%%-*-latex-*-

\begin{figure}[t]
\framebox[\columnwidth]{
  \begin{tabular}{cc}
    \subfloat[Tree\hyp{}like pattern\label{fig:pat_tree}]{
      \includegraphics[bb=61 601 157 721]{pat_tree_ex}
    }
    &
    \subfloat[Source parse tree\label{fig:src_tree}]{
      \includegraphics[bb=61 601 166 721]{match_ex}
    }
  \end{tabular}
}
\caption{Tree pattern matching \texttt{a = a - b*c - d}}
\label{fig:match_ex}
\end{figure}


\section{A Backtracking Algorithm}
\label{backtracking}

Before presenting a precise algorithm for unparsed\hyp{}pattern
matching, let us discuss informally an example. Consider the problem
of matching the pattern \texttt{\%x = \%y - \%z} against the \Clang
expression \texttt{a = a - b * c - d}. Pattern\hyp{}tree matching
would first parse the expression into the parse tree in
Figure~\ref{fig:src_tree}, then parse the pattern using an extended
parser into the parse tree in Figure~\ref{fig:pat_tree}, and then
match the latter against the former. As a result, all the
meta\-variables are correctly bound with respect to the grammar in the
substitution \(\{\textsf{x} \mapsto \texttt{"a"}, \textsf{y} \mapsto
\texttt{"a-b*c"}, \textsf{z} \mapsto \texttt{"d"}\}\). The key idea of
unparsed patterns is to avoid parsing the pattern by going the other
way around, i.e., by unparsing the source parse tree and comparing the
result with a textual pattern. However, if the parse tree is simply
unparsed into a string, matching would fall back to the case of
matching between two strings, which is very imprecise, because it
would yield both the substitution \(\{\textsf{x} \mapsto \texttt{"a"},
\textsf{y} \mapsto \texttt{"a-b*c"}, \textsf{z} \mapsto
\texttt{"d"}\}\), which is correct, and \(\{\textsf{x} \mapsto
\texttt{"a"}, \textsf{y} \mapsto \texttt{"a"}, \textsf{z} \mapsto
\texttt{"b*c-d"}\}\), which is incorrect, since the subtraction
operator is left\hyp{}associative. Moreover, meta\-variables are bound
to strings (i.e., concrete syntax), rather than being bound to
subtrees of the parse tree. This is not suitable when using pattern
matching to process the matched subtrees, which is a common case
within parsing\hyp{}based tools.

The technical issue is that the whole parse tree is fully unparsed
(i.e., de\-structured) at once, dropping the references to all the
subtrees. In order to avoid that, the parse tree should be unparsed
level by level (in a breadth\hyp{}first traversal) and the unparsed
pattern (which is a list of lexemes and meta\-variables here) should
either be partially matched against the current unparsed forest or the
latter should be further unparsed. These two alternatives are
sometimes possible for a given parse forest and unparsed pattern. In
the first option, i.e., partial matching, can be tried first and if it
leads to a failure, the second option, i.e., unparsing, is tried
instead. If both options lead to a failure, then the whole matching is
deemed a failure. This technique is called \emph{backtracking} and
does not lead to a linear\hyp{}time algorithm in the worst case (in
the size of the pattern plus the size of the source tree). Also, the
order in which matching or unparsing are tried is not significant as
there is no way to guess which would me more likely to be successful
\emph{a priori}. More precisely, firstly, the source parse tree is
pushed on an empty analysis stack. This stack is, in general, a parse
forest. We shall speak of the ``left of the forest'' instead of the
``top of the stack.'' Secondly, the textual pattern, here the string
\texttt{\%x = \%y - \%z}, is transformed into a list of lexemes and
meta\-variables.
%% \footnote{In practice, this is not necessary
%%   because the leaves of the source parse tree contain all the lexical
%%   information needed to realise a match step (read further). But, as
%%   much as the theory of unparsed patterns is concerned, it is simpler
%%   to assume some tokenisation of the pattern, hence avoiding the use
%%   of regular expressions in the formal model or the algorithm itself.}
Thirdly, given an initial empty substitution \(\sigma\), the algorithm
non\hyp{}deterministically chooses one of the two following actions,
and backtrack in case of failure.
\begin{enumerate}

  \item \textbf{Matching.} The first element \(e\) of the pattern is
    matched against the leftmost tree \(h\) of the forest. This can be
    achieved in two different situations:
    \begin{enumerate}
    
      \item \textbf{Elimination.} If \(h\) and \(e\) are the same
        lexeme, then the remaining pattern is matched against the
        remaining forest, with the same substitution \(\sigma\).

      \item \textbf{Binding.} If \(h\) is not a lexeme (i.e., it is
        not a leaf) and \(e\) is the meta\-variable \(x\), which is
        either already bound to a subtree equal to \(h\) (Unparsed
        patterns are not linear, i.e., a meta\-variable can occur more
        than once.) or unbound in \(\sigma\), then the remaining
        pattern is matched against the remaining forest, with
        \(\sigma\) updated with \(x\) bound to \(h\).

  \end{enumerate}

  \item \textbf{Unparsing.} If the forest starts with a tree \(t\),
    unparsing consists in replacing \(t\) by the forest of its direct
    subtrees (in other words, the root of \(t\) is cut out) and trying
    again with the same pattern and the same substitution.

\end{enumerate}
The algorithm always stops because either the pattern length or the
forest size strictly decreases at each step. It fails if and only if
the final pattern is not empty. In case of success, the final
substitution is the result (it contains all the bindings of the
meta\-variables to subtrees of the source parse tree).


\subsection{Pattern Matching}

Unparsed patterns are noted \(\overline{p}\) and the set of unparsed
patterns is inductively defined as the smallest set \(\overline{\cal
  P}\) such that
\begin{itemize}

  \item \(\el \in \overline{\cal P}\);

  \item if \(l \in {\cal L}\) and \(\overline{p} \in \overline{\cal
    P}\), then \(\cons{l}{\overline{p}} \in \overline{\cal P}\);

  \item if \(x \in {\cal V}\) and \(\overline{p} \in \overline{\cal
    P}\), then \(\cons{\meta{x}}{\overline{p}} \in \overline{\cal P}\).

\end{itemize}
Let us extend the substitutions defined in section~\ref{model}, in
order to cope with unparsed patterns, not just meta\-variables. The
effect of a substitution on a pattern will be to replace every
occurrence in the pattern of the meta\-variables in its domain by the
corresponding parse trees. The substitutions computed by any of our
matching algorithms are total, i.e., they replace \emph{all} the
meta\-variables of the pattern. It is handy to distinguish the forests
which contain no meta\-variables by calling them \emph{closed forests}
or \emph{closed patterns}, and their contents \emph{closed trees}. In
order to distinguish a substitution applied to a meta\-variable \(x\)
from a substitution on an unparsed pattern \(\overline{p}\), we shall
note \(\sigma(x)\) the former and \(\subst{\overline{p}}{\sigma}\) the
latter. Consider the formal definition of substitutions in
Figure~\ref{x_subst_def}.  The first equation \((\eqn{1})\) means that
the substitution on the empty pattern is always the empty forest. The
second equation \((\eqn{2})\) defines the substitution of a
meta\-variable by its associated tree: the tree is added to the left
of the resulting forest and the substitution proceeds recursively over
the remaining unparsed pattern. The third equation \((\eqn{3})\)
specifies that the substitutions always leave lexemes unchanged.
\begin{figure}[t]
\framebox[\columnwidth]{%
\(
\begin{aligned}
\subst{\el}{\sigma} 
&\eqn{1} \el\\
\subst{\cons{\meta{x}}{\overline{p}}}{\sigma}
&\eqn{2} \cons{\sigma(x)}{\subst{\overline{p}}{\sigma}}\\
\subst{\cons{l}{\overline{p}}}{\sigma}
&\eqn{3} \cons{l}{\subst{\overline{p}}{\sigma}} 
\end{aligned}
\)
}
\caption{Substitutions on unparsed patterns\label{x_subst_def}}
\end{figure}

\begin{figure}[b]
\framebox[\columnwidth]{%
\begin{mathpar}
\inferrule{}{\smj{\el}{\el}{\sigma_\varnothing}}
\quad\TirName{\text{\sf END}}

\inferrule
{\smj{f}{\overline{p}}{\sigma}}
{\smj{\cons{l}{f}}{\cons{l}{\overline{p}}}{\sigma}}
\,\TirName{\text{\sf ELIM}}

\inferrule*[right=\text{\sf BIND}]
{\smj{f}{\overline{p}}{\sigma}\\
 \sigma \subseteq \sigma \oplus x \mapsto t}
{\smj{\cons{t}{f}}%
     {\cons{\meta{x}}{\overline{p}}}%
     {\sigma \oplus x \mapsto t}}

\inferrule*[right=\text{\sf UNPAR}]
  {\smj{f_1 \cdot f_2}{\overline{p}}{\sigma}}
  {\smj{\cons{c(f_1)}{f_2}}{\overline{p}}{\sigma}}
\end{mathpar}
}
\caption{A backtracking pattern matching\label{x_match_def}}
\end{figure}

Pattern matching is defined by the inference system given in
Figure~\ref{x_match_def}, where the rules are unordered and in
\Prolog, in Figure~\ref{x_match_prolog}. Let us call a
\emph{configuration} the pair \(\smat{f}{\overline{p}}\). In case the
forest contains only one tree \(h\), let us write
\(\smat{h}{\overline{p}}\) instead of
\(\smat{[h]}{\overline{p}}\). The pattern matching associates a
configuration to a substitution. This system of inference rules is not
syntax\hyp{}directed, because the conclusions of rules \textsf{BIND}
and \textsf{UNPAR} overlap: a non\hyp{}deterministic choice between
binding and unparsing must be done. This dilemma cannot be decided
solely based on the shape of the configuration and thus the
implementation must rely on a backtracking mechanism, as we said
before. Note that no rule has more than one premise involving the
\((\twoheadrightarrow)\) relation, hence the proof trees (i.e.,
derivations, when read deductively) are actually lists. Rule
\textsf{END} rewrites the empty configuration to the empty
substitution; this happens as the last rewrite step---from whence its
name. Let us read the rules inductively, since this reading
corresponds to an algorithm.
\begin{itemize}

  \item Rule \textsf{ELIM}: if the pattern and the tree start with the
    same lexeme, then remove the lexemes and try to rewrite the
    remaining configuration.

  \item Rule \textsf{BIND}: a meta\-variable \(x\) is bound to a tree
    \(t\), i.e., \(x \mapsto t\), if the remaining configuration
    rewrites to a substitution \(\sigma\) which either already
    contains the binding or whose domain does not contain \(x\) (i.e.,
    \(\sigma \subseteq \sigma \oplus x \mapsto t\)); the resulting
    substitution is the updating of \(\sigma\) with the new binding.

  \item Rule \textsf{UNPAR}: to match the same pattern against the
    same tree \(t = c(f_1)\) whose root has been cut off (i.e.,
    \(f_1\) remains); this is \emph{unparsing}. Note that the
    configuration
    \(\smat{\cons{c(f_1)}{f_2}}{\cons{\meta{x}}{\overline{p}}}\) can
    lead both to an unparsing or a binding.
\end{itemize}
\begin{figure}[t]
\VerbatimInput[frame=single]{backtrack.pl}
\caption{The backtracking algorithm of Figure~\ref{x_match_def} in
  \Prolog\label{x_match_prolog}}
\end{figure}
The encoding in \Prolog is shown in Figure~\ref{x_match_prolog}. Patterns
are noted \texttt{P}, variables \texttt{X}, forests \texttt{F},
substitutions \texttt{S}, trees \texttt{T}. A binding between a
meta\-variable and a tree is a pair \texttt{\{X,T\}} and a
substitution is a list of such
bindings. \(\smj{f}{\overline{p}}{\sigma}\) is \texttt{match(P,F,S)};
\(c(f)\) is \texttt{node(C,F)}; \(\meta{x}\) is \texttt{meta(X)};
\(l\) is \texttt{lex(L)} and `\(\sigma \oplus x \mapsto t\) if
\(\sigma \subseteq \sigma \oplus x \mapsto t\)' is implemented by
\texttt{add(S1,\{X,T\},S2)}, where \(\sigma\) is \texttt{S1}, \(x
\mapsto t\) is \texttt{\{X,T\}} and \(\sigma \oplus x \mapsto t\) is
\texttt{S2}.


\subsection{Closed\hyp{}Tree Inclusion}

\begin{figure}[b]
\framebox[\columnwidth]{%
  \includegraphics[scale=0.5]{sqsubseteq}
}
\caption{Closed\hyp{}tree inclusion \([a,b,c] \sqsubseteq h\)}
\label{fig:sqsubseteq}
\end{figure}

We need to defined what it means for a closed tree to be included in a
parse tree. This is the \emph{closed\hyp{}tree matching}, which is a
variation on the classic tree matching, as, in the latter, the pattern
tree may embed meta\-variables and the roots are matched. This way, it
becomes possible to compare the expressive power of the backtracking
pattern matching with respect to the more familiar tree matching, used
by many existing tools. Informally, let us say that a tree \(h_1\) is
included in a tree \(h_2\) if and only if \(h_1\) is included in
\(h_2\) and the fringe of \(h_1\) is included in the fringe of
\(h_2\), as shown in Figure~\ref{fig:sqsubseteq}. Let us note \(h_1
\sqsubseteq h_2\) this relationship. Technically, we only need to
define \((\sqsubseteq)\) between a forest \(f\) and a tree \(h\) in
such a manner that \(f \sqsubseteq h\) implies that there exists a
constructor \(c\) such that \(c(f) \sqsubseteq h\). But we shall not
precise this further. The formal definition we propose here is based
on partially ordered inference rules (see section~\ref{model}),
displayed in Figure~\ref{x_tree_matching_def}. We give now a logical
reading of the rules.

\begin{figure}[t]
\framebox[\columnwidth]{
\begin{mathpar}
\inferrule*{}{\el \sqsubseteq \el}
\;\TirName{\text{\sf EMP}}
\quad
\inferrule
  {f_1 \sqsubseteq f_2}
  {\cons{h}{f_1} \sqsubseteq \cons{h}{f_2}}
\,\TirName{\text{\sf EQ}}
\quad
\inferrule
  {f \sqsubseteq f_1 \cdot f_2}
  {f \sqsubseteq \cons{c(f_1)}{f_2}}
\,\TirName{\text{\sf SUB}}
\quad
\inferrule
  {f \sqsubseteq [h]}
  {f \sqsubseteq h}
\,\TirName{\text{\sf ONE}}
\end{mathpar}
}
\caption{Closed\hyp{}forest inclusion\label{x_tree_matching_def}}
\end{figure}
\begin{itemize}

  \item Rule \textsf{ONE} states that if the forest \(f\) is included
    in a forest made of a single tree \(h\), then it is included in
    \(h\). This relates a closed forest and a tree.

  \item Axiom \textsf{EMP} says that the empty forest is included in
    the empty forest.

  \item Rule \textsf{EQ} specifies that if a non\hyp{}empty forest
    \(f_1\) is included in another non\hyp{}empty forest (possibly the
    same), then the forest \(\cons{h}{f_1}\) is included in the forest
    \(\cons{h}{f_2}\), where \(h\) is a tree (possibly a lexeme).

  \item Rule \textsf{SUB} states that if a forest \(f\) is included in
    the catenation of forests \(f_1\) and \(f_2\), such that the
    constructor \(c\) can have \(f_1\) as direct subtrees, then \(f\)
    is included in \(\cons{c(f_1)}{f_2}\) (i.e., grouping some trees
    into a new tree changes nothing). When reading the rules
    inductively, i.e., bottom\hyp{}up, or, in other words,
    algorithmically, we must add the constraint that rule \textsf{SUB}
    must always be considered last, i.e., if a derivation (i.e., a
    proof tree) ends with \textsf{SUB}, its conclusion has the shape
    \(f \sqsubseteq \cons{c(f_1)}{f_2}\) \emph{and it is implied that
      there is no forest} \(f'\) \emph{such that} \(f =
    \cons{c(f_1)}{f'}\) (which would conclude \textsf{EQ}).

\end{itemize}

\subsection{Soundness}

The soundness of the pattern matching means that all computed
substitutions, once applied to the original pattern, yield a closed
forest which is included, in the sense above, in the original
tree. Informally: all successful pattern matchings lead to successful
closed\hyp{}tree matchings. Formally:
\begin{thm}[Soundness]\hfill
\label{backtracking:soundness}
\begin{center}
If \(\smat{h}{\overline{p}} \twoheadrightarrow \sigma\),
then \(\subst{\overline{p}}{\sigma} \sqsubseteq h\).
\end{center}
\end{thm}

%%-*-latex-*-

\noindent\textsc{Proof~\ref{backtracking:soundness}} (Soundness).\\
\noindent Let \(\ind{R}(\overline{p}, f, \sigma)\) be the proposition
\begin{center}
\emph{If} \(\smj{f}{\overline{p}}{\sigma}\),
\emph{then} \(\subst{\overline{p}}{\sigma} \sqsubseteq f\).
\end{center}
Then \(\ind{R}(p, [h], \sigma)\) is equivalent to the soundness
property (by rule \textsf{ONE}). Firstly, let us assume that
\begin{equation}
  \smj{f}{\overline{p}}{\sigma} \label{x:sound:head}
\end{equation}
is true (otherwise the theorem is trivially true). This means that
there exists a pattern\hyp{}matching derivation \(\Delta\) whose
conclusion is \(\smj{f}{\overline{p}}{\sigma}\). This derivation is a
list, which makes it possible to reckon by induction on its structure,
i.e., one assumes that \(\ind{R}\) holds for the premise of the last
rule in \(\Delta\) (this is the \emph{induction hypothesis}) and then
proves that \(\ind{R}\) holds for \(\smj{f}{\overline{p}}{\sigma}\). A
case by case analysis on the kind of rule that can end the derivation
guides the proof.
\begin{enumerate}

   \item Case where \(\Delta\) ends with \textsf{END}.\\ In this case,
     \(\overline{p} = \el\) and \(f = \el\). Therefore
     \begin{gather*}
       \subst{\overline{p}}{\sigma} = \subst{\el}{\sigma} \eqn{1} \el
       \sqsubseteq \el = f.
     \end{gather*}
     We conclude that \(\ind{R}(\el, \el, \sigma)\) holds.
 
     \medskip

   \item Case where \(\Delta\) ends with \textsf{ELIM}.
        \begin{mathpar}
          \inferrule*[right=\text{\sf ELIM}]
            {\inferrule*
               {\inferrule*[vdots=1.5em]{}{ }}
               {\smj{f'}{\overline{p}'}{\sigma}}
            }
            {\smj{\cons{l}{f'}}{\cons{l}{\overline{p}'}}{\sigma}}
       \end{mathpar}
       where, since we assumed \eqref{x:sound:head},
       \begin{enumerate}
          
         \item \label{x:sound:13} \(f \triangleq \cons{l}{f'}\),

         \item \label{x:sound:14} \(\overline{p} \triangleq
           \cons{l}{\overline{p}'}\).

       \end{enumerate}
       Let us assume that the induction hypothesis holds for the
       premise of \textsf{ELIM}, i.e., \(\ind{R}(\overline{p}', f',
       \sigma)\) holds:
       \begin{gather}
         \subst{\overline{p}'}{\sigma} \sqsubseteq f'.
         \label{x:sound:2}
       \end{gather}
        Besides, we have
        \begin{align}
          \subst{\overline{p}}{\sigma}
          &= \subst{\cons{l}{\overline{p}'}}{\sigma}\notag
          & \text{by} \,\; \text{\ref{x:sound:14}}\\
          &\eqn{3} \cons{l}{\subst{\overline{p}'}{\sigma}}\notag
          &\text{cf. Fig.~\ref{x_subst_def}}\\
          &\sqsubseteq \cons{l}{f'}
          &\text{by} \,\; \eqref{x:sound:2} \,\; \text{and} \,\;
          \textsf{EQ} \, \text{(Fig.~\ref{x_tree_matching_def})}\notag\\
          &= f
          & \text{by} \,\; \text{\ref{x:sound:13}}\notag\\
            \subst{\overline{p}}{\sigma}
          &\sqsubseteq f. \label{x:sound:5}
        \end{align}
        As a conclusion, the induction hypothesis and
        \eqref{x:sound:head} imply \eqref{x:sound:5}, so
        \(\ind{R}(\cons{l}{\overline{p}'}, \cons{l}{f'}, \sigma)\)
        holds.

     \medskip

   \item Case where \(\Delta\) ends with \textsf{BIND}.
        \begin{mathpar}
          \inferrule*[right=\text{\sf BIND}]
            {\inferrule*
                {\inferrule*[vdots=1.5em]{}{ }}
                {\smj{f'}{\overline{p}'}{\sigma'}}}
            {\smj{\cons{t}{\!f'}}%
                {\cons{\meta{x}\!}{\!\overline{p}'}}%
                {\sigma' \oplus x \mapsto t}}
        \end{mathpar}
        where, because we assumed \eqref{x:sound:head},
        \begin{enumerate}

          \item \label{x:sound:17} \(f \triangleq \cons{t}{f'}\),

          \item \label{x:sound:18} \(\overline{p} \triangleq
            \cons{\meta{x}}{\overline{p}'}\),

          \item \label{bind_x} \(\sigma' \subseteq \sigma' \oplus x
            \mapsto t\),

            \item \label{bind_rho} \(\sigma \triangleq \sigma' \oplus
              x \mapsto t\).

        \end{enumerate}
        Let us assume that the induction hypothesis holds for the
        premise of \textsf{BIND}, i.e., \(\ind{R}(\overline{p}', f',
        \sigma')\) holds:
        \begin{gather}
          \subst{\overline{p}'}{\sigma'} \sqsubseteq f'.
          \label{x:sound:7}
        \end{gather}
        We also have
        \begin{align}
           \sigma(x) 
          &\triangleq (\sigma' \oplus x \mapsto t)(x)
          &\text{by} \,\; \text{\ref{bind_rho}}\notag\\
          &= t
          &\text{by} \,\; \eqref{model:oplus}\notag\\
           \sigma(x)
          &= t \label{x:sound:8}\\
           \subst{\overline{p}'}{\sigma'}
           &= \subst{\overline{p}'}{(\sigma' \oplus x \mapsto t)}
           &\text{by} \,\; \text{\ref{bind_x}} \,\; \text{and} \,\;
           \text{\ref{minimality}}\notag\\
           \subst{\overline{p}'}{\sigma'}
           &= \subst{\overline{p}'}{\sigma}
           &\text{by} \,\; \text{\ref{bind_rho}} \label{x:sound:19}
        \end{align}
        \begin{remark}
          The lemma~\ref{minimality}, about minimal substitutions, can
          be applied because the meta\hyp{}parsed patterns
          \(\overline{p}\) it is defined upon are a strict superset of
          the unparsed patterns \(\overline{p}\) of this section.
        \end{remark}
        Besides, we have
        \begin{align}
           \subst{\overline{p}}{\sigma}
        &\triangleq \subst{\cons{\meta{x}}{\overline{p}'}}{\sigma}
        &\text{by} \,\; \text{\ref{x:sound:18}}\notag\\
        &\eqn{2} \cons{\sigma(x)}{\subst{\overline{p}'}{\sigma}}\notag
        &\text{cf. Fig.~\ref{x_subst_def}}\\
        &= \cons{t}{\subst{\overline{p}'}{\sigma}}
        &\text{by} \,\; \eqref{x:sound:8}\notag\\
        &= \cons{t}{\subst{\overline{p}'}{\sigma'}}
        &\text{by} \,\; \eqref{x:sound:19}\notag\\
        &\sqsubseteq \cons{t}{f'}
        &\text{by} \,\; \eqref{x:sound:7} \,\; \text{and} \,\;
           \textsf{EQ} \, \text{(Fig.~\ref{x_tree_matching_def})}\notag\\
        &\triangleq f
        &\text{by} \,\; \text{\ref{x:sound:17}}\notag\\
           \subst{\overline{p}}{\sigma}
        &\sqsubseteq f. \label{x:sound:9}
        \end{align}
        In the end, the induction hypothesis and \eqref{x:sound:head}
        imply \eqref{x:sound:9}, so
        \(\ind{R}(\cons{\meta{x}}{\overline{p}'}, \cons{t}{f'},
        \sigma)\) holds.

     \medskip

   \item \label{unpar} Case where \(\Delta\) ends with
       \textsf{UNPAR}.
       \begin{mathpar}
         \inferrule*[right=\text{\sf UNPAR}]
           {\inferrule*
              {\inferrule*[vdots=1.5em]{}{ }}
              {\smj{f_1 \cdot f_2}{\overline{p}}{\sigma}}}
           {\smj{\cons{c(f_1)}{f_2}}{\overline{p}}{\sigma}}
       \end{mathpar}
       where, since we assumed \eqref{x:sound:head},
       \begin{enumerate}

         \item \label{x:sound:15} \(f = \cons{c(f_1)}{f_2}\).
       
       \end{enumerate}
       Let us assume that the induction hypothesis holds for the
       premise of \textsf{UNPAR}, i.e., \(\ind{R}(\overline{p}, f_1
       \cdot f_2, \sigma)\) holds:
       \begin{align}
         \subst{\overline{p}}{\sigma} &\sqsubseteq f_1 \cdot f_2.\notag\\
          &\sqsubseteq \cons{c(f_1)}{f_2}
          &\text{by} \,\; \textsf{SUB} \,
         \text{(Fig.~\ref{x_tree_matching_def})}\notag\\
           &= f
           &\text{by} \,\; \text{\ref{x:sound:15}}\notag\\
            \subst{\overline{p}}{\sigma}
           & \sqsubseteq f. \label{x:sound:I}
       \end{align}
        As a conclusion, the induction hypothesis and
        \eqref{x:sound:head} imply \eqref{x:sound:I}, so
        \(\ind{R}(\cons{l}{\overline{p}'}, \cons{c(f_1)}{f_2},
        \sigma)\) holds.\hfill \(\Box\)

\end{enumerate}


\subsection{Completeness}

The completeness of our algorithm means that every time a complete
substitution on a pattern matches a tree, our algorithm computes a
substitution which is included in the first one. Indeed, the computed
substitution never contains useless bindings, but the other one
may. Therefore, the completeness property is perhaps better stated by
referring to \emph{minimal substitutions}: all minimal substitutions
that enable a closed\hyp{}tree inclusion are computed by our pattern
matching. Formally, this can be expressed as follows.
\begin{thm}[Completeness]\hfill
\label{backtracking:completeness}
\begin{center}
If   \(\subst{\overline{p}}{\sigma} \sqsubseteq h\),
then \(\smat{h}{\overline{p}} \twoheadrightarrow \sigma'\) 
and  \(\sigma' \subseteq \sigma\).
\end{center}
\end{thm}

\noindent\textsc{Proof Sketch~\ref{backtracking:completeness}}
(Completeness). By structural induction on the patterns.\(\Box\)

\begin{lem}[Minimality]\hfill
\label{minimality}
\begin{center}
If \(\sigma \subseteq \sigma'\) 
then \(\subst{\overline{p}}{\sigma} = \subst{\overline{p}}{\sigma'}\).
\end{center}
\end{lem}
\noindent In other words, for a given substitution, there exists a
minimal substitution yielding the same result (if defined) for any
pattern.

\noindent\textsc{Proof Sketch~\ref{minimality}} (Minimality) By
structural induction on the patterns.\(\Box\)

\subsection{Compliance}

As a corollary, the algorithm is sound and complete with respect to
the closed\hyp{}tree inclusion:
\begin{cor}[Compliance]\hfill
\label{backtracking:compliance}
\begin{center}
\(\subst{\overline{p}}{\sigma} \sqsubseteq h\) if and only if
\(\smat{h}{\overline{p}} \twoheadrightarrow \sigma'\) 
and  \(\sigma' \subseteq \sigma\).
\end{center}
\end{cor}
\noindent In other words, the concept of closed\hyp{}tree inclusion
coincides exactly with the backtracking algorithm.

\noindent\textsc{Proof~\ref{backtracking:compliance}}
(Compliance). The way from left to right is the completeness. From the
soundness, it comes that
\emph{If \(\smat{h}{\overline{p}} \twoheadrightarrow \sigma'\), then
\(\subst{\overline{p}}{\sigma'} \sqsubseteq h\).}
The minimality lemma~\ref{minimality} implies then
\(\subst{\overline{p}}{\sigma} \sqsubseteq h\). \(\Box\)

\subsection{Further Discussion}

One solution to overcome the inefficiency of non\hyp{}determinacy is
to make explicit the tree structure in the textual pattern by adding
some kind of parentheses, so each step becomes uniquely
determined. For example, to match the pattern \texttt{\%x = \%y - \%z}
against the parse tree in Figure~\ref{fig:src_tree}, we would add to
the pattern meta\-lexemes called \emph{meta\-parentheses} (i.e.,
parentheses that do not belong to the object language), which are here
represented as escaped parentheses: \texttt{\%(} and \texttt{\%)}. For
example, we may use the pattern \texttt{\%(\%x = \%(\%(\%y\%) -
\%z\%)\%)}. Note that in general we cannot use plain parentheses to
unveil the structure because the language may already contain
parentheses which may have a completely different meaning other than
grouping. For instance, in the following \textsf{Korn shell}
(\textsf{ksh}) pattern:
\begin{verbatim} 
case %x in [yY]) echo yes;; *) echo no;; esac
\end{verbatim}
we must use meta\-parentheses to explicit the tree structure, because
parentheses would not pair the way we want---in fact, they would not
pair at all. Fully meta\-parenthesized patterns enable
linear\hyp{}time pattern matching. However, the explicit structure
comes at the price of seriously obfuscating the pattern. Clearly, if
we may say, fully meta\-parenthesized patterns are quite difficult to
read and write. The legibility of the pattern can be improved if the
matching algorithm allows some of the meta\-parentheses to be
dropped. This is what is shown in the next section, where the system
is syntax\hyp{}directed.
