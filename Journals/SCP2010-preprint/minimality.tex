%%-*-latex-*-

\subsection{Minimal Substitution}

A substitution on a pattern is a function which replaces all the
meta\-variables in the pattern by abstract syntax tree. It is
formally defined as
\begin{align*}
  \subst{\el}{\sigma} 
&\eqn{1} \el\\
  \subst{\cons{\cstr{meta}{x}}{\overline{p}}}{\sigma}
&\eqn{2} \cons{\sigma(x)}{\subst{\overline{p}}{\sigma}}\\
  \subst{\cons{l}{\overline{p}}}{\sigma}
&\eqn{3} \cons{l}{\subst{\overline{p}}{\sigma}}\\
  \subst{\cons{\pat{\overline{p}_1}}{\overline{p}_2}}{\sigma}
&\eqn{4}
  \cons{\pat{\subst{\overline{p}_1}{\sigma}}}%
       {\subst{\overline{p}_2}{\sigma}}
\end{align*}
Note that we distinguish the substitution on a meta\-variable
\(x\) from the substitution on a pattern \(\overline{p}\) by noting
the former \(\sigma(x)\) and the latter
\(\subst{\overline{p}}{\sigma}\).

\medskip

\noindent\textsc{Lemma}~\ref{minimality} (Minimal Substitution)\hfill
\label{minimality_proof}
\begin{center}
If \(\sigma \subseteq \sigma'\) 
then \(\subst{\overline{p}}{\sigma} = \subst{\overline{p}}{\sigma'}\).
\end{center}
\noindent In other words, for a given substitution, there exists a
minimal substitution yielding the same result (if defined) for any
pattern.

\medskip

\noindent\textsc{Proof}. Let \(\beth(\sigma, \sigma', \overline{p})\)
be the proposition
\begin{center}
\emph{If} \(\sigma \subseteq \sigma'\) 
\emph{then} \(\subst{\overline{p}}{\sigma} =
  \subst{\overline{p}}{\sigma'}\).
\end{center}
Let us prove it by induction on the structure of the pattern
\(\overline{p}\), that is to say, we shall assume that \(\beth(\sigma,
\sigma', \overline{p}')\) holds for all the immediate subpatterns
\(\overline{p}'\) of \(\overline{p}\) (that is the \emph{induction
  hypothesis}) and then prove that \(\beth(\sigma, \sigma',
\overline{p})\) holds too. The immediate subpatterns are the
subpatterns used to build the pattern in one step, by means of the
definition. More precisely, we can define the relation \((\prec)\)
such that \(\overline{p}' \prec \overline{p}\) holds if
\(\overline{p}'\) is an immediate subpattern of \(\overline{p}\) as:
\begin{gather*}
\overline{p}' \prec \cons{e}{\overline{p}'} \qquad\qquad 
\overline{p}' \prec \cons{\pat{\overline{p}'}}{\overline{p}_1}
\end{gather*}
We shall prove, for all \(\sigma\), \(\sigma'\) and \(\overline{p}\),
\begin{gather*}
(\forall \overline{p}'.(\overline{p}' \prec \overline{p} \Rightarrow
  \beth(\sigma, \sigma', \overline{p}'))) \Rightarrow \beth(\sigma,
  \sigma', \overline{p})
\end{gather*}
which, by the structural\hyp{}induction principle, implies \(\forall
\sigma, \sigma', \overline{p}.\beth(\sigma, \sigma',
\overline{p})\). We shall proceed by distinguishing firstly the case
when \(\overline{p}\) is the minimal pattern, i.e., \(\overline{p} =
\el\), and secondly the other cases (for which \(\overline{p}'\)
exists).
\begin{enumerate}

  \item Case \(\overline{p} = \el\).\\ We have \(\subst{\el}{\sigma}
    \eqn{1} \el \eqn{1} \subst{\el}{\sigma'}\), i.e., \(\beth(\sigma,
    \sigma', \el)\) holds.

  \item Case \(\overline{p} = \cons{e}{\overline{p}'}\).\\ Let us
    assume the induction hypothesis
    \[
    \forall \overline{p}'.(\overline{p}' \prec \overline{p}
    \Rightarrow \beth(\sigma, \sigma', \overline{p}')).
    \]
    Since in this case we have a \(\overline{p}'\) such that
    \(\overline{p}' \prec \overline{p}\), then we deduce
    \(\beth(\sigma, \sigma', \overline{p}')\). Let us suppose now
    \(\sigma \subseteq \sigma'\) (otherwise the theorem is trivially
    true). Therefore, we have
    \begin{gather}
      \sigma \subseteq
      \sigma' \label{eq:subst:00}\\ \subst{\overline{p}'}{\sigma} =
      \subst{\overline{p}'}{\sigma'}. \label{eq:subst:01}
    \end{gather}
    Let us now consider the different kinds of \(e\).
    \begin{enumerate}

      \item Case \(e = \meta{x}\).
        \begin{align}
          \subst{\cons{e}{\overline{p}'}}{\sigma} &\triangleq
          \subst{\cons{\meta{x}}{\overline{p}'}}{\sigma}\notag\\ 
          &\eqn{1}
            \cons{\sigma(x)}{\subst{\overline{p}'}{\sigma}}\notag 
            \intertext{The definition of \(\subseteq\) and
              \eqref{eq:subst:00} imply that 
            \(\sigma(x) = \sigma'(x)\), therefore} 
          &= \cons{\sigma'(x)}{\subst{\overline{p}'}{\sigma}}\notag\\ 
          &= \cons{\sigma'(x)}{\subst{\overline{p}'}{\sigma'}} 
          & \text{by} \,\;
          \eqref{eq:subst:01}\notag\\ &\eqn{1}
          \subst{\cons{\meta{x}}{\overline{p}'}}{\sigma'}\notag\\ 
          &\triangleq
          \subst{\cons{e}{\overline{p}'}}{\sigma'}\notag\\
            \subst{\overline{p}}{\sigma}
          &= \subst{\overline{p}}{\sigma'}. \label{eq:subst:02}
        \end{align}
        Finally, \eqref{eq:subst:02} and \eqref{eq:subst:00} imply
        \(\beth(\sigma, \sigma', \overline{p})\).

      \item Case \(e = l \in {\cal L}\).
        \begin{align}
          \subst{\cons{e}{\overline{p}'}}{\sigma}
          &\triangleq \subst{\cons{l}{\overline{p}'}}{\sigma}\notag\\
          &\eqn{2} \cons{l}{\subst{\overline{p}'}{\sigma}}\notag\\
          &= \cons{l}{\subst{\overline{p}'}{\sigma'}}
          & \text{by} \,\; \eqref{eq:subst:01}\notag\\
          &\eqn{2} \subst{\cons{l}{\overline{p}'}}{\sigma'}\notag\\
          &\triangleq \subst{\cons{e}{\overline{p}'}}{\sigma'}\notag\\
          \subst{\overline{p}}{\sigma}
          &= \subst{\overline{p}}{\sigma'}. \label{eq:subst:03}
        \end{align}
        Finally, \eqref{eq:subst:03} and \eqref{eq:subst:00} imply
        \(\beth(\sigma, \sigma', \overline{p})\). 

  \item Case \(e = \pat{\overline{p}_1}\).\\ Since \(\overline{p}_1
    \prec \overline{p}\), the induction hypothesis also implies here
    \(\beth(\sigma, \sigma', \overline{p}_1)\), that is
    \begin{gather}
      \subst{\overline{p}_1}{\sigma} =
      \subst{\overline{p}_1}{\sigma'} \label{eq:subst:05}
    \end{gather}
    since we already assumed \eqref{eq:subst:00}.
    \begin{align}
      \subst{\cons{e}{\overline{p}'}}{\sigma}
       &\triangleq
      \subst{\cons{\pat{\overline{p}_1}}{\overline{p}'}}{\sigma}\notag\\
      &\eqn{4}
      \cons{\pat{\subst{\overline{p}_1}{\sigma}}}%
           {\subst{\overline{p}'}{\sigma}}\notag\\
      &= \cons{\pat{\subst{\overline{p}_1}{\sigma}}}%
              {\subst{\overline{p}'}{\sigma'}}
      & \text{by} \,\; \eqref{eq:subst:01}\notag\\
      &= \cons{\pat{\subst{\overline{p}_1}{\sigma'}}}%
              {\subst{\overline{p}'}{\sigma'}}
      & \text{by} \,\; \eqref{eq:subst:05}\notag\\
      &\eqn{4} \subst{\cons{\pat{\overline{p}_1}}{\overline{p}'}}%
                     {\sigma'}\notag\\
      &\triangleq \subst{\cons{e}{\overline{p}'}}{\sigma'}\notag\\
         \subst{\overline{p}}{\sigma}
      &= \subst{\overline{p}}{\sigma'}. \label{eq:subst:04}
    \end{align}
    Finally, \eqref{eq:subst:04} and \eqref{eq:subst:00} imply
    \(\beth(\sigma, \sigma', \overline{p})\).\hfill \(\Box\)

    \end{enumerate}

\end{enumerate}
