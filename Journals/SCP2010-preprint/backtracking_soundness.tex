%%-*-latex-*-

\noindent\textsc{Proof~\ref{backtracking:soundness}} (Soundness).\\
\noindent Let \(\ind{R}(\overline{p}, f, \sigma)\) be the proposition
`\emph{If} \(\smj{f}{\overline{p}}{\sigma}\),
\emph{then} \(\subst{\overline{p}}{\sigma} \sqsubseteq f\).'
Then \(\ind{R}(p, [h], \sigma)\) is equivalent to the soundness
property (by rule \textsf{ONE}). Firstly, let us assume that
\begin{equation}
  \smj{f}{\overline{p}}{\sigma} \label{x:sound:head}
\end{equation}
is true (otherwise the theorem is trivially true). This means that
there exists a pattern\hyp{}matching derivation \(\Delta\) whose
conclusion is \(\smj{f}{\overline{p}}{\sigma}\). This derivation is a
list, which makes it possible to reckon by induction on its structure,
i.e., one assumes that \(\ind{R}\) holds for the premise of the last
rule in \(\Delta\) (this is the \emph{induction hypothesis}) and then
proves that \(\ind{R}\) holds for \(\smj{f}{\overline{p}}{\sigma}\). A
case by case analysis on the kind of rule that can end the derivation
guides the proof.
\begin{enumerate}

   \item Case where \(\Delta\) ends with \textsf{END}.\\ In this case,
     \(\overline{p} = \el\) and \(f = \el\). Therefore
     \(\subst{\overline{p}}{\sigma} = \subst{\el}{\sigma} \eqn{1} \el
       \sqsubseteq \el = f.\)
     We conclude that \(\ind{R}(\el, \el, \sigma)\) holds.
 
%     \medskip

   \item Case where \(\Delta\) ends with \textsf{ELIM}.
        \begin{mathpar}
          \inferrule*[right=\text{\sf ELIM}]
            {\inferrule*
               {\inferrule*[vdots=1.5em]{}{ }}
               {\smj{f'}{\overline{p}'}{\sigma}}
            }
            {\smj{\cons{l}{f'}}{\cons{l}{\overline{p}'}}{\sigma}}
       \end{mathpar}
       where, since we assumed \eqref{x:sound:head},
       \begin{enumerate}
          
         \item \label{x:sound:13} \(f \triangleq \cons{l}{f'}\),

         \item \label{x:sound:14} \(\overline{p} \triangleq
           \cons{l}{\overline{p}'}\).

       \end{enumerate}
       Let us assume that the induction hypothesis holds for the
       premise of \textsf{ELIM}, i.e., \(\ind{R}(\overline{p}', f',
       \sigma)\) holds:
       \begin{gather}
         \subst{\overline{p}'}{\sigma} \sqsubseteq f'.
         \label{x:sound:2}
       \end{gather}
        Besides, \ref{x:sound:14}, \eqref{x:sound:2}, \textsf{EQ}, and
        \ref{x:sound:13} imply
        \begin{align}
          \subst{\overline{p}}{\sigma}
          &= \subst{\cons{l}{\overline{p}'}}{\sigma}
          \sqsubseteq \cons{l}{f'}
          \eqn{3} \cons{l}{\subst{\overline{p}'}{\sigma}} =
          f.\label{x:sound:5} 
        \end{align}
        As a conclusion, the induction hypothesis and
        \eqref{x:sound:head} imply \eqref{x:sound:5}, so
        \(\ind{R}(\cons{l}{\overline{p}'}, \cons{l}{f'}, \sigma)\)
        holds.

%     \medskip

   \item Case where \(\Delta\) ends with \textsf{BIND}.
        \begin{mathpar}
          \inferrule*[right=\text{\sf BIND}]
            {\inferrule*
                {\inferrule*[vdots=1.5em]{}{ }}
                {\smj{f'}{\overline{p}'}{\sigma'}}}
            {\smj{\cons{t}{\!f'}}%
                {\cons{\meta{x}\!}{\!\overline{p}'}}%
                {\sigma' \oplus x \mapsto t}}
        \end{mathpar}
        where, because we assumed \eqref{x:sound:head},
        \begin{enumerate}

          \item \label{x:sound:17} \(f \triangleq \cons{t}{f'}\),

          \item \label{x:sound:18} \(\overline{p} \triangleq
            \cons{\meta{x}}{\overline{p}'}\),

          \item \label{bind_x} \(\sigma' \subseteq \sigma' \oplus x
            \mapsto t\),

            \item \label{bind_rho} \(\sigma \triangleq \sigma' \oplus
              x \mapsto t\).

        \end{enumerate}
        Let us assume that the induction hypothesis holds for the
        premise of \textsf{BIND}, i.e., \(\ind{R}(\overline{p}', f',
        \sigma')\) holds:
        \begin{gather}
          \subst{\overline{p}'}{\sigma'} \sqsubseteq f'.
          \label{x:sound:7}
        \end{gather}
        Also, \ref{bind_rho} and \eqref{model:oplus} imply
        \begin{align}
           \sigma(x) 
          &\triangleq (\sigma' \oplus x \mapsto t)(x) = t,
          \label{x:sound:8}\\
          \intertext{and \ref{bind_x}, Lem.~\ref{minimality} and
            \ref{bind_rho} yield}
           \subst{\overline{p}'}{\sigma'}
           &= \subst{\overline{p}'}{(\sigma' \oplus x \mapsto t)} =
           \subst{\overline{p}'}{\sigma}.\label{x:sound:19}
        \end{align}
        Besides, \ref{x:sound:18} and Fig.~\ref{x_subst_def} imply
        \begin{align}
           \subst{\overline{p}}{\sigma}
        &\triangleq \subst{\cons{\meta{x}}{\overline{p}'}}{\sigma}\notag
         \eqn{2}
         \cons{\sigma(x)}{\subst{\overline{p}'}{\sigma}}\notag\\
        &= \cons{t}{\subst{\overline{p}'}{\sigma}}
         = \cons{t}{\subst{\overline{p}'}{\sigma'}},
        \;\,\text{by} \,\; \eqref{x:sound:8} \,\; \text{and} \,\;
        \eqref{x:sound:19}\notag\\
        &\sqsubseteq \cons{t}{f'} \triangleq f;\,\;
        \text{by} \,\; \eqref{x:sound:7},
           \textsf{EQ} \,\; \text{and} \,\;
           \text{\ref{x:sound:17}}.\label{x:sound:9}
        \end{align}
        In the end, the induction hypothesis and \eqref{x:sound:head}
        imply \eqref{x:sound:9}, so
        \(\ind{R}(\cons{\meta{x}}{\overline{p}'}, \cons{t}{f'},
        \sigma)\) holds.

%     \medskip

   \item \label{unpar} Case where \(\Delta\) ends with
       \textsf{UNPAR}.
       \begin{mathpar}
         \inferrule*[right=\text{\sf UNPAR}]
           {\inferrule*
              {\inferrule*[vdots=1.5em]{}{ }}
              {\smj{f_1 \cdot f_2}{\overline{p}}{\sigma}}}
           {\smj{\cons{c(f_1)}{f_2}}{\overline{p}}{\sigma}}
       \end{mathpar}
       where, since we assumed \eqref{x:sound:head},
       \begin{enumerate}

         \item \label{x:sound:15} \(f = \cons{c(f_1)}{f_2}\).
       
       \end{enumerate}
       Let us assume that the induction hypothesis holds for the
       premise of \textsf{UNPAR}, i.e., \(\ind{R}(\overline{p}, f_1
       \cdot f_2, \sigma)\) holds. Then, \textsf{SUB}
       (Fig.~\ref{x_tree_matching_def}) and \ref{x:sound:15} imply
       \begin{align}
         \subst{\overline{p}}{\sigma} &\sqsubseteq f_1 \cdot f_2
          \sqsubseteq \cons{c(f_1)}{f_2} = f.
          \label{x:sound:I}
       \end{align}
        As a conclusion, the induction hypothesis and
        \eqref{x:sound:head} imply \eqref{x:sound:I}, so
        \(\ind{R}(\cons{l}{\overline{p}'}, \cons{c(f_1)}{f_2},
        \sigma)\) holds.\hfill \(\Box\)

\end{enumerate}
