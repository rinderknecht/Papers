\begin{abstract}

  Merge sort is an algorithm widely taught because it epitomises the
  solving strategy called \emph{divide and conquer}, the computational
  version of analysis and synthesis in mathematics, which is of
  general interest in computer programming. Merge sort comes in two
  common flavours, top-down and bottom-up, and an analysis of their
  asymptotic costs (number of comparisons) shows that both belong to
  the class \(O(n\log{n})\), where \(n\)~is the number of sorted
  keys. Neither precise bounds on their costs tell them apart because
  these bounds overlap, and students tend to confuse both
  variants. Using only induction and algebra, we show that the
  top-down variant is always to be preferred.

\end{abstract}
