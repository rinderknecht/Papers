\section{Definitions and Assumptions}

In this section, we gather the facts necessary to compare top-down and
bottom-up merge sorts, from our previous article in this
journal~\cite{Rinderknecht_2013}.

\subsection{Merging}

\emph{Merging} consists in combining two ordered series of keys into
one ordered series. Without loss of generality, we shall be only
interested in sorting keys in increasing order. One way to achieve
this consists in comparing the two smallest keys, output the smallest
and repeat the procedure until one of the series becomes empty, in
which case the other is wholly appended. 

By definition, the cost of merging is the number of comparisons
performed.

\paragraph{Minimum cost}
\label{merge:best_case}

Let \(\B{\Join}{m,n}\) be the minimum cost when merging series of
length \(m\)~and~\(n\). We have
\begin{equation}
\B{\Join}{m,n} = \min\{m,n\}.\label{eq:best_merge}
\end{equation}

\paragraph{Maximum cost}

The maximum cost \(\W{\Join}{m,n}\) when merging series of length
\(m\) and~\(n\) is
\begin{equation}
\W{\Join}{m,n} = m + n - 1.\label{eq:worst_merge}
\end{equation}

\paragraph{Average cost}

The average cost \(\M{\Join}{m,n}\) of merging series of size \(m\)
and~\(n\) is
\begin{equation}
\M{\Join}{m,n} = m + n - \frac{m}{n+1} - \frac{n}{m+1}.
\label{eq:mean_merge}
\end{equation}

\paragraph{Sorting}

Merging can be used to sort \emph{one} series of keys by splitting in
two the initial series of keys, then the two pieces are split again
etc. until only singletons remain. These are then merged pairwise
etc. until only one series remains, which is inductively sorted, since
a singleton is a sorted series on its own and the merger of~\(s\)
and~\(t\) is sorted if~\(s\) and~\(t\) are. This scheme leaves open
the choice of a splitting strategy:
\begin{itemize}

  \item \emph{top-down merge sort} consists in cutting a series of
  \(n\)~keys into two series of equal or almost equal lengths, that
  is, \(\ceiling{n/2}\) and \(\floor{n/2}\), where \(\floor{x}\)
  (\textsl{floor of~\(x\)}) is the greatest integer less than or equal
  to~\(x\), and \(\ceiling{x}\) (\textsl{ceiling of~\(x\)}) is the
  least integer greater than or equal to~\(x\);

  \item \emph{bottom-up merge sort} consists in cutting the series in
    two pieces of lengths \(2^{\ceiling{\lg n}-1}\) and
    \(n-2^{\ceiling{\lg n}-1}\), where \(\lg n\)~is the binary
    logarithm of~\(n\) --~in other words, we cut at the largest power
    of~\(2\) smaller than~\(n\).

\end{itemize}
The difference between the two variants can be easily seen in the
\fig~\ref{fig:comp}.
\begin{figure}
\centering
\subfloat[Bottom-up\label{fig:bot_up}]{%
\includegraphics{bot_up}}
\qquad
\subfloat[Top-down\label{fig:top_down}]{%
\includegraphics{top_down}}
\caption{Comparing merge sorts on \((6,3,2,4,1,5)\)\label{fig:comp}}
\end{figure}

\subsection{Sorting $2^n$ keys}

These two strategies agree in the case of \(2^p\)~keys, as seen in
\fig~\ref{fig:bot_up1}.
\begin{figure}[b]
\centering
\includegraphics[bb=71 633 244 721]{bot_up1}
\caption{Sorting \((7,3,5,1,6,8,4,2)\)
\label{fig:bot_up1}}
\end{figure}

\paragraph{Minimum cost}

Let us note \(\B{}{2^p}\) the minimum number of comparisons to sort
\(2^p\)~keys. We have
\begin{equation*}
\B{}{2^p} = p2^{p-1}.
\end{equation*}

\paragraph{Maximum cost}

Let~\(\W{}{2^p}\) be the maximum number of comparisons for sorting
\(2^p\)~keys. We have
\begin{equation*}
\W{}{2^p} = (p-1)2^p + 1.
\end{equation*}

\paragraph{Average cost}

Let \(\M{}{2^p}\) be the average number of comparisons for sorting
\(2^p\)~keys, and let \(\alpha := \sum_{k \geqslant 0}\frac{1}{2^k+1}
\simeq 1.264499\). We have
\begin{equation*}
\M{}{2^p}
  = p2^p - \alpha 2^p + \sum_{k \geqslant 0}\frac{1}{2^{k}+2^{-p}}.
\end{equation*}


\subsection{Top-down merge sort}

%% The cost \(\C{\top}{n}\) for top-down merge sort satisfies the
%% equations
%% \begin{equation}
%% \C{\top}{0} = \C{\top}{1} = 0,\quad
%% \C{\top}{n} = \C{\top}{\floor{n/2}}
%% + \C{\top}{\ceiling{n/2}}
%% + \C{\Join}{\floor{n/2},\ceiling{n/2}}.
%% \label{eq:Ctop}
%% \end{equation}

\paragraph{Minimum cost}

Let \(\B{\top}{n}\) be the minimum cost of using top-down merge
sort on \(n\)~keys. We have \(\B{\top}{n} = \B{\top}{\floor{n/2}} +
\B{\top}{\ceiling{n/2}} + \floor{n/2}\). Let \(\nu_k\)~be the number
of \(1\)-bits in the binary notation of~\(k\). We have
\begin{equation}
\B{\top}{n} = \sum_{k=0}^{n-1}{\nu_k}.
\label{eq:Btop}
\end{equation}

\paragraph{Maximum cost}

Let \(\W{\top}{n}\) be the maximum cost of using top-down merge
sort on \(n\)~keys. We have \(\W{\top}{n} = \W{\top}{\floor{n/2}} +
\W{\top}{\ceiling{n/2}} + n - 1\) and
\begin{equation}
\W{\top}{n} = n\ceiling{\lg n} - 2^{\ceiling{\lg n}} + 1.
\label{eq:Wtop}
\end{equation}

\paragraph{Average cost}

Let \(\M{\top}{n}\) be the average cost of using top-down merge
sort on \(n\)~keys. We have
\begin{equation}
\M{\top}{n} = \M{\top}{\floor{n/2}} +
\M{\top}{\ceiling{n/2}} + n -
\frac{\floor{n/2}}{\ceiling{n/2}+1} 
- \frac{\ceiling{n/2}}{\floor{n/2}+1}.
\label{eq:Mtop}
\end{equation}
