%%-*-latex-*-

\section{A stack-based algorithm}
\label{stack}

We seek to design an algorithm inputting a query and a stream, reading
the elements from the latter one by one and outputting as soon as
possible all the matches of the stream so far against the query
pattern. If no match is found after reading an element, the algorithm
can be run again on the remaining stream with a data structure that
keeps the intermediary results. It is an on\hyp{}line algorithm,
alternatively, it can be conceived as creating a stream of matches
from a stream of elements and a query.

\paragraph{The rightmost branch as a stack of elements}

Since only the rightmost branch changes throughout insertions, we can
keep it in memory instead of the whole \XML tree. The way it changes
also allows us to implement it as a stack, as Al-Khalifa et~al.\@
observed~\cite{AlKhalifa:2002}: the bottom of the stack is the root
element and the top is the end of the rightmost branch, which is
always the last inputted element. An extension is thus implemented by
simply pushing the new element on the stack; an expansion is achieved
by popping elements until an extension becomes possible (two phases).

\paragraph{Attributes of stack elements}

Each element in the stack is associated with sets, called
\emph{attributes}, containing \emph{partial matches}, that is,
incomplete matches of the query. The implementation of extensions and
expansions on a stack yields two kinds of attributes: the
\emph{inherited attributes} result from an extension (or the second
phase of an expansion), and the \emph{synthesised attributes} result
from the first phase of an expansion. (We reuse the terminology of
attributed grammars.) As an illustration, let us assume a query
\(\query{a}{b,c}\) and the current state of the \XML tree shown in
\fig{fig:initial_branch}. A complete match \([1 \mapsto a_1, 2 \mapsto
  b_1, 3 \mapsto c_2]\) appears as \(\mathbf{a_1b_1c_2}\) for the sake
of brevity. Element~\(d_1\) has no attribute since its kind is not in
the query. Element~\(a_1\) has no inherited attributes and it has some
synthesised attributes which are partial matches made of itself and
its contained elements which are not in the stack, that is,
\(c_1\)~and~\(b_1\). Element~\(c_2\) has some inherited attributes
made by combining itself with the synthesised attributes of its
ancestors in the stack, that is, \(a_1\)~and~\(d_1\), and it has no
synthesised attributes itself since it has no contained elements in
the stack (yet).

\paragraph{Stack specification}

Formally, let \proc{Empty} denote any empty stack; \(\proc{Push}
(\id{x}, S)\) the stack whose top element is~\id{x} and remaining
stack is~\(S\); \(\proc{Pop} (S)\) is the pair whose first component
is the top element of stack~\(S\) and the second is the remaining
stack, assuming \(S \neq \proc{Empty}\) (this definition amounts to
say that \proc{Pop} is the exact inverse function of \proc{Push}, that
is to say \(\id{S'} = \proc{Push} (\id{x}, S) \Leftrightarrow
\proc{Pop}(\id{S'}) = (\id{x}, S)\)). In order to model the rightmost
branch of an \XML tree, we need a stack whose elements are triples
\((\varepsilon, \iota, \sigma)\), where~\(\varepsilon\) is an element,
\(\iota\)~denote inherited attributes of~\(\varepsilon\),
and~\(\sigma\) are the synthesised attributes of~\(\varepsilon\). For
example, \(\proc{Push} ((d_1, \varnothing, \varnothing),
\proc{Empty})\) denotes the stack after reading~\(d_1\) from the
stream (namely, the first line of the table in
\fig{fig:stack_after_c2}).

\paragraph{Insertion}

Let us assume a query \(q = \query{\kappa_1}{\kappa_2, \dots,
  \kappa_n}\). Let \(\proc{Insert} (\varepsilon, S)\) be a pair
whose first component is a (possibly empty) series of matches and the
second component is the stack resulting from the insertion of
element~\(\varepsilon\) into the stack~\(S\).

\begin{wrapfigure}[9]{r}{0pt}
\centering
\(
\begin{array}{ccc}
\toprule
& \multicolumn{2}{c}{\text{Attributes}}\\
\cmidrule(l){2-3}
\text{Element} & \text{Inherited} & \text{Synthesised}\\
\toprule
d_1 & \varnothing & \varnothing\\
a_1 & \varnothing & a_1c_1, a_1b_1\\
c_2 & \mathbf{a_1b_1c_2}, a_1c_2 & \varnothing\\
\bottomrule
\end{array}
\)
\caption{Stack after adding~\(c_2\)\label{fig:stack_after_c2}}
\end{wrapfigure}
The (complete) matches are obtained by reading element~\(\varepsilon\)
from the input stream and combining it with previous partial matches,
while inserting it into the stack~\(S\). They are streamed out by
increasing roots and, if matching with order, by increasing
descendants, else the order is undefined (for matches with same
root). For the sake of brevity, we do not give here the loop that
calls \proc{Insert}, and instead focus directly on \proc{Insert}
itself. (We use the algorithmic language defined by Cormen et~al.\@ in
their textbook~\cite{Cormen:2001}. For additional clarity, we write
the code in static single\hyp{}assignment style.) For example,
consider \fig{fig:stack_after_c2}. If~\(S\) is the stack before
adding~\(c_2\), \(S'\) the stack after adding~\(c_2\), and
\(\mathcal{C}\) the complete matches using~\(c_2\), then \(({\cal
  C}, S') = \proc{Insert}(c_2, S)\), and \(\mathcal{C} = \{[1
  \mapsto a_1, 2 \mapsto b_1, 3 \mapsto c_2]\}\). Moreover,
\fig{fig:stack_after_c2} shows~\(S'\), except that, for the sake
of clarity, we will continue to show the matches in the stack in bold
\emph{even if they are actually streamed out}. For example, in
\fig{fig:stack_before_b2} and~\fig{fig:stack_after_b2}, the matches
\(\mathbf{a_1b_1c_2}\) and \(\mathbf{a_1b_2c_1}\) are shown inside the
stack as inherited attributes.
\begin{codebox}
\Procname{\(\proc{Insert}(\varepsilon, \id{S})\)}
\li \If \(\id{S} = \proc{Empty}\) \label{insert:li:empty_stack}
\RComment \(\varepsilon\) is the first element.
\li \Then \(\Return \, \varnothing, \proc{Push} ((\varepsilon, \varnothing,
                        \varnothing), \id{S})\)
          \label{insert:li:not_matches}
\RComment No match yet.
\End
\li \((\varepsilon_1, \iota_1, \sigma_1), \id{S_1} 
      \gets \proc{Pop}(\id{S})\) 
    \label{insert:li:first_pop}
\RComment Let us destructure the top of stack \(\id{S}\). 
\li \If \(\varepsilon \sqsubset \varepsilon_1\)
    \label{insert:li:containment_in_top}
\RComment Does the top~\(\varepsilon_1\) of the stack contain \(\varepsilon\)?
\li \Then \({\cal C}, {\cal P} \gets \proc{Complete} (\varepsilon, \id{S})\)
    \label{insert:li:complete_stack}
\RComment New partial and complete matches.
\li \Return \({\cal C}, \proc{Push} ((\varepsilon, {\cal P},
                                          \varnothing), 
                                         \id{S})\)
    \label{insert:li:extension}
\RComment Matches and rightmost extension.
    \End
\li  \((\varepsilon_2, \iota_2, \sigma_2), \id{S_2}
       \gets \proc{Pop} (\id{S_1})\)
     \label{insert:li:second_pop}
\RComment Destructure one more layer of stack \(\id{S}\).
\li \Return \(\proc{Insert}
  (\varepsilon, 
   \proc{Push} ((\varepsilon_2,
                 \iota_2,
                 \sigma_2 \cup \proc{Filter}(\varepsilon_1, 
                                             \iota_1 \cup \sigma_1),
                \id{S_2}))\)
     \label{insert:li:expansion}
\RComment Expansion.
\end{codebox}
\noindent (In the pseudo\hyp{}code, we write pairs without parentheses
when there is no ambiguity.) There are three cases when defining
\(\proc{Insert} (\varepsilon, S)\): (\emph{i}) the stack~\(S\)
is empty because \(\varepsilon\) is the first element in the stream,
at lines~\ref{insert:li:empty_stack}--\ref{insert:li:not_matches};
(\emph{ii}) the top element of~\(S\) contains~\(\varepsilon\) (which
implies the extension of the rightmost branch), at
lines~\ref{insert:li:containment_in_top}--\ref{insert:li:extension};
%\begin{wrapfigure}[7]{r}{0pt}
\begin{figure}[b]
\centering
\(
\begin{array}{c>{\hspace*{5pt}}c>{\hspace*{5pt}}c}
\toprule
& \multicolumn{2}{c}{\text{Attributes}}\\
\cmidrule(l){2-3}
\text{Element} & \text{Inherited} & \text{Synthesised}\\
\toprule
\varepsilon & \varnothing & \varnothing\\
\bottomrule
\end{array}
\)
\caption{\(\varepsilon\) is the first element}
\label{fig:first_elem}
%\end{wrapfigure}
\end{figure}
(\emph{iii}) \(\varepsilon\)~and the top of~\(S\) are disjoint (which
implies a rightmost expansion), at
lines~\ref{insert:li:second_pop}--\ref{insert:li:expansion}. If~\(S\)
is empty, the resulting stack only contains~\(\varepsilon\) with
neither inherited nor synthesised attributes, as shown in
\fig{fig:first_elem}. Otherwise, we pop up element~\(\varepsilon_1\)
with its inherited and synthesised attributes,
\(\iota_1\)~and~\(\sigma_1\), and we get the remaining
stack~\id{S_1}, at line~\ref{insert:li:first_pop}. If the top of the
stack, \(\varepsilon_1\), contains \(\varepsilon\)
(line~\ref{insert:li:containment_in_top}), then it means that we have
to extend the rightmost branch, which is achieved by pushing
\(\varepsilon\) on~\(S\) (line~\ref{insert:li:extension}). The
inherited attributes~\(\cal P\) of~\(\varepsilon\) are computed by
completing the partial matches in~\(S\) and complete matches can also
be found in the process, at line~\ref{insert:li:complete_stack}.  The
element~\(\varepsilon\) has no synthesised attributes since it
contains no elements yet. See the table in
\fig{fig:rightmost_extension_table}.
\begin{figure}
\centering
\(
\begin{array}{ccc}
\toprule
& \multicolumn{2}{c}{\text{Attributes}}\\
\cmidrule(l){2-3}
\text{Element} & \text{Inherited} & \text{Synthesised}\\
\toprule
\multicolumn{3}{c}{\framebox[55mm][c]{\id{S}}}\\
\varepsilon & \proc{Complete} (\varepsilon, \id{S}) & \varnothing\\
\bottomrule
\end{array}
\)
\caption{Rightmost extension}
\label{fig:rightmost_extension_table}
\end{figure}

Otherwise, the top of the stack does not contain~\(\varepsilon\),
which means that we have to perform a rightmost expansion. This
implies that the stack~\id{S_1} must be non\hyp{}empty; in other
words, \(S\)~must contain at least two elements. Indeed, a rightmost
expansion means that we grow another branch rooted on the rightmost
bran\-ch, but not at its end (else it would be an extension), so the
branch contains at least two nodes. Let us call~\(\varepsilon_2\) the
top element of~\id{S_1}, and~\(\iota_2\) and~\(\sigma_2\) its
inherited and synthesised attributes; \id{S_2}~is the remaining stack,
at line~\ref{insert:li:second_pop}. Please consider the table at
\fig{fig:rightmost_expansion_0}.
\begin{figure}[b]
\centering
\subfloat[Stack before expansion\label{fig:rightmost_expansion_0}]{
\(
\begin{array}{@{}c>{\qquad}cc@{}}
\toprule
& \multicolumn{2}{c}{\text{Attributes}}\\
\cmidrule(l){2-3}
\text{Element} & \text{Inherited} & \text{Synthesised}\\
\toprule
\multicolumn{3}{c}{\framebox[55mm][c]{\id{S_2}}}\\
\varepsilon_2 & \iota_2 & \sigma_2\\
\varepsilon_1 & \iota_1 & \sigma_1\\
\bottomrule
\end{array}
\)}
\;
\subfloat[Recursive expansion\label{fig:rightmost_expansion_1}]{
\(
\begin{array}{@{}c>{\qquad}cc@{}}
\toprule
& \multicolumn{2}{c}{\text{Attributes}}\\
\cmidrule(l){2-3}
\text{Element} & \text{Inherited} & \text{Synthesised}\\
\toprule
\multicolumn{3}{c}{\framebox[55mm][c]{\id{S_2}}}\\
\varepsilon_2 & \iota_2 & \sigma_2 \cup \iota_1 \cup \sigma_1\\
\\
\bottomrule
\end{array}
\)}
\caption{Rightmost expansion}
\end{figure}

The rightmost expansion is realised by means of an extension at the
node where the new rightmost branch is rooted, i.e., when the biggest
element containing the new element~\(\varepsilon\) is on the top of
the stack. Then the first step of the rightmost expansion consists in
inserting recursively~\(\varepsilon\) in a stack equal to~\(S\)
\emph{without its top element} \(\varepsilon_1\), until the condition
for an extension is satisfied. The attributes of~\(\varepsilon_1\) are
not lost: they are added (by a set union) to the synthesised
attributes of the new top element~\(\varepsilon_2\) (see
line~\ref{insert:li:expansion}). In terms of the \XML tree, this
amounts to move up the attributes of the rightmost leaf and cut this
leaf, and so on until an extension is possible. Check
\fig{fig:rightmost_expansion_1}.

As an example, consider the transition from \fig{fig:expansion} to
\fig{fig:extension} by inserting~\(b_2\). The stack before the
insertion is shown in \fig{fig:stack_before_b2}.
\begin{figure}
\centering
\subfloat[Stack before adding \(b_2\)\label{fig:stack_before_b2}]
         {\(
\begin{array}{ccc}
\toprule
& \multicolumn{2}{c}{\text{Attributes}}\\
\cmidrule(l){2-3}
\text{Element} & \text{Inherited} & \text{Synthesised}\\
\toprule
d_1 & \varnothing & \varnothing\\
a_1 & \varnothing & a_1c_1,a_1b_1 \\
c_2 & \mathbf{a_1b_1c_2},a_1c_2 & \varnothing\\
a_2 & \varnothing & \varnothing\\
\bottomrule
\end{array}
\)}
\;
\subfloat[Stack after adding \(b_2\).\label{fig:stack_after_b2}]
         {\(
\begin{array}{ccc}
\toprule
& \multicolumn{2}{c}{\text{Attributes}}\\
\cmidrule(l){2-3}
\text{Element} & \text{Inherited} & \text{Synthesised}\\
\toprule
d_1 & \varnothing & \varnothing\\
a_1 & \varnothing & a_1c_1,a_1b_1 \\
c_2 & \mathbf{a_1b_1c_2},a_1c_2 & \varnothing\\
b_2 & \mathbf{a_1b_2c_1},a_1b_2 & \varnothing\\
\bottomrule
\end{array}
\)}
\caption{The stack before and after inserting element~\(b_2\)}
\end{figure}
\Fig{fig:stack_after_b2} shows the result of the rightmost expansion
caused by the insertion of \(b_2\). Note that we assume in this figure
that we search also for unordered matches of \(\query{a}{b,c}\) thus
\([1 \mapsto a_1, 2 \mapsto b_2, 3 \mapsto c_1]\), i.e.,
\(\mathbf{a_1b_2c_1}\), is valid in spite of \(c_1 < b_2\).


\paragraph{Completion}

In order to understand how the inherited attributes are computed (in
case of a rightmost extension), we must now define precisely function
\proc{Complete}.
\begin{codebox}
\Procname{\(\proc{Complete} (\varepsilon, \id{S})\)}
\li\If \(\id{S} = \proc{Empty}\) \label{complete:li:empty_stack}
\RComment If the stack is empty
\li\Then \Return \(\varnothing, \varnothing\)
\label{complete:li:no_matches}
\RComment then there is no completion.
\End
\li\((\varepsilon_1, \iota_1, \sigma_1), \id{S_1}
          \gets \proc{Pop} (\id{S})\)
    \label{complete:li:pop}
\RComment Otherwise, destructure the top of the stack
\zi
\RComment and combine~\(\varepsilon\) with the synthesised attributes
\(\sigma_1\) of the top:
\li \If \(T(\varepsilon_1) = \kappa_1\)
\RComment if the top~\(\varepsilon_1\) matches the query root,
\label{complete:li:top_matches_root}
\li \Then \({\cal C}_1, {\cal P}_1
             \gets \proc{Combine} (\varepsilon, \sigma_1 \cup \{[1
             \mapsto \varepsilon_1]\})\)
\RComment it may combine with \(\varepsilon\)
\label{complete:li:add_combination}
\li \Else \({\cal C}_1, {\cal P}_1
             \gets \proc{Combine} (\varepsilon, \sigma_1)\)
\RComment or not.
    \label{complete:li:combine}
    \End
\li \({\cal C}_2, {\cal P}_2
       \gets \proc{Complete} (\varepsilon, \id{S_1})\)
    \label{complete:li:recursion}
\RComment Complete the remaining partial matches.
\li \Return \({\cal C}_1 \cup {\cal C}_2, {\cal P}_1 \cup {\cal P}_2\)
    \label{complete:li:union}
\RComment New complete and partial matches.
\end{codebox}
\noindent If stack~\(S\) is empty, then let us return no new matches
(lines~\ref{complete:li:empty_stack}--\ref{complete:li:no_matches}). Otherwise,
let us pop element~\(\varepsilon_1\), whose synthesised attributes
are~\(\sigma_1\), and let~\id{S_1} be the remaining stack, at
line~\ref{complete:li:pop}. The next step is to try to
combine~\(\varepsilon\) with the partial matches in~\(\sigma_1\) and
get new matches~\(\mathcal{C}_1\) and new partial matches~\({\cal
  P}_1\) (line~\ref{complete:li:combine}) but we must not forget a
possible new partial match involving only~\(\varepsilon\)
and~\(\varepsilon_1\). Thus we first check whether the top of the
stack, \(\varepsilon\), matches the query root, at
line~\ref{complete:li:top_matches_root}. If so, we also must try to
combine \(\varepsilon_1\) and \(\varepsilon\), at
line~\ref{complete:li:add_combination}. For example, consider the
partial match \(a_1c_2\) in \fig{fig:stack_after_c2}. Next, we
complete the remaining stack \id{S_1} with \(\varepsilon\) and obtain
new matches \({\cal C}_2\) and new partial matches \({\cal P}_2\)
(line~\ref{complete:li:recursion}). We merge these new matches and we
merge separately the partial matches (line~\ref{complete:li:union}).


\paragraph{Combination}

Let us define now \proc{Combine}, which is the function that tries to
complete a set of partial matches with a given element. Since we deal
here with partial matches, we must be more precise here. Let us note
\(\cal D\) the function that returns the kind indexes of a given match
(the \emph{domain} of the match). For example, if \(q =
\query{a}{b,c}\) and \(\mu_q = [1 \mapsto a_1, 2 \mapsto b_1, 3
  \mapsto c_2]\), then \({\cal D}(\mu_q) = \{1,2,3\}\). Let us note
\(\overline{\cal D}(\mu_q)\) the complementary set, e.g., if \(\mu_q = [1 \mapsto a_1, 3 \mapsto c_2]\), then \({\cal D}(\mu_q) = \{1,3\}\) and \(\overline{\cal D}(\mu_q) = \{2\}\). Then the match~\(\mu_q\) is
complete if and only if \(\overline{\cal D}(\mu_q) =
\varnothing\). Let us note \(\mu_q \oplus i \mapsto \varepsilon\) the
extension of a match~\(\mu_q\) with the binding \(i \mapsto
\varepsilon\). The operator~\(\oplus\) can be formally defined, for
all \(j \in {\cal D}(\mu_q) \cup \{i\}\), as
\begin{equation*}
(\mu_q \oplus i \mapsto \varepsilon)(j) \triangleq
\left\{
\begin{aligned}
\varepsilon && \text{if} \, i = j\\
\mu_q(j)    && \text{otherwise}
\end{aligned}
\right.
\end{equation*}
Moreover, let us assume we have a function \proc{Choose} that takes a
set of matches and returns one of them paired with the complementary
matches. (This is a way to defer to the implementation the actual
iteration order.) The following definition of \proc{Combine} returns
both ordered and unordered matches. For ordered matches only, see
further.
\begin{codebox}
\Procname{\(\proc{Combine} (\varepsilon, \sigma)\)}
\li	\If \(\sigma = \varnothing\)
\label{combine:li:empty}
\RComment If there are no partial matches
\li	\Then \Return \(\varnothing, \varnothing\)
\label{combine:li:nothing}
\RComment then return no new matches.
	\End
\li	\(\mu_q, \sigma' \gets \proc{Choose} (\sigma)\)
\label{combine:li:arbitrary_selection}
\RComment Pick a partial match~\(\mu_q\); the remaining ones are \(\sigma'\).
\li \({\cal C}, {\cal P} \gets 
      \proc{Combine} (\varepsilon, \sigma')\)
\label{combine:li:recursion}
\RComment Combine \(\varepsilon\) with the remaining partial matches.
\li	\If \(\exists i \in \overline{\cal D}(\mu_q).T(\varepsilon) = \kappa_i\)
\label{combine:li:new_match}
\RComment If~\(\varepsilon\) completes \(\mu_q\) at index \(i\),
i.e., \(\varepsilon\) matches \(\kappa_i\),
\li	\Then \If \(|\overline{\cal D}(\mu_q)| = 1\)
\label{combine:li:one_index_remaining}
\RComment then if~\(i\) was the sole unused index
\li       \Then \Return \({\cal C} \cup \{\mu_q \oplus i \mapsto
                          \varepsilon\}, {\cal P}\)
\label{combine:li:make_complete_match}
\RComment we make a complete match,
          \End
\li       \Return \({\cal C}, 
                     {\cal P} \cup \{\mu_q \oplus i \mapsto
                     \varepsilon\}\)
\label{combine:li:make_partial_match}
\RComment else it is a partial match.
\End
\li\Return \({\cal C}, {\cal P}\)
\label{combine:li:ignore}
\RComment If \(\varepsilon\) does not extend \(\mu_q\), ignore the
partial match \(\mu_q\).
\end{codebox}
If the set of partial matches \(\sigma\) is empty
(line~\ref{combine:li:empty}), then let us return an empty set of new
matches (line~\ref{combine:li:nothing}). Otherwise, let us arbitrarily
choose any partial match \(\mu_q\) in \(\sigma\) and name the
remaining partial matches \(\sigma'\)
(line~\ref{combine:li:arbitrary_selection}). Next, let us recursively
combine \(\varepsilon\) with the remaining partial matches
(line~\ref{combine:li:recursion}). If \(\mu_q\) can be extended by
\(\varepsilon\), that is to say, at least one of its index \(i\) is
unused and \(\varepsilon\) matches \(\kappa_i\)
(line~\ref{combine:li:new_match}), then a new partial or complete
match can be made. If index \(i\) was the last unused in \(\mu_q\)
(line~\ref{combine:li:one_index_remaining}), then \(\mu_q \oplus i
\mapsto \varepsilon\) is a complete match
(line~\ref{combine:li:make_complete_match}), otherwise it is partial
(line~\ref{combine:li:make_partial_match}). If \(\varepsilon\) can not
extend \(\mu_q\) (either because it matches no node of the query or
because the sole matching nodes in the query are already matched by
nodes of the \XML tree read so far), we just ignore it
(line~\ref{combine:li:ignore}).

%\paragraph{Sharing roots.}

%\paragraph{Ordered matches only.}
