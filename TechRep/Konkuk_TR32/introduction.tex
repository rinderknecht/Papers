%%-*-latex-*-

\section{Introduction}

\noindent The \XPath expression \texttt{\small a//b[//c]} is
interpreted as ``\texttt{\small a//b} and \texttt{\small a//c}'', if
we omit the unique target semantics, according to which \XPath
expressions evaluate to a 
\begin{wrapfigure}[6]{l}{40pt}
\centering
\includegraphics[bb=71 694 100 721]{binary_query}
\caption{\textsf{\small a//b[//c]}\label{fig:binary_query}}
\end{wrapfigure}
sequence of nodes with same tag, here~\texttt{\small b}. Many research
papers~\cite{Cooper:2001,Li:2001,Zhang:2001} are devoted to matching
efficiently these expressions, considered as tree patterns or queries,
against \XML trees as a whole or streams of \XML elements, either by
solving the binary joins and then stitching them back to answer the
original query~\cite{AlKhalifa:2002,Chien:2002,Jiang:1:2003,Wu:2003}
or by considering them as a
whole~\cite{Bruno:2002,Jiang:2:2003,Jiao:2005}.  The standard
interpretation of the \XPath expressions implies that, in the previous
example, \texttt{\small b} may be an ancestor or a descendant
of~\texttt{\small c}, but it is striking that most authors assume in
their examples that the nodes matched by~\texttt{\small b}
and~\texttt{\small c} are \emph{not} in an ancestor or descendant
relationship. We surmise that this tacit assumption for picking
examples is often closer to what the query specifier or reader has in
mind, rather than the usual interpretation. In other words, the reader
of an \XPath query probably tends to conceive the query as a tree, as
in \fig{fig:binary_query}, and to assume that the matching subtree is
isomorphic, instead of reading the query as a set of binary joins
that, once stitched together, may yield a degenerate case, that is, a
path. In particular, the expression \texttt{\small a//b[//b]} is
equivalent to \texttt{\small a//b} and there is no way in \XPath to
actually denote two different \texttt{\small b} nodes on two divergent
paths, i.e., no path including the other, as the graphic
representation seems to suggest.

But can we rely solely on this psychological assumption to convey
another interpretation of the \XPath expressions? We believe that,
beyond the useful purpose of avoiding misunderstandings by making more
assumptions explicit, it is indeed useful to suppose sometimes that
the matched subtree is isomorphic to the query tree, because this
allows some additional constraints to be implicitly included in the
query.

As an illustration, consider in \fig{fig:mathml}
\begin{figure}
\centering
\subfloat[\(\cos x - \sin y \)\label{fig:cos_minus_sin}]
         {\includegraphics[bb=71 632 168 721]{cos_minus_sin}}
\qquad
\subfloat[\(\cos(\sin x) - y\)\label{fig:cos_sin_minus}]
         {\includegraphics[bb=71 632 142 721]{cos_sin_minus}}
\qquad
\subfloat[Pattern with variables \(a\)~and~\(b\). \label{fig:mathml_query}]%
         {\includegraphics[bb=63 632 187 721]{mathml_query}}
\caption{Two \MathML trees and a query.\label{fig:mathml}}
\end{figure}
the \MathML trees corresponding to the expressions \(\cos x - \sin y\)
and \(\cos(\sin x) - y\). These trees have very different
interpretations, despite having a structure close enough to be matched
by the same pattern. By interpreting the query pattern in a more
restrictive way than in \XPath, we can get less matches. For example,
if we want to match the difference between a possibly nested sine
application and a possibly nested cosine application, \emph{but not
  their composition}, we can say that the two nodes \textsf{apply} in
the query in \fig{fig:mathml_query} cannot match nodes which are in a
descendant or ancestor relationship, and then the query only matches
the tree in \fig{fig:cos_minus_sin}. Furthermore, since we know that
the subtraction is not commutative when interpreting these formulas,
we can impose that the two matched nodes \textsf{apply} must follow
the \emph{same order} as in the pattern, i.e., the sibling order,
yielding here no match at all.

The paper is structured as follows. We start with the next
section~\ref{problem} by stating mode precisely the problem we want to
solve; then we propose a naive solution in section~\ref{naive};
follows a stack\hyp{}based algorithm, in section~\ref{stack}, and an
index\hyp{}based procedure, in section~\ref{table}; this paper being
concluded in section~\ref{conclusion} with a short comparison of the
proposed algorithms and possible future work.
