%%-*-latex-*-

\section{Naive approach}
\label{naive}
 
\noindent Given a query \(\query{\kappa_1}{\kappa_2, \dots,
  \kappa_n}\), a naive approach to solve our problem consists in
storing the elements read so far in lists according to their kind (as
in the multiple streams framework) and, when a new element is read, we
check for all the combinations of elements that make up a match. This
procedure is inefficient because it reconsiders previous invalid
combinations after an element is read. We can easily improve it by
storing all the partial matches found so far and then trying to
complete \emph{only them} with one more element each time, until a
complete match is found. For example, let us reconsider the stream
[\(d_1\), \(a_1\), \(c_1\), \(b_1\), \(c_2\), \(a_2\), \(b_2\),
  \(d_2\), \(a_3\), \(b_3\), \(c_3\)] from the \XML tree in
\fig{fig:xml_tree}, and let us seek both ordered and unordered matches
to the query \(\query{a}{b,c}\). In \fig{fig:input_lists}, the first
line corresponds to the
\begin{wrapfigure}[8]{r}{0pt}
\centering
\fbox{\(
\begin{array}{c|c|c|c|c|c|c}
\varnothing & a_1 & a_1 & a_1 & a_1 & a_1,a_2 & a_1, a_2\\
\varnothing & \varnothing & \varnothing & b_1 & b_1 & b_1 & b_1,b_2\\
\varnothing & \varnothing & c_1 & c_1 & c_1,c_2 & c_1,c_2 & c_1,c_2\\
d_1 & d_1 & d_1 & d_1 & d_1 & d_1 & d_1
\end{array}
\)
}
\caption{Input lists\label{fig:input_lists}}
\end{wrapfigure}
sorted list of \(a\)-elements read so far, the second to the
\(b\)-elements etc. Each column shows the state of the lists after
each element is read from the input stream.  At each step, i.e., for
each column, the new valid combinations are respectively
\(\varnothing\), \{\(a_1\)\}, \{\(a_1\), \(a_1c_1\)\}, \{\(a_1\),
\(a_1c_1\), \(a_1b_1\)\}, \{\(a_1\), \(a_1c_1\), \(a_1b_1\),
\(\mathbf{a_1b_1c_2}\), \(a_1c_2\)\}, \{\(a_1\), \(a_1c_1\),
\(a_1b_1\), \(\mathbf{a_1b_1c_2}\), \(a_1c_2\), \(a_2\)\}, \{\(a_1\),
\(a_1c_1\), \(a_1b_1\), \(\mathbf{a_1b_1c_2}\), \(a_2\), \(a_1c_2\),
\(\mathbf{a_1b_2c_1}\), \(a_1b_2\)\} ---~solutions in bold. Note that,
since the kinds in this query are pairwise distinct, we can simplify
the notation for complete and partial matches by just enumerating
elements, like \(a_1c_1\) and \(a_1b_2\) etc. The drawbacks of this
procedure are that it uses too much memory and it is too slow. Indeed,
it needs to keep in memory the complete lists of elements and it tries
to form new combinations with elements which are not disjoint, like
\(a_1b_1c_1\) in our example (\(b_1 \sqsubset c_1\)).
