%%-*-latex-*-

\section{Context and problem}
\label{problem} 

\paragraph{Streaming of \XML elements} 

A \emph{tag} is a pair made of a \emph{kind}, noted~\(\kappa\), and a
status ``opening'' or ``closing''. An \emph{element}, noted
\(\varepsilon\), is a triple made of an opening tag, some contents and
a closing tag of the same kind as the opening tag. The contents is
made of a series of plain text and other (embedded) elements. Elements
can be empty, in which case the contents is empty. By extension, an
element is said to be of kind~\(\kappa\) if its tags are of
kind~\(\kappa\). Let us define the function~\(K\) on elements which
gives the kind of its argument. A set of kinds is noted~\(\cal
K\). Kinds are totally ordered. For example, let us consider the
following original \XML file (left column) where there is only one tag
per numbered line.
{\small\begin{alltt}
 \(1\) <c> \emph{text\(\sb{1}\)}                          <c range="1--22"> \emph{text\(\sb{1}\)} </c>
 \(2\)   <a> \emph{text\(\sb{2}\)}                        <a range="2--13"> \emph{text\(\sb{2}\)} </a>
 \(3\)     <b> \emph{text\(\sb{3}\)}                      <b range="3--6"> \emph{text\(\sb{3}\)} </b>
 \(4\)       <a> \emph{text\(\sb{4}\)}                    <a range="4--5"> \emph{text\(\sb{4}\)} </a>
 \(5\)       </a>              
 \(6\)     </b>
 \(7\)     <b> \emph{text\(\sb{5}\)}                      <b range="7--8"> \emph{text\(\sb{5}\)} </b>
 \(8\)     </b>
 \(9\)     <c> \emph{text\(\sb{6}\)}                      <c range="9--12"> \emph{text\(\sb{6}\)} </c>
\(10\)       <a> \emph{text\(\sb{7}\)}                    <a range="10--11"> \emph{text\(\sb{7}\)} </a>
\(11\)       </a>
\(12\)     </c>
\(13\)   </a>
\(14\)   <a> \emph{text\(\sb{8}\)}                        <a range="14--15"> \emph{text\(\sb{8}\)} </a>
\(15\)   </a>
\(16\)   <b> \emph{text\(\sb{9}\)}                        <b range="16--21"> \emph{text\(\sb{9}\)} </b>
\(17\)     <c> \emph{text\(\sb{10}\)}                     <c range="17--20"> \emph{text\(\sb{10}\)} </c>
\(18\)       <b> \emph{text\(\sb{11}\)}                   <b range="18--19"> \emph{text\(\sb{11}\)} </b>
\(19\)       </b>
\(20\)     </c>
\(21\)   </b>
\(22\) </c>
\end{alltt}}
\noindent Here, the kinds of elements are~\(a\), \(b\)~and~\(c\),
i.e., \({\cal K} = \{a, b, c\}\). Elements are streamed by increasing
order of their opening tag line number in the \XML file, as shown in
the right column, where we recorded the line numbers of the opening
tags and the matching closing tags into an attribute \texttt{\small
  range}. This scheme works if we assume one tag per line, otherwise
tags must be numbered by order of appearance in the file. Finally, let
us assume that there is a function~\(T\) which takes an element and
returns its textual contents (a series of pieces of text corresponding
to the text nodes). This allows us to abstract away the text in the
elements.

Let \(L(\varepsilon)\) and \(U(\varepsilon)\) be respectively the
lower bound of the range of element~\(\varepsilon\) and the upper
bound of the range of~\(\varepsilon\). By definition, element
\(\varepsilon_1\) is said to be \emph{lower than} element
\(\varepsilon_2\), noted \(\varepsilon_1 < \varepsilon_2\), if and
only if \(L(\varepsilon_1) < L(\varepsilon_2)\). Elements are streamed
by increasing lower bounds of their range.

Al\hyp{}Khalifa et al.~\cite{AlKhalifa:2002} assume that the
database provides multiple streams of sorted elements of the same kind
of tag. In our example, we have the following three streams: [\(a_1\),
\(a_2\), \(a_3\), \(a_4\)], [\(b_1\), \(b_2\), \(b_3\), \(b_4\)] and
[\(c_1\), \(c_2\), \(c_3\)]. We can interleave these streams into one
stream by picking repeatedly the minimum element of the streams, like
merge sort: [\(c_1\), \(a_1\), \(b_1\), \(a_2\), \(b_2\), \(c_2\),
\(a_3\), \(a_4\), \(b_3\), \(c_3\), \(b_4\)]. (This unique stream does
not need to be statically constructed by the database.) In this paper,
we prefer to assume a unique stream of sorted elements, although, for
the sake of clarity, we keep the subscripting of tags as in the
multiple\hyp{}streams framework. The following table on the left side
shows the stream as the interleaving of multiple streams, whilst the
right table displays the same stream with a single\hyp{}stream view,
i.e., considering a generic series of elements [\(\varepsilon_1\),
\(\varepsilon_2\), \(\dots\), \(\varepsilon_{11}\)], with a
supplementary column \(K\) for the kind of element.
\begin{equation*}
\begin{array}{|c||c|c|c|c|c||c|c|c|c|}
\hhline{~---~~----}
\multicolumn{1}{c|}{} & T & L & U &
\multicolumn{1}{c}{} & \multicolumn{1}{c|}{} & K & T & L & U\\
\hhline{-::===~-::====}
c_1 & \emph{text}_1  & 1 & 22 & & \varepsilon_1 & c & \emph{text}_1 & 1 & 22\\
\hhline{-||---~-||----}
a_1 & \emph{text}_2  & 2 & 13 & & \varepsilon_2 & a & \emph{text}_2 & 2 & 13\\
\hhline{-||---~-||----}
b_1 & \emph{text}_3  & 3 & 6 & & \varepsilon_3 & b & \emph{text}_3 & 3 & 6\\
\hhline{-||---~-||----}
a_2 & \emph{text}_4  & 4 & 5 & & \varepsilon_4 & a & \emph{text}_4 & 4 & 5\\
\hhline{-||---~-||----}
b_2 & \emph{text}_5  & 7 & 8 & & \varepsilon_5 & b & \emph{text}_5 & 7 & 8\\
\hhline{-||---~-||----}
c_2 & \emph{text}_6  & 9 & 12 & & \varepsilon_6 & c & \emph{text}_6 & 9 & 12\\
\hhline{-||---~-||----}
a_3 & \emph{text}_7  & 10 & 11 & & \varepsilon_7 & a & \emph{text}_7 & 10 & 11\\
\hhline{-||---~-||----}
a_4 & \emph{text}_8  & 14 & 15 & & \varepsilon_8 & a & \emph{text}_8 & 14 & 15\\
\hhline{-||---~-||----}
b_3 & \emph{text}_9  & 16 & 21 & & \varepsilon_9 & b & \emph{text}_9 & 16 & 21\\
\hhline{-||---~-||----}
c_3 & \emph{text}_{10} & 17 & 20 & & \varepsilon_{10} & c & \emph{text}_{10} & 17 & 20\\
\hhline{-||---~-||----}
b_4 & \emph{text}_{11} & 18 & 19 & & \varepsilon_{11} & b & \emph{text}_{11} & 18 & 19\\
\hhline{-||---~-||----}
\end{array}
\end{equation*}


\paragraph{Containment and disjointedness}

\hspace*{-2pt} Since the stream of elements is computed from a valid
\XML file, any two elements in the stream are either in a
\emph{containment} or else a \emph{disjointedness}
(non\hyp{}overlapping) relationship. By definition, an element
\(\varepsilon_1\) is contained in \(\varepsilon_2\) (or ``is a
descendant of~\(\varepsilon_2\)''), noted \(\varepsilon_1 \sqsubset
\varepsilon_2\), if \(L(\varepsilon_2) < L(\varepsilon_1)\) and
\(U(\varepsilon_1) < U(\varepsilon_2)\). Containment is neither
reflexive (it is strict inclusion) nor symmetric, but it is
transitive. Elements \(\varepsilon_1\) and \(\varepsilon_2\) are
disjoint, noted \(\varepsilon_1 \, \sharp \, \varepsilon_2\), if and
only if \(U(\varepsilon_1) < L(\varepsilon_2)\) or \(U(\varepsilon_2)
< L(\varepsilon_1)\), i.e., \(\varepsilon_1 \nsqsubset \varepsilon_2\)
and \(\varepsilon_2 \nsqsubset \varepsilon_1\). Disjointedness is
neither reflexive nor transitive, but it is symmetric.


\begin{wrapfigure}[7]{r}{0pt}
\fbox{\includegraphics[bb=71 693 142 721]{pattern}}
\caption{Query \(\query{\kappa_1}{\kappa_2, \dots, \kappa_n}\)}
\label{fig:pattern}
\end{wrapfigure}

\paragraph{Problem statement}

%% Let~\(\cal A\) be the function that returns the arity of a query,
%% i.e., the number of kinds it contains.

Let us note \(\cal E\) the infinite set of all possible \XML elements.
A \emph{pattern}, or \emph{query}, graphically represented in
\fig{fig:pattern}, is a tuple of kinds~\(\kappa_i\), noted
\(\query{\kappa_1}{\kappa_2, \dots, \kappa_n}\). A \emph{stream} of
\XML elements is a series [\(\varepsilon_1\), \(\varepsilon_2\),
  \(\dots\)], such that \(i < j \Rightarrow \varepsilon_i <
\varepsilon_j\). A \emph{complete match} of a query~\(q\),
noted~\(\mu_q\), is a finite mapping from the indexes of the kinds in
the query \(q = \query{\kappa_1}{\kappa_2, \dots, \kappa_n}\) to
\(\cal E\), such that
\begin{align}
1 \leqslant i \leqslant n & \Rightarrow K \circ \mu_q(i) = \kappa_i,\label{match:tags}\\
2 \leqslant i \leqslant n & \Rightarrow \mu_q(i) \sqsubset \mu_q(1),\label{match:desc}\\
2 \leqslant i,j \leqslant n & \Rightarrow \mu_q(i) \nsqsubset \mu_q(j).\label{match:div}
\end{align}
For example, let \(q = \query{a}{b,b}\) and the input stream [\(d_1\),
  \(a_1\), \(c_1\), \(b_1\), \(c_2\), \(a_2\), \(b_2\)], then there is
a match \(\mu_q\) which satisfies \(\mu_q(1) = a_1\), \(\mu_q(2) =
b_1\), \(\mu_q(3) = b_2\). Alternatively, we can write instead \(\mu_q
= [1 \mapsto a_1, 2 \mapsto b_1, 3 \mapsto b_2]\) (the order of the
bindings in the map is not meaningful). This is an \emph{ordered
  match} because it enjoys the additional condition: \(i < j
\Rightarrow \mu_q(i) < \mu_q(j)\). Otherwise, a match is said to be
\emph{unordered}. For example, \([1 \mapsto a_1, 2 \mapsto b_2, 3
  \mapsto b_1]\) is an unordered match. Conditions~\eqref{match:tags}
and~\eqref{match:desc} are the standard interpretation of \XPath
queries. We add here condition~\eqref{match:div}, which constrains the
elements matching the pattern descendants to be pairwise disjoint.

This problem is perhaps more intuitive as a property on \XML trees.

\begin{wrapfigure}[11]{r}{0pt}
\centering
\fbox{\includegraphics[bb=71 651 141 721]{xml_tree}}
\caption{\XML tree with tags \(a\), \(b\), \(c\) and \(d\)}
\label{fig:xml_tree}
\end{wrapfigure}

\paragraph{The \XML trees}

We make no assumption here on whether the outgoing stream is generated
or not from an \XML file actually stored in the database, but thinking
in terms of that tree is intuitive. For example, \fig{fig:xml_tree}
displays a tree from which the stream [\(d_1\), \(a_1\), \(c_1\),
  \(b_1\), \(c_2\), \(a_2\), \(b_2\), \(d_2\), \(a_3\), \(b_3\),
  \(c_3\)] is produced by a preorder traversal. Note that two elements
are in a containment relationship if and only if their corresponding
nodes both belong to the same rooted path, i.e., a path including the
root. Now, let us assume the tree in \fig{fig:xml_tree}, and a query
\(\query{a}{b,c}\), then the unique ordered match is \([1 \mapsto a_1,
  2 \mapsto b_1, 3 \mapsto c_2]\).
We can think of a match as a map from the nodes of the query
 tree to some nodes of the \XML tree (condition~\eqref{match:tags})
 such that the descendants in the query tree are mapped to descendants
 of the mapping of the query root (condition~\eqref{match:desc}) and
 such that these descendants are located in paths diverging from the
 mapping of the query root (condition~\eqref{match:div}). Coming back
 to our example, if we are interested in the unordered matches as
 well, we must add \([1 \mapsto a_1, 2 \mapsto b_2, 3 \mapsto
 c_1]\). By contrast, if we interpret the query as in \XPath, i.e.,
 allowing containment between descendants, we would have to add \([1
 \mapsto a_1, 2 \mapsto b_1, 3 \mapsto c_1]\), \([1 \mapsto a_1, 2
 \mapsto b_2, 3 \mapsto c_2]\) and \([1 \mapsto a_3, 2 \mapsto b_3, 3
 \mapsto c_3]\).

\begin{wrapfigure}[10]{r}{0pt}
\vspace{-\baselineskip}
\centering
\subfloat[Initial]{
  \fbox{\includegraphics[bb=71 651 104 721]{rightmost_branch}}
  \label{fig:initial_branch}}
\quad
\subfloat[Extension]{
  \fbox{\includegraphics[bb=71 651 118 721]{rightmost_extension}}
  \label{fig:extension}}
\quad
\subfloat[Expansion]{
  \fbox{\includegraphics[bb=71 651 122 721]{rightmost_expansion}}
  \label{fig:expansion}}
\caption{Rightmost branches (in bold)\label{fig:rightmost_branches}}
\end{wrapfigure}

\paragraph{Rightmost branches}

The \emph{rightmost branch} of a tree is the longest rooted path made
of the successive rightmost children. In the example at
\fig{fig:xml_tree}, the rightmost branch is [\(d_1\), \(a_3\),
  \(b_3\), \(c_3\)]. Adding a new node, i.e., the next available
element in the input stream, changes only the rightmost branch because
the elements are ordered by increasing positions of their opening tags
(preorder traversal). Consider previous stages of our \XML tree in
\fig{fig:rightmost_branches}: before adding~\(a_2\)
(\fig{fig:initial_branch}), after adding~\(a_2\) (\fig{fig:extension})
and after adding~\(b_2\) (\fig{fig:expansion}). This sequence
illustrates the fact that inserting a node either extends the
rightmost branch (as adding~\(a_2\)) or creates a new branch rooted in
the previous rightmost branch (as adding~\(b_2\)). We call the first
case a \emph{rightmost extension} and the second is known as
\emph{rightmost expansion} of a tree in the mining literature (see
Asai et al.~\cite{Asai:2003}).

%% \begin{figure}
%% {\footnotesize
%% \begin{verbatim}
%% let $doc := fn:doc("file:///tmp/doc.xml")
%% for $a in $doc//a,
%%     $b in $a//b,
%%     $c in $a//c except $a//$b//c
%%                 except $a//c[descendant::b=$b
%%                              and descendant::b/@order=$b/@order]
%% return <disjoint_elements>
%%        {(element {fn:node-name($a)}{$a/@order},
%%          element {fn:node-name($b)}{$b/@order},
%%          element {fn:node-name($c)}{$c/@order})}
%%        </disjoint_elements>
%% \end{verbatim}
%% }
%% \caption{\XQuery encoding of \(\query{a}{b, c}\)}
%% \label{problem:xquery}
%% \end{figure}


%% \paragraph{\XQuery}

%% The kind of query we propose to answer in this article cannot be
%% expressed by the standard semantics of \XPath, but it can be expressed
%% in \XQuery. For example, consider the query \(\query{a}{b, c}\) in
%% \fig{fig:binary_query}. It can be matched against an \XML file named
%% \texttt{\small /tmp/doc.xml} corresponding to the \XML tree of
%% \fig{fig:xml_tree} in the following manner, assuming that the
%% attribute \texttt{\small order} records the subscripts, e.g.,
%% \texttt{\small <a order="3"> ... </a>} stands for~\(a_3\). See
%% \fig{problem:xquery}.
