%%-*-latex-*-

\section{Coding and decoding}

\textbf{BER codes.} The structure of a BER code is based on the triple
(\emph{tag}, \emph{length}, \emph{contents}). The \emph{tag} field
corresponds to the tag of the value type in \ASN, the \emph{length} is
the length of the contents field and the \emph{contents} field is
either another code (in which case the code is said
\emph{constructed}) or the encoding of a primitive type (in which case
the code is said \emph{primitive}). A primitive type is an \ASN
built-in type which is not defined in terms of other types,
\emph{e.g.,} the \texttt{\small INTEGER} type. If the contents length
is unknown at the encoding-time, it is possible for the coder to
provide a special dummy length and then close the code with an
\emph{ending octet}, in which case the code is said to be in
\emph{indefinite form}, as opposed to \emph{definite form}. Definite
form requires that the sender computes the whole code before sending
it (in order to be able to compute the contents length) and it allows
the receiver to allocate a bounded amount of memory to store the
incoming code. The indefinite form allows the sender to encode the
value coming from the upper application as it comes throughout a
buffer (i.e. faster encoding within a bounded space) but it requires
the receiver to handle carefully the incoming stack size. Indeed, the
BER codes have a recursive structure and one of the advertised
vulnerabilities was due to a deeply embedded code in indefinite form
which overflowed the receiver's stack because the
\emph{implementation} was mishandling the memory.

\medskip

\textbf{Abstract BER codes.} A complete formalisation of the BER first
requires a model of the codes \emph{at the octet level}, by means of a
context-free grammar for instance, and the proof of some relevant
properties on it. For example, from a soundness point of view, it is
important to prove that the grammar is not ambiguous, i.e. a given
code cannot be described in more than one way (exactly one derivation
tree); from the decoder's efficiency point of view, it is important to
prove that the grammar can be recursively analysed without
backtracking and with a small constant amount of
look-ahead. Unfortunately, due to the limited room, we have to skip
this interesting stage. We shall assume that we already deal with
\emph{abstract codes}, which correspond to the abstract syntax trees
of the compilers: an abstract code does not model the octets, but
rather the structure of the codes. As a consequence, the length field
is not included in an abstract code since, conceptually, an abstract
code is a tree, not a series as the original codes. Moreover, the
concepts of definite and indefinite form are not relevant for abstract
codes, since they apply to octet streams only. The abstract codes are
thus modeled with an OCaml type since these types correspond to trees
with user-defined nodes and leaves.

\vspace*{-10pt}

\begin{tabbing}
\Xtype \type{primitive\_code} \equal \= \Pint{} \vbar~\Preal{}
\vbar~\Pminusinf{} \vbar~\Pplusinf \vbar~\Pstring{}\\ 
\> \vbar~\Pbitstr{} \vbar~\Pbool{} \Xof \type{int}
\vbar~\Pnull\\
\Xtype \type{code} \equal \lpar\type{tag\_class} $\times$
\type{int}\rpar{} $\times$ \type{contents}\\ 
\Xand \type{contents} \equal \Primitive{} \Xof \type{primitive\_code}
\vbar~\Constructed{} \Xof \type{code} \type{list}
\end{tabbing}

\vspace*{-10pt}

The type \type{primitive\_code} captures the codes of the values from
types \texttt{\small INTEGER}, \texttt{\small REAL}, \texttt{\small
  BIT STRING}, \texttt{\small OCTET STRING}, \texttt{\small BOOLEAN},
\texttt{\small NULL} and the numerous character string types. The
abstract primitive codes carry little discriminative information for a
given type; for example, \emph{all} the \texttt{\small INTEGER} values
are encoded into the same abstract code \Pint, but codes of
\texttt{\small REAL} values are still different (\Preal). This way we
abstract away octet-level details which would otherwise bring us too
far. We nevertheless keep the \texttt{\small BOOLEAN} standard
encoding: value \texttt{\small FALSE} is encoded as \lpar\cst{Pbool
  0}\rpar{} and \texttt{\small TRUE} is encoded as \lpar\cst{Pbool
  $n$}\rpar{} for any $n>0$. \emph{This allows to maintain the
  non\hyp{}determinism of the BER in the modeling.} A \type{code} is a
triple made of a \type{tag\_class}, a tag number (\emph{int}) and
\emph{contents}. The latter is either a primitive or a constructed
code. A constructed code is a list of codes.

\medskip

\textbf{Inference rules.} We define the encoding with a \emph{system
of inference rules}. These are logical implications $P_1 \wedge P_2
\wedge \ldots \wedge P_n \Rightarrow C$ graphically represented as
\begin{mathpar}
\inferrule{P_1\\ P_2\\ \ldots\\ P_n}{C} 
\end{mathpar}
where the $P_i$ are the \emph{premises} and $C$ is the
\emph{conclusion}. When there is no premise, $C$ is an \emph{axiom}
and is simply noted $C$. An inference rule can be interpreted also
from a computational point of view: in order to compute $C$, we need
to compute the $P_i$ first (order is not specified). The rules and
axioms can contain unquantified variables (\emph{free variables}). In
this case they are implicitly universally quantified ($\forall$) at
the beginning. For instance
\inferrule{P_1(x)\\ P_2(y)}{P(x,y)}\;\TirName{\textsc{Prop}} actually
denotes the property \textsc{Prop} which is $\forall x,y.P_1(x) \,
\wedge \, P_2(y) \Rightarrow P(x,y)$. A system of inference rules is
an unordered set of rules. A theorem is a judgement, i.e. a formal
statement. A demonstration is a proof tree whose root (the conclusion)
is the theorem, the inner nodes are the conclusions of its subtrees
and the leaves are axioms.

\medskip

%% \inferrule*[right=\quad \textsc{Int}]
%%   {}
%%   {\coding{\Gamma}{\Int \,
%%    (n)}{([\tau,p],\Integer)}{(\tau,\Primitive \, (\Pint))}
%%   }

\textbf{Abstract BER.} Let us note $\coding{\Gamma}{v}{(\Psi,\T)}{c}$
the judgement ``In the environment $\Gamma$, the value $v$ is encoded
into the code $c$, following the type \T{} with the tags $\Psi$.'' The
environment models the module and is mandatory because recursive types
are allowed, thus type references do exist. Given a type name $x$, the
referred type is $\Gamma(x)$. Using a system of inference rules to
define the encoding relation means that the successful encoding of a
value matches a proof tree made with the following rules:
\begin{mathpar}
\inferrule*[right=True]
  {n > 0}
  {\coding{\Gamma}{\True}{([\tau,p],\Boolean)}{(\tau, \Primitive \, (\Pbool \, n))}
  }
\and
\inferrule[Ref]
  {\coding{\Gamma}{v}{\Gamma(x)}{c}}
  {\coding{\Gamma}{v}{([\,],\TRef \, (x))}{c}}
\and
\inferrule
  {\textnormal{$\pi$ is a permutation on \type{components}}\\\\
   \coding{\Gamma}{v}{(\Psi,\SEQUENCE \; (\pi (\Phi)))}{c}}
  {\coding{\Gamma}{v}{(\Psi,\SET \; \Phi)}{c}}\;\TirName{Set}

\inferrule*[right=Tags]
  {\coding{\Gamma}{v}{(\Psi,\T)}{c}}
  {\coding{\Gamma}{v}{((\tau,\Explicit)\!::\!\Psi,\T)}
          {(\tau, \Constructed \, [c])}
  }
\end{mathpar}

\begin{mathpar}
\inferrule*[right=SeqOptOut]
  {\varphi = (l, \overline{\T},\Some \, \Optional)\\
   \coding{\Gamma}{\Seq \, M}{([\psi\,], \SEQUENCE \; \Phi)}{c}\\
  }
  {\coding{\Gamma}{\Seq \, ((l,v)\!::\!M)}
          {([\psi\,],\SEQUENCE \, (\varphi\!::\!\Phi))}{c}
  }
\end{mathpar}
\begin{mathpar}
\inferrule*[right=SeqOptIn]
  {\varphi = (l,\overline{\T},\Some \, \Optional)\\
   \coding{\Gamma}{v}{\overline{\T}}{c}\\
   \coding{\Gamma}{\Seq \, M}{([\psi], \SEQUENCE \; \Phi)}
          {(\tau,\Constructed \, C)}\\
   \overline{c} = (\tau, \Constructed \, (c\!::\!C))
  }
  {\coding{\Gamma}{\Seq \, ((l,v)\!::\!M)}
          {([\psi],\SEQUENCE \, (\varphi\!::\!\Phi))}{\overline{c}}}
\end{mathpar}

\medskip

Due to the lack of space, we only presented the more interesting
rules, of which we shall comment the conclusions before the
premises. Lists are noted between brackets and $a::A$ is a list whose
head is~$a$ and sub-list is~$A$. A pair is either noted $(a,b)$ or
$a,b$. Rule \textsc{True} illustrates a primitive encodings which is
non\hyp{}deterministic (variable $n$ is free). Pattern $[\tau,n]$
matches a list of a single element which is a pair whose first
projection is named $\tau$ and the second is named $n$. Since we
operate on \core, this tag is compulsorily the predefined
\texttt{\small UNIVERSAL} and \texttt{\small IMPLICIT} tag of
\texttt{\small INTEGER}. Rule \textsc{Ref} matches the encoding of a
type reference $\TRef (x)$ with no tags: we encode the referenced type
$\Gamma(x)$. Rule \textsc{Tags} apply when an \texttt{\small EXPLICIT}
tag occurs first. Note that $\Psi$ cannot be empty, i.e. [\,], since
an \texttt{\small IMPLICIT} tag only apply to a core type. Rule
\textsc{Set} models the non\hyp{}determinism of the BER with respect
to the \texttt{\small SET} type: any permutation of the sub-codes is
allowed.

\medskip

Rules \textsc{SeqOptOut} and \textsc{SeqOptIn} model another
non\hyp{}deterministic behaviour: a component value whose type is
\texttt{\small OPTIONAL} may not be encoded, as a sender's
option. Hence these \emph{two} rules have the same conclusion (it is
the only case), contrary to rule \textsc{Set} in which
non\hyp{}determinism is modeled by a free variable ($\pi$). We did not
model the encoding errors: at any time, given an environment $\Gamma$,
a tagged type $(\Psi, \T)$ and a value $v$, if no conclusion
$\coding{\Gamma}{v}{(\Psi,\T)}{\star}$ matches then it is a run-time
error (we can build no code $c$ in place of $\star$) and the
implementation must handle properly this situation in an unspecified
way. If the typing is statically done by the \ASN compiler, this
should not happen, but since we decided not to model the typing, the
typing is partly included in the encoding (i.e. at run-time).

\medskip

\textbf{Abstract decoding.} As we said in section~\ref{modeling}, the
BER decoding process is not published, is up to the \ASN compiler
implementors and can be modeled by a non\hyp{}injective function. We
propose the following equational definition we expect to be
faithful. Let us note $\decoding{\Gamma}{c}{(\Psi,\T)}$ the decoding
of $c$ in the environment $\Gamma$ according to type \T{} tagged
$\Psi$.

\begin{gather*}
\decoding{\Gamma}{(((\Universal,1), \Primitive (\Pbool \,
0)))}{([\,],\Boolean)} = \False\\
\begin{align*}
\decoding{\Gamma}{(((\Universal,1), \Primitive (\Pbool \,
n)))}{([\,],&\Boolean)} = \True\\
& \ \textnormal{for all} \ n > 0
\end{align*}\\
\decoding{\Gamma}{c}{([\,],\TRef \, (x))} = \decoding{\Gamma}{c}{\Gamma(x)}\\
\decoding{\Gamma}{(\tau,\Constructed \,
  [c])}{((\tau,\Explicit)\!::\!\Psi,\T)} =
\decoding{\Gamma}{c}{(\Psi,\T)}\\
\begin{align*}
\decoding{\Gamma}{(\tau,\kappa)}{([\,],\Choice \, F)} = & \ 
\decoding{\Gamma}{(\tau,\kappa)}{F(l)}\\
& \ \textnormal{where} \ F(l) = ((\tau,m)\!::\!\Psi,\T) 
\end{align*}
\end{gather*}

We do not provide the full definition for we lack of space and do not
wish to drown the reader into too much technical details anyway.
