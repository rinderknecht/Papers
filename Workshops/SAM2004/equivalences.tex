%%-*-latex-*-

\section{Equivalences}

\textbf{Value equivalence.} It is possible to present a complete
definition of the value equivalence because we shaped \core with this
goal in mind. We note $A@B$ the catenation of lists~$A$
and~$B$. We have
\begin{mathpar}
\inferrule
  {}
  {v \approx v}\quad\TirName{Reflexivity}
\and
\inferrule
  {v_1 \approx v_2\\
   \Seq \, M_1 \approx \Seq \, M_2}
  {\Seq \, ((l,v_1)\!::\!M_1) \approx \Seq \,
  ((l,v_2)\!::\!M_2)}\;\TirName{Seq}
\and
\inferrule[Transitivity]
  {v_1 \approx v_2\\
   v_2 \approx v_3}
  {v_1 \approx v_3}%\;\TirName{Trans}
\and
\inferrule
  {\exists l,v_2,M'_2,M_2.M=M'_2 @\, (l,v_2)\!::\!M_2\\\\
    v_1 \approx v_2\\
   \Set \, M_1 \approx \Set \, M_2}
  {\Set ((l,v_1)\!::\!M_1) \approx \Set \, M}\;\TirName{Set}
\and
\inferrule
  {v_1 \approx v_2}
  {v_2 \approx v_1}\;\TirName{Symmetry}
\and
\inferrule
  {v_1 \approx v_2}
  {\Chosen \, (l,v_1) \approx \Chosen \, (l,v_2)}\;\TirName{Choice}
\and
\inferrule*[right=SeqOf]
  {v_1 \approx v_2\\ \SeqOf \, V_1 \approx \SeqOf \, V_2}
  {\SeqOf \, (v_1\!::\!V_1) \approx \SeqOf \, (v_2\!::\!V_2)}
\and
\inferrule*[right=SetOf]
  {\exists v_2,V_2,V'_2.V=V'_2 @\, v_2\!::\!V_2\\
   v_1 \approx v_2\\
   \SetOf \, V_1 \approx \SetOf \, (V'_2 @\, V_2)} 
  {\SetOf \, (v_1\!::\!V_1) \approx \SetOf \, V}
\end{mathpar}
Our value equivalence amounts to a structural equality modulo
permutations on sub-values of \texttt{\small SET} and \texttt{\small
SET OF} types.

\medskip

\textbf{Code equivalence.} The BER embed a lot of the type information
into the codes through the use of tags and a structure isomorphic to
types. This makes possible to define an equivalence relationship
between codes that relies on two codes only --- no further context is
needed.
\begin{mathpar}
\inferrule[Reflexivity]
  {}
  {c \sim c}
\and
\inferrule[Symmetry]
  {c_1 \sim c_2}
  {c_2 \sim c_1}
\and
\inferrule[Transitivity]
  {c_1 \sim c_2\\ c_2 \sim c_3}
  {c_1 \sim c_3}
\and
\inferrule*[right=True]
  {m > 0\\ n > 0}
  {(\tau, \Primitive \, (\Pbool \, m)) \sim (\tau, \Primitive \,
  (\Pbool \, n))}
\and
\inferrule*[right=Seq/SeqOf]
  {\tau = (\Universal, 16)\\
   c_1 \sim c_2\\\\
   (\tau, \Constructed \, C_1) \sim (\tau, \Constructed \, C_2)}
  {(\tau, \Constructed \, (c_1\!::\!C_1)) 
   \sim (\tau, \Constructed \, (c_2\!::\!C_2))}
\and
\inferrule*[right=SeqOptOut]
  {\tau = (\Universal, 16)\\
   (\tau, \Constructed \, C_1) \sim (\tau, \Constructed \, C_2)}
  {(\tau, \Constructed \, (c_1\!::\!C_1)) \sim (\tau, \Constructed \, C_2)}
\end{mathpar}
Contrary to value equivalence, there are two many cases and hence we
cannot present them all. Rule \textsc{True} defines the equivalence of
two possibly different encodings of the value \texttt{\small
TRUE}. Rule \textsc{Seq/SetOf} specifies when (and how, in fact) codes
from values of types \texttt{\small SEQUENCE} and \texttt{\small
SEQUENCE OF} are equivalent. By the way, note that the tags of these
two types are identical, hence, in theory, this rule makes equivalent
the encodings of, say, values of types \texttt{\small SEQUENCE \{a
INTEGER\}} and \texttt{\small SEQUENCE OF INTEGER}, as soon as the
integer value is the same. Rule \textsc{SeqOptOut} is dual to the
homonym rule of the abstract BER where an optional value component is
not encoded. Here, it is allowed to skip a sub-code when
decoding. \emph{We do not specify when a sub-code has to be skipped or
in which code.} We leave this to a more refined specification and/or
algorithm.

\medskip

\textbf{Equivalence properties.} The properties we expect to hold in
our BER model can now be restated in a formal way. First of all,
proposition~\ref{all_BER_codes}, which states that all the BER
encodings of a given value, according to a given type, are equivalent,
becomes through the use of formal notations:

\begin{proposition}
If $\coding{\Gamma}{v}{\overline{\T}}{c_1}$ and
$\coding{\Gamma}{v}{\overline{\T}}{c_2}$ then $c_1 \sim c_2$.
\end{proposition}

Next, proposition~\ref{entailment} which states that the decoding of
two equivalent codes lead to two equivalent values is now restated in
the following way:

\begin{proposition}[Equivalence entailment]\\
$c_1 \sim c_2 \Longrightarrow \decoding{\Gamma}{c_1}{\overline{\T}}
  \approx \decoding{\Gamma}{c_2}{\overline{\T}}$
\end{proposition}

Finally, the soundness theorem~\ref{soundness}, which says that the
encoding and decoding of a \core value $v$, following a \core tagged
type $\overline{\T}$, leads to a value which is equivalent to $v$, is
now formally rephrased:

\begin{theorem}[Soundness]
If $\coding{\Gamma}{v}{\overline{\T}}{c}$ then $v \approx
\decoding{\Gamma}{c}{\overline{\T}}$.
\end{theorem}

We have no room to show the proofs of these properties because they
contain a great number of cases. One tricky aspect is the correct
handling of sub-code permutations when dealing with \texttt{\small SET
OF} and \texttt{\small SET} values: for a given unknown permutation on
the sender's side, we must explicitly construct the reverse
permutation on the receiver's side.
