%%-*-latex-*-

\abstract{The Abstract Syntax Notation One (\ASN) can be used to model
types of values carried by signals in SDL or MSC but is also directly
used by network protocol implementors. In the last few years, the
press has reported several alleged vulnerabilities of \ASN and the
Basic Encoding Rules (BER) related to network protocols like SNMP and,
more recently, OpenSSL. In reality it has been shown that the security
issues (theoritically denial of service attacks) were due to
low-quality and poorly-tested compiler implementations. We use some
formal methods to go further. We review formally the design of the BER
themselves and prove that, under some assumptions, it is flawless
whatever the network protocol is and whatever the values to be
transmitted are. More precisely, we start with a formal modeling of
the BER which abstracts away low-level details but captures the design
principles. Then we define a soundness property stating that the
composition of encoding and decoding yields a value which is
equivalent to the original. Finally we prove that this property holds
for all values specified with \ASN.}
