%%-*-latex-*-

\section{Introduction}

The wide variety of software and hardware architectures in distributed
systems and telecommunications makes it valuable to use a common
high-level data notation in protocol specifications. To fulfill this
need, the ISO organization and the International Telecommunication
Union (ITU) defined the Abstract Syntax Notation One (\ASN) series of
standards. \ASN~\cite{Dubuisson:2000,X.680:2002,X.681:2002,X.682:2002,X.683:2002}
is a language for data types allowing the protocol designer to capture
numerous networking concepts, such as protocol data units, without
worrying about the possible environment and implementation
heterogeneity of the peers. The peers share a set of \ASN modules and
agree upon a set of \emph{encoding rules}, such
as~\cite{X.690:2002,X.691:2002}, which is a method for encoding values
produced at run-time by the communicating applications, into series of
bits. \ASN has been adopted for a wide range of applications, such as
network management, secure e-mail, mobile telephony, air traffic
control etc.
%, video conferencing over the internet, electronic commerce,
%digital certificates, radio paging, financial service systems etc.

\medskip

In the last few years, the press has reported several alleged
vulnerabilities of \ASN and the Basic Encoding Rules (BER) related to
network protocols like SNMP and, more recently, OpenSSL. Each time, an
accurate description of the problem has been finally published,
showing that the weakness lay in \emph{implementations} poorly written
and insufficiently tested. The real vulnerabilities were almost all
related to improper decoding of ill-formed BER encodings (or
\emph{codes}) causing buffer overflows, unspecified
(non-deterministic) behaviours, stack corruptions and, in the end, a
possible denial of service.

\medskip

From now on, it is important to understand and remember that \ASN and
the BER, intrinsically, have nothing to do with security or
cryptographic protocols. Both are used for modeling and handling the
data part of protocols, not the control. As a consequence, the
soundness property we aim at in this article must not be considered as
a security property about \emph{control} but as mere correctness of
composition of encoding and decoding with the BER of \emph{values}
specified by means of \ASN. For instance, there are no attackers, no
nonces etc. here. Nevertheless, the difficulty is not lesser.

\medskip

More precisely, in this work we want to prove that the design of the
BER themselves is flawless, whatever the network protocol is and
whatever the values to be transmitted are. To achieve this goal we
need the support of formal methods. We start by a formal modeling of
the BER which abstracts away low-level details but captures the design
principles. Then we define a soundness property representing the
security warranty we require and finally we prove that this property
holds for all values that can be specified with \ASN.

