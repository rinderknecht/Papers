%%-*-latex-*-

\section{Core \ASN}

\ASN syntax is involved because it aims at allowing the specification
of as many networking concepts as possible. For instance types, values
and subtyping constraints may depend on each other: a type may contain
constraints (on components) and values (\emph{e.g.,} default values),
a value has a type and constraints rely upon types (\emph{e.g.,}
inclusion constraint) and values (\emph{e.g.,} value constraint). We
define \core such that

\medskip

\begin{itemize}

  \bitem the default tagging mode of the module is \texttt{\small EXPLICIT
  TAGS};

  \bitem tags obey the standard rules, like alternative types in
  \texttt{\small CHOICE} having distinct tags etc.;

  \bitem tags appear only at the top-level, i.e. just after the symbol
  `{\small \verb+::=+}';

  \bitem the built-in types are explicitly tagged \texttt{\small
  IMPLICIT} and \texttt{\small UNIVERSAL};

  \bitem tags are explicitly either \texttt{\small IMPLICIT} or
  \texttt{EXPLICIT};

  \bitem \texttt{\small IMPLICIT} tags apply only to untagged
  types;

  \bitem tag values are numeric \texttt{\small INTEGER} values (not
  value references);

  \bitem component types are references and references are component
  types;

  \bitem there are no \texttt{DEFAULT} component types;

  \bitem there is no \texttt{\small COMPONENTS OF} clause;

  \bitem constraints appear only at the top-level (not in component
  types);

  \bitem there are no \texttt{\small ABSENT}, \texttt{\small PRESENT}
  or \texttt{\small OPTIONAL} component constraints;

  \bitem there is no selection type, \emph{e.g.,} no {\small \verb+T ::= i < U+};

  \bitem \texttt{\small SET OF} and \texttt{\small SEQUENCE OF} apply
  to references;

  \bitem the \texttt{\small BIT STRING} and \texttt{\small INTEGER}
  type do not define constants, \emph{e.g.,} no {\small \verb+INTEGER {c(1)}+}
  or {\small \verb+BIT STRING {a(x)}+};

  \bitem the only \texttt{\small BIT STRING} values are series of
  bits, \emph{e.g.,} \verb+'1110'B+;

  \bitem \texttt{\small ENUMERATED} types define constants with
  explicit numeric integers;

  \bitem \texttt{\small REAL} values are not legal tokens for
  \texttt{\small INTEGER} values and conversely (\emph{e.g.,} \texttt{\small
  0} is \emph{only} of type \texttt{\small INTEGER});
  
  \bitem \texttt{\small REAL} values do not use the
  \emph{(mantissa, base, exponent)} form;

  \bitem there are no references in values (thus no recursive values).

\end{itemize}

\medskip

We relax the first assumption we made in section~\ref{modeling} and
assume now that we have one \ASN module, syntactically correct with
respect to \mbox{X.680}. It is reduced to \core by applying the
following series of rewritings which do not commute in general.  For
we lack of room to give the formal rewriting rules, we only illustrate
the process on short examples.

\medskip

\begin{enumerate}

  \item \label{selection} We remove the selection types, taking care
  of tags:

        \begin{center}
        {\small
          $\left\{
             \begin{tabular}{l}
                \verb+A ::= [0] i < [1] B+\\
                \verb+B ::= [2] C+\\
                \verb+C ::= [3] CHOICE{i [4]D}+\\
                \verb+D ::= [5] INTEGER+
             \end{tabular}
            \right.
          \rightarrow
          \left\{
               \begin{tabular}{l}
                  \verb+A ::= [0][4][5] INTEGER+\\
                  \verb+B ::= [2] C+\\
                  \verb+C ::= [3] CHOICE{i [4]D}+\\
                  \verb+D ::= [5] INTEGER+
               \end{tabular}
             \right.$
        }
        \end{center}

        Note that the selection types that do not define a unique type
        lead to recursive type definitions whose pattern is
        \texttt{\small X ::= X}, as in

        \begin{center}
        {\small 
        \verb+T ::= CHOICE {a a < T}+\\
        $\longrightarrow
         \left\{
            \begin{tabular}{l}
               \verb+T ::= CHOICE {a A}+\\
               \verb+A ::= a < T+
            \end{tabular}
          \right.
         \longrightarrow
         \left\{
             \begin{tabular}{l}
                \verb+T ::= CHOICE {a A}+\\
                \verb+A ::= A+
             \end{tabular}
          \right.$
        }
        \end{center}

  \item \label{type_references}
        The top-level type references are unfolded, i.e. the type
        references at the declaration level are replaced by the type
        they reference, as in

        \begin{center}
        {\small
         $\left\{
             \begin{tabular}{l}
                \verb+T ::= U (C)+\\
                \verb+U ::= REAL (D)+
             \end{tabular}
          \right.
          \longrightarrow
          \left\{
             \begin{tabular}{l}
                \verb+T ::= REAL (D ^ C)+\\
                \verb+U ::= REAL (D)+
             \end{tabular}
          \right.$}
        \end{center}

Beware of the case of constrained references
        to \texttt{\small SET OF} types:

        \begin{center}
        {\small
         $\left\{
            \begin{tabular}{l}
              \verb+A ::= SET OF C+\\
              \verb+B ::= A (SIZE (7))+
            \end{tabular}
          \right.
          \longrightarrow
          \left\{
            \begin{tabular}{l}
              \verb+A ::= SET OF C+\\ 
              \verb+B ::= SET (SIZE (7)) OF C+
            \end{tabular}
          \right.$}
        \end{center}

       The result {\small \verb+B ::= SET OF C (SIZE (7))+} would be
       wrong!

       This step is difficult because it removes all recursive types
       declarations that do not lead to a uniquely defined type, like
       \texttt{\small T ::= T} or {\small
       \verb+T ::= CHOICE {a a < T}+} etc. (See step~\ref{selection}.)

  \item The default values are expanded and the \texttt{DEFAULT}
  annotation is replaced by \texttt{OPTIONAL}, like in the following
  example

        \begin{center}
        {\small
         $\left\{
            \begin{tabular}{l}
               \verb+v T ::= {}+\\
               \verb+T ::= SET {a U DEFAULT w}+
            \end{tabular}
          \right.
          \rightarrow
          \left\{
             \begin{tabular}{l}
               \verb+v T ::= {a w}+\\
               \verb+T ::= SET {a U OPTIONAL}+
             \end{tabular}
           \right.$}
        \end{center}

  \item \label{COMPONENTS_OF_unfolding} The \texttt{\small COMPONENTS
  OF} clauses are expanded:

        \begin{center}
        {\small
         $\left\{
            \begin{tabular}{l}
               \verb+T ::= SET {COMPONENTS OF [6] A}+\\
               \verb+A ::= SET {a REAL}+
            \end{tabular}
          \right.
         \rightarrow
         \left\{
            \begin{tabular}{l}
              \verb+T ::= SET {a REAL}+\\
              \verb+A ::= SET {a REAL}+
            \end{tabular}
          \right.$}
         \end{center}

        If the tagging mode is \texttt{\small AUTOMATIC TAGS}, we must
        \emph{previously} compute the current component tags and then
        insert the components referred by \texttt{\small
        COMPONENTS OF}.

        \begin{center}
        {\small
         $\left\{
          \begin{tabular}{l}
             \verb+PDU DEFINITIONS AUTOMATIC TAGS ::=+\\
             \verb+  A ::= SET {a SET OF B, COMPONENTS OF B}+\\
             \verb+  B ::= SET {b [2] INTEGER}+\\
             \verb+END+
          \end{tabular}
          \right.$
          $\longrightarrow
          \left\{
          \begin{tabular}{l}
             \verb+PDU DEFINITIONS AUTOMATIC TAGS ::=+\\
             \verb+  A ::= SET {a [0] SET OF B, b [1][2] INTEGER}+\\
             \verb+  B ::= SET {b [2] INTEGER}+\\
             \verb+END+
          \end{tabular}
          \right.$
        }
        \end{center}

  \item \label{constants_unfolding}
        \texttt{\small INTEGER} and \texttt{\small BIT STRING}
        constants are replaced by their definition and removed from
        their defining type:

        \begin{center}
        {\small
           $\left\{
              \begin{tabular}{l}
                 \verb+T ::= INTEGER {c(x)}+\\
                 \verb+v T ::= c+
              \end{tabular}
            \right.
           \longrightarrow
           \left\{
              \begin{tabular}{l}
                \verb+T ::= INTEGER+\\
                \verb+v T ::= x+
              \end{tabular}
           \right.$}
        \end{center}

        This step may reveal some recursive values, as in

        \begin{center}
        {\small
           $\left\{
              \begin{tabular}{l}
                 \verb+T ::= INTEGER {c(v)}+\\
                 \verb+v T ::= c+
              \end{tabular}
            \right.
           \longrightarrow
           \left\{
              \begin{tabular}{l}
                \verb+T ::= INTEGER+\\
                \verb+v T ::= v+
              \end{tabular}
            \right.$}
        \end{center}

  \item For \texttt{\small BIT STRING} values which are specified by
  means of a series of bit names, we unfold their associated
  references and replace the value by an equivalent string of bits:

  \begin{center}
        {\small
           $\left\{
              \begin{tabular}{l}
                \verb+T ::= BIT STRING {msb(x),lsb(y)}+\\
                \verb+v T ::= {msb,lsb}+
              \end{tabular}
            \right.\!\!
            \rightarrow\!
            \left\{
              \begin{tabular}{l}
                \verb+T ::= BIT STRING+\\
                \verb+v T ::= '10000001'B+
              \end{tabular}
            \right.$
        }
  \end{center}

   assuming the excerpt {\small \verb+  x INTEGER ::= 7  y INTEGER ::= 0+}

   Also, values in hexadecimal form are translated into binary form:
   \begin{center}
     {\small \verb+x U ::= 'A'H+ $\longrightarrow$
        \verb+x U ::= '1010'B+}
   \end{center}

  \item \label{rec_values} We unfold the value references, disallowing
  at the same time the recursive values, like {\small
  \verb+v T ::= {v}+}

  \item We unfold the {\small \verb+ENUMERATED+} constants and add the
  missing integers:

  \begin{center}
  {\small
   $\left\{
    \begin{tabular}{l}
      \verb+T ::= ENUMERATED{a(v),b}+\\
      \verb+v INTEGER ::= 3+
    \end{tabular}
    \right.
    \!\!\rightarrow\!\!
    \left\{
    \begin{tabular}{l}
      \verb+T ::= ENUMERATED{a(3),b(4)}+\\
      \verb+v INTEGER ::= 3+
    \end{tabular}
    \right.$    
  }  
  \end{center}

  \item We unfold the tag values (this always terminates because there
  are no more recursive values since step~\ref{rec_values}), checking
  that they are syntactically integers:

  \begin{center}
  {\small 
   $\left\{
    \begin{tabular}{l}
      \verb+T ::= [APPLICATION v] IMPLICIT REAL+\\
      \verb+v INTEGER ::= 3+
    \end{tabular}
    \right.$
    $\longrightarrow
    \left\{
    \begin{tabular}{l}
      \verb+T ::= [APPLICATION 3] IMPLICIT REAL+\\
      \verb+v INTEGER ::= 3+
    \end{tabular}
    \right.$
  }
  \end{center} 


  \item The tagging mode becomes \texttt{EXPLICIT TAGS}, like

  \begin{center}
   {\small
    $\left\{
     \begin{tabular}{l}
       \verb+PDU DEFINITIONS IMPLICIT TAGS ::=+\\
       \verb+  A ::= SET {a [0] SET OF B}+\\
       \verb+  B ::= [1] CHOICE {b [2] REAL}+\\
       \verb+END+
     \end{tabular}
     \right.$
    $\longrightarrow
     \left\{ 
     \begin{tabular}{l}
       \verb+PDU DEFINITIONS EXPLICIT TAGS ::=+\\
       \verb+  A ::= SET {a [0] IMPLICIT SET OF B}+\\
       \verb+  B ::= [1] EXPLICIT CHOICE {b [2] IMPLICIT REAL}+\\
       \verb+END+
     \end{tabular}
     \right.$
   }
  \end{center}

  \item We make explicit the tags of the built-in types:
   \begin{center}
    {\small 
     \verb+A ::= INTEGER+ $\longrightarrow$
     \verb+A ::= [UNIVERSAL 2] IMPLICIT INTEGER+
    }
   \end{center}

  \item We reduce the \texttt{\small IMPLICIT} tags, as
   \begin{center}
    \small\tt
    T ::= [0] IMPLICIT [1] EXPLICIT [UNIVERSAL 9] IMPLICIT REAL
    $\longrightarrow$ T ::= [0] IMPLICIT REAL
   \end{center}
   
  \item \label{structural_constraints} We apply and reduce the
  structural subtyping constraints \texttt{\small ABSENT},
  \texttt{\small PRESENT} and \texttt{\small OPTIONAL}, like
  \begin{alltt}
  \small
T ::= CHOICE \{a REAL, b REAL\}
             (WITH COMPONENTS \{a(PRESENT)\})
  \(\longrightarrow\) T ::= CHOICE \{a REAL\}
  \end{alltt}
  (General case complex but tractable.)

\end{enumerate}

\medskip

It is important to understand that in \core it is still possible that

\begin{enumerate}
  
  \bitem \label{finiteness} types have only infinite values:
  \texttt{\small T ::= SET} \verb+{+\texttt{a} \texttt{\small
  T}\verb+}+

  \bitem \label{typing} values are ill-typed: \texttt{v}
  \texttt{\small REAL ::= ""}

  \bitem values do not conform to all additional \mbox{X.680}
  requirements, like
  \begin{center}
  {\small
    $\left\{
    \begin{tabular}{l}
      \verb+T ::= SEQUENCE {a BOOLEAN, b INTEGER}+\\
      \verb+t T ::= {b 7, a TRUE}   -- illegal+
    \end{tabular}
    \right.$
  }
  \end{center}

  \bitem \label{constraint_consistence} subtyping constraints are
  inconsistent: \texttt{\small T ::= REAL (SIZE(7))}

  \bitem \label{subtype_non_emptiness} subtypes are empty:\\
  {\small \verb+T ::= SET ((SIZE(1))^(SIZE(2))) OF REAL+}
        
  \bitem \label{solvability} subtypes have no value set:
  \texttt{\small T ::= REAL (ALL EXCEPT T)}

\end{enumerate}

The reason why this is not a problem is that \core has been defined
with the BER modeling in mind, in particular we do not aim here at a
full validation of \ASN.

\medskip

\textbf{Abstract grammar.} We formally define the constructs of \core
by means of an \emph{abstract grammar} implemented with the algebraic
data types of the functional programming language OCaml
\cite{ChaillouxManouryPagano:2000}, which is a full-fledged
programming language, as well as, historically, a logic
meta-language. The \core parser output is a pair of a type environment
and a value environment. The former is a mapping from type names to
types, corresponding to the type declarations in the \ASN
specification, and the latter is a mapping from value names to values,
corresponding to the value declarations. The types and values are
\emph{abstract syntax trees}, complying with the abstract grammar. We
except from the abstract grammar the \texttt{\small OBJECT IDENTIFIER}
and \texttt{\small RELATIVE-OID} types and values for the sake of
brevity. We also ignore the extension markers and the subtyping
constraints beacause they play no role in the
BER~\cite[\S{8.1.1.4}]{X.690:2002} (however we considered some
constraints at step~\ref{structural_constraints}).

\medskip

\textbf{Values.} The abstract grammar for \core values is defined as
follows. Firstly, we assume that the parser removes the ambiguity
between enumeration constants~\cite[\S{19}]{X.680:2002} and value
references~\cite[\S{11.4}]{X.680:2002}. For instance, in \texttt{a}
\texttt{\small T} \verb+::=+ \texttt{b}, the token \texttt{b} can
denote either an enumeration constant or a value reference, depending
on the definition of type \texttt{\small T}. The ambiguity can always
be removed just by looking at the type definition (this is easy in
\core). The type \ocamltypename{item} is used later in the enumerated
constants and the type \ocamltypename{label} denotes component names.

\smallskip

\noindent 
\Xtype \type{item} \equal \type{string} \Xand \type{label} \equal
\type{string}

\noindent 
\Xtype \type{core\_value} \equal \lbra\SetOf{} \Xof \type{core\_value}
\type{list} \vbar~\SeqOf{} \Xof \type{core\_value} \type{list}
\vbar~\Set{} \Xof \lpar\type{label} $\times$ \type{core\_value}\rpar{}
\type{list} \vbar~\Seq{} \Xof \lpar\type{label} $\times$
\type{core\_value}\rpar{} \type{list} \vbar~\True{} \vbar~\False{}
\vbar~\Enum{} \Xof \type{item} \vbar~\Int{} \Xof \type{int}
\vbar~\Real{} \Xof \type{float} \vbar~\Null{} \vbar~\MinusInfinity{}
\vbar~\Chosen{} \Xof \type{label} $\times$ \type{core\_value}
\vbar~\pvString{} \Xof \type{string} \vbar~\BitStr{} \Xof \type{bool
array} \vbar~\PlusInfinity{}\rbra

\smallskip

\noindent
where \SetOf{} corresponds to values of \texttt{\small SET OF} and
\SeqOf{} to values of \texttt{\small SEQUENCE OF} types~\cite[\S{25},
\S{27}]{X.680:2002}; \Set{} models values of the \texttt{\small SET}
type and \Seq{} models values of \texttt{\small SEQUENCE}
types~\cite[\S{24}, \S{26}]{X.680:2002} (the argument is a mapping
from labels to values); \True{} and \False{} are obvious; \Enum{}
models enumerated constants; \Int{} and \Real{} stand for
\texttt{\small INTEGER} and \texttt{\small REAL} values (for
simplicity, we assume they fit the built-in arithmetic of OCaml);
\Null{} models the special \texttt{\small NULL}
value~\cite[\S{23}]{X.680:2002}; \PlusInfinity{} and \MinusInfinity{}
correspond to \texttt{\small PLUS-INFINITY} and \texttt{\small
MINUS-INFI\-NI\-TY}; \Chosen{} corresponds to \texttt{\small CHOICE}
values~\cite[\S{28}]{X.680:2002} (thus its argument is a pair of a
label and a value); \pvString{} concentrates all kinds of character
strings; \BitStr{} represents \texttt{\small BIT STRING}
constants~\cite[\S{21}]{X.680:2002} and \texttt{\small OCTET STRING}
values~\cite[\S{22}]{X.680:2002}. OCaml values of type
\ident{core\_value} will be noted $v$. 
%The argument of the \Set{} and \Seq{} constructors is noted $\IdEnv{}$.

\smallskip

\Xtype \type{tagged\_type} \equal \type{tag} \type{list} $\times$
\type{core\_type}\\
\Xand \type{tag} \equal \lpar\type{tag\_class} $\times$
\type{int}\rpar{} $\times$ \type{tag\_mode}\\
\Xand \type{tag\_class} \equal \Universal{} \vbar~\Private{}
\vbar~\Application{} \vbar~\Context\\ 
\Xand \type{tag\_mode} \equal \Explicit \vbar~\Implicit\\
\Xand \type{core\_type} \equal \lbra\Choice{} \Xof
\type{label} $\rightarrow$ \type{tagged\_type} 
\vbar~\OctetString{}\\
\vbar~\SET{} \Xof \type{components} 
\vbar~\SEQUENCE{} \Xof \type{components} 
\vbar~\BitString{}\\
\vbar~\SETOF{} \Xof \type{tagged\_type} 
\vbar~\SEQUENCEOF{} \Xof \type{tagged\_type}
\vbar \Null{}\\
\vbar~\Enumerated{} \Xof \type{item} $\rightarrow$ \type{int}
\vbar~\Integer{} 
\vbar~\Boolean{} 
\vbar~\REAL{}\\
\vbar \pvString{} 
\vbar~\TRef{} \Xof \type{string}\rbra\\ 
\Xand \type{components} \equal \lpar\type{label} $\times$
\type{tagged\_type} $\times$ \lbra\Optional\rbra{}
\type{option}\rpar{} \type{list}

\smallskip

The type \type{tagged\_type} models the tagged types of \core, in
which a type (\type{core\_type}) can be preceded by a list of
tags. Constructor names of type \type{core\_type} are almost
self-explanatory, except \TRef{} which denotes type references. The
type \ocamltypename{components} defines the components of
\texttt{\small SET} and \texttt{\small SEQUENCE} \core types: it is a
triple made of a label, a tagged type and an optional \texttt{\small
OPTIONAL} component's attribute. OCaml values of type
\type{core\_type} are noted \T{} and \type{tagged\_type} values
$\overline{\T}$. The mapping of type $\type{label} \rightarrow
\type{tagged\_type}$, which is the argument of \Choice, is noted
$F$. Values of type \type{components} are lists noted $\Phi$ of
components noted $\varphi$, \emph{e.g.,} $\SET \, (\varphi\!::\!\Phi)$. An
\ASN module is modeled by a type environment which is modeled by a
function $\Gamma$ from type names to tagged types, since there are no
more value references in \core. Values of type \type{tag} are noted
$\psi$ and lists of tags $\Psi$.
